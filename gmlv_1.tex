\chapter{Нэгдүгээр хэсэг}

\section{ХАРАГДАХ БОЛОН ҮЛ ХАРАГДАХ ЗҮЙЛИЙН ТУХАЙ}
ЭНД ӨНГӨЦ АЖИГЛАЛТ НЬ СУДЛААЧИЙГ БУРУУ ЗАМ РУУ ОРУУЛДАГ ГЭДГИЙГ БАТЛАХ БА ӨӨРИЙГӨӨ ХЯНАХ БОЛОН ӨӨРИЙГӨӨ ШАЛГАХ АРГУУДЫГ САНАЛ БОЛГОНО
Орчуулагчийн бяцхан тайлбар: Энэ болон удаах бүх бүлгүүдэд Хүннү болон Сяньби гэсэн хятад гаралтай дэлхий нийтэд түгсэн нэр томъёог би ХҮН хийгээд СҮМБЭ гэж орчуулсан. Энэ нь Б.Даш-Ёндон багш Л.Н.Гумилевыг орчуулах үедээ анх гаргасан санаа бөгөөд надад их таалагддаг юм. Хүн гүрэн, Сүмбэ улс гэхэд илүү сонсголонтой байдаг юм.
I. Угсаатны судлалын ашиг тусын тухай
УГСААТНЫ ТӨСГҮЙ БАЙДАЛ
Ямар нэг ард түмэн эх орондоо удаан бөгөөд тайван амьдарвал түүний төлөөлөгчдөд амьдралын арга зан авир, зан үйл, сонирхол, үзэл бодол, социал харилцамж, өөрөөр хэлбэл эдүгээ “зан үйлийн тогтсон үзэл” хэмээн нэрлэж буй бүх зүйлс нь цорын ганц боломжтой бөгөөд зөв мэт санагддаг.
1. Овгийн холбоо, ястан, үндэстэн, ард түмэн зэрэг эдгээр ойлголтуудыг этнологид угсаатан гэсэн нэр томъёогоор тэмдэглэдэг бөгөөд үүнийг тайлбарлахад энэхүү номыг зориулж байгаа юм. Дурын нэр томъёоны ач холбогдлын тухайд тохиролцоход амархан боловч маш бага зүйл өгдөг. Энэ нь судлаачийн анхдагч байр суурь л болж чадна. Нэр томъёог нээж олох нь хүнд, учир нь энэ бол байгал болон түүхэнд энэхүү үзэгдлийн байр суурийг үзүүлнэ гэсэн үг юм. Надад “Энгийнээр яриач” гэхэд нь би “Тэгвэл гэрэл гэж юу вэ? Энгийнээр хэлээч” гэдэг юм. Үүнд хэн ч хариулдаггүй юм. Чухам ийм учраас би уншигчаас миний хэллэгийн хүнд байдлыг уучилж, номыг маань юугий нь ч алгасалгүй эхнээс дуустал уншихыг хүсч байна.
Хэрэв хаа нэгэн газар ямар нэг гажилт болбол зүгээр л өөртэйгөө төсгүй болсон “боловсролгүй” байдлаас болсон гэж үзнэ. Намайг жааханд Майн Ридийг сонирхон шүтэж байхад нэгэн соёлтой хатагтай “Негрүүд манайхантай адилхан эрчүүд шүү дээ, тэд зөвхөн хар л байдаг” гэж билээ. Малаитын эргийн меланезийн удган мөн л ийм үндэслэлээр “Англичууд бидний л адил толгой агнадаг, тэд зөвхөн цагаан байдаг” гэж хэлж чадна гэдэг нь түүний толгойд ч орохгүй юм. Энгийн дүгнэлт нь бодит байдлыг үгүйсгэхэд үндэслэгддэг хэдий ч заримдаа дотоод учир зүйтэй мэт санагдана. Гэхдээ энэ нь асар удалгүй өөр зүйлтэй тулгарахдаа сарнидаг.
Баруун Европын Дундад зууны шинжлэх ухаанд угсаатны зүй тийм чухал биш байв. Европчуудын өөр соёлуудтай харилцах явдал нь “Римийн эзэнт гүрний” хамжлагуудын үр удам амьдардаг, ислам руу хэсэгчлэн эргэсэн Газрын дундад тэнгисийн сав газар байв. Энэ нь мэдээжийн хэрэг тэднийг “франк”, “латин”, өөрөөр хэлбэл франц, итали гэж хоёр хуваасан бөгөөд гэхдээ соёлын нийтлэг үндэс ялгаа нь харилцан ойлголтыг үгүйсгэхээргүй бага байлгаж байв. Гэвч газар зүйн их нээлтийн эрин үед эл байдал сууриараа өөрчлөгдөв. Хэрэв негр, папуас болон хойд америкийн индианчуудыг “зэрлэгүүд” гэж нэрлэж болж байсан ч энэ үгийг хятад, энэтхэг, ацтек, инкүүдийн тухайд хэлж болохгүй байв. Өөр тайлбарлал хайх хэрэгтэй болов.
XYI зуунд Европын аялагчид алс холын орон газрыг нээж, эндээс хүссэн ч, хүсээгүй ч өөрсдийн амьдралын дассан хэлбэртэй адил зүйлийг хайх болов. Испаний конкистадорууд индианчуудын касик гэдэг овгийн удирдагчид загалмай бүхий “дон” гэдэг цолыг олгож, тэднийгээ индианчуудын тайж хэмээн үзэж байв. Негрүүдийн овгийн ахлагч “хаан” гэсэн цол ч авч байв. Тунгуссын бөө нар хэдийгээр жирийн эмч нар байж, өвчний шалтгааныг зэрлэг амьтан, харь гаригийнхан шиг материаллаг оршин буй муу ёрын “сүнс”–ний нөлөөллөөс хайдаг атал бөө нарыг санваартан гэж үзэж байв.
Харилцан үл ойлголцол ойлгоод байх юм юу ч үгүй гэсэн итгэлтэйгээр гүнзгийрч, уугуулуудын мэдрэхүйг доромжилсон, европчуудыг алан хядах явдлыг төрүүлсэн зөрчилдөөн үүссэн юм. Үүний хариуд англи, францчууд харгис хатуу залхаан цээрлүүлэх экспедици зохион байгуулав.
Австралийн соёлжсон абориген Вайпулданья буюу Филипп Роберто үзэгдэх шалтгаангүй үүсэн гардаг бүр илүү аймшигтай гунигт явдлыг өгүүлж байна. Жишээлбэл, нэг абориген гал доторхи сүнсийг авсан гэж үзээд тамхи татаж байсан цагаан арьстныг алсан байна. Өөр нэг цагаан арьстныг халааснаасаа цагаа гаргаж, нар руу харсан хэмээн жадаар нэвт сүлбэжээ. Аборигенүүд энэ хүнийг халаасандаа нар хадгалж явсан гэж үзсэн байна. Иймэрхүү үл ойлголцлоос болоод залхаан цээрлүүлэх экспедици явуулж, бүхэл бүтэн овгийг устгадаг байв. Зөвхөн цагаан хүмүүс төдийгүй, австралийн абориген, шинэ Гвинейн папуасууд, малайчуудад гунигт зөрчилдөөн тулгарч, ялангуяа халдварт өвчний тархалт байнга дэгддэг байв.
2 Локвуд Д. Я – абориген. М., 1971. С. 142-145.
1968 оны 10 дугаар сарын 30–нд Амазонк мөрний цутгал Манаус голын эрэг дээр атроари овгийн индианчууд миссионер Кальяри болон түүний найман дагуулыг өөрсдийнх нь үзлээр нэн бүдүүлэг аашилсан хэмээн үзэж алсан байна. Тэд атроарийн нутагт ирээд хий буудан ирснээ мэдэгдэн тэдний заншлыг доромжилж, дараа нь эзэд нь дургүйцсээр байтал овоохойд нь орж, хүүхдийн чихнээс татаж, хориглосон байдаг шөлтэй савыг авахыг оролджээ. Энэ отряд По голын эргийн хүмүүс Амазонкийн эрэг дэх хүмүүстэй огт төсгүй гэж хэлсэн зөвлөгөөг анхааралгүй мартсан аж, харин индианчуудын зан заншлыг мэддэг Кальярийг орхиж явсан зөвхөн ганц ойчин амьд үлдсэн байна.
3 Фесуненко И. С четками и счетчиком Гейгера //Вокруг света. 1972. № 3. С. 14-17.
Аборигенүүдийг устгаж байхын оронд тэдэнтэй дотно байсан нь дээргүй юу гэсэн асуулт тавигдахаас өмнө багагүй хугацаа өнгөрчээ. Гэхдээ үүний тулд бусад соёлын ард түмэн европчуудаас, мөн өөр хоорондоо хэл, шүтлэг төдийгүй, илүү маргаанаас зайлсхийхийн тулд судлавал зохих “зан үйлийн” бүхий л шинжээрээ ялгаатайг хүлээн зөвшөөрөх ёстой болов. Ингэж ард түмний хоорондын ялгааны тухай шинжлэх ухаан угсаатны зүй үүссэн билээ.
Үндэсний эрх чөлөөний хөдөлгөөний цохилтын улмаас колоничлол устаж байна, гэвч угтсаатан хоорондын харилцаа үлдэж, өргөсөж байна. Эндээс харилцан ойлголцол тогтоох асуудал нь дэлхийн бодлогын даяар түвшинд ч, бидэнд таатай, мөн бидэнтэй төсгүй хүмүүстэй уулзах үед хувийн болон микро түвшний харилцаанд ч улам бүр чухал болсоор байна. Ингэлээ ч гэсэн практик ач холбогдол бүхий хэдий ч яагаад хүмүүс бид өөр хоорондоо ийм их төсгүй байна вэ, бие биендээ юу “хийвэл” зохих вэ, бидэнд жам ёсны хэмээн төсөөлөгдөж буй, угсаатны дотоод харилцаанд бүрэн хангалттай, хөршүүдтэйгээ харилцахад хангалттай зүйлүүдийн оронд харилцах зохистой замыг хайх, бусдын зан авир, ёс заншлыг судлах ёстой юм бэ ? гэсэн онолын шинэ асуулт гарч ирнэ. Зарим тохиолдолд угсаатны төсгүй байдлыг газар зүйн нөхцлийн олон янз байдлаар тайлбарлаж болно. Гэхдээ энэ нь цаг агаар, орон нутаг нь өөр хоорондоо төстэй тийм газар л ажиглагддаг. Түүхийг үзэхгүйгээр цааш явахгүй нь тодорхой юм.
Үнэн хэрэг дээрээ янз бүрийн ард түмэн янз бүрийн эрин үед үүсч, тодорхой хүний шинж чанарыг бүрдүүлэгч хувийн намтар лугаа тийм арилшгүй ул мөр үлдээдэг түүхийн янз бүрийн хувь заяатай байдаг. Мэдээжийн хэрэг угсаатнууд өөрийг нь тэжээж буй байгальтай харилцах өдөр тутмын харилцаагаараа дамжуулан газар зүйн орчиндоо нөлөөлж байдаг, гэхдээ энэ бол бүх зүйл биш юм. Өвөг дээдсээс өвлөн авсан уламжлал өөрийн үүрэгтэй, хөршүүдтэйгээ (угсаатны хүрээлэл) дайсагнах болон найрамдах дадсан харилцаа бас өөрийн үүрэгтэй, мөн соёлын үйлчлэл, шашин нь мөн л өөрийн гэсэн ач холбогдолтой, гэхдээ энэ бүхнээс гадна байгалийн дурын үзэгдэл лугаа угсаатнуудад хамаарах хөгжлийн хууль гэж байдаг. Ард түмнүүд үүсэх, устан алга болох олон янзын үйл явцад энэ хууль илрэхийг бид угсаатны нийлэгжилт гэж нэрлэж байна.
Бид материйн хөдөлгөөний энэ хэлбэрийн онцлогийг харгалзахгүйгээр практикийн хувьд ч, онолын хувьд ч угсаатны сэтгэл зүйн оньсгыг тайлах түлхүүрийг олж чадахгүй. Бидэнд аль аль нь хэрэгтэй байгаа бөгөөд чингэхдээ бидний сонгосон замд гэнэтийн бэрхшээл үүсч байна.
\section{ХЭРЭГЛЭН БУЙ НЭР ТОМЪЁОНЫ БУДЛИАН}
Анхдагч мэдээллийн илүүдэл, системчлэх зарчмын сул боловсруулалт зэрэг нь түүх болон угсаатны зүйд нэн харамсмаар тусдаг. Зөвхөн ном зүй гэхэд л боть ботиороо байх бөгөөд үүний учрыг олох нь заримдаа эрдэм шинжилгээний ажлаасаа ч илүү их түвэгтэй болдог. Уншигчдад үйл явдлын бүхий л нийлбэрийг нэгэн зэрэг харах (актуализмын буюу идэвхийн зарчим) буюу тэдгээрийн бий болсон бүх аргуудыг (хувьслын зарчим) үзэх хэрэгцээ байдаг болохоос биш, ихэнхи хэсэг нь хуучирчихсан өгүүллүүдийн нэрсийн жагсаалт бүхий олон боть хэрэггүй юм. Марксизмын үндэслэгчдийн бүтээлүүдэд түүхэн үйл явдлуудыг ойлгох системийн хандлагын хөтөлбөр хадгалагдаж байдаг, гэхдээ түүнийг угсаатны нийлэгжилтэд хараахан хэрэглээгүй байгаа билээ.
Хуучны, зарим талаар мартагдсан түүх судлалд энэхүү салбарт системийн аргыг нэвтрүүлэх гэсэн хэд хэдэн оролдлого байсан нь үнэн юм, гэхдээ байгал судлалын шинжлэх ухааны төлөөлөгчдөөс ялгаатай нь эдгээрийг зохиогчид ойлгох, өрөвдөхийн алинтай нь тааралдаагүй юм. Полиби-ийн концепцийг өнөөдөр гайхамшигт ховор нандин зүйл, Ибн Халдуныг (XIY зуун) нэн сонирхолтой гэж үзэж, Жамбаттист Виког зөвхөн шинжлэх ухааны түүхэнд дурдаж байна. Н.Я.Данилевский, О.Шпенглер, А.Тойнби нарын хэдийгээр агуу боловч азгүй бүтээлүүд нь ерөөсөө түүхэн загвар бүтээхээс татгалзахын шалтгаан болж байна. Энэ үйл явцын үр дүн нь нэн тодорхой юм. Түүхэн үйл явдлын бүхий л нийлбэрийг санах боломжгүй учраас, систем байхгүй байгаа учраас нэр томъёо байж чадахгүй юм. Иймээс түүхчдийн хоорондын харилцаа жилээс жилд хүндэрсээр байна.
Түүхчид нэр томъёонд янз бүрийн өнгө аяс өгч, мөн янз бүрийн агуулга оруулсаар байтал тэдгээрийгээ олон утгат үгс болгон хувиргаж орхижээ. Энэ үйл явцын эхний шатанд хэлж буй өнгө аяс, маргалдах үеийн нөхцөл байдал зэргээр нь ярилцагчаа яаж ийгээд ойлгож болж байлаа, гэвч дараагийн үгэнд энэхүү (хангалтгүй) ойлголтын түвшин алга болдог аж. Жишээлбэл, “овог” гэдэг үгийг голдуу “овгийн байгуулал” хэмээх ойлголтод хэрэглэдэг. Харин “Шуйн боярын овог” гэх ойлголт нь энд илт үл тохирно. Эдгээрийг орчуулах үед бүр долоон дор болно. Хэрэв “кельтийн клан” гэдгийг овог гэж үзвэл Дунд болон Бага Жусын (ру) буюу алтайн “яс” (соек), ямар нэг казакийн салбарыг ингэж нэрлэж болохгүй, учир нь эдгээр нь үүрэг болон нийлэгжилтээрээ өөр өөр зүйл юм.
Огт төсгүй эдгээр бүх үзэгдлийг адилханаар нэрлэж, тэр ч байтугай энэ үндсэн дээр өөр хооронд нь тэнцүүлэн үздэг байна. Хүссэн ч, эс хүссэн ч түүхч юмыг биш, аль хэдийнээ утгаа алдсан үгийг судладаг, чухам энэ үед л бодит үзэгдэл түүнээс гулган алга болдог. Одоо энэ асуудлаар гурван түүхч маргаж байна гэж бодоод үзье. Тэдний нэг нь “Овог” гэдгийг “клан”, хоёрдахь нь “сеок”, гурав дахь нь “боярын ураг” хэмээн ойлгож байг. Энд тэд бие биенээ ойлгохгүйгээр үл барам, юу ярьж байгаагаа ч мэдэхгүй юм.
Мэдээжийн хэрэг нэр томъёог тохирч болно гэж бидэнд татгалзаж болох авч ойлголтын тоо хэмжээ мэдээллийн хуримтлалтай шууд пропорционалаар өсч, систем байхгүй үед олон утгат (полисемантик) болж, улмаар анализ синтез хийхэд тохирохгүй болдог. Гэхдээ энд гарц олж болно.
Бид одоог хүртэл судалгааны хэм хэмжээ, түүний хэтийн төлөвийн тухай ярьж ирсэн. Дурын юмыг судлах нь уг зүйлийг бүхэлд нь харах боломжтой болсон үед л практик ач холбогдолтой болдог. Жишээлбэл цахилгаан техникч ионжилт, дулааны өгөлт, цахилгаан соронзон талбай гэх мэтийн үйлчлэлийн хэмжээг нэгэн адил биш ч гэсэн төсөөлөх ёстой. Физик–газар зүйч хүн Дэлхийн бүрхүүлийн тухай ярихдаа тропо хүрээ, гидро хүрээ, лито хүрээ, тэр ч бүү хэл био хүрээний тухай санадаг. Түүхч хүн харилцан холбоотой үйл явдлын хэсэг бүрдлийг нэгдсэн гаргалгаанд хамран базаж чадвал уншигчдын хувьд илүү жинтэй, сонирхолтой дүгнэлт хийж чадах юм. Энэ нь хүнд боловч боломжгүй зүйл биш юм. Хамгийн гол нь энэхүү гаргалгаа бүхий л тооцож үзсэн баримттай нийцсэн байх ёстой. Хэрэв хэн нэгэн хүн энэхүү номонд дурдсан баримтуудыг тайлбарлах илүү гоё, илүү итгэж болох концепцийг санал болгох ахул би түүнд мэхийн ёслох байна.
Үүний урвуугаар хэн нэгэн хүн миний гаргалгааг эцсийн, дахин нягтлах, цаашид дахин боловсруулах шаардлагагүй хэмээн мэдэгдэх аваас би түүнийг үл зөвшөөрнө. Харамсалтай нь олон ном хүнээс урт насалдаггүй. Харин шинжлэх ухааны хөгжил бол хүн төрөлхтөн бий болох язгуурын хууль юм. Иймээс би өөрийнхөө зорилгыг хүмүүсийг Дэлхийн био хүрээ хэмээх дээд эхтэй хамаатуулж буй Түүх хэмээх эрхэмсэг сайхан Хатагтай, түүний мэргэн эгч Газар зүй хатагтайд чадлынхаа хэрээр тус болох явдал гэж үзэж байна.
4 В.И.Вернадскийн шинжлэх ухаанд оруулсан био хүрээ хэмээх нэр томъёо нь өөртөө амьд организмын нийлбэр цогцос, тэдний өмнөх амьдралын бүх үр шимээс бусад: хөрс, тунамал чулуулаг, агаарын хүчилтөрөгч зэргийг багтаасан Дэлхийн бүрхэвчийн нэг юм. Ийм маягаар угсаатны нийлэгжилтийг био хүрээний био үйл явцуудтай холбон тогтоосон нь намайг шүүмжлэгчдийн хэлдгээр “биологизм” биш, харин “газар зүйн зүйл” юм. Хэдийгээр ийм хаяг зүүх нь таарах эсэхийг мэдэхгүй боловч Дэлхийн гадаргуу дээр буй бүх зүйлс аль нэг байдлаар эсвэл эдийн засгийн, эсвэл түүхийн гэхчлэн газар зүйн хүрээнд ордог.
НЭГТГЭН ДҮГНЭХҮЙ ХИЙГЭЭД ЭГЭЛ УХАГДАХУУНУУД (СКРУПУЛЮС)
Дэлхийн бүхий л хуурай газар, далайн гадаргуугийн ихээхэн хэсэгт тархсан Homo sapiens нь бага хэмжээний геологийн эргэлттэй дүйцүүлэхүйц тийм ихээхэн өөрчлөлтийг дэлхийн хэлбэр дүрсэд оруулсан байна.
5 Вернадский В. И. Химическое строение биосферы Земли и ее окружения. С. 273.
Эндээс бид судалгааны түүхэн болон газар зүйн аргуудыг хамтатгасан, онцгой арга зүйг авч үзэх, судлахыг шаардсан түүх–газар зүйн зүй тогтлын онцгой категорийг бий болгох ёстой болж байна. Энэ нь өөрөө шинэ зүйл биш боловч асуудалд хандах хандлага нь одоо болтол хэсэг бусаг байгаа юм. Жишээлбэл, археологийн дурсгалын он цагийг тогтоохын тулд С14 аргаар шинжилгээ хийж, цахилгаан хайгуул (энэ ажил нь практикт хэрэглэхэд ихээхэн түвэгтэй) хийж, “чулуун эмгэнийг” судлах үед кибернетик арга (харж тооцсонтой адил үр дүн өгсөн) зэргийг хэрэглэжээ. Энд хамгийн гол зүйлийг алдсан байна. Бидний бодлоор энэхүү “гол зүйл” нь хөшөө дурсгалын дуугүйгээс мэдээлэл гарган авах чадвар болно. Индукцийн зам нь харь үгийг энгийн буюу шүүмжлэлт байдлаар дахин өгүүлэх түүхчийн боломжийг хязгаарлаж, чингэхдээ тухайн эх сурвалжид үл итгэх явдлыг судалгааны хязгаарлалт болгож байна. Гэхдээ энэ үр дүн нь төгсөөгүй учраас сөрөг шинжтэй юм. Эерэг зүйл нь гэвэл эх сурвалжаас ялган авсан, маргаангүй баримтуудын зарим хэмжээг тогтоож, он цагийн хүснэгтэд оруулах буюу түүхийн зураг дээр байрлуулж болно. Эдгээрийг тайлбарлахын тулд философема буюу гүн ухаанчлал, постулат буюу үүсгэл үнэн хэрэгтэй болно. Мухардал бий болов.
Ингэхэд газар зүйч, геолог, амьтан судлагч, хөрс судлагч нарт хэзээ ч илүү өгөгдөхүүн байдаггүй, гэтэл шинжлэх ухаан нь хөгжөөд л байдаг юм. Энэ нь байгал шинжээчид гүн ухааны постулатын оронд “эмпирик нэгтгэн дүгнэлт” хэрэглэснээс болдог бөгөөд үүнийг В.И.Вернадский ажигласан баримттай тэнцэх үнэн магад шинж гэж үзсэн юм.
6 Вернадский В. И. Избр. соч.: В 6 т. Т. V; Биосфера. С. 19.
Өөрөөр хэлбэл байгалийн шинжлэх ухаанууд түүхчдийн дуугай байдлыг даван туулж, тэр байтугай үүнээс шинжлэх ухаандаа ашиг гаргаж чаджээ. Учир нь тэд эх сурвалжид ямагт агуулагдаж байдаг буюу бид өөрөө буруу хүлээн авах замаар оруулж ирсэн хуурамч зүйлээс ангижирч чадсан байна. Тэгвэл түүхчид яагаад үүнээс татгалздаг юм бэ ? Байгалийг эх сурвалж хэмээн үзэхдээ бид судалгааны харгалзан нийцэх арга зүйг татан оруулах ёстой бөгөөд, энэ нь бидэнд Исида буюу амьдралын бурхны хаалтыг нээх агуу их хэтийн төлөвийг өгөх юм.
Шинжлэх ухааны зорилтуудын нэг бол хамгийн олон янзийн үзэгдлүүдийг ойлгох нэгдсэн үзэл бий болгох нарийн зүй тогтлыг ялган гаргах боломжтой болох, цаашдаа эдгээрийн дунд баримжаалж сурахын тулд хамгийн бага баримтуудаас хамгийн их мэдээлэл авах явдал юм. Энэхүү зүй тогтлууд нь харагдахгүй, гэхдээ бодож олоогүй учраас тэдгээрийг нэгтгэн дүгнэх замаар нээдэг юм. Биологоос жишээ авч үзье. “Тэнгэрээр гариг болон одод хөдөлнө. Агаарын бөмбөрцөг хөөрч байна, харин уулнаас салсан чулуу хавцалд унаж байна. Голууд тэнгист цутгаж, далайд тундас бууж, тундсийн хөрсөн давхраа үүсгэж байна. Хулгана маш жижигхэн савартай, заан асар том хөлтэй, Дэлхий дээрх амьтад халим болон аварга наймаалжны хэмжээнд хүрэхгүй. Эдгээр баримтын хооронд юу нийтлэг байна вэ? Энэ бүхэн нь бүх ертөнцийн таталцлын зүй тогтолд үндэслэсэн бөгөөд энэ нь мөн л бодитой, үл харагдах, гэхдээ оюун ухаанаар хүрч болох бусад зүй тогтлуудтай сүлжилдсэн байдаг юм.”
7 Малиновский А. А. Путь творческой биологии. М., 1969. С. 7.
Дэлхийн татах хүч ямагт байсаар ирсэн бөгөөд хүмүүс түүний байгааг мэдэхийн тулд мөчрөөс алим унахыг ажиглаж байсан Ньютоны онгод хэрэгтэй байсан юм. Биднийг хүрээлэн, бидний хувь заяаг удирдаж буй байгалийн олон хүчирхэг хүчин бидний оюун сэтгэлгээний чанадад байгаа юм. Бид бүрэн нээгээгүй ертөнцөд амьдарч, голдуу хар таамгаар хөдөлж, ингэх болгондоо гамшигт үр дагаварт хүрдэг. Ийм учраас л суут эрдэмтдийн онгод хэмээн миний ойлгож явдаг шинжлэх ухааны шидэт нүдний шил биднийг хүрээлэн буй ертөнц, түүнд бидний эзэлж буй байр суурийг ойлгох, ядахдаа өөрсдийн үйлдлийн ойрын үр дагаврыг харж сурахад хэрэгтэй болоод байгаа юм.
Төв Азийн түүхийн болон доод Волгийг археологийн түүхийн материал дээр физик газар зүй болон эртний угсаатан судлалын үзэгдлийн функционал холбоог тогтооход зориулагдсан судалгаанууд нь дараах гурван дүгнэлт хийх боломжийг олгож байна. 1. Хүний өөрийнх нь үйл ажиллагааны үр дүн болсон угсаатны түүхэн хувь заяа нь тэдгээрийг багтаагч ландшаф буюу байгал орчны хөдлөнги төлөв байдалтай шууд холбоотой байна, 2. Түүхэн хувь заяаны нь талсжсан ул мөр болох тухайн угсаатны археологийн соёл нь маш нарийн тооцоолон гаргаж болох тухайн эрин үеийнхээ ландшафтын эртний буюу палео газар зүйн төлөв байдлыг тусгаж байна. 3. Түүхэн болон археологийн материалуудыг хослуулан үзсэнээр аль нэг эрин үеийн багтаагч ландшафтын шинж чанар, улмаар түүний өөрчлөлтийн шинж чанарыг дүгнэх боломж олгож байна.
8 Гумилев Л. Н. 1) Хазария и Каспий //Вестник ЛГУ. 1964. № 6. С. 95, 2) Хазария и Терек //Там же. № 24. С. 78.
Энд нарийн цаг хугацаа гэдэг нь харьцангуй бөгөөд зааг ялгаа нь бүдэг үед алдаа нь нэмэх, хасах 50 жил болдог нь дүгнэлтэд нөлөөлөхгүй бөгөөд улмаар гэм хоргүй юм. Үгийн шууд утгаар бол нягт нямбай занд тэмүүлэх нь хавьгүй илүү аюултай. Scrupulus хэмээх латин үг нь эртний римчүүдийн шаахайд орсон жижиг чулууг хэлнэ. Тэд шаахай доторхи эдгээр чулууны байрлалыг судлахыг учир утгагүй хэрэг гэж үзэж, зүгээр л гутлаа тайлж шаахайгаа сэгсэрдэг байна. Иймээс “нягт нямбай” гэсэн үг нь хэрэггүй жижиг зүйлийг харгалзахыг хэлнэ. Эдүгээ энэ үг “хэт нарийн хандах” гэсэн утгаар хэрэглэгдэх болжээ.
Харамсалтай нь “нягт нямбай”–д тавигдах шаардлага нь ямагт аятайхан байдаггүй, жишээлбэл, байгалийн үзэгдлийг түүхэн үйл явдалтай жишиж үзэхэд боломжит хязгаарыг 50–60 жилээс илүү багасгаж болдоггүй ба учир нь хайж буй холбоо нь эртний орнуудын аж ахуйн системээр нөхцөлдсөн байдаг.
9 Гумилев Л. Н. 1) Истоки ритма кочевой культуры Срединной Азии (опыт историко-географического синтеза) //Народы Азии и Африки. 1966. № 4. С. 85-94; 2) Роль климатических колебаний в истории народов степной зоны Евразии //История СССР. 1967. № 1. С. 53-66; 3) Изменения климата и миграции кочевников //Природа. 1972. № 4. С. 44-52.
Газар тариалан, мал аж ахуй, тэр ч бүү хэл анчдын аж ахуйн систем нь өөрийн гэсэн инерцтэй байдаг. Хэрэв аж ахуйн энэ системийг ган болж сулрууллаа гэж бодоход түүнд тулгуурласан төр улс доройтох явдал нь бүх нөөц дуусч, байнгын өлөн зэлмүүн байдал (энэ нь түр хугацааны өлсгөлөн биш) өсөн төрж буй шинэ үеийнхний хүчийг сульдаасан үед л болдог. Энэ үйл явцыг байгалийн болон түүхийн үзэгдлийг нягт нямбай хамааруулах замаар яагаад ч олохгүй, харин түүхэн үйл явдлын цувралыг өргөнөөр нэгтгэх буюу интеграцчлах замаар л нээн олж болно. Үүнтэй холбогдуулан нэгэн байгал судлаачийн “ Хэрэв та хулганын тусгай тусгай эсийг микроскопоор маш нарийн судлах аваас хулгана юутай төстэйг хэзээ ч мэдэж чадахгүй. Готын сүмийн чулуу бүрийг химийн анализаар шинжилсэн ч гэсэн та түүний гайхамшгийг олж харж чадахгүй” гэсэн алдарт үгийг сануулах ёстой.
10 Selye Н. From Dream to Discovery. New York, 1964 (цит. по: Мирская Е. 3. Противоречивость научного творчества //Научное творчество /Под ред. С. Р. Микулинского, М. Г. Ярошевского. М., 1969. С. 298). Ср.: Советская археологии. 1969. № 3. С. 282-283.
Бусдаас нь салангид байдлаар нэг буюу хоёр баримт авч үзлээ ч гэсэн бид оюуны хувьд чадамгай, авъяас чадвараараа өөрийн үнэлгээг уншигчдад тулгаж чаддаг эртний зохиогчдын олзонд байсаар байх нь ойлгомжтой юм. Хэрэв бид эх сурвалжаас шууд мэдээллийг ялган авч, хоёр мянган баримтын оронд хоёр баримт авч, өөр хоорондоо төдийгүй, бидний санал болгосон загвартай хамааралтай хэд хэдэн шалтгаан-үр дагаврын гинжин холбоо олж болох юм. Энэ нь XYIII зууны үед газар зүйн детерминизмын талынхан, жишээлбэл Монтескье–ийн хайж байсан шиг жирийн бус функционал хамаарал юм. Энд бид хүн байгалийн харилцамжийн тухай шинжлэх ухааны үндэс болсон системийн холбоо олж байна.
Бидний тэмдэглэсэн түгээмэл болон өвөрмөц шинж чанар нь түүнийг шинжлэх ухааны бие даасан, завсрын салбар байдлаар, түүхийг газар зүйтэй хослуулсан–этнологи байдлаар судлах боломж олгож байна. Гэсэн ч энд шинэ хөндүүр асуудал гарч ирнэ. Бид угсаатны талаарх мэдэгдэхүйц тодорхойлолт олж чадах болов уу ?
ЖААЗ
Угсаатны тухайд бид юуг нарийн мэдэх вэ? Маш их зүйл мэднэ, мөн маш бага зүйл мэднэ. Бид угсаатан үзэгдлийнхээ хувьд доод палеолитийн үед байсан гэж батлах үндэслэл байхгүй юм. Хөмсөгний дээдэх гүн хөнхөр, асар том гавлын ясаараа бол неандертал хүн сэтгэж, мэдэрч байсан байх. Гэхдээ тэднийг ямар хүмүүс байсан гэдгийг хэрэв бид хэрэв шинжлэх ухааны үнэн магадын суурин дээрээ үлдэхийг хүсч байвал одоохондоо таамаглах ч эрхгүй юм.
Дээд палеолитийн эрин үеийн хүмүүсийн тухай бид илүү мэднэ. Тэд гайхамшигтай ан хийж, жад, муна бэлтгэж, амьтны арьсан хувцас өмсөж, парижийн импресионистуудаас дутуугүй зурдаг байжээ. Тэдний хамтын ахуйн хэлбэр нь бидэнд тодорхой буй зарим зүйлтэй төстэй мэт, гэхдээ энэ нь шинжлэх ухааны таамаглал дэвшүүлж ч болохооргүй, зөвхөн санал төдий юм. Эртний үеүдэд манай үе хүртэл хадгалагдаагүй ямар нэг онцлог байсан байж магадгүй.
Хожуу неолит болон хүрлийн үеийг (НТӨ III–II зуун) бол бид түүхтэй төстэй өндөр магадлалт хувиар тооцож болно. Харамсалтай нь энэ үеийн угсаатны ялгааны талаарх бидний мэдлэг хэсэг бусаг, ядмаг бөгөөд эдгээрт тулгуурласан ч гэсэн энэ үед бидний сонирхож буй зүй тогтлыг саланги онцлогоос нь ялгаж чадахгүй, ганц зүйлийг ерөнхий мэт үзэж, алдаанд унах эрсдэлд байна.
Шинжилгээний үнэн магад материалыг түүхэн эрин үе гэж бидний нэрлэдэг, бичмэл сурвалжууд угсаатны түүх болон тэдгээрийн харилцамжийг гийгүүлж эхэлсэн үе өгч эхлэв. Энэ сэдвийн эхний бүлгийг судалмагцаа гаргаж авсан ажиглалтаа илүү эртний үед хэрэглэж, экстраполяцийн замаар судалгааны эхний үед үүссэн мэдлэгийнхээ цоорхойг бөглөхийг оролдох нь бидний зөв юм. Ийм маягаар бид цаашдаа түүхэн шүүмжлэлд хамгийн олон гардаг аберрацийн буюу гажилтын алдаанаас зайлсхийх юм. .
Дээд тооллыг бид XIX зууны эхээр авах нь зохимжтой бөгөөд учир нь зүй тогтлыг тогтоохын тулд бидэнд дууссан үйл явц хэрэгтэй. Дуусаагүй үйл явцыг тухайд бол таамаглах маягаар л ярьж болох ба ингэж таамаглахын тулд бидний хайж буй яг тэрхүү зүй тогтлын томъёолол гарт бэлэн байх хэрэгтэй. Түүнээс гадна ХХ зууны үзэгдлийг судлахад үзэгдлүүд масштабаа буюу хэмжээгээ алддаг ойрын аберраци буюу гажилт гарч болно. Энэ нь алсын аберрацитай мөн адил юм. Иймээс бид асуудлыг тавихдаа 3 мянган жилийн эрин үе, НТӨ XII зуунаас НТ –ын XIX зуун буюу ойлгомжтой байдлаар бол Трой хотын уналтаас Наполеоны буулт хүртэлх үеэр хязгаарлаж байна.
Эхлээд бид манай арвин баялаг материалыг үнэн магад шинж нь эргэлзээ үл төрүүлэгч мэдээллүүдийг харьцуулан жишихэд үндэслэсэн синхроник хэмээх арга зүйг хэрэглэх замаар судалж үзнэ. Бидний оруулах гэж буй шинэ зүйл гэвэл дэвшүүлсэн аспектдаа баримтуудыг хослуулах явдал болно. Учир нь цаг хугацааны хүснэгтэд буй он цагийн хэлхээс нь түүхэн амьдралын туршид ард түмнүүдэд юу тохиолдсон талаар уншигчдад ямар ч төсөөлөл өгдөггүй учраас энэ нь тун чухал юм. Санал болгон буй арга зүй маань нийгмийн гэхээсээ байгалийн ухаанд илүү хэрэглэгддэг бөгөөд энд статистик магадлалын үндсэн дээр баримтуудын хоорондын холбоог тогтоох, үзэгдлийн дотоод логикийг эмпирик нэгдсэн дүгнэлт хийх цорын ганц зам бөгөөд энэ нь ажигласан баримтын нэгэн адил үнэн магадтай гэж үздэг юм.
11 Вернадский В. И. Избр. соч. Т. V. С. 19.
Эмпирик нэгдсэн дүгнэлт нь хэдийгээр анхдагч материал дээр (туршлага, ажиглалт, эх сурвалжийг унших зэрэг) тулгуурлаагүй, таамаглал ч биш, хялбаршуулалт ч биш авч, шалгасан баримтуудын цуглуулга дээр бий болдог. Материалыг системд оруулах, концепци бүтээх, асуудлыг эрэгцүүлэх зэрэг нь дунд шат бөгөөд гүн ухааны нэгдсэн дүгнэлтийн өмнө хийгддэг юм. Бидний зорилгын хувьд чухамхүү энэ дунд шат хэрэгтэй юм.
Аль нэг юманд хамаарах мэдээллүүд хэчнээн тодорхой, хэчнээн олон тоотой байх тусам түүний тухай шавхагдашгүй төсөөлөл бүрдүүлэх явдлыг төдий чинээ хөнгөвчлөх мэт санагддаг. Үнэн хэрэг дээрээ ийм байж болох уу ? Ер нь бол үгүй юм. Зураглалыг бүхэлд нь өөрчилж чаддаггүй, Илүүдэл, хэт жижиг мэдээлэл нь кибернетик болон системологийн шинжлэх ухаануудад “шуугиан” буюу “саад” гэж нэрлэгдсэн зүйлийг бий болгодог. Гэхдээ бусад зорилгын хувьд чухамхүү сэтгэл санааны эв дан хэрэгтэй байдаг. Товчоор хэлбэл, үзэгдлийн мөн чанарыг ойлгохын тулд шинжлэх ухааны мэдэлд буй бүх мэдээллийг биш, харин судлан буй асуудалд хамаарах бүх баримтуудын бүх нийлбэрийг олбол зохино.
Гэхдээ “асуудалд хамаарах” гэж юуг хэлэх вэ ? Янз бүрийн тохиолдолд янз бүрийн хариулт байх нь ойлгомжтой байна. Хүн төрөлхтний түүх, алдарт хүмүүсийн намтар хоёр адил тэнцүү биш, аль ч тохиолдолд хөгжлийн зүй тогтол нь өөр өөр байна, мөн тэдгээрийн хооронд хэд л бол хэдэн ялгаа бий. Дайн, хууль гаргах, хөшөө, уран барилга бүтээх, ханлиг буюу бүгд найрамдах улс байгуулах зэрэг дурын түүхэн үйл явдал дээр асуудал улам хүндрэх бөгөөд эдгээрийг ойртолтын хэд хэдэн үе шатаар авч үзэх ёстой, чингэхэд энэ нь анх харахад зөрчилтэй үр дүн үзүүлнэ.
Нийтэд тодорхой Европын түүхээс жишээ авъя. Реформацийн дараа протестант Уни, католик Лигийн хооронд тэмцэл үүсчээ (ойртолт a. ) Эндээс Баруун европын бүх протестантууд бүх католикуудын эсрэг дайтах ёстой байв. Гэвч католик Франц протестант Унийн гишүүн байв, харин 1643 онд протестант Дани протестант Шведийн ар талд цохилт өгөв. Өөрөөр хэлбэл, улс төрийн ашиг сонирхол үзэл суртлынхаас дээгүүр тавигджээ. (ойртолт b). Үүгээр анхны нотолгоог буруу гэж үзэх үү ? Огтхон ч үгүй. Энэ нь зөвхөн илүү нэгтгэн дүгнэсэн баримт юм. Түүнээс гадна хоёр талын цэрэгт хөлснийхөн тулалдаж, тэдний дийлэнхи олонхи нь холимог шашинтай, дээрмийн шуналтай байв. Эндээс дараагийн ойртолтоор (с) Гучин жилийн дайныг дээрэм гаарсан гэж тодорхойлж болно. Энэ нь ч ямар нэг хэмжээгээр бас л зөв юм. Эцэст нь шашны лоозон, хааны алтан титмийн цаана хэрэв харгалзахгүй бол буруу болох ангийн бодит ашиг сонирхол нуугдаж байжээ. (ойртолт d). Үүний дээр эртний угсаатны зүйн замаар илрүүлсэн тодорхой салбарын салан тусгаарлах хандлагыг нэмж болно. (ойртолт e).
Энд татсан жишээнээс дэс дараалсан ойртолтын систем нь тусгаар нэг үзэгдлийг задлан үзэх үед ч ихээхэн нарийн ажил болох нь харагдаж байна. Ийм хэдий боловч амжилт олох найдвараа алдах хэрэггүй бөгөөд учир нь бидэнд шинжлэх ухааны дедукцийн зам үлдсэн юм. Дэлхийн хөдөлгөөн зүй тогтолт олон хөдөлгөөний олон бүрэлдэхүүнт хөдөлгөөнөөс (тэнхлэгээ тойрон эргэх, нарыг тойрон эргэх, туйлаа солих, галактик дотроо гаригуудтайгаа хамт шилжих болон бусад олон хөдөлгөөн) бүтдэг лугаа адилаар хүн төрөлхтөн болон антропо хүрээ нь нэг бус, тусгай тусгай шинжлэх ухаануудын судалдаг хэд хэдэн үйлчлэлийн нөлөөгөөр хөгжиж байдаг. Нийгмийн хөгжилд туссан байнгын хөдөлгөөнийг түүхэн материализм судладаг бол, хүний физиологийг биологийн салбар, хүн болон орчны харилцааг газар зүйн шинжлэх ухааны хүрээнд орших түүхэн газар зүй, дайн, хууль, байгууллагыг улс төрийн түүх, харин үзэл бодол, сэтгэлгээг соёлын түүх, хэл судлалыг лингвистика, мөн уран сайхны бүтээлийг гүн ухаан гэх мэтээр судалдаг. Тэгвэл бидний асуудал хаана нь байна вэ ?
Угсаатанг (аль нэгэн) жишээлбэл, хэлийг нийгмийн үзэгдэл биш гэдгээс эхлэе. Учир нь энэ нь хэд хэдэн формацид оршиж болно. Угсаатан бий болоход нийгмийн аяндаа хөгжих нөлөөлөл гаднын шинжтэй байна. Нийгмийн хөгжил нь угсаатан бүрэлдэх юмуу задрахад тэр нь зөвхөн улс төрийн хувьд ч, соёлын хувьд ч түүхэнд биелсэн тохиолдолд л нөлөөлж чадна. Иймээс угсаатны нийлэгжилтийн асуудал нь социал асуудлууд нь аажмаар байгалийн болон шилжиж буй тэр газарт түүхийн шинжлэх ухааны зааг дээр оршин байна гэж хэлж болно.
Угсаатны нийлэгжилтийн бүхий л үйл явц нь Дэлхийн гадаргуу дээр газар зүйн аль нэг нөхцөлд болдог учраас хүний хамтлаг болох угсаатны жам ёсоор бүрэлдүүлсэн эдийн засгийн боломжийг төлөөлөх хүчин зүйл болох ландшафтын ролийн тухай асуудал гарч ирж байна.
12 Калесник С. В. Основы общего землеведения. М., 1955. С. 412-416.
Гэхдээ бид асуудлаа судлах гэж түүхийг газар зүйтэй хослуулсан нь хангалтгүй юм. Учир нь энд цаг ямагт эсвэл хувьслын төлөв байдалд байдаг, эсвэл инволюци буюу урвуу хувьслын байдалд байдаг, эсвэл моноформизм буюу зүйлийн доторхи тогтвортой байдалд байдаг амьд организм бусад амьд организмтай харилцан үйлчилж, геобиоценез гэдэг хамтын нийгэмлэгийг бүрдүүлж байгаа тухай яригдаж байна.
Ийм маягаар манай асуудлыг түүх, газар зүй (амьдрах орчны судлал), биологи (экологи ба генетик) гэсэн гурван шинжлэх ухааны уулзвар дээр байршуулбал зохино. Ингэвээс “угсаатан” хэмээх нэр томъёонд хоёр дахь ойртолтын тодорхойлолт өгч болно. “Угсаатан гэдэг бол Homo sapiens зүйлийн оршин байх өвөрмөц хэлбэр, угсаатны нийлэгжилт бол түүхэн болон хрономи (орчны) хүчин зүйлийн хослолоор тодорхойлогдох зүйлийн дотоод хэлбэр үүсэх салаа хувилбар” юм.
Энд хүн төрөлхтний хөгжлийн хөдөлгөгч хүчний нэг нь тачъяàл, сэрэл болдог гэсэн хачин жигтэй асуудал харагдаж болно, гэвч судалгааны ийм хэв маягийн эхлэлийг Ч.Дарвин, Ф.Энгельс нар тавьсан билээ.
13 К. Маркс К., Энгельс Ф. Соч. 2-е изд. Т. 21. С. 176.
Бид шинжлэх ухааны уламжлалыг мөрдсөөр манай өмнөх олонхи судлаачдын анхаарлын гадна үлдсэн хүний үйл ажиллагааны тэрхүү тал руу анхаарлаа хандуулж байна.
ГАЗАР ЗҮЙГҮЙ ТҮҮХЧ “НЭВТ ЦОХИЛТТОЙ” ТУЛГАРДАГ
Өөрийг нь хүрээлэн буй байгаль орчноос хүний хамаарах, нарийн яривал газар зүйн орчноос хамаарах хамаарлыг, хэдийгээр энэ хамаарлын хэмжээг янз бүрийн эрдэмтэд янз бүрээр үнэлж байсан ч хэзээ маргаж байгаагүй юм. Гэхдээ ямар ч тохиолдолд Дэлхийд дээр амьдарч байсан болон амьдран буй ард түмнүүдийн аж ахуйн амьдрал нь тэдний амьдран буй газар нутгийн ландшафт болон цаг агаартай нягт холбоотой юм. Эртний эрин үеийн эдийн засгийн сэргэлт, уналтыг мөшгөх нь ихээхэн хүнд байдаг ба энэ нь бас дахиад л эх сурвалжаас олж авсан мэдээллүүд бүрэн бус байдгаас болдог. Гэхдээ цэргийн хүчин гэсэн нэг шалгуур үзүүлэлт бий. Шинэ үеийн хувьд гэвэл энэ нь хэний ч эргэлзээг төрүүлэхгүй бөгөөд хоёр мянган жилийн өмнө ч гэсэн энэ нь суурин ард түмэн төдийгүй, нүүдэлчдийн хувьд ч мөн л ийм байжээ.
Нүүдэлчид аян дайн хийхэд цатгалан, хүчтэй, ядраагүй төдийгүй, хатуу нумыг “чихээ хүртэл” татах (нумыг “нүдээ хүртэл” татахад сум 350–400 м зайд тусдаг бол “чихээ хүртэл” татахад 700 м зайд сумаа тусгаж болдог байна), мөн хүнд сэлэм буюу түүнээс ч хэцүү муруй сэлмээр дайтах хүмүүс хэрэгтэй байдаг. Түүнээс гадна тэрэг, ачаа зэргийг харгалзан нэг хүнд 3–5 морь хэрэгтэй. Мөн нөөц сум шаардагдах бөгөөд үүнийг бэлтгэхэд ихээхэн хөдөлмөр орно. Хоол хүнсний нөөц, жишээлбэл, нүүдэлчийн хувьд хонин сүрэг, улмаар дэргэд нь байх малчин хэрэгтэй. Бас хүүхэд, эмэгтэйчүүдийг хамгаалах нөөц манаа хэрэг болно. Товчоор хэлбэл тэр үед ч дайнд их мөнгө, чингэхдээ асар их мөнгө ордог байжээ. Дайныг эхлээд анхны, том биш ялалтын дараа явуулж болно, харин түүнийг ялахын тулд бат бөх ар тал, цэцэглэж буй аж ахуй, мөн эдгээрт нийцсэн байгалийн таатай нөхцөл хэрэгтэй байдаг.
Газар зүйн нөхцлийн, жишээлбэл цэргийн түүхэнд газар орны саадны ач холбогдлын тухай эртнээс ярьсаар ирсэн бөгөөд цаашид ч ямагт ярих биз. Эртний түүхээс хэдэн жишээ санахад л хангалттай юм. Тразимен нуурын дэргэдэх тулалдаанд римийн цэргийн явж буй замтай 90 градус өнцгөөр байрлах нуурын эрэг дэх байх хэд хэдэн гүнзгий хөндийг ашиглан Ганнибал ялалт байгуулсан юм. Энэ байрлалын ачаар тэр римийн цэргийг тэр гурван газраас зэрэг довтлон тулалдаанд ялжээ. Киноскефалын дэргэд македоны цэргийн жигүүр бартаатай газарт сарнисан бөгөөд жагсаалаа алдсан хүнд зэвсэгт энэ дайчдыг римийнхэн хялбархан бут цохисон билээ. Энэ болон бусад жишээнүүд нь түүхчдийн анхаарлын төвд ямагт байсаар ирсэн ба И.Болдин “Гартаа газар зүйгүй түүхч нэвт цохилттой тулгардаг“ гэсэн алдарт санамжаа хэлэх шалтаг болсон буй за.
14 Болтин И. Н. Примечания на историю древния и нынешния России г. Леклерка, сочиненные генерал-майором Иваном Болтиным; В 2 т. Т. 2. СПб., 1788. С. 20.
ХХ зуунд ийм тодорхой асуудал дээр зогсох нь утгагүй юм, учир нь өнөөгийн түүх нь урьд өмнөхөөс илүү гүнзгий зорилтууд дэвшүүлэх болж, газар зүй ч гэсэн манай гаригийн зэрлэг байдлыг энгийнээр дүрслэхээс холдож, манай өвөг дээдэс хүрч ч чадахгүй байсан боломжуудыг олж àвсан байна.
Иймээс бид асуудлаа өөрөөр тавьж үзье; зөвхөн газар зүйн орчин хүмүүст хэрхэн нөлөөлдгийг төдийгүй, өнөөдөр хүмүүс ямар хэмжээгээр био хүрээ хэмээн нэрлэж буй Дэлхийн бүрхэвчийн тэр хэсгийн бүрэлдэхүүн хэсэг болдог вэ ?
15 Вернадский В. И. Избр. соч. Т. V.
Хүн төрөлхтний амьдралын чухам ямар ямар зүй тогтолд газар зүйн орчин нөлөө үзүүлж, ямарт нь нөлөөлдөггүй вэ? Асуудлыг ингэж тавих нь анализ, өөрөөр хэлбэл судалгаанд тохиромжтой болгохын тулд асуудлыг зохиомлоор хэсэгчлэн хуваахыг шаардана. Улмаар энэ нь түүхийг ойлгоход маань туслах л ач холбогдолтой болох агаад уг нь манай ажлын зорилго бол синтез хийх явдал билээ. Гэхдээ суурьгүйгээр байшин барьж болохгүйн адил урьдчилаад хэсэгчлэн хуваахгүйгээр нэгтгэн дүгнэх хийх боломжгүй юм. Бага зүйлээр хязгаарлая. Бид хүн төрөлхтний түүхийн тухай ярихдаа голдуу түүхийн хөдөлгөөний нийгмийн хэлбэрийг ойлгодог, өөрөөр хэлбэл хүн төрөлхтний дэвшилтэт хөгжлийг спиралаар явж буй бүхэл зүйл мэт үздэг. Энэ хөдөлгөөн аяндаа явагддаг, ганц үүгээр л гэхэд гаднын ямар ч хүчин зүйл байлаа гэсэн үүрэг гүйцэтгэж чадахгүй юм.
Түүхийн энэ тал руу газар зүйн ч, биологийн ч үйлчлэл нөлөөлж чадахгүй. Тэгвэл эдгээр нь юунд нөлөөлдөг юм бэ ? Организмуудад, түүний дотор хүмүүст нөлөөлдөг байна. Энэ гаргалгааг 1922 онд Л.С.Берг бүх организм, түүний дотор хүмүүсийн хувьд: “Газар зүйн ландшафт нь организмд албадан үйлчилж, бүх амьтдыг зүйлийн зохион байгуулалтын хийдэг шиг тодорхой чиглэлд хувиран өөрчлөгдөхийг тулгадаг юм. Тундр, ой, тал, цөл, уул, усан орчин, арал дээрх амьдрал-энэ бүхэн нь организмд онцгой ул мөр үлдээдэг” хэмээн хийчихсэн юм.
Дасан зохицох чадваргүй зүйлүүд нь газар зүйн өөр ландшафтад нүүн суурьших буюу эсвэл мөхөж сөнөх ёстой аж.
16 Берг Л. С. Номогенез. Пг., 1922. С. 180-181.
Харин “ландшафт” гэдгийг “бусад хэсгээс чанарын хувьд ялгаатай, байгалийн хилээр хөвөөлөгдсөн, юмс үзэгдлийн бүхэллэг, харилцан нөхцөлдсөн зүй тогтлыг төлөөлсөн, ихээхэн хэмжээний орон зайн хэв шинжээр илэрхийлэгдсэн, ландшафтын бүрхүүлтэй бүх талаараа салшгүй холбоотой дэлхийн гадаргуугийн хэсэг” гэж ойлгож байна.
17 Калесник С. В. Основы общего землеведения. С. 455.
Энэ бүхнийг нийлүүлэн “хөгжих орон” гэж нэрлэж болно. Энд томъёолсон сэдвийг Л.С.Берг хувьслын хромикийн (“хорос” гэдэг нь байр, орон гэсэн грек гаралтай үг юм) зарчим гэж нэрлэсэн бөгөөд ийнхүү газар зүйг биологитой холбосон билээ.
18 Савицкий П.Н. Географические особенности России (1). Праге, 1927. С. 30-31.
Бидний үзэж буй аспектад дээр дурдсан хоёр шинжлэх ухаан дээр түүх нэмэгдэж байгаа бөгөөд ингэсэн хэдий ч зарчим өөрчлөлтгүй үлдэх юм. Түүнээс гадна энэ нь шинэ хүлээж байгаагүй баталгаатай болсон бөгөөд энэ нь бидний угтсаатны хөгжлийн зүй тогтлыг үргэлжлүүлэн авч үзэхдээ шинэ угсаатан бий болох эрчимт үе, угсаатны нийлэгжилтийн үеийг тодорхойлсны үндсэн дээр угсаатны нийлэгжилтийг харгалзан үзэхийг даалгаж байна. Гэхдээ энэ нь өөр бүлгийн сэдэв болно.
II. Байгал ба түүх
\section{БАЙГАЛ СУДЛАЛ БА ТҮҮХИЙН ХОСЛОЛ}
Харагдах ялгаа, саланги мэт санагдах харилцан холбоо байсан хэдий ч хүн ертөнцийг бүхэллэг байдлаар төсөөлж байсан эрт үед байгал судлал, түүх хоёрыг холбох асуудал үүсч ч чадахааргүй байв. Мөнхжүүлж болно гэж үзсэн бүхий л үйл явдлыг түүхийн сударт оруулсан. Дайн, үер, төрийн эргэлт, өвчин тахал, суут хүн төрөх, солир нисэх энэ бүхнийг хойч үеийнхний сонирхолд адил тэгш ач холбогдолтой үйл явдлууд гэж үзэж байжээ. Тэр үед шинжлэх ухааны сэтгэлгээнд байгалийн үзэгдэл болон ард түмний буюу тодорхой хүний хувь заяаны хоорондын холбоог өргөн холбооны замаар илрүүлэн гаргадаг “төстэй зүйл төстэй зүйлийг төрүүлдэг” гэсэн ид шидийн зарчим ноёрхож байлаа. Энэ зарчим нь астралоги болон мантикт (мэргэ төлөгийн тухай шинжлэх ухаан) хөгжлөө олсон бөгөөд тусгай шинжлэх ухаан хөгжиж, мэдлэг хуримтлахын хэрээр тэдгээрийг явцгүй, практикт хэрэглэх үед өөрийгөө харуулж чадаагүйн улмаас гээсэн билээ.
XYIII–XIX зууны үед шинжлэх ухааны ялгарлын ачаар асар их мэдээлэл хуримтлагдаж, ХХ зууны эх гэхэд зах хязгааргүй болсон юм. Дүрслэн өгүүлбээс, Шинжлэх ухааны хүчирхэг голыг усжуулах суваг руу цутгаж эхэлсэн юм. Амь оруулах чийг өргөн газар нутгийг тэжээсэн авч өмнө нь түүний тэжээдэг байсан нуур, өөрөөр хэлбэл бүхэллэг ертөнцийг бясалгах үзэл хатах болов. Ингээд намрын салхи ёроолын хөрсийг үлээж, талбайн хөвсөрсөн шороог давсархаг салхи бордох болов. Удалгүй энэ талын оронд хуурай ч гэсэн мал тэжээх хужир марз үүсэх бөгөөд био хүрээ бүдүүлэг бодист байраа мэдээжийн хэрэг үүрд биш ч, удаан хугацаагаар тавьж өгнө. Сүйрсэн газрыг хүмүүс орхин явахад сувгуудад лаг сууж, гол мөрөн гольдрилоо тавьж, байгалийн санг дүүргэнэ. Хужир марзыг салхи нарийн цэвэр тоосоор хучиж, түүн дээр мал идэлгүй үлдсэн ногооны үр унана. Хэдэн зууны дараа энэ тал газар ялзмагийн давхраа, нууранд нялцгай биетэн үүснэ, улмаар өвс идэштэн ирж, усанд хөвөгч шувуу савраараа загасны түрс авчирч нууранд орхино…Ингээд л олон янз байдлаараа амьдрал цэцэглэнэ.
Шинжлэх ухаанд ч мөн ийм байдаг. Шинжлэх ухааны нарийсал нь мэдлэг хуримтлуулах хэрэгслийнхээ хувь л ашигтай байдаг. Дифферениаци хэмээх ялгарал нь хэрэгтэй бөгөөд гарцаагүй үе шат боловч хэрэв удаанаар сунжирвал хөнөөлтэй болдог. Мэдээллийг системчлэлгүй хуримтлуулж, түүнийгээ өргөнөөр нэгтгэн дүгнэх зүйл болгох нь угаасаа учир утгагүй явдал юм. Тэгвэл эртний шинжлэх ухааны зарчмууд хуурамч байсан уу? Магадгүй л юм. Тэдгээрийн чадваргүй шинж нь постулатуудад биш, харин тэдгээрийг хэрэглэж чадахгүйд байсан юм биш үү ? Хуримтлагдсан мэдлэгийн нийлбэр, бидний нүдний өмнө хөгжиж байгаа судалгааны арга зүйг ашиглан барин авч болох “байгалийн түүх ба хүмүүсийн түүх”–ийн харилцамж байгаа л шүү дээ. Одоо энэ замаар явахыг оролдож, түүхийг байгалийн үзэгдлийг тайлбарлах үед ашиг тус өгөхөөр судалж болох уу гэж зорилтоо томъёолъё.
Нийгмийн болон байгалийн үзэгдэл нь адилтгашгүй боловч хаа нэгтээ шүргэлцлийн цэгтэй байгаа нь тодорхой юм. Түүнийг л олох хэрэгтэй бөгөөд яагаад гэвэл энэ нь антропо хүрээ бүхэлдээ байж болохгүй билээ. Антропо хүрээг био масс хэмээн үзлээ ч гэсэн үзэгдлийн хоёр талыг тэмдэглэх нь зүйтэй. Энд : а) мозаичный буюу шигтгэмэл шинж, учир нь хүмүүсийн янз бүрийн хамт олон хүрээлэн буй орчинтой янз бүрээр харилцдаг, хэрэв сүүлийн таван мянган жилийн сайтар мэдэгдэж буй түүхийг харгалзан үзвэл энэхүү олон янз байдал, түүний шалтгааныг тайлбарлах нь тавигдсан асуудлын түлхүүр болж болно, 2) хүн төрөлхтөн хэмээх судлах зүйлийн олон талт шинж, Үүнийг дараах утгаар ойлгох хэрэгтэй. Хүн бүр (буюу хүн төрөлхтөн бүхэлдээ) бодит бие, организм, ямар нэг биоценезийн дээд мөчир, мөн нийгмийн гишүүн, ястны төлөөлөгч гэх мэт болж байдаг. Дээр дурдсан жишээ тус бүрт уг зүйлийг (тухайн тохиолдолд хүн) харгалзах шинжлэх ухаанууд судалгааны бусад талуудыг үгүйсгэлгүй судалж байдаг. Манай асуудлын хувьд гэвэл бүхэл хүн төрөлхтний чухамхүү угсаатны тал чухал болно.
Одоо танин мэдэхүйгээр бага шиг аялая. Шууд ажиглалтад юу өртдөг вэ? гэж өөрөөсөө асууя. Энэ нь юмс биш, харин юмсын зааг байдаг байна. Бид далайн ус, газар дээрх тэнгэрийг хардаг, учир нь эдгээр нь эрэг, агаар, уулстай хиллэж байдаг аж.
Гэхдээ далайн загас зөвхөн түүнийг барьж, агаарт татан гаргахад л усны оршин байгаа тухай таамаглаж чадах байх. Энэ лугаа бид цаг хугацаа гэсэн ойлголтыг мэднэ, гэхдээ түүний хилийг харахгүй болохоороо нийтээр хүлээн зөвшөөрсөн тодорхойлолт өгөх боломжгүй. Мөн ялгарал нь хэчнээн хүчтэй байвал бидний хараагүй байгаа зүйл бидний хувьд улам тодорхой болдог, энэ талаар бид гүйцээн бодож, сэтгэн дүрсэлдэг.
Түүхийг үйл явдлын гинж байдлаар бид байнга ажигладаг. Эндээс түүх бол зааг хил юм, аз болоход тэр нь материйн болон байгалийн дөрвөн улирлын хөдөлгөөний хэлбэр гэдгийг мэднэ. Хэрэв ийм байх ахул социо хүрээ, түүний бий болгосон техно хүрээний зэрэгцээгээр зөвхөн хүмүүсийн эргэн тойронд байх бус, харин тэдэнд өөрт нь байх ямар нэг амьд мөн чанар бий. Энэ жам нь хүний ухамсар өчүүхэн ч түвэг гаргалгүйгээр барьж болохуйц ихээхэн ялгаатай. Чухам ийм учраас л хүмүүнлэгийн концепцууд хэрэггүй, тодруулж хэлбэл хангалтгүй болдог ба эдгээр нь түүхэн үйл явц буюу газар зүй, биологи, нийгмийн (идеалист системд) буюу оюун санааны хүчин зүйлийн нөлөөллийн асуудлыг тавьснаас биш, аль алийг нь хослуулах, үүний ачаар үйл явцыг өөрийг нь болон түүний бүрдүүлэгч хэсгүүдийг эмпирик байдлаар нэгтгэн дүгнэх боломжтой болгох асуудлыг дэвшүүлээгүй билээ.
Энд санал болгож буй хандлага нь анализаас өөр зүйл биш, өөрөөр хэлбэл түүхэн дэх тодорхойгүй байрыг “тайлах”-ад зайлшгүй ”хэсэгчлэн задаргаа” хийж, дараа нь судалгааны янз бүрийн арга зүйн үр дүнг харгалзсан синтез хийх явдал болно.
XIX зууны түүх судлалд нийгмийн болон байгалийн харилцамжийг үргэлж харгалзаж байсангүй.
19 Плеханов Г. В. Нечто об истории //Соч.: В 24 т. Т. 8. М., Л., 1923. С. 227.
Гэхдээ одоо байгалийн үйл явцын өөрчлөлтийг судалж, түүнийг түүхийн үйл явдлуудтай жиших болсон нь тодорхой юм. Биоценологи нь хүн ландшафтын биоценезэд дээд төгс салбар болон орсон бөгөөд учир нь тэр том махчин агаад ийм байдлаараа байгалийн хувьсалд захирагдана, гэхдээ энэ нь техно хүрээг бүтээж, өөрийн хөгжлөөс ангижирсан, зөвхөн эвдрэн сүйрэх чадвартай үйлдвэрлэх хүчний хөгжил гэсэн нэмэлт агшин байхыг үгүйсгэхгүй гэдгийг харуулсан.
\section{ФОРМАЦИ БА УГСААТАН}
Дашрамд дурдахад хэрэв бид дэлхийн бүх түүхийг авч үзвэл формаци халагдах болон шинэ ард түмэн бий болох явдлууд давхцах нь угаасаа л ховор явдал, харин нэг формацийн хүрээнд өөр хоорондоо төстэй угсаатнууд байнга үүсч, хөгжиж байдгийг ажигдаж болно.
Феодализм Атлатикаас Номхон далай хүртэл цэцэглэж агсан XII зууны хоёр жишээ авъя. Францын баронууд Скандинавын чөлөөт тариачид, Египетийн боол-цэрэг-мамлюкуудтай оросын бүүдгэр хотуудын омголон хүмүүс, дэлхийн талыг эзэлсэн ядуу зүдүү монголын нөхрүүдтэй, эсвэл Сүн гүрний хятадын газрын эзэдтэй адил байсан гэж үү ? Эдгээр бүх хүмүүст үйлдвэрлэлийн феодалын арга л нэгдмэл юм нь байлаа, харин үлдсэн бүх зүйлд тэдний хооронд нийтлэг гэхээр юм бага байв. Тариаланч, нүүдэлчин хоёрын байгальд харьцах харьцаа таардаггүй байсан, Европ дахь бусдын юманд халдах буюу соёлын зүйл хуулбарлах чадвар нь загалмайлтны аян дайныг урамшуулагч газар нутаг булаан эзлэх эрмэлзэлтэйгээ адилаар Хятадаас хамаагүй дээгүүр байсан, Оросын гар тариалан хөдөлмөрийн бага зардлаар арвин ургац авдаг Сири, Пелопоннесийн усан үзмийн тариалангаас хялбар бөгөөд бүдүүлэг байсан, улмаар хэл, шашин, урлаг, боловсрол зэрэг бүх зүйл нь төсгүй байв. Гэхдээ энэ бүх олон янз байдал нь огт замбараагүй биш байсан юм. Амьдралын хэвшил тус бүр тодорхой ард түмний хүртээл байлаа. Энэ нь ялангуяа угсаатны бүтээж, тэжээгдэж байсан ландшафтын хувьд илт тодорхой байдаг.
Зөвхөн байгал л угсаатны давтагдашгүй байдлын хэмжээг тодорхойлдог хэмээн бодож болохгүй юм. Эрин зуун солигдож, угсаатны харьцаа өөрчлөгдөж, тэдний зарим нь алга болж, зарим нь үүсч байдаг, энэ үйл явцыг зөвлөлтийн шинжлэх ухаанд этногенез буюу угсаатны нийлэгжилт хэмээн нэрлэдэг болжээ. Дэлхийн нэгдмэл түүхэн дэх угсаатны нийлэгжилтийн хэмнэл нь нийгмийн хөгжлийн лугшилттай холбоотой байсан ба энэхүү холбоотой гэдэг нь давхцана, тэр тусмаа нэгдэнэ гэсэн үг биш юм. Түүхийн үйл явцын хүчин зүйл янз бүр учраас бидний зорилт бол угсаатны нийлэгжилтэд шууд хэвшмэл байх цаад үзэгдлийг ялган үзэх, ингэснээрээ угсаатан гэж юу вэ, хүн төрөлхтний амьдралд тэр нь ямар ямар үүрэг гүйцэтгэдэг вэ гэдгийг өөртөө тайлбарлахад оршино.
Эхлээд нэр томъёоны утга учир, судалгааны зааг хүрээг тохирох хэрэгтэй. “Угсаатан” хэмээх грек үг нь толь бичигт олон утгатай байдаг авч бид ”зүйл, үүлдэр”гэсэн нэг л утгыг сонгон авсан. Энэ нь хүнийх гэдэг нь ойлгомжтой. Бидний дэвшүүлж буй сэдэвт “аймаг”, “үндэстэн” гэсэн ойлголтуудыг ялган гаргах санаа байгаагүй, яагаад гэвэл бид хашилтаас гаргаж болох тийм үгийг, өөрөөр хэлбэл англичуудад ч, масаичуудад ч, эртний грекүүд болон орчин үеийн цыгануудад ч нийтлэг байх үгийг сонирхсон юм. Homo sapiens зүйлийн энэ чанарыг өөрийн болон үлдэх бүх ертөнцөд “өөрийн” (заримдаа ойрхон, гэхдээ голдуу холын) хэмээн эсрэгцүүлэн тавьж болох бүлэглэлд оруулсан юм. “Бид–тэд” (conditiosine qua non est) гэж сөргүүлэн тавих нь бүх эрин үеийн бүх улс орнуудад хэвшмэл байдаг. Эллин, иудей, үс сахалгүй хятадууд (Дундад улсын хүмүүс), мөн ху (бүдүүлэг зах хязгаар, түүний дотор оросууд), анхны халифын үеийн араб–мусульманууд, “бурхангүйчүүд”, Дундад зууны евро–католикууд (“Христианы ертөнц” хэмээх нэгдэл), шударга бусчууд, түүний дотор грекүүд, оросууд, “үнэн алдартнууд” (энэ эрин үед), мөн католикуудыг оруулсан “христ бусчууд”, туареги, туареги бусчууд, цыганууд болон үлдэх бүх хүмүүс гэх мэт. Ингэж сөргүүлэн тавих үзэгдэл нь түгээмэл байсан бөгөөд энэ нь түүний гүн гүнзгий дэд үндсийг зааж байна, гэхдээ энэ нь өөрөө бол их уст голын хөөс төдий бөгөөд үүний мөн чанарыг бид олох ёстой. Угсаатны (“үүлдэр” гэсэн утгаар) гэж нэрлэж болох, социал, соёл, улс төр, шашин, болон бусад олон зүйлийг бий болгодогтой адилаар хүн төрөлхтний угсаатны түүхийг бүтээх аспект болж чадахуйц энэ үзэгдлийн нарийн нийлмэлийг цохон дурдахад урд өмнө хийсэн ажиглалт хангалттай билээ. Ийм учраас бидий зорилт бол юуны өмнө үйл явцын зарчмыг барьж авах явдал мөн.
Угсаатны соёл болон газар зүйн холбоо эргэлзээгүй бөгөөд гэхдээ түүнээс байгалийн олон талт үзэгдлүүд угсаатны түүхийн эргэлттэй харилцах харилцааны бүхий л нарийн ширийнийг шавхан гаргаж болохгүй юм. Түүнээс гадна угсаатны ангиллын үндэст байгаа дурын шинж тэмдэг нь тодорхой орчинд дасан зохицсоны шинж гэсэн сэдэв нь угсаатны нийлэгжилтийн үйл явцын зөвхөн нэг л талыг тусгадаг. Гегель “Ионийн цаг агаарыг Гомерийн уран бүтээлийн шалтгаан хэмээн зааж болохгүй” гэж бичсэн байдаг.
20 Гегель Ф. Философия истории //Соч.: В 14 т. Т. 8. М., 1935. С. 72.
Гэхдээ ландшафтад дасан зохицох нь дээд зэргийн байгаа, тодорхой бүрэлдсэн нутагт угсаатан нь нүүх үедээ уугуул угсаатнаас ялгагдах анхдагч олон шинжээ хадгалан үлддэг. Тухайлбал, испаничууд Мексикт нүүн суурьшихдаа ацтек буюу майн индианчууд болчихоогүй юм. Хэдийгээр Юкатаны чийглэг бүс, Анаукагийн хагас цөл нь Андалуси болон Кастилаас ихээхэн ялгаатай байсан хэдий ч испаничууд хот, бэхэлсэн далан бүхий микро ландшафт байгуулж, өөрийн материалын ч, оюуны ч соёлоо хадгалан үлдсэн юм. XIX зуунд Мексик (тэр үед түүнийг Шинэ Испани гэж нэрлэж байв) Испаниас тусгаарласан нь испани хэл, католик шашныг авсан индианчуудын хойч үеийнхнээс болон тэд Рио–Грандаас хойгуур хэрэн хэсэгч комончей хэмээх чөлөөт овгоор дэмжүүлснээс ихээхэн хэмжээгээр болсон хэрэг байлаа.
Одоо цаашдын тайлбарлалд эхлэл болох анхны дүгнэлтийг хийе. Дэлхий гаригийн ландшафттай харилцан үйлчилж, түүхэн цаг хугацаанд байнга өөрчлөгдөж байдаг зүйсэн мэт антропо хүрээ нь этно хүрээнээс өөр зүйл биш юм. Учир нь хүн төрөлхтөн хуурай газрын гадаргууд хаа сайгүй боловч жигд бус тархсан, мөн Дэлхийн байгал орчинтой ямагт, гэхдээ янз бүрээр харилцдаг энэ явдлыг Дэлхийн бүрхүүлүүдийн нэг, чингэхдээ угсаатны ялгааны засвар заавал оруулсан байдлаар авч үзэх нь зохистой. Ийм маягаар бид “угсаатны хүрээ” гэсэн нэр томъёог гарган ирж байна. Угсаатны хүрээ нь газар зүйн бусад үзэгдлийн адил биологийн болон нийгмийнхээс ялгаатай өөрийн гэсэн зүй тогтолтой байх ёстой. Угсаатны зүй тогтлыг орон зайд (угсаатны зүй), мөн цаг хугацаанд (угсаатны нийлэгжилт болон антропоген ландшафтын палеогеографи буюу хүний өсөн үржих газар орчны эртний газар зүй ) авч үздэг.
ТҮҮХЭН ЭХ СУРВАЛЖИД ИТГЭЖ БОЛОХ УУ ?
XY–XYIII зууны газар зүйн сэтгэлгээний гайхамшигт тоймын зохиогч В.К. Яцунский: “Түүхэн газар зүй нь өнгөрсний тухай хүмүүсийн газар зүйн төсөөллийг судалдаггүй, харин өнгөрсөн зуунуудын тодорхой газар зүйг судалдаг” гэж шударгаар тэмдэглэсэн байдаг.
21 Яцунский В. К. Историческая география. М., 1955. С. 3.
Энэхүү хайгуулын анхдагч өгөгдөхүүнийг өнгөрсөн эрин үеүдийн түүхэн бүтээлүүд дундаас хайх ёстой нь мэдээж юм. Тэгэхдээ яаж ? Харамсалтай нь судалгааны боломжит арга зүйн талаар ямар ч заалт байдаггүй юм. Учир нь ийм юм аа.
Эртний цаг агаарын нөхцлийг сэргээн тогтоох сурвалж болсон түүхэн материалуудыг маш өргөнөөр хэрэглэдэг. Ийм маягаар түүхэн үед Төв ази хуурайшсан тухай асуудлаар Л.С.Берг болон Г.Е. Грумм –Гржимайло нарын хооронд болсон алдарт маргаан хөгжсөн байдаг.
22 Берг Л. С. Климат и жизнь. М., 1974. 23 Грумм-Гржимайло Г. Е. Рост пустынь и гибель пастбищных угодий и культурных земель в Центральной Азии за исторический период //Изв. ГО. 1933. Т. XI. Вып. 5.

Энэ асуудалтай холбоо бүхий НТ-ын I зуунд Каспийн тэнгисийн түвшин хэлбэлзсэн асуудлыг эртний зохиогчдын бүтээлүүдээс ишлэл сонгон авах замаар шийдвэрлэхийг оролдсон юм.
24 Берг Л. С. Уровень Каспийского моря в историческое время: Очерки по физической географии. М.; Л., 1949. С. 205- 279; Шнитков А. В. Ритм Каспия // Доклады АН СССР. 1954. Т. 94. № 4; Аполлов Б. А. 1) Доказательство прошлых низких состояний уровня Каспийского. М., 1951; 2) Колебания уровня Каспийского моря //Труды института океанологии. 1956. Т. XV.
Дорнод европын цаг агаарын өөрчлөлтийн талаар дүгнэлт хийхийн тулд Оросын он цагийн бичгүүдийн мэдээллээс тусгайлан түүвэрлэлт хийсэн байна.
25 Бетин В. В., Преображенский К). В. Суровость зим в Европе и ледовитость Балтики. Л., 1962; Бучинский И.Е. Очерки климата Русской равнины а историческую эпоху. Л., 1957.
Гэтэл олон тооны, ихээхэн хөдөлмөр зарсан судалгааны дүн найдлага биелүүлсэнгүй. Заримдаа эх сурвалжийн мэдээлэл батлагдаж, заримдаа өөр замаар шалгахад үгүйсгэгдэж байлаа. Гарган авсан өгөгдөхүүн нь үнэнтэй давхцах нь тохиолдлын явдал болох нь илт байсан ба энэ нь арга зүй төгс бусыг харуулсан юм. Үнэн хэрэг дээрээ эртний буюу дундад зууны зохиогчийн гэрчилснээс зүгээр иш татах зам нь буруу буюу сайндаа л гэхэд бүрхэг дүгнэлтэд хүргэж байв. Ийм ч байх ёстой юм.
Эртний түүх бичигчид байгалийн үзэгдлийн талаар эсвэл далимаар, эсвэл тухайн үеийнхээ шинжлэх ухааны төсөөллөөс ургуулан бичиж, аянга цахилгаан, үер ус, ган зэргийг зөн совин буюу нүглийн залхаалт мэтээр тайлбарласан байдаг. Аль ч тохиолдолд байгалийн үзэгдлийг сонголттой бичдэг байсан агаад тэдгээрээс зохиогчийн анхааралд өртөхдөө хэд нь орхигдсоныг бид таамаглаж ч чадахгүй юм. Нэг зохиогч нь байгалийг анхаарч байсан бол дараагийн зууны өөр нэг зохиогч нь эс анхаарна, магадгүй хуурай үеийг бороог чийгтэй үеийнхээс илүү олон дурдсан байж болно. Түүхэн шүүмжлэл туслах чадваргүй бөгөөд учир нь шалтгаан-үр дагаврын хамаарлаар холбогдоогүй үйл явдлуудыг алгассан тухайд энэ нь хүчгүйдэх билээ.
Эртний зохиогчид бүтээлээ ямагт тодорхой зорилгын тулд бичдэг байсан ба сонирхол татаж буй үйл явдлынхаа ач холбогдлыг ёс мэт дөвийлгэдэг байжээ. Дэврүүлсний болон дарсны хэмжээг тодорхойлоход маш хүнд, ийм боломж цөөн байдаг.
26 Бетин В. В., Преображенский К). В. Суровость зим в Европе и ледовитость Балтики. Л., 1962; Бучинский И.Е. Очерки климата Русской равнины а историческую эпоху. Л., 1957.
Ингээд Л.С.Берг түүхэн бичиглэлийн үндсэн дээр тариалангийн газар цөл болон хувирах нь дайны үр уршиг байдаг гэсэн дүгнэлт хийжээ. 27 Берг Л. С. Климат и жизнь.
Эдүгээ энэ концепцийг шүүмжлэлгүйгээр хүлээн авч, П.К. Козловын олдвор болох Хар Хот нэрээр алдаршсан тангадын сөнөсөн хот Эзэн–Айг жишээ болгон юунаас ч илүү авах болжээ. Энэ агшин нь ихэд сургамжтай учраас бид энэ хотын газар зүйн байрлал болон түүний мөхлийн нөхцөл гэсэн ганцхан асуудалд анхаарлаа хандуулъя. 28 Мерперт Н.Я., Пашуто В. И., Черепин Д.В. Чингис-хан и его наследие //История СССР. 1962. № 5. С. 56.
Тангадын хаант улс эдүгээ элсэн цөл болсон тэр газар буюу Ордос болон Алшаад нутаглаж байжээ. Энэ улс ядуу, цөөн хүнтэй байх ёстой мэт боловч үнэн хэрэг дээрээ 150 мянган морьтон бүхий армитай, их сургууль, академи, сургууль, шүүх, тэр ч байтугай ховор зүйлийн худалдаа хийдэг байв. Тэр нь гаргаснаасаа илүү ихийг оруулдаг байжээ. Алдагдлыг төвдийн эзэмшлээс авсан алтан элсээр, хамгийн гол нь Тангадын хаант улсын баялгийг бүрдүүлдэг амьд мал гарган хаадаг байжээ.
29 Грумм-Гржимайло Г. Е. Рост пустынь…
П.К. Козловын нээсэн хот нь эдүгээ усгүй болсон Эзэн голын доод савд оршино. Хотыг баруун болон зүүнээс нь хүрээлэн буй хоёр далан нь энд устай байсныг харуулдаг, гэвч гол баруун тийш гольдрилоо сольж, эдүгээ хоёр салаалан давст Гашуун нуур, цэнгэг Согоо нуур хоёрт цутгадаг аж. П.К.Козлов Согоо нуурын хөндийг хүрээлэн буй цөлийн дундах гайхамшигт баян бүрд гэж үзэхийн хамт олон тооны хүнийг тэжээх боломжгүй гэж тэмдэглэсэн байдаг. Зөвхөн Эзэн–Ай хотын цайз гэхэд л 400 м тал бүхий тэгш дөрвөлжин байжээ. Эргэн тойронд нь арай бага шиг байгууламжийн ул мөр, суурингууд байсныг үзүүлэгч шаазангийн хэлтэрхий байна. Үнэхээр ч 1227 онд Чингис хаан Тангадын нийслэлийг эзлэн авч, монголчууд хүн амыг нь хядсан юм. Гэхдээ П.К.Козловын нээсэн хот бүр XIY зуун гэхэд амьдарсаар байсан бөгөөд үүнийг түүний ахалсан экспедицийн ажилтнуудын олсон олон тооны баримт бичгийн цаг хугацаа гэрчилдэг юм. Түүнээс гадна хотын мөхөл нь торгууд ардын домгоор бол бүслэгчид шороо дүүргэсэн шуудайгаар далан барьсны улмаас голын урсгал өөрчлөгдсөнтэй холбоотой гэж үздэг аж. Энэ далан нь одоо болтол товгор хэлбэрээрээ хадгалагдан байна. Бичиглэлүүдэд Урахай (монгол) болон Хэчуйчен (хятад) хотыг эзлэн авсан тийм баримт байдаггүй юм. Бүр Дундад зууны үед эхэлсэн муу муухай болгоныг монголчуудад тохох тэнэг уламжлалын дагуу энэ хотын мөхлийг монголчуудад ногдуулсан юм. Үнэн хэрэг дээрээ Тангадын энэ хот 1373 онд мөхсөн. Хотыг тэр үед Чингисийн сүүлчийн удмынхантай дайн хийж байсан Мин гүрний цэргүүд эзлэн авч, Хятадыг баруун талаас нь заналхийлж байсан монголчуудын тулгуур цэгийнх нь хувьд сүйрүүлсэн билээ.
30 Руденко С. И., Гумилев Л. Н. Археологические исследования П. К. Козлова в аспекте исторической географии //Изв. ВГО. 1966. Выл 3. С. 244.
Тэгвэл яагаад хот сэхээгүй юм бэ ? Голын урсгал өөрчлөгдөх нь шалтгаан биш, учир нь хотыг Эзэн голын өөр цутгаланд шилжүүлэн авчирч болох байсан. Энэ асуултын хариуг П.К.Козловын номоос олж болно. Тэр өөрт байх нямбай зангаараа Эзэн голын ус татарч, Согоо нуурын ус ширгэж, зэгс ургаж эхэлснийг тэмдэглэсэн байдаг. Голын гольдрил баруун тийш шилжсэн нь энд зарим нэг үүрэг гүйцэтгэсэн боловч энэ нь XIII зууны үед асар олон хүнийг тэжээж байсан улс орон яагаад ХХ зууны эх гэхэд элсэн цөл болон хувирсан бэ? гэдгийг тайлбарлаж чадахгүй юм.
Ингээд Азийн тариалангийн газар цөлжсөний буруутан нь монголчуудад биш, харин бидний дараагийн ажилдаа дүрслэн үзүүлэх үзэгдэл, цаг агаарын өөрчлөлтөд ногдох болж байна.
31 Гумилев Л. Н. 1) Гетерохронность увлажнения Евразии в древности (Ландшафт и этнос. IV) //Вестник ЛГУ. 1966. № 6. С. 64-71; 2) Гетерохронность увлажнения Евразии в Средние века (Ландшафт и этнос. V) //Там же. 1966. № 18. С. 81- 90.
ХӨШӨӨ ДУРСГАЛД ИТГЭЖ БОЛОХ УУ ?
Чухам яагаад Чингис хаан болон түүний хүүхдүүдэд Азийг цөлмөсөн явдлыг тохдог мөртлөө үүнээс илүү том цар хүрээтэй үйл явдал, жишээлбэл, 841-846 онуудад хиргисүүд уйгаруудыг бут ниргэсэн, эсвэл 1756–1758 онуудад Манжийн эзэн хаан Цянь Лун (монголоор Тэнгэрийг тэтгэсэн хаан–Орч) хаан ойрадуудыг толгой дараалан хүйс тэмтэрсэн явдал түүхчдийн анхаарлын гадна үлдээд байдаг юм бэ ?
32 Хятадын эзэн хаан Цянь Лун Ойрдуудыг нийтээр нь хүйс тэмтэрч, чингэхдээ хүүхэд, эмэгтэйчүүд, хөгшдийг ч хүртэл өршөөгөөгүй юм. Албан ёсны Хятадын түүх: “Ойролцоогоор сая орчим ойрдыг алсан” гэсэн товч лавлагаагаар хязгаарлагддаг. Агуу том үйл явдал албаны явдалд живсэн бөгөөд энэ нь харамсалтай нь цорын ганц тохиолдолд биш аж ?! Харамсалтай нь хүн төрөлхтний түүх бидэнд тодорхой байдлын хувьд янз бүр байдаг, энэ нь газар зүйч нь нэг нь 1: 200 000, нөгөө нь 1:100 харьцаатай зургийг зураглалын нэг самбарт зурсантай адил юм.

Энэ асуултын хариуг ард түмнүүдийн түүхээс биш, харин түүх судлалаас хайвал зохино. Түүхийн гайхамшигтай номыг үргэлж, шалтгаан бүрээр бичээд байдаггүй, түүнээс гадна эдгээр ном бүгдээрээ бидний үед хүрч ирээгүй. XIY–XY зууны үе бол Ойрхи Дорнодод утга зохиолын цэцэглэлтийн эрин үе байлаа, гэтэл энэ үед Перс болон Оросод монголын дарлалтай хийх тэмцэл нь хамгийн чухал асуудал байсан юм. Ийм учраас ч манай үеийг хүртэл бүтнээрээ үлдсэн энэ явдалд зориулсан олон бүтээл байдаг. Тэдгээрийн дунд авъяаслаг, хурц бүтээлүүд байсан бөгөөд эдгээрийг бид мэдэх юм. Эдгээрийг дуурайж, давтсаар байгаад тухайн асуудлаарх бүтээлийн нийт тоог ихээхэн нэмэгдүүлжээ. Ойрадуудыг бас түүхийн хуудсанд бутниргэсэн явдал нь өөрийн түүхчийг олоогүй буюу эсвэл тэр нь алаан хядаанд үрэгджээ. Ийм маягаар үйл явдлуудыг жигд биш тусгадаг, тэдгээрийн ач холбогдлыг гажуудуулдаг байсан аж, учир нь эдгээрийг янз бүрийн масштабаар бичсэн мэт төсөөлөл төрдөг.
Чухам эндээс л эзэлж авсан орныхоо хүн амыг бараг толгой дараалан хүйс тэмтэрдэг, газар нутгийг нь бүрэн өөрчилдөг гэсэн үнэнд огт үл нийцэх Чингисийн цэргүүдэд тохдог таамаглал үүссэн билээ. Хамгийн их хуурайшилд дайнд нэрвэгдсэн орнууд биш, харин огт дайн байгаагүй Уйгар, өвслөг талыг нь хэн ч эзлэх гээгүй Зүүнгар өртсөнийг тэмдэглэвэл зохино. Иймээс эх сурвалжийн түүх– газар зүйн баримтууд нь найдваргүй юм.
Эцэст нь агуу их түүхэн үйл явдал, жишээлбэл, XIII зууны монголчуудын аян дайныг миграци буюу их нүүдэл гэж үздэг нь сонирхолтой байдаг. Нэрт эрдэмтэн Э.Хантингтон, Э.Брукс нар ийм үзэлд автсан, гэхдээ монголчуудын аян дайн их нүүдэлтэй холбоогүй байжээ. Ялалтыг морьтны бөөгнөрөл биш, харин багавтар, сайтар зохион байгуулагдсан хөдөлгөөнт отрядууд байгуулж, үүнийхээ дараа төрөлх талдаа эргэн ирдэг байв. Нүүн гарагсдын тоо XIII зуунд ч гэсэн тун өчүүхэн байжээ. Тухайлбал, Зүчийн удмын Бат, Орд–Ичин (Шейбан) нар Чингисийн гэрээслэснээр ердөө л 4 мянган морин цэрэг, өөрөөр хэлбэл, Карпатаас Алтай хүртэлх нутагт тархан байршсан 20 мянга орчим хүн хүлээн авсан байна. Үүний эсрэгээр XYII зуунд болсон халимагуудын жинхэнэ их нүүдлийг Дэлхий дахины түүхийн бүтээлүүдэд том дуулиан үүсгэж чадаагүйн улмаас ихэнхи түүхчид тэмдэглэгдээгүй үлдээжээ. Иймээс тавигдсан асуудлыг шийдвэрлэхийн тулд бэлэн эмхтгэсэн бүтээлээс амархан олж авахаас илүүгээр түүхийн илүү ул суурьтай мэдлэг хэрэгтэй байна, голдуу түүхчид буюу хөдөө аж ахуйн эдийн засагчдаар хязгаарлагддагаас илүү нарийн газар зүйн мэдлэг хэрэгтэй байна. Хамгийн гол нь Геротоос эхлээд манай үе хүртэлх бичмэл эх сурвалжийн олон зохиогчдод хэвшмэл байдгаа субъектив ойлголтоос үнэн магад мэдээллийг ялган авах нь зайлшгүй болоод байна.
Бид үнэн зөв мэдээлэл гэж түүхэн шүүмжлэлийн гал дөлийг туулан гарсан, эргэлзээ төрүүлэхээргүй тайлбарлал олсон эх сурвалжийн мэдээллийг нэрлэж байна. Энэ нь олон байдаг, гэхдээ дийлэнхи хэсэг нь улс төрийн түүхэнд хамаардаг. Бид их тулалдаан, энхийн хэлэлцээр, ордны эргэлт, агуу их нээлтүүдийн он цаг, нарийн ширийнийг сайн мэддэг, гэвч эдгээр өгөгдөхүүнийг байгалийн үзэгдлийг тайлбарлахад хэрхэн хэрэглэх вэ ? Түүхийн баримтыг байгалийн өөрчлөлттэй жишин харьцуулах арга зүйг зөвхөн ХХ зуунд боловсруулж эхэлсэн юм.
Цаг уурын түүхч Э.Леруа Ладюри: Европын янз бүрийн орнуудын аж ахуйн сэргэлт, уналтыг чийгийн их бага, хүйтрэл дулаарлын үеүдтэй холбох гэсэн эрмэлзэл нь эргэлзэхийн аргагүй зүйл болох эдийн засаг, нийгмийн хямралыг үгүйсгэхэд үндэслэгдсэн байдаг. Ингээд Балти орчмын үр тариаг (өөрөөр хэлбэл Оросын) Газар дундын тэнгисийн сав газар нийлүүлэх явдал ихэссэн, XYI, ялангуяа XYII зуунд Испанид хонины тоо толгой багассан зэргийг жилийн температурын үл ялиг өөрчлөлттэй холбох гэхээсээ Европын орнуудад Шинэчлэл (Реформаци) болон Сөрөг шинэчлэлийн учруулсан сүйрэлтэй харьцуулан жиших нь амар юм. Энэ хүн яг үнэн хэлж байна.
33 Леруа Лодюри Э. История климата за 1000 лет. Л., 1971. С. 14-15.
Газар нутаг дээр нь Гучин жилийн сүйрэлт дайн (1618–1648) болсон Герман төдийгүй, үгүйрэлд нэрвэгдээгүй Испани улсад энэ зуунд хүн ам нь 1600 онд 8,0 сая, 1700 онд 7,3 сая болж болж сөрөг өсөлттэй гарсныг тэмдэглэхэд хангалтай юм. Үүнийг залуу эрчүүдийн ихэнхи хэсэг нь Америк буюу Нидерландад дайчлагдаж, үүний улмаас энэ оронд аж ахуй, гэр бүлээ тэтгэх ажлын гар хүрэлцэхгүй болсноор хялбархан тайлбарлаж болно.
“Европын эдийн засгийн хөгжлийг Альпад гарцаагүй тогтсон мөстлөг татарснаар тайлбарлах гэсэн түүхчийн тухай юу бодож болох вэ?” гэхэд Э. Леруа Ладюри түүнийг зөвшөөрөхгүй байх боломжгүй гэж бичжээ.
Улмаар манай зохиогчийн бодлоор нямбай бөгөөд цаг хугацааг нь нарийн гаргасан, дур зоргын тайлбарлалаас чөлөөлсөн баримтуудыг зүгээр л цуглуулах ёстой болж байна. Өөр үгээр бол бидний сонирхож буй хүчин зүйлүүдийг тайлбарлах явдал нь эдийн засаг, нийгэм, угсаатны зүйн хүчин зүйлүүдийн оронд тохиолдлын зүйлээс ангид байгаа гэдэгт бид итгэлтэй байх ёстой аж.
34 Миграци буюу их нүүдэл нь зөвхөн хүний маш нарийн өдөөлт, хөдөлгөгч хүчинд хариулдаг. Өлсгөлөн нь үр тарианы үйлдвэрлэлд хүнд нөхцөл бий болсон үед үүсдэг бөгөөд цаг агаарын хувьд тэдгээрийг хэзээ ч a priori дахин тайлж болдоггүй, цаг агаарын утгаараа заримдаа богино хэмжээний бөгөөд үл ялих цаг агаарын үйл явдлын тухай … ярьж болох юм. (Мөн тэнд. С. 17).
Яг таг он цагийг тодорхойлох нарийн арга зүй газар зүйд байхгүй. Мянган жилийн алдааг бол энд бүрэн боломжтой гэж үздэг. Жишээлбэл, ийм ийм нутагт лаг шавар хөрсийг бүрхсэн байлаа гэхэд эндээс үер болсныг тэмдэглэж болно, харин энэ нь хэзээ вэ? 500 жилийн өмнө үү, 5 мянган жилийн өмнө үү гэдгийг хэлэх боломж байхгүй. Тоосонцрын шинжилгээгээр, жишээлбэл, эдүгээ чийгэнд дуртай ургамал ургаж байгаа тэр газарт хуурай газрын ургамал байсныг илрүүлж болно. Гэхдээ энэ нь уг хөндийн намагшил нь ойролцоох голын гольрил өөрчлөгдсөнөөс болоогүй, ерөөсөө цаг уурийн өөрчлөлтөөс болоогүй гэх баталгаа байхгүй юм. Монгол болон Казакстаны тал нутагт төгөл шугуйн үлдэгдэл илрүүлсний хувьд гэвэл эдгээрийг хуурайшлаас болж үхсэн буюу хүн цавчиж авсан гэж хэлж болохгүй юм. Хэрэв сүүлчийг нь нотоллоо ч гэсэн энэ нь хүн ландшафттай тооцоо бодсон тодорхойгүй эрин үе хэвээрээ л үлдэнэ.
Энд археологичид тус болж магадгүй юм. Материаллаг соёлын өв дурсгалууд нь ард түмнүүдийн сэргэлт, уналтыг харуулж, нэлээд нарийн он цагийг өгүүлдэг. Газарт байгаа юмс буюу эртний шарил нь судлаачдыг төөрөгдөлд оруулах юмуу баримтыг гажуудуулахад нөлөөлдөггүй. Гэвч юмс дуугардаггүй болохоороо археологич хүний сэтгэлгээнд бүрэн орон зай олгодог юм. Хэдийгээр манай үеийн хүний сэтгэлгээний хэв маяг дундад зууныхаас ихээхэн ялгаатай, манай хүн ч гэсэн бас л зөгнөн мөрөөдөхөөс ангид биш учраас түүнийг бодит байдалд хавьгүй ойрхон байгаа гэж үзэх ямар ч итгэл байхгүй.
ХХ зуунд бид заримдаа манай түүхийн мартагдсан хуудсуудыг нээж, судлахад хүргэсэн, Египет, Вавилон, Энэтхэг, тэр ч бүү хэл уулын Алтайд олдсон үнэхээр азтай олдворуудад үндэслэн археологийн малтлагын хүч чадалд мухар сохроор итгэх явдалтай тулгарч байна. Энэ бүхэн нь тохиолдол бөгөөд археологичдын ихэнхи хэсэг нь улайдмал талын хуурай тоосноос гарган авсан гавлын яс, ухсан шарил доторх ясны хэлтэрхий, нэгэн тоосгоны хир өндөр ханын үлдэгдэл зэрэгт сэтгэл ханах ёстой болдог. Үүний дээр энэхүү олсон зүйл нь алга болсон зүйлсийн тун өчүүхэн хэсэг гэдгийг санах хэрэгтэй. Дэлхийн ихэнхи нутагт арьс, даавуу, цаас (эсвэл түүнийг орлох үйс) гэх мэт бараг бүх хөнгөн материалууд үлдээгүй бөлгөө. Чухам юу алга болсон нь хэзээ ч тодорхойгүй, харин алга болсон зүйлийг оршин байгаагүй гэж үзэх, үүнд засвар хийхгүй байх нь буруу дүгнэлтэд анхнаасаа хүргэх алдаа болно. Товчлон хэлбэл, түүхгүй археологи бол судлаачийг төөрөгдөлд оруулж болно. Одоо асуудлыг өөрөөр шийдэхээр оролдоод үзье.
III. УГСААТАН ГЭЖ БАЙГАА ЮУ
УГСААТНЫГ ТОДОРХОЙЛОХ ШИНЖ БАЙХГҮЙ
Бидний санал болгож буй тодорхойлолтоор Homo sapiens хэмээх зүйлийн оршихуйн хэлбэр нь бусад бүх хамтлагийг сөрөн зогссон орь амьтдын хамтлаг юм. Энэ нь түүхэн цаг хугацаанд үүсч, устаж байдаг боловч ямар нэг хэмжээгээр тогтвортой байж угсаатны нийлэгжилтийн асуудлыг бүрдүүлж байдаг юм. Гэсэн хэдий эдгээр хамтлагууд нь заримдаа хэлээрээ, заримдаа ёс заншлаараа, заримдаа үзэл суртлын системээрээ, заримдаа гарал үүслээрээ, харин цаг ямагт түүхэн хувь заяагаараа их бага хэмжээгээр ялгаатай болдог. Эндээс угсаатан нь нэг талаас, түүхэн үйл явцаас уламжлагдсан болно, нөгөө талаас, үйлдвэрлэлийн үйл ажиллагаа – аж ахуйгаараа дамжин өөрийн бүрэлдсэн тэрхүү ландшафтын биоценезтэй холбогдоно. Дараа нь угсаатан энэ харьцаагаа өөрчилж болох ба чингэхдээ тэр нь танигдахгүй болтлоо зүсээ өөрчилдөг, үүний залгамжлалыг түүхэн арга зүй, эх сурвалжийн хамгийн чанд хатуу шүүмжлэлийн тусламжтайгаар л мөшгөж болдог, учир нь үг бол хуурамтгай байдаг.
Цааш явахаасаа өмнө бид хараахан тайлбарлаагүй байгаа “угсаатан” хэмээх ойлголтоо ядахдаа тохирох ёстой. Хэдийгээр дэлхий дээр угсаатны гадуур байсан орь хүн байгаагүй, байхгүй хэдий ч бидэнд дурын угсаатныг байгаа байдлаар нь тодорхойлох бодитой нэг ч шинж тэмдэг байхгүй юм. Дурдсан бүхий л шинж тэмдгүүд угсаатныг “заримдаа” л тодорхойлдог, харин нийлбэр байдлаараа бол ерөөс юуг ч тодорхойлдоггүй. Энэ сэдвийг сөрөг аргаар шалгаж үзье.
Түүхэн материализмын онолд нийгмийн үндсийг нийгэм–эдийн засгийн формаци дотор хэрэгжиж буй үйлдвэрлэлийн арга гэдгийг хүлээн зөвшөөрдөг. Чухам иймээс энд шийдвэрлэх ролийг өөрийн хөгжил, экзогенийн буюу гадаад шалтгаант хүчин зүйлс, түүний дотор нийгмийн дэвшлийн нийлэгжилтэд үндсэн зүйл байж чадахгүй байгалийн хүчин зүйлс гүйцэтгэнэ. “Нийгэм” гэдэг ойлголт нь материаллаг амьдралын түүхэн тодорхой нийтлэг нөхцлөөрөө нэгдсэн хүмүүсийн нийлбэр цогцос юм. Нөхцлийн энэхүү систем дэх гол хүч нь материал баялгийн үйлдвэрлэлийн арга байдаг. Хүмүүс үйлдвэрлэлийнхээ үйл явцад нэгдэж, энэ нэгдлийнхээ үр дүнд нийгмийн харилцаанд орж, энэ нь хүй нэгдэл, боол эзэмшил, феодал, капиталист, социалист гэсэн нэр бүхий таван формацийн аль нэгэнд хэлбэршдэг.
“Угсаатанд нэгдэж” болдоггүй, учир нь аль нэг угсаатанд хамаарахыгаа субъект өөрөө шууд хүлээн авах бөгөөд тойрон хүрээлсэн хүмүүс ч гэсэн үүнийг ямар ч эргэлзээгүй баримт гэж үздэг. Энд угсаатны шинжилгээний үндэс нь мэдрэмж болж байна. Хүн өөрийн угсаатанд бүр багаасаа хамаарагддаг. Заримдаа өөр гаралтныг инкорпораци буюу татан оруулах боломжтой байдаг ч үүнийг их хэмжээгээр хийвэл энэ нь угсаатныг задалдаг. Түүхэн тодорхой нөхцлүүд нь угсаатны амьдралын туршид нэг бус удаа өөрчлөгддөг. Үүний урвуугаар угсаатны дивергенци буюу салалт үйлдвэрлэлийн нэг арга ноёрхож буй үед их ажиглагддаг. Түүхэн үйл явцыг байгалийн түүх болон хүний түүхийн харилцан үйлчлэл хэмээн авч үзсэн К.Марксын санаанаас үүдэн анхны, хамгийн нийтлэг хуваагдал бол техно хүрээнд үүссэн социал өдөөгч, газар зүйн орчноос байнга ирж байдаг байгалийн өдөөгч хэмээн үзэж болно.
35 К.Маркс К., Энгельс Ф. Соч. Т.З. С. 16. 36 Семевский Б. Н. Методологические основы географии //Вестник ЛГУ. 1968. № 24. С. 58-60; Калесник С. В. Проблема географической среды //Там же. 1968. № 12. С. 94.
Хүн бүр аль нэгэн нийгмийн гишүүн төдийгүй, гормоны үйлчлэлээр тодорхойлогдох насандаа оршин байгаа юм. Социал утгаараа янз бүрийн шинж чанар бүхий ангит төр буюу овгийн холбоог (социал организм) бүрдүүлж, байгалийн утгаараа угсаатан ( ястан, үндэстэн ) бүрдүүлдэг өнө удаан амьдрагч хамт олны тухайд мөн ийм зүйл хэлж болох юм. Аль алинийх нь үл давхцал тодорхой.
\section{УГСААТАН БОЛ НИЙГЭМ БИШ}
Гэхдээ “угсаатан бол нийгэм-түүхийн категори бөгөөд түүний хөгжил, дэвшил нь байгалийн биологийн хуулиар биш, нийгмийн хөгжлийн өвөрмөц хуулиар тодорхойлогддог” хэмээн үздэг өөр үзэл ч байдаг.
37 Козлов В. И., Покшишевский В. В. Этнография и география //Советская этнография. 1973. № 1. С. 9-10.
Үүнийг хэрхэн ойлгох вэ? Түүхэн материализмын онолын ёсоор үйлдвэрлэх хүчний дэвших хөгжил нь анги үүсэх, мөн устах харилцан солигдох диалектик үйл явцыг төрүүлэгч үйлдвэрлэлийн харилцааны өөрчлөлтийг бий болгодог. Энэ нь материйн хөгжлийн нийгмийн хэлбэрт хэвшмэл байх даяар үзэгдэл юм. Энд угсаатны нийлэгжилт ямар хамаатай юм ? Франц, англи мэтийн ихэд нэрд гарсан угсаатан бий болсон нь цаг хугацаа буюу газар нутгийн хувьд феодалын формацийн бий болохтой давхцаж байгаа бус уу? Эсвэл эдгээр угсаатнууд феодализм мөхөж, капитализмд шилжсэнээр алга болчихоо юу? Мөн энэхүү Францад “нийгэм–түүхийн” гэх категори байдаг ба XIY зуун гэхэд Францын эзэнт улс францчуудаас гадна кельт–бретон, баск, провенсаль, бургундыг эзлэн авсан бус уу? Энэ хүмүүс угсаатан биш байсан хэрэг үү? В.И. Козловын оносон тодорхойлолт өрөөсгөл байдлаараа ялгагдаж байгаа тухай маш олон зүйлээс нэг нь ч энэ баримтыг өгүүлэхгүй байна гэж үү? Хэрэв ийм байх ахул энэ нь шинжлэх ухааны маргааны шалтаг л болох юм.
Диалектик материализм материйн хөдөлгөөний янз бүрийн хэлбэрүүдийг механик, физик, химийн болон биологийн гэж ангилан үзэж, эдгээрийг байгалийн зүйлийн хэсэгт оруулдаг. Материйн хөдөлгөөний нийгмийн хэлбэр нь өөрт нь байх онцлогийн улмаас онцгой шинжтэй байдаг. Энэ нь өөрийн бүхий л илрэлээрээ хүн төрөлхтөнд л байдаг юм. Хүн бүр, техник бүхий хүмүүсийн хамт олон, доместикат (гаршсан амьтан, таримал ургамал) зэрэг нь материйн хөдөлгөөний цаг хугацаа (түүх), болон орон зайд (газар зүй) байнгын хамааралт байдалд байдаг нийгмийн ч, байгалийн ч хэлбэрийн үйлчлэлд өртөж байдаг. Материалыг нэгдмэл ажиглалт болон судалгаанд хүртэцтэй бүрдэл (түүхэн газар зүй) болгон нэгтгэн дүгнэхдээ бид түүнийг нийгмийн талаас, байгалийн талаас гэсэн хоёр өнцгөөр авч үзэх ёстой. Эхний өнцгөөр бол бид овгийн холбоо, төр, эзэнт улс, улс төрийн нам, гүн ухааны сургуулиуд гэх мэтээр нийгмийн байгууллагуудыг олж харна. Удаах өнцгөөр бол угсаатан, өөрөөр хэлбэл, харьцангуй богино хугацаанд үүсэж, сарниж байдаг, гэхдээ тохиолдолдол бүртээ өвөрмөц бүтэцтэй, гомеостаз буюу зогсонги байдлын хязгаар дотроо давтагдашгүй урьдач бодолтой, өвөрмөц хэмнэлтэй хүмүүсийн хамт олон байдаг.
Ангиуд бол нийгэм–түүхийн категори болох нь тодорхой юм. Ангиас өмнөх нийгэмд түүний адилтгал нь овгийн буюу аймгийн холбоо, жишээлбэл, кельтуудын клан байжээ. Энэ нь “социал категори” гэсэн ойлголтын өргөн утгаар жишээлбэл, төр, сүм хийдийн байгууллага, полис (Эллада) хэмээх хот улс, феод хэмээх эзний газар зэрэг тогтвортой институтуудад тархсан байж болох юм. Гэхдээ түүх гадарлах хүн бүрт иймэрхүү категори нь угсаатны хил хязгаартай тун ховор тохиолдолд давхцдаг, өөрөөр хэлбэл энд шууд холбоо байхгүй аж. Үнэн хэрэг дээрээ Москвад ажилчид, албан хаагчид, татарууд амьдардаг гэж хэлэх нь хэр зөв бэ? Бидний бодлоор бол энэ нь утгагүй зүйл бөгөөд харин В.И. Козловын логикоор зөвхөн ийм болж таарч байна. Энд алдаа нь постулат буюу анхдагч үндэслэлдээ байна. Энэ ч бага хэрэг, эдийн засаг нь бүхэлдээ материйн хөдөлгөөний нийгмийн хэлбэрт багтаж, үндэсний хүрээг даван гардаг. Нэгдмэл европын зах зээл, нэг төрлийн техник, орон тус бүрийн боловсролын төстэй байдал, хөрш хэлийг амархан судлах зэрэг нь ХХ зууны Европт угсаатны ялгааг арилгах ёстой мэт санагдана. Хэрэг дээрээ юу болов ? Ирландчууд хүч чадлаа ч хайрлалгүй, өөрийнхөө эртний, бараг мартагдсан хэлийг ч судлалгүй Их Британиас аль хэдийнээ салж байна. Хэдийгээр сүүлийн 300 жилд өөрсдийгөө дарлагдсан гэж үзээгүй ч Шотланд, Каталони хоёр автономи эрхийн төлөө тэмцэж байна. Бельгид саяхан болтол зохицол дунд амьдарч байсан фламанд болон валлончууд галзуу тэмцэл эхэлж, энэ хоёр угсаатны оюутнуудын дунд гудамжны зодоон болж байна.
38 Кон И. С. Диалектика развития наций: Ленинская теория наций и современный капитализм //Новый мир. 1970. № 3. С. 133-149.
Дээр үед ч гэсэн мөн л нийгэм–улс төрийн болон угсаатны цухалдал (уналт) тохиолдлоор л давхцаж байжээ. Бид хөгжлийн хоёр шугамын интерференци буюу холилдох үзэгдлийг буюу математикийн хэлээр яривал үл хамаарах хоёр хувьсагч нийлж буйг ажиглаж байна. Зөвхөн маш хүчтэй хүсэлтэй баймаажин л үүнийг анзаарахгүй байж болно.
“Бид”, “Биднээс бусад” хэмээн бусад бүх зүйлд өөрийгөө сөргүүлэн тавьсан угсаатан байгаагийн үзэгдэх илрэлийн мөн чанарыг нээхийг оролдоод үзье. Энэ сөргөлдөөнийг юу төрүүлж, юу тэтгэж байдаг вэ ? Энэ нь хэлний нэгдэл биш, учир нь хоёр буюу гурван хэлтэй угсаатан олон байдаг ба үүний урвуугаар нэг хэлээр ярьдаг янз бүрийн угсаатан байдаг. Францчууд гэхэд л франц, кельт, баск, провансаль гэсэн дөрвөн хэлээр ярьдаг боловч тэдний нэгдлийн түүх, нарийн яривал Парижийн хаадын Францыг Рейнээс эхлэн Пиреней хүртэл байлдан дагуулсан явдал нь удаан бөгөөд цус урсгасаныг үл харгалзан өнөөгийн угсаатны нэгдэлд нь саад болдоггүй. Үүний хамт мексикчүүд, аргентинчууд, перучууд испаниар ярьдаг боловч тэд испаничууд биш юм. Тэд XIX зууны эхэнд дайнд л сүйрэх гэж цусаа урсгаагүй юм.
Латин Америк нь Англи, АНУ–ын худалдааны компаниудын гарт орсон байлаа. Нортумберлендийн англичууд викингийн удмынхан учраас норвегитой төстэй хэлээр ярьдаг байв. Харин ирландчууд сүүлийн үеийг хүртэл зөвхөн англи хэлийг мэддэг байв, гэхдээ тэд англи хүн болоогүй юм. Араб хэлээр янз бүрийн хэд хэдэн ард түмэн ярьдаг. Олонхи узбек хүний хувьд төрөлх хэл тажик байдаг гэх мэт. Түүнээс гадна язгуур угсааны хэл гэж бий, XII–XIII зууны Англид францаар, НТӨ II-I зууны Парпид грек хэлээр, YII–XI зууны Персэд арабаар гэх мэтээр ярьдаг байсан юм. Ард түмний бүхэллэг байдлыг зөрчөөгүй учраас асуудал хэлэндээ биш гэсэн дүгнэлт хийх хэрэгтэй юм.
Түүнээс гадна олон янз байдлыг практикт хэрэглэж, чингэхдээ энэ практик нь олон хэлт ард түмнийг ойртуулдаг. Жишээлбэл, Номхон далай дахь америк–японы дайны үед япончууд америкийн радио нэвтрүүлгийг тайлан уншиж сурснаар америкчууд радиогоор нууц мэдээлэл дамжуулах боломжоо алдсан байна. Гэхдээ тэд ухаалаг, ер бусын гарц олж, цэргийн албанд дайчлагдсан индианчуудыг морзын үсэгт сургажээ. Алач хүн мэдээллээ навах хүнд атабасс хэлээр дамжуулж, ассинибойн–сиусу хүн дакот хэлээр дахин дамжуулж, хүлээн авсан хүн нь мэдээг англи руу орчуулдаг байжээ. Япончууд нууц түлхүүрийг нээсэн хэдий ч нээлттэй сэдвийг яаж ч чадахгүй байсан юм. Цэргийн алба нь хүмүүсийг ямагт ойртуулдаг учраас индианчууд “цайвар царайт” дайчин нөхөдтэй болон гэртээ харьцгаасан байна. Гэвч энэ үед индианчуудын уусах буюу ассимиляци үйл явц болоогүй, учир нь командлал нь чухамхүү тэдний угсаатны онцлог, түүний дотор хоёр хэлийг үнэлсэн юм. Ингээд хэдийгээр тодорхой тохиолдолд хэл нь угсаатны нийтлэгийн шалгуур үзүүлэлт болж чадах авч энэ нь түүний шалтгаан биш юм.
Вепс, удмурт, карел, чуваш нар одоо болтол гэртээ өөрийнхөө хэлээр ярьж, сургууль дээрээ орос хэл судалж, цаашдаа төрөлх тосгоноо орхих үед оросуудаас бараг ялгагддаггүй. Төрөлх хэлний нь мэдлэг тэдэнд огт саад болоогүй юм.
Сүүлд нь турк–османыг үзье. XIII зуунд туркмений удирдагч Эртогрул монголчуудаас зугтаж, Бага Азид 500 орчим морьтныг гэр бүлтэй нь дагуулан иржээ. Иконийн султан ирсэн хүмүүсийг “буруу номтон” грекуудтэй хилийн дайн хийхийн тулд хязгаарын Никой, Брусс хэмээх газар суурьшуулжээ. Эхний султаны үед Бруссад бүх Ойрхи Дорнодоос суурьших газар, олз омог хайсан “газии” хэмээх сайн дурынхан цутгах болжээ. Тэд “спаги” хэмээх морьт цэргийг бүрдүүлсэн байна. XIY зууны үед Болгар, Македонийг эзэлсэн нь туркийн султанд гэр бүлээс нь салгасан христийн шашинт хөвгүүдийг цэрэг болгон зохион байгуулж, исламын шашин, цэргийн хэрэгт сургаж, янычар хэмээх “шинэ цэргийн” гвардийн байдалд оруулах боломж олгосон байна. XY зуунд Газрын дундад тэнгисийн бүх эргийн түрэмгийчүүдээс бүрдүүлсэн усан флот байгуулав. XYI зуунд Диарбекра, Ирак, Курдистаныг эзлэн авснаар “акинжи” хэмээх хөнгөн морин цэрэгтэй болов. Эдний дипломатууд нь францын урвагчид, санхүүч болон эдийн засагчаар грек, армян, еврей нар болцгоожээ. Энэ хүмүүс эхнэрүүдээ боолын захаас худалдан авч байв. Тэнд нь польш, украйн, немц, итали, грузин, грек, бербер, негр гэх мэтийн хүүхнүүд байдаг байлаа. Энэ эмэгтэйчүүд XYII–XYIII зуунуудад туркийн цэргийн ээж, эмээ нар болсон юм. Туркууд угсаатан байсан, гэхдээ залуу цэрэг тушаалыг туркээр сонсож, ээжтэйгээ польшоор ярьж, эмээтэйгээ италиар хүүрнэж, зах дээр грекээр наймаа хийж, шүлгийг персээр уншиж, залбирлаа арабаар хийдэг байна. Гэхдээ энэ бол осман бөгөөд биеэ осман лугаа авч яван, исламын тэнгэрлэг зоригт дайчин болдог байв.
Энэхүү угсаатны бүхэллэгийг XIX зуунд олон тооны европын урвагчид болон Парижид суралцсан младо–туркууд сүйтгэсэн юм. ХХ зуунд Османы гүрэн унаж, угсаатнууд задрав: энэ хүмүүс өөр угсаатны бүрэлдэхүүнд орсон юм. Шинэ Туркийг Бага Азийн гүнээс гарсан сельжукийн удмынхан босгосон, харин османуудын үлдэгдлүүд өөрийн зууныг Стамбулын мухар гудмуудад өнгөрөөсөн билээ. Ингэхлээр 600 жилд осман угсаатныг хэл биш, харин шашны нийтлэг нэгтгэж байжээ.
ҮЗЭЛ СУРТАЛ БА СОЁЛ
Үзэл суртал, соёл хоёр нь заримдаа шинж тэмдэг болдог боловч энэ нь заавал биш юм. Жишээлбэл, Зөвхөн үнэн алдарт христиан хүн визант хүн болж болно, мөн бүх үнэн алдартнууд Константинополийн эзэн хааны албатууд буюу “өөрийн” гэж үзнэ. Гэхдээ загалмайтан болгарууд грекүүдтэй дайн өдөөж, харин үнэн алдартны шашин авсан Орос нь Царьградад захирагдахыг бодохоо баймагц энэ байдал зөрчигдөж эхэлсэн. Мухаммедийг залгамжлагч халифуудын тунхагласан нэг сэтгэлгээний энэ зарчим амьд амьдралын өрсөлдөөнийг давж чадсангүй, исламын нэгдлийн дотоодод ахиад л угсаатнууд үүсэв. Бидний үзсэнчлэн номлол нь угсаатан болж буй хүмүүсийн бүлгийг жишээлбэл, турк–османыг (х.72) буюу Энэтхэгийн баруун хойт зүгийн сикхи нарыг нэгтгэж чаддаг. Дашрамд дурдахад османы эзэнт гүрэнд султанд захирагддаг мусульман–суннитууд, бас өөрсдийгөө турк гэж үздэггүй–араб, крымийн татарууд байсан юм. Крымийн татааруудын хувьд османчуудтай хэлний ойр байдал нь ч хүртэл үүрэг гүйцэтгэж байгаагүй. Шашин шүтлэг ч гэсэн угсаатны шинжилгээний шинж тэмдэг болж болж чадахгүй юм байна.
Гуравдахь жишээ нь сикхи, энэтхэг гаралтай сектантууд холимог байдлаар угсаатан гэж өөрийгөө бататгасан явдал юм. Энэтхэгт тогтоосон кастын системийг бүх индусын хувьд заавал мөрдөх зүйл гэж үздэг. Энэ бол угсаатны онцгой бүтэц юм. Индус байна гэдэг бол тэр ч байтугай үл хөндөгдөх зэрэглэлээс хамгийн ядуу нь хүртэл кастын гишүүн байна гэсэн хэрэг бөгөөд бусад бүх хүмүүс, түүний дотор олзлогдсон англи хүртэл амьтдаас доогуур тавигдана. Энд улс төрийн нэгдэл байгаагүй, гэхдээ зан үйлийн урьдач үзлийг хатуу сахиж, энэ нь маш хатуу чанд байдаг байв. Каст бүр тодорхой ажил төрөл эрхлэх эрхтэй, цэргийн алба хаахыг зөвшөөрсөн тэр хүмүүст энэ нь бага байв. Энэ нь мусульман–афганчуудад Энэтхэгийг эзэмших, хамгаалалтгүй хүн амыг элэг барих боломж олгосон юм, Чингэхдээ Пенжабын хүн ам хамгийн их нэрвэгдэж байв.
Тэнд XYI зуун болоход эхлээд муу зүйлийг үл эсэргүүцэх, дараа нь мусульманчуудтай дайн хийх зорилго тавьсан сургааль бий болов. Индусуудаас сикхи (шинэ шүтлэгийн зориулсан нэр) нар салан гарснаар кастын систем хүчингүй болов. Тэд Энэтхэгийн бүхэллэгээс эндогамийн буюу дотроо гэр бүл болох замаар тусгаарлаж, зан үйлийн өөрийн гэсэн урьдач бодол бий болгож, өөрийн нийтлэгийн бүтцийг тогтоожээ. Бидний мөрдөн буй зарчмаар сикхи нарыг индусын эсрэг өөрийгөө сөргүүлэн тавьсан үүсч буй угсаатан гэж үзэх ёстой болно. Тэд ч өөрсдийгөө ингэж үздэг. Шашны концепци нь тэдний хувьд билэгдэл, харин бидний хувьд бол угсаатны дивергенци буюу салалт юм.
Сикхүүдийн сургаалийг номлол гэж үзэж болохгүй, учир нь хэрэв Москвад хэн нэг нь энэ шашныг бүрэн хүлээн авсан байлаа ч тэр хүн сикхи болохгүй, сикхүүд ч түүнийг “өөрийн “ гэж үзэхгүй юм. Сикхүүд шашны үндсэн дээр, монголчууд төрөл садны үндсэн дээр, швейцарчууд австрийн феодалуудтай амжилттай дайтан, дөрвөн хэл дээр ярьдаг улс орны хүн амыг гагнаж чадсаны үр дүнд угсаатан болж чаджээ. Угсаатнууд янз бүрийн аргаар бүрэлддэг ба бидний зорилго бол үүний ерөнхий зүй тогтлыг олох явдал юм.
Ихэнхи томхон ард түмэн зохицолт систем бүрдүүлэгч хэд хэдэн угсаатны газар зүйн хэв шинжүүдтэй байдаг ба эдгээр нь өөр хоорондоо цаг хугацааны хувьд ч, социал бүтцийн хувьд ч ихээхэн ялгагддаг. Ядахдаа боярын малгай, том сахал, цантай цонхны цаана юм нэхэж буй эмэгтэйчүүд бүхий XYIII зууны Москва, хиймэл үс, хантаазаар гоёсон дээдэс эхнэрүүдээ үдшийн цэнгүүнд урьж буй XYIII зууны Москва, сахал үс болсон үгүйсгэгч оюутан аль хэдийнээ хоорондоо холилдож эхэлсэн бүхий л давхаргын авхайчуудыг гэгээрүүлж, дээр нь декадент гэх соёлын гутранги үзэлтнүүд бүхий ХХ зууны Москва харьцуулаад үзье.
Манай эрин үеийн энэ бүх зүйлийн харьцуулж, энэ бүгд нэг л угсаатан гэдгийг мэдэж аваад түүх, угсаатны зүйг мэдэхгүйгээр судлаач төөрөгдөлд орохоор байна гэдгийг бид харж авлаа. Нэг орон зайн зүсэлтэд, тухайлбал 1869 оныг авч үзвэл багагүй сонирхолтой байх болно. Поморууд, питерийн ажилчид, Завольжийн хуучин шүтлэгтнүүд, алт хайгчид, ойн тариачид, талын губернийн тариачид, доны казахууд, уралын казахууд энэ бүх хүмүүс гадна талаасаа өөр хоорондоо огт адилгүй, гэхдээ энд ард түмний нэгдэл эвдрээгүй байгаа. Ахуйн хувьд ойрхон байгаа нь жишээлбэл, гребений казахуудыг чеченчүүдтэй нэгтгээгүй юм.
Хачирхалтай нь энэхүү санал болгож буй үзэл нь анхааралд орох ёстой байсан чухам тэр газарт идэвхитэй эсэргүүцэлтэй тулгарч байна. Зарим угсаатны судлаачид угсаатан судлал газар зүйтэй харилцах, мөн асуудлын түүх, өөрөөр хэлбэл түүх судлалын талаар өөрийн үзлийг зохиогчид сөргүүлэн тавьж байна. Тэдэнтэй маргаанд орохоос зайлсхийсэн хэдий ч би хангалттай үндэслэлгүйгээр дээд жаяг тогтоохоор өрсөлдөж буй үзэл баримтлалуудыг үгүйсгэхгүй байж чадахгүй юм. Энэ бол сонгодог талаасаа эвгүй зүйл бөлгөө.
Угсаатны зүй шинжлэх ухааны хувьд бий болсон нь В.И.Козлов, В.В.Покшииевский нарт ингэж төсөөлөгдөж байна. XIX зууны дунд үе болтол газар зүй, угсаатан судлал хоёр нийлэн хөгжиж, дараа нь угсаатан судлал нийгэм–түүхийн болон газар зүйн чиглэлүүдэд хуваагджээ. Эхний чиглэлд Л.Г.Морган, И.Я. Бахофен, Э.Тейлор, Ж. Фрезэр, Л.Я.Штеренберг нар орон, удаах чиглэлд Ф.Ратцель, Л.Д. Синицкий, А.А.Купер, түүнчлэн “хүний газар зүйн” францын сургууль орно. Санал болгон буй энэ ангилалд түүнийг бараг үгүй болгож буй ноцтой гажиг байна. “Чиглэлүүдийн” төлөөлөгчид янз бүрийн зүйлийг сонирхож, янз бүрийн сэдвүүдэд анхаарлаа тавьдаг. Хэрэв ингэвэл тэдгээрийг сөргүүлэн тавих нь цагаатгаж болшгүй явдал юм. Ф.Ратцель угсаатны зүйн бүсчлэлийн газар зүйн шинжийг үндэслэх гэж оролдох үедээ таатай ид шид буюу адууг ёслон алах, өөрөөр хэлбэл Ж.Фрезерийн “Алтан мөчир” хэмээх алдарт номоо зориулсан зүйлс болох анимизмын үзэл баримтлалтай огтхон ч маргалдаагүй билээ. Гэхдээ чухамхүү олон талт эрдэмтдийн олон янзийн ашиг сонирхол байгаа учраас л зохиогчид газар зүйгээс угсаатны зүйг салгаж, түүнийг шинээр нийгмийн шинжлэх ухаан болгох гэж үзээд байгаа юм. Энд хэтэрхий харамсалтай үр дагавартай зарим будлиан илт байна. Ямар ч шинжлэх ухаан сэдэвчлэлээ зүгээр л өөрчлөх замаар биш, харин судалгааныхаа хүрээг өргөтгөх замаар хөгждөг. Улмаар хэрэв газар зүйн угсаатны зүйн ололт амжилт дээр түүхийн асуудлыг нэмбэл энэ нь шинжлэх ухааны дэвшил болно. Хэрэв нэг зүйлийг нөгөөгөөр соливол харин энэ нь хэзээ ямагт хор хохиролтой байдаг дороо дэвхцэх явдал болно.
Энэ нь хоёр шинжлэх ухааны зааг дээр байж, гэхдээ угсаатны газар зүйд өөрийгөө оруулалгүй, хүн амын газар зүйн ээлжит нээлтэд өөрийгөө зориулан буй тэрхүү эрдэмтдэд нэн ойлгомжтой юм. Тэдний бодлоор ялгаа нь “эдийн засгийн газар зүйчийн хувьд хүн нь үйлдвэрлэл, хэрэглээний чухал субъект”, “угсаатны зүйчийн хувьд тэр нь угсаатны тодорхой онцлогийг (соёл, хэл гэх мэт) тээгч” 39 байгаад оршино.
39 Козлов В. И., Покшишевский В. В. Этнография и география. С. 3-13
Энд дээрх өгүүллийн зохиогчийг зөвшөөрч яагаад ч болохгүй юм. Эскимосчуудыг тэдний далайн ан хийдгийг нь орхичихоод үйл үгийн дүрмийн хэлбэр буюу далай болон тундрын хорт сүнсний тухай төсөөллөөр хязгаарлан судалж болно гэж үү дээ ? Эсвэл индусуудыг цагаан будааны талбайд хийдэг хөдөлмөрийг нь орхичихоод карма болон дахин төрөх онолыг нь нарийчлан үзэж болно гэж үү ? Үгүй ээ, хөдөлмөрийн үйл явцын шинж чанар, хэрэглээ, дайн, төр улс байгуулах буюу түүний уналт зэрэг нь хуримын ёслол буюу тухайлсан ёслол лугаа адилаар угсаатны зүйн судалгааны объект болдог. Ард түмнийг хөгжлийнх үе шатанд нь хөршүүдтэй нь зэрэгцүүлэн судлах явдал газар зүйн орчныг харгалзахгүйгээр сэтгэшгүй юм.
Угсаатны зүйг нийгэм-эдийн засгийн хөгжлийн ойролцоогоор ижил түвшинд байгаа, ижил төстэй байгал–газар зүйн нөхцөлд (жишээлбэл, “далайн ан хийдэг арктикийн анчин”, “хуурай талын малчид” гэх мэтийн хэв шинжтэй) амьдардаг ард түмнүүдэд хэвшмэл байх аж ахуй-соёлын хэв маяг”–ийн тухай сургаалиар сольж болохгүй.
40 Андрианов Б. В., Чебоксаров Н. Н. Хозяйственно-культурные типы и проблемы их картографирования //Советская этнография. 1972. № 5.
Энэ чиглэл нь эдийн засгийн газар зүйд үр нөлөөтөй байж болох авч угсаатны зүйд ямар ч хамаагүй, хамаатай байж ч чадахгүй. Жишээлбэл, энэхүү ангиллаар бугатай чукча (өөрөөр хэлбэл малчид), далайн анчин чукча (буга нь алга болсон үед тэд юу хийх вэ) нарыг хэдийгээр тэд нэгдмэл угсаатан боловч янз бүрийн хэсэгт тараагаад орхичихож байна. Москва орчмын орос тариачид, поморууд, сибирийн харцгын анчин зэрэг нь нэг угсаатан биш бил үү ? Ийм жишээ тоо томшгүй. Козловын санал нь угсаатны зүйг устгаж, түүнийг хүн амын ажил эрхлэлтийг харгалзсан хүн ам зүйгээр солиход хүргэж байна. Гэхдээ энэ сэдэв нь бидний сонирхлыг татахгүй бөлгөө.
Угсаатныг биологийн таксоном буюу дэс дараалсан нэгж болох арьстан, болон популяци буюу бүлийн хөгжилтэй зэрэгцүүлэх нь бас л буруу юм. Арьстан нь өөр хоорондоо хүний амьдралд чухал ач холбогдол байхгүй бие шинжээрээ ялгагддаг. Бүл нь нь чөлөөтэй эвцэлдэн үржиж, хөрш бүлээс аль нэг хэмжээгээр тусгаарлагдмал мэт байдаг, тодорхой газар нутагт оршин амьдарч буй амьтны нийлбэр юм.
41 Рогинский Я. Я., Левин М. Г. Основы антропологии. М., 1955. С. 325-329. 42¯Тимофеев-Ресовский Н. В. Микроэволюция. Элементарные явления. Материал и факторы микроэволюционного процесса //Ботанический журнал. 1958. Т. 43. № 3.
Бидний санал болгож буй ойлголтоор угсаатан бол үл давтагдах дотоод бүтэцтэй, зан үйлийн өвөрмөц тогтсон бодолтой, энэ хоёр нь хөдлөнги бүрдэл болж байгаа хүний хамт олон юм. Эндээс угсаатан бол социологийн ч, биологийн ч, газар зүйн ч үзэгдлүүдэд ордоггүй эгэл үзэгдэл мөн гэж хэлж болно.
Угсаатны нийлэгжилтийг “хэл–соёлын үйл явц”–д оруулах нь бодит байдлыг гажуудуулж, угсаатны түүхийн нарийн нийлмэлийн зэргийг бууруулна. Энэ талаар Ю.В.Бромлейн санал болгосноор асуудлыг ойлгохын тулд этникос ба эсо (угсаатан нийгмийн байгууллага) гэсэн нэмэлт нэр томъёо оруулах санал гаргасан юм. Түүний шийдвэрт сэтгэл дундуур байж болох авч бүрэн үгүйсгэх нь эвгүй юм.
43 Бромлей К). В. Опыт типологизации этнических общностей //Советская этнография. 1972. № 5. С. 61.-И, видимо, не случайно в том же журнале год спустя напечатано еще одно филологическое исследование о термине “этнос”, обосновавшее употребление его в том смысле, в каком оно фигурирует у Л. Н. Гумилева и у Ю. В. Бромлея (см.: Поплинский Ю. К. К истории возникновения термина “этнос” //Советская этнография. 1973. № 1).
Түүний үндэслэлийн логикийн дагуу бол хэл сурах чадвартай хүмүүс хэд хэдэн угсаатанд нэгэн зэрэг хамаарагдах ёстой болж байна. Энэ нь утга учир багатай. Хэдийгээр хоёр, тэрч байтугай гурван хэлтэй угсаатан олон байдаг боловч тэд хэлний ангиллын суурь дээр уусан нийлдэггүй. А.С.Пушкин болон түүний нөхдүүд франц хүмүүс болоогүй шүү дээ. Харин ч урвуугаар мексик, перучууд испаниар ярьж, католик шашин шүтэж, Сервантесийн уншдаг атлаа өөрсдийгөө испаничууд гэж үздэггүй. Үүнээс гадна тэд “чөлөөлөх” гэж нэрлэгдсэн дайнд сая сая хүний амь насыг хөнөөсөн билээ. Мөн энэ үед Дээд Перу, Чакогийн хөндийн индианчууд өөрсөдтэй нь соёл, эдийн засаг, хэлний хувьд нийтлэг зүйл юу ч байхгүй Испаничуудын төлөө тулалдсан юм. Хэрэв индианчуудын дайсан нь алс холын испаничууд биш, харин хуучин овог нэгтийхээ эсрэг сөрөн зогссон, хагас испанижсан эрлийз- нутгийнхан байсан, мөн тэд XIX зууны эхээр бие даасан угсаатан болж бүрэлдсэн зэргийг харгалзах юм бол энэ нь бүрэн ойлгомжтой юм. В.И.Козловын байр сууринаас бол ийм хожмын угсаатны нийлэгжилтийг тайлбарлах боломжгүй болно.
НЭГ ӨВГИЙН ГАРАЛ
Эрт үед үүнийг угсаатны хувьд зайлшгүй гэж үздэг байсан. Бодит хүн байхгүй бол өвөг дээдсийн үүргийг ямагт тотем болоод байдаггүй байсан амьтан гүйцэтгэх нь олонтаа байв. Турк, римчуудын хувьд энэ нь гичий чоно, уйгаруудад–хатныг жирэмслүүлсэн чоно, төвдүүдэд сармагчин болон эм ракша (ойн савдаг) зэрэг байлаа. Гэхдээ энэ нь голдуу домгоор танигдахгүй болтол нь гажуудуулсан хүн голдуу байсан юм. Еврейчүүдийн дээд эцэг Авраам, арабуудын өвөг–түүний хүү Исмаил, Фивийг үндэслэгч, беотийчүүдийг эхлүүлэгч Кадм гэх мэт.
Хачирхалтай нь энэхүү балар эртний үзэл мөхсөнгүй, манай үед онго биетийн оронд эдүгээ оршин буй угсаатны өвөг дээдэс хэмээн ямар нэг эртний овгийг тавихыг оролдох болжээ. Гэхдээ энэ нь бас л буруу юм. Зөвхөн эцэг буюу зөвхөн эхээс гарсан хүн байдаггүйн адил янз бүрийн өвөг дээдсээс гараагүй угсаатан гэж байхгүй билээ. Мөн угсаатныг арьстантай хольж болохгүй атал үүнийг байнга шахам, ямар ч үндэслэлгүй хийдэг. Энэхүү сонирхол татам зүйлийн үндэс нь арьстны нийлэгжилтийн үйл явц нь дэлхий тодорхой районуудад хөгжсөн ба эдгээр районуудын өвөрмөц байгал орчин, өөрөөр хэлбэл, газар зүйн бүсийн цаг агаар, ургамал, амьтнаар нөхцөлдсөн байх магадлалтай гэж үздэг урьдач бодол байдаг. Үүний учрыг олъё.
44 Козлов В. И., Покшишевский В. В. Этнография и география. С. 10.
Дээд палеолитийн үеийн Европод субарктикийн нөхцөл ноёрхож, Роны хөндийн цаг агаар өндөр чийгтэй байхад тэнд Гримальди арьстны негржүү хүмүүс суурьшиж, харин Африкийн тропикийн халуун ойд монголжуу болон негржүү шинжийг хослуулсан койсан арьстан амьдарч байлаа. Энэ арьстан эртний, түүний нийлэгжилт тодорхойгүй, гэхдээ түүнийг эрлийз гэх үндэс байхгүй юм. Негржүү банту нар койсан арьстныг түүхийн бүрэн үе–НТӨ I зууны орчим Африкийн өмнөд хязгаар руу шахан хөөжээ, хожим нь энэ үйл явц бечуанууд бушменуудыг Калахарын цөл рүү хөөсөн XIX зуун хүртэл үргэлжилжээ. Үүний хамт Эквадорын Америкт хэдийгээр байгалийн нөхцөлд нь африкийнхтай ойролцоо боловч негржүү хүмүүс үүсээгүй юм.
Евроазийн чийглэг бүсэд кроманьон хэв шинжийн европжуу болон монголжуу арьстан амьдарч, арьстны төстэй байдлын шинж байгаагүй билээ. Төвдөд монголжуу ботууд европжуу дар, памирчуудтай хаяа нийлж, харин Гималайд гургууд патанчуудтай зэрэгцэн амьдарч байв. Товчоор хэлбэл эдгээр хүмүүсийн оршин амьдарч буй бүс нутгийн газар зүйн нөхцөл болон янз бүрийн бүлийн хоорондын антропологийн онцлогийн функциональ холбоо тодорхойгүй гэдгийг хүлээн зөвшөөрбөл зохино. Түүнээс гадна энэ нь ерөөс байгальд байдаг эсэх, тэр тусмаа энэхүү үзэл бодол нь арьстны ангиллыг өргөргийн бүсээр биш, уртрагийн бүсээр хийж, атлантын бүсэд европжуу болон африкийн негржүү арьстныг багтааж, номхон далайн бүсэд Зүүн ази болон Америкийн монголжуу арьстныг оруулан бүтээдэг орчин үеийн палеоантопологийн ололт амжилттай эрс зөрж байгаа юм. Энэ үзэл нь арьстны нийлэгжилтэд байгалийн нөхцлийн үйлчлэлийг үгүйсгэж байгаа юм. Учир нь энэ хоёр бүлэг цаг агаарын янз бүрийн бүсэд бүрэлдсэн билээ.
45 Алексеев В.П. В поисках предков. М., 1972.
Үүний урвуугаар угсаатнууд нь аж ахуйн үйл ажиллагааныхаа идэвхийн ачаар байгалийн хүрээлэлтэй холбоотой байдаг. Аж ахуйн үйл ажиллагаа нь ландшафтад өөрөө дасан зохицох, ландшафтыг өөртөө үйлчлүүлэх гэсэн хоёр чиглэлээр илэрдэг юм. Аль ч тохиолдолд нь бид хэдийгээр үүслийнх нь шалтгаан тодорхойгүй авч, бодитой оршин буй үзэгдэл болсон угсаатнуудтай тулгарч байна.
Судлан буй сэдвийн бүхий л олон янз байдлыг ямар нэгэн ганц зүйл рүү овоолох угаасаа хэрэггүй байдаг. Зүгээр л аль аль хүчин зүйлийнх нь ролийг тогтоох нь сайн. Жишээлбэл, ландшафт нь угсаатны хамт олныг үүсэх үеийнх нь боломжийг тодорхойлж өгдөг, харин шинээр төрсөн угсаатан ландшафтыг өөрийн шаардлагад тохируулан өөрчилдөг. Ийм харилцан зохицол нь үүсэн буй угсаатан хүч чадлаар дүүрэн, түүнийгээ хэрэглэхээр эрэлхийлэн буй үед л боломжтой болдог. Харин дараа нь бүтээн босгосон байрандаа дасал болох бөгөөд энэ нь хойч үеийнхний хувьд ойр бөгөөд үнэ цэнэтэй байна. Үүнийг үгүйсгэвэл энд бүхий л сэтгэл зүрхээрээ хайрласан ландшафтын элементүүдийн хослол хэмээн ойлгогдох эх орон гэж энэ ард түмэнд байхгүй гэсэн дүгнэлтэд гарцаагүй хүрнэ. Хэн нэг нь үүнийг зөвшөөрөх нь юул бол.
Зөвхөн ганц энэ л гэхэд угсаатны нийлэгжилт бол нийгмийн үйл явц биш, учир нь социо хүрээний ердийн хөгжил байгалийн үзэгдлүүдтэй харилцан үйлчилж, чингэхдээ түүний бүтээгдэхүүн болоогүйг л харуулж байна.
Угсаатны нийлэгжилт нь үйл явц, харин шууд ажиглагдах угсаатан нь угсаатны нийлэгжилтийн шат (үе), улмаар тогтворгүй систем гэсэн чухамхүү энэ баримт нь угсаатныг антропологийн арьстан, тэр тусмаа арьстны дурын онолтой харьцуулан үзэх явдлыг үгүйсгэж байна. Үнэн хэрэг дээрээ антропологийн ангиллын зарчим нь адил төст байдал юм. Мөн угсаатныг бүрдүүлэгч хүмүүс нь олон янз байдаг. Угсаатны нийлэгжилтийн үйл явцад хоёр буюу түүнээс дээш бүрдэл хэсэг оролцдог. Янз бүрийн угсаатныг эвцэлдүүлэхэд заримдаа шинэ тогтвортой хэлбэр өгдөг, заримдаа үүлдэр доройтдог. Жишээлбэл, славян, угор, алан, түргүүдийн хольцоос их оросын ястан үүсэн хөгжсөн, харин сүүлийн хоёр мянган жил Хятадын Их Цагаан хэрмийн шугамын дагууд ямагт үүсч байсан монгол–хятад, манж–хятадын эрлийзүүдийг багтаасан бүрдлүүд тогтворгүй болж, бие даасан угсаатны нэгжийн хувьд устдаг байв.
YII зууны Дундад азид согдууд амьдарч байлаа, харин “тажик” гэсэн нэр томъёо бүр YII зууны үед “араб”, өөрөөр хэлбэл халифын цэрэг гэсэн утгатай байсан. 733 онд Наср ибн Сейяр согдуудын бослогыг дараад цөөрсөн цэргүүдээ исламын шашныг аль эрт авсан хорасаны персүүдээр сэлбэсэн юм. Тэднийг олон тоогоор авсан учраас перс хэл нь түүний араб армид ноёрхох болжээ. Ялалтын дараа согд эрчүүдийг хядаж, хүүхдүүдийг нь боол болгон зарж, царайлаг хүүхнүүд, цэцэглэн буй цэцэрлэгийг нь ялагчид хуваан авснаас Согд болон Бухарт хорасанчуудтай төстэй перс хэлт хүн ам бий болсон байна.
46 Гумилев Л. Н. Древние тюрки. М., 1967. С. 359-360.
Гэхдээ 1510 онд Иран ба Дундад азийн хувь заяа салан одов. Ираныг итгэлтэй шиит турк угсааны Измаил Сефери эзэлж, перс, шиизм рүү хандаж байв. Харин Дундад Ази узбек–суннитуудад ногдож, тэнд амьдран байсан перс хэлт хүн ам “тажик” гэсэн хуучин нэрээ хадгалан үлдээд 1918 онд Бухарын Мангытын улс унах хүртэл ямар ч утга холбогдол олж байгаагүй юм. Хуучны Туркестаны хягаарт Узбек болон Тажикийн бүгд найрамдах улс бүрэлдэх үед YII зууны булаан эзлэгч, Бухар болон Самаркандад амьдарч байсан хорасаны персийн хойч үеийнхэн хүн амын тооллогоор узбекууд, харин XI–XYI зууны булаан эзлэгч, Душанбе болон Шахрисябзед амьдарч байсан туркийн хойч үеийнхэн тажик хэмээн бүртгүүлжээ. Тэд хоёр хэлийг хүүхэд байхаасаа мэддэг, мусульманчууд байсан болохоор хэн гэж бүртгүүлэх нь тэдэнд ялгаагүй байлаа. Сүүлийн дөчин жилд энэ байдал ихээхэн өөрчлөгдөж, тажик болон узбекууд социалист үндэстэн болон бүрэлджээ. Харин шашны хамаарал угсаатныг (мусульман ба кафир) тодорхойлж, тажикуудад овог байгаагүй 20–иод он хүртэл тэднийг хэрхэн авч үзэх вэ? Турк болон иран хэмээх угсаатны хоёр хольц Дундад Азид дасан зохицоход хангалттай урт мянган жилийн өмнө “импортоор” авчирсан угсаатнууд байсан юм. Энд нээж, дүрслэвэл зохих тодорхой зүй тогтол үйлчилж байгаа нь илт байна. Гэхдээ гарал үүслийн нийтлэг нь угсаатныг тодорхойлох шалгуур болж чадахгүй нь ойлгомжтой юм. Учир нь энэ нь хүй нэгдлийн үеийн бүдүүлэг шинжлэх ухаанаас бидний ухамсрын уламжилж авсан домог болно.
УГСААТАН ХИЙ ҮЗЭГДЭЛ БОЛОХ НЬ
Гэхдээ “угсаатан” нь аль нэг нийгмүүдийн нийлэх үед бүрэлддэг зүгээр л социал ойлголт байж болох бус уу?
47 Козлов В. И. Динамика численности народов. М., 1969. С. 56.
Тэгвэл “угсаатан” нь хуурамч хэмжигдэхүүн, харин угсаатны зүй нь учир утгагүй цагийн барууш байх нь, ингэж байснаас социал нөхцлүүдийг судлах нь хялбархан билээ. Энэ үзэл алдаатай гэдэг нь, хэрэв юм боддог хүнд заль мэхийг хүрэлцээтэй ажиглалтаар сольбол бүр ч тодорхой болно. Үүнийг бодит жишээн дээр авч үзье. Францад кельт–бретончууд болон ибер–гаскончууд амьдардаг. Вандейн ой болон Пиренейн уулс дунд тэд өөрийн хувцсаа өмсөж, өөрийн хэлээр ярьж, өөрийн эх орондоо байж, өөрсдийгөө франчуудаас шууд ялгадаг. Гэхдээ Францын маршал Мюрат буюу Ланн нарын тухайд тэднийг франц биш, баск хүн гэж ярьж болох уу? Эсвэл Дюмагийн романы баатар, түүхэн хүн д’ Артаньяны тухайд франц биш гэж болох уу? Жанна Д’Аркийн хамтран зүтгэгч бретоны тайж Шатобриан болон Жиля де Ретцийг франц хүнд тооцохгүй байж болох уу? Ирланд Уайлдыг английн зохиолч биш гэх үү? Алдарт Дорно дахин судлаач Чокан Валиханов өөрийгөө орос-казак гэж тэнцүүгээр үздэг тухайгаа ярьж байсан. Ийм жишээ тоо томшгүй боловч энэ нь бүгдээрээ угсаатны хамаарал бол ухамсрын өөрийнх нь бүтээгдэхүүн биш, харин хүмүүсийн ухамсарт илрэн гарсан хамаарал юм гэдгийг харуулж байна. Энэ нь бидний дээд мэдрэлийн үйл ажиллагааны хэлбэр гэж ойлгодог хүний мөн чанарын илүү гүнзгий, ухамсар болон сэтгэл зүйд харьцахдаа гаднын шинжтэй ямар нэг талыг тусгаж байгаа нь илт байна. Бусад тохиолдлуудад угсаатнууд яагаад ч юм хүрээллийн үйлчлэлд асар их эсэргүүцэл үзүүлэх ба уусан шингэдэггүй билээ.
Цыганууд гэхэд л Энэтхэг ба өөрийн нийгмээс салаад аль хэдийн олон мянган жил болж, төрөлх нутгаасаа холбоо тасарсан хэдий испани, франц, чех буюу монголчуудын алинтай нь ч уусан нийлээгүй юм. Тэд Европын нийгмийн феодалын институтыг хүлээн аваагүйгээр барахгүй нүүдэллэн очсон бүх орнууддаа өөр овгийн бүлэг хэвээрээ үлдсэн байна. Ирокезууд одоо болтол хэт аварга капитализмаар хүрээлэгдсэн угсаатны жижиг бүлэг болон амьдарч “америкийн аж төрөх ёсыг” хүлээж аваагүй байна. Бүгд Найрамдах Монгол ард Улсад согд (тува), казах зэрэг олон түрэг угсаатан амьдардаг ч “нийгмийн материаллаг болон оюуны хөгжил” нь төстэйг үл харгалзан монголчуудтай уусан нэгдэлгүй, бие даасан угсаатнаа бүрдүүлсээр байна. Үүний зэрэгцээ XYIII зуунд Францчууд Канад руу шилжин суурьшсан бөгөөд тэдний ойн суурин болон Францын үйлдвэржсэн хотуудын хөгжил нэн ялгаатай боловч одоо болтол тэд өөрийн угсаатны нүүр царайгаа хадгалсаар байна. Еврейчүүд Испаниас хөөгдсөний дараа эндогам бүлгээрээ Саликонд 400 гаруй жил амьдарсаар байна. Харин 1918 оны тооллогоор тэд хөрш грекчүүдтэй гэхээсээ арабчуудтай илүү төстэй болсон аж. Үүний нэгэн адил Унгараас гаралтай немцүүд гадаад төрхөөрөө Герман дахь ижил угсааныхантайгаа төстэй, харин цыганууд индусуудтай төстэй байдаг. Шалгарал нь шинж тэмдгийн харьцааг удаанаар өөрчилж, харин мутаци буюу бүлийн өөрчлөлт ховор байдаг нь тодорхой байна. Ийм учраас өөрт нь зохицсон ландшафтад амьдран буй дурын угсаатан бараг тэнцвэрийн төлөв байдалд оршдог байна.
Оршихуйн нөхцлийн өөрчлөлт нь угсаатанд хэзээ ч нөлөөлдөггүй гэж бодох хэрэггүй юм. Заримдаа энэ нь шинэ шинжүүд бий болж, угсаатны шинэ хувилбар бүтээхүйцээр тийм хүчтэй нөлөөлдөг, их бага хэмжээгээр тогтвортой байдаг юм. Бидний хувьд энэ үйл явц хэрхэн явагддаг, яагаад эдгээр нь янз бүрийн үр дүнд хүрдгийн учрыг олох л үлдэж байна.
Зөвлөлтийн нэрт судлаач С.А. Токарев угсаатны нийтлэг хэмээх ойлголтын тодорхойлолтын оронд “дөрвөн формацид ард түмний түүхийн дөрвөн хэв маяг таарах бөгөөд хүй нэгдлийн байгуулалд тухайн нутаг дэвсгэр дээрх бүх хүмүүсийг садан төрлийн холбоотой нь нэгтгэсэн хүмүүсийн бүлэг, боол эзэмшлийн үед боолчуудыг оруулалгүй, зөвхөн чөлөөт хүн амыг нэгтгэсэн демос байдаг, феодализмын үед ноёрхогч ангийг үл оруулсан улс орны бүх хөдөлмөрчид болох ястан байна, харин капитализм болон социализмд антагонист анги болон салсан хүн амын бүх давхарга болох үндэстэн байдаг” гэсэн социологийн үзэл баримтлалыг дэвшүүлсэн юм.
48 Токарев С. А. Проблема типов этнических общностей //Вопросы философии. 1964. № II. С. 52- 53. См. также: Агаев А. Г. Народность как социальная общность //Вопросы философии. 1965. № II. С. 30.
Энэхүү ишлэл нь “угсаатны нийтлэг” гэсэн ойлголтонд огт өөр учир холбогдол оруулсан учраас өөр ямар нэг зүйлд тус болж магадгүй авч, түүхэн газар зүй, ерөөсөө байгалийн шинжлэх ухааны анхаарлын гадна байгаа юм. Ийм учраас угсаатан гэж нэрлэх зүйл рүү түүнийг орууллаа ч гэсэн энэ үзэл баримтлалтай маргах нь үр ашиггүй бөлгөө. Үгээр маргах нь ямар ашигтай байх билээ ?
ӨРНӨ ДОРНЫН ХООРОНД
Газар дундын тэнгисийн соёлуудтай танилцаж байхдаа бид дасаж сурсан ойлголт, үнэлгээнийхээ орчинд дунд байдаг. Шашин нь Бурханд итгэх итгэл байж, төр улс нь тодорхой дэг журам, эрх мэдэлтэй, улс орон ба нуурууд нь тодорхой газартаа байж л байдаг.
Зөвхөн энэхүү дасаж дадсан “Өрнө”, “Дорно” нэрийг л бид огтхон ч газар зүйн утгагүй хэрэглэдэг, тухайлбал, Марокког “Дорнод”, Унгар ба Польшыг “Өрнөд” гэх жишээтэй. Гэхдээ энэ нөхцөлд байдлыг хэрэглээд амжчихсан бөгөөд энд ойлголтын будлиан гардаггүй юм. Үүнд уран зохиол уншиж, амьд уламжлал байгаагийн ачаар мэргэжилтэн бус хүнд ч танил болсон зүйлийн судлагдсан байдал ихээхэн дэм болдог.
Гэвч бид Дундад болон Дорнод азийг зааглан хувааж буй уулын давааг давмагцаа л тооллын огт өөр систем бүхий ертөнцтэй учирдаг. Энд бид зөвхөн бурхан төдийгүй, биднийг хүрээлэн буй ертөнц оршин буйг үгүйсгэдэг шашинтай учирна. Дэг журам, нийгмийн байгууламж нь төр болон эрх мэдлийн зарчмуудтай зөрчилдөнө. Нэр нь тодорхойгүй олон оронд хэл, эдийн засаг, тэр ч байтугай газар нутаггүй угсаатантай учирч, гол нуурууд нь малчидтайгаа хамт нүүдэллэдэг газрыг үзнэ. Бидний нүүдлийн гэж үзэж дассан тэр овгууд суурин байдаг хийгээд цэргийн хүч нь тооноосоо хамаардаггүйг олж үзнэ. Өөрчлөлтгүй юм гэвэл зөвхөн угсаатны нийлэгжилтийн зүй тогтол бөлгөө.
Ийм өөр материал өөр хандлагыг, улмаар судалгааны өөр цар хүрээг шаарддаг юм. Үүний эсрэгээр бол тэр нь ойлгомжгүй үлдэж, ном маань уншигчдад хэрэггүй болно. Манай уншигчид европ нэр томъёонд дассан байдаг. Тэд “хаан”, “граф”, “канцлер” юмуу “хөрөнгөтний коммун” гэж юу болохыг мэднэ. Ойкумен буюу дэлхий ертөнцийн Дорно талд энд дүйцэх нэр томъёо байдаггүй. Жишээ нь “Хаган” буюу хаан гэдэг нь король, император биш, өвөг дээдсээ хүндлэх ёс заншлыг хавсран гүйцэтгэдэг, насаараа сонгогдсон цэргийн удирдагч байдаг. Өөрөө зүрхийг нь хагарахад хүргэсэн II Генрихийн буяны хоолонд Арслан зоригт Ричард үйлчилж байна гэж төсөөлж болох уу? Мөн энэ ёслолд гаскон болон английн дээдсийн төлөөлөгчид оролцож болох уу? Энэ бол шал дэмий юм. Харин Их талын дорнодод тэрээр үүнийг гүйцэтгэх ёстой бөгөөд ингэхгүй бол түүнийг тэр дор нь алах байсан.
“Хятад” буюу “индус” зэрэг нэр нь баруун европынхны хувь бүхэлдээ угсаатны систем болсон “франц” болон “немц” гэдэгтэй дүйдэггүй. Харин соёлын өөр зарчим дээр нэгдсэн индусуудыг кастын систем холбож, хятадуудыг дүрс бичиг, хүмүүнлэгийн боловсрол холбож байдаг. Индостаны унаган хүн мусульман шашин авмагцаа л индус байхаа болих ба учир нь тэрээр тэрслүү болж, үл хөндөгсдийн зэрэглэлд ордог. Күнзийн үзлээр бүдүүлэгчүүдийн дунд амьдардаг хятад хүнийг мөн л бүдүүлэг гэж үздэг. Харин хятадын ёс заншлыг мөрддөг харийн хүнийг хятад гэж үнэлдэг.
Дорно, Өрнийн угсаатнуудын харьцуулахын тулд бид адил жинтэй хуваагдалд тохирох зөв нийцлийг олох ёстой. Үүний тулд бүх орон, эрин зуунд байдаг байгалийн үзэгдэл болох угсаатны мөн чанарыг судалж үзье.
Тавьсан зорилгодоо хүрэхийн тулд бид манай орчин үеийн төсөөлөлд нийцэхгүй хэмээн урьдчилан үгүйсгэлгүйгээр ертөнцийн тухай эртний уламжлалт мэдээлүүдэд туйлын анхааралтай хандах ёстой Хэдэн мянган жилийн өмнө амьдарч байсан хүмүүс орчин үеийн бидний л адил ухамсар, чадвар, үнэн болон мэдлэгт тэмүүлдэг байсан гэдгийг бид ямагт мартдаг. Энэ тухай янз бүрийн үед янз бүрийн ард түмнээс бидний үед ирсэн бичиг ном гэрчилж байна. Чухам ийм учраас л угсаатан судлал буюу этнологи нь практикийн хувьд зайлшгүй болоод байгаа бөгөөд түүний арга зүйн туслалцаагүйгээр эрт үеийн соёлын өвийн ихээхэн хэсэг нь бидний хувьд хүрч чадахгүй зүйл болж үлдэх юм.
Дорнод азийн түүх, соёлыг ойлгоход ердийн хандлага тохирохгүй. Европын түүхийг бид судлахдаа Франц, Герман, Английн түүх гэх мэт, эсвэл эртний, дундад зууны зэргээр салбаруудад хувааж болно. Энэ түүхээ, жишээлбэл, Римийг судлаад бид дараа нь Римтэй тулгарч байсан болохоор нь л хөрш ард түмнүүдийг нь хөндөж эхэлдэг. Барууны орнуудын хувьд ийм хандлага нь урьдаас заагдсан үр дүнгээрээ цагаатгагддаг, харин Дунд азийг судлахад ийм аргаар бид хангалттай үр дүн олж чадахгүй. Үүний шалтгаан нь их гүнзгий юм. “Ард түмэн” гэх нэр томъёоны азийн ойлголт түүний европ ойлголтоос эрс ялгаатай байдагт оршино. Азид өөрт нь гэсэн угсаатны нийтлэгийг янз бүрээр ойлгодог, Хэрэв бид манай сэдэвт шууд холбоо байхгүй гээд Левант болон Энэтхэг Хятадыг авч хаялаа ч гэсэн хятад, иран, нүүдлийн гэсэн гурван өөр ойлголт үлдэх л болно. Энэ үед нүүдлийн гэх сүүлчийн ойлголт нь эрин үеэсээ хамаарч нэн хүчтэй өөрчлөгдөн байдаг. Хүннүгийн үед байсан ойлголт уйгар болон монголын үед өөр болдог.
Этноним буюу ард түмний нэр нь Европт тогтвортой, Дундын азид их багаар шилжимтгий, Хятадад шингээгч, Иранд түлхэгч шинжтэй байдаг. Өөрөөр хэлбэл Хятадад хятад хүн гэж тооцогдохын тулд хятадын ёс суртахууны үндэс, боловсрол, зан үйлийн дүрэм журмыг сахидаг байх ёстой, энд гарал үүслийг тооцдоггүй, мөн хятадууд эртнээс олон хэлээр ярьдаг байсан учраас хэлийг ч мөн адил харгалздаггүй байв. Ийм учраас л хятадууд жижиг ард түмэн, ястныг шингээн гарцаагүй өргөсдөг байсан нь ойлгомжтой юм. Үүний урвуугаар Иранд хүн перс болж төрөхөөс гадна заавал Агурамаздийг уншиж, Ариманыг үзэн ядах ёстой аж. Үүнгүйгээр “арий” хүн болох боломжгүй. Дундад зууны персүүд (сасанидын) өөрсдийгөө “сайн угсаат” (номдорон) гэж нэрийдэн, бусад хүмүүсийг үүнд оруулаагүйгээс гадна хэн нэгнийг эгнээндээ оруулах боломжийг ч сэтгэдэггүй байжээ. Үүний үр дүнд ард түмний тоо тасралтгүй унаж байв. Парфянчуудын ойлголтыг дүгнэх нь их хүнд авч энэ нь персийнхнээс зарчмын хувьд ялгаагүй, гэхдээ зөвхөн нэлээд өргөн байсан нь ойлгогдож байна.
Харин хүн гэж хэлэгдэхийн тулд эсвэл гэр бүлийн тусламжтайгаар овгийн гишүүн болох, эсвэл өөрийн хүн болох үедээ шаньюйн зарлигаар болох ёстой байв. Хүнгийн залгамжлагчид болох түргүүд бүхэл бүтэн овгийг эгнээндээ нэгтгэх болжээ. Ингэж авсан суурин дээр жишээлбэл, казак, якут зэрэг овгийн холимог холбоод үүсчээ. Угаасаа түрэг, хүн нартай нэн ойр монголчуудад орд, өөрөөр хэлбэл сахилга бат, удирдлагаараа нэгдсэн хүмүүсийн бүлэг зонхилох болов. Энд гарал үүслийг ч, хэлийг ч, сүсэг бишрэлийг ч биш, зөвхөн эр зориг, захирагдахад бэлэн байх чадварыг л шаарддаг байв. Орд хэмээх нэр нь этноним буюу ард түмний нэр биш бөгөөд орд гэж байгаа үед этнонимууд ерөөсөө алга болдог, учир нь тэдний хэрэгцээ байхгүй болж, улмаар “ард түмэн” гэх ойлголттой “төр” гэх ойлголт давхцдаг.
Үүнтэй холбогдуулан дээр дурдсан бүх тохиолдолд “төр” гэсэн ойлголт нь янз бүр байдаг бөгөөд орчуулж ч болдоггүй юм. Хятадын “го” гэдэг нь саад болон жадтай хүн бүхий дүрс үсгээр тэмдэглэгдэнэ, Энэ нь английн “state” буюу францын “etate“, тэрч байтугай латины “imperium“ болон “respublicae“ зэрэг үгстэй огтхон таардаггүй. Ираны “шах” буюу дээр дурдсан “орд” хэмээх нэр томъёо нь агуулгаараа асар хол.
Эдгээр ялгааны нарийн ширийн нь заримдаа төстэйн элементүүдээс илүү ач холбогдолтой байдаг, энэ нь ч европчуудад гайхалтай тэр зүйл монголчуудад жам ёсны, эсвэл үүний урвуугаар байх зэргээр түүхэнд оролцогчдын зан үйлийг тодорхойлдог юм. Үүний шалтгаан нь янз бүрийн ёс зүйд байгаа бус, харин тухайн тохиолдолд төр хэмээх зүйлийг адилтгашгүй байгаад оршино. Ийм учраас бид судлан буй ард түмнээ хүчээр тааруулсан хуурамч бүдүүвчид оруулчихгүйн тулд зөвхөн төстэй шинжийг төдийгүй, мөн ялгааг тэмдэглэж байх болно.
Европынхтой төсөөгүй төрийн хэлбэр, нийгмийн институт, угсаатны хэм хэмжээ, тэр ч байтугай ярианы хэв маяг зэрэг бүх юмыг угаасаа хоцрогдсон, төгс бус, ач холбогдолгүй мэтээр үздэг нэн тархсан үзэл бодол биднийг гонсойлгож чадахгүй нь мэдээж юм. Улиг болсон европ төвт үзэл нь бүдүүлэг ойлгоцдоо л хангалттай болохоос биш, ажиглаж буй үзэгдлийн олон янз байдлыг шинжлэх ухаанчаар сэтгэн бодоход тохирохгүй юм. Хятад юмуу арабчуудын бодлоор баруун европчууд яс чанар муутай байж болно шүү дээ. Энэ ч гэсэн буруу бөгөөд шинжлэх ухааны хувьд явцгүй болно. Иймээс бид бүх ажиглалтыг яг адил хэмжээний нарийвчлалтай хийж чадах тооцооллын тийм систем олох ёстой гэдэг нь тодорхой. Зөвхөн ийм хандлага л өөр өөр үзэгдлүүдийг харьцуулж, ингэснээрээ үнэн магад дүгнэлт хийх боломжийг олгоно. Судалгааны дээр дурдсан бүх нөхцлүүд нь зөвхөн түүхэнд төдийгүй, мөн газар зүйд ч заавал хэрэгтэй, учир нь энэ нь хүн болон газар зүйн нэрүүдийг холбон өгч байна. Өрнөдөд улс орнуудыг нэрээр нь ялгадаг, харин Дорнодод яадаг бол ?
НЭРГҮЙ УЛС ОРОН БА АРД ТҮМЭН
Бидний Хятад гэж нэрлэдэг мусульманы ертөнцийн дорнод хил, Дундын эзэнт гүрний баруун хойт хязгаарын хооронд оршдог тодорхой нэр үгүй нэг орон оршдог. Энэ орны газар зүйн хил нь маш нарийн тэмдэглэгдсэн, түүний байгаль–газар зүйн нөхцөл нь нэн өвөрмөц бөгөөд давтагдашгүй, хүн ам нь цөөн, эртнээс соёлд холбогдолтой байдаг нь бүр ч хачирхалтай билээ. Энэ орныг хятад, грек, арабын газар зүйчид сайн мэддэг байсан ба энд орос, баруун европын аялагчид ирж, археологийн малтлага олон удаа хийсэн авч…түүний бүх зүйлийг нь яаж ийгээд дүрслэн нэрлэж байсан ч энэ орон өөрийнхээ нэрийг хэлээгүй юм. Ийм учраас энэ орон хаана жаргадагийг зааж өгье.
Памираас дорно зүгт уулын хоёр нуруу үргэлжлэх ба энэ бол өмнө зүгт нь Төвд орших Кунь–лунь болон Тянь–Шань хоёр бөлгөө. Энэ хоёр нурууны хооронд Таримын их уст голоор хэрчигдсэн Такла–Маканы элсэн цөл оршино. Энэ гол эх ч байхгүй, адаг ч байхгүй. Харин эхлэлийг нь “Арал” буюу өөрөөр хэлбэл Яркенд–дарьи, Аксу–дарьи, Хотан–дарьи хэмээх гурван голын ханцуй дунд байх “арал” гэж үздэг. Харин түүний адаг нь заримдаа элсэнд алга болж, заримдаа Карабуранкель нуур хүрч, заримдаа байрлалаа байнга сольж байдаг Лоб нуурыг дүүргэдэг юм.
49 Мурзаев Э. М. Природа Синьцзяна и формирование пустынь Центральной Азии. М., 1968. С. 185-190.
Энэ хачирхалтай оронд гол нуурууд нь нүүн шилжиж, хүмүүс нь уулын сархиагт шавалдан сууна. Уулнаас цэнгэг горхиуд урсан гарах боловч тэр дороо л сэвсгэр хөрсөнд шингэн алга болж, энэ нуруудаас алс хол газраар дахин хөрсөн дээр гарч ирэх бөлгөө. Тэр нь баян бүрдүүд болох ба дараа нь гол ахиад л энэ удаад элсэнд алга болно. Энэ хачин оронд ёроол нь далайн түвшингөөс доош 154 метрт байх хамгийн гүн хүрхрээ байдаг. Энэхүү хүрхрээнд эртний соёлын төв–Турфаны баян бүрд оршино. Зундаа 48 градус халуун, өвөлдөө- 37 градус хүйтэн, намрын ер бусын хуурай агаартай, хаврын хүчтэй салхитай тул хэрхэн шинжлэх ухаан, урлагийн ажил хийх билээ? Гэхдээ багагүй амжилттай хийсэн л юм даа.
Энэ орны эртний хүн ам өөрийн гэсэн нэргүй байсан. Одоо эдгээр хүмүүсийг тохар гэж нэрлэх болжээ, гэхдээ энэ нь этноним биш, “цагаан толгой” (цайвар үст) гэсэн төвдийн tha gar гэсэн хоч юм. Янз бүрийн баян бүрдийн оршин суугчид Европт мэдэх хэлтэй төсгүй, энэтхэг-европын, түүний дотор баруун арийн бүлгийн янз бүрийн хэлээр ярилцана. Энэ орны баруун өмнөдөд, Куньлун уулын бэлд Хотон, Яркендынхантай нягт холбоотой байдаг, гэхдээ тэдэнтэй холилдоогүй төвдийн овгууд нүүдэллэн амьдарна.
50 Гумилев Л. Н. Терракотовые фигурки обезьян из Хотана (опыт интерпретации) // Сообщения Эрмитажа. 1959. Т. XVI. С. 55-57.
Манай тооллын эхний зуунд энэ оронд Кашгарийн өмнөдөөс Хотан хүртэл амьдарч байсан сакууд болон эх оронд нь болсон иргэний дайны гамшгаас зугтсан хятадын дүрвэгчид баруун талаас нь нэвтрэн орсон байна. Хятадууд Турфаны баян бүрдэд Гаочан хэмээх өөрийн колонийг байгуулжээ. Энэ нь IX зууны үе хүртэл байж байгаад ул мөргүй алга болсон байна. Энэ орны этноним нэрийг сонгон олох боломжгүй нь тодорхой байна. Учир нь энэ бол Эртний ертөнцөд хамгийн шилдэгт тооцож болох аж ахуйг эрхэлж асан соёлт хүмүүс байсан билээ.
Төв азийн баян бүрдийн шинж чанарыг эртнээс авахуулан хүний хэрэгцээтэй нийцүүлж байжээ. Турфанчууд газар доорх усан хангамжийн ираны кяриз хэмээх системийг эзэмшиж, үүний ачаар усалгаатай талбай нь олон хүн амыг тэжээж байв. Тэд жилд хоёр удаа ургац авдаг байжээ. Турфаны усан үзмийг зүй ёсоор дэлхийн шилдэг гэж тооцон, шийгуа, гуа, гүйлс хавраас эхлэн намар орой болтол байж, урт ширхэгт хөвөнгийн суулгацыг шовх улиангар, торгомсог моддоор салхинаас хамгаалж байв. Эргэн тойрны хад асганы нуранги, сайр, бул чулуудаас бүрдэх чулуун цөл нь модыг бутыг ч ургуулдаггүй байжээ.
51 Прекрасное описание природы этих мест см.: Мурзаев Э. М. Путешествия без приключений и фантастики. М., 1962. С. 52- 58.
Энэ найдвартай хамгаалалт нь баян бүрдийг том армиас ч хамгаалж чадна. Цөл гатлуулан явган цэргийг тийш хаяхад биедээ хүнс төдийгүй, ус авч явахад хүрэхгүй ба энэ нь жин хөсгийг хязгааргүй нэмэгдүүлдэг учраас энэ нь маш түвэгтэй. Мөн нүүдэлчдийн хөнгөн морин цэргийн довтолгоо ч гэсэн цайзын хананд аюул учруулахгүй. Энэ улсын хоёр дахь том төв–Карашар нь Баграш–куль хэмээх цэнгэг нуурын ойролцоо ууланд оршино. Энэ хот бол “тарган хүмүүсийн газар бөгөөд загасаар хахаж цацна… Байгалаар өөрөөр нь сайтар хамгаалагдсан, хот дотроос хамгаалахад хялбар” байсан аж. Баграш–кулиас Лоб нуурыг тэтгэгч Кончедарья гол урсана. Түүний эргийн дагуу цангах зовлонгүйгээр гоёмсог буга, зэрлэг гахайн сүргийг халхлан буй өндөр зэгс, чацаргана, сухай, улиангараар хөвөөлөгдсөн их уст Тарим голд хүрч болно.
52 Бичурин Н. Я. (Иакимф). Собрание сведений по исторической географии Восточной и Срединной Азии /Сост. Л. Н. Гумилев и М. ф. Хван. Чебоксары, 1960. С. 558. 53 Мурзаев Э. М. Путешествия без приключений и фантастики. С. 113-129.
Энэ орны суурин амьдрагсдын эртний үзэл суртал нь шашин гэж нэрлэж болохооргүй хинаяна хэлбэрийн буддизм (“бага хөлгөн” буюу “бага дугуй” өөрөөр хэлбэл Буддын хольц байхгүй, хамгийн ортодоксаль сургааль) байжээ. Хинаянчууд Бурхныг үгүйсгэж, түүний оронд кармын (шалтгаацлын дараалал) ёс суртахууны хуулийг тавьдаг байв. Төгс байдалд хүрсэн хүн зовлонгоос ангижирч, дахин төрөх замаар төгс тайван төлөв байдал буюу нирванд хүрэхийг хүссэн дурын ямар ч хүнд үлгэр жишээ болж байдаг юм. Нирванд бурхны ивээлээс ч, бусдын тусламжаас ч үл хамаарах зорин тэмцсэн хүн буюу архад (ариун) л хүрч чадна. “Өөрөө өөртөө дэнлүү бол” гэж хинаянчууд ярьдаг.
“Төгс замд хүрэх” гэдэг нь цөөн хүний асуудал гэдэг нь ойлгомжтой. Харин бусад нь юу хийдэг байсан вэ? Тэр зүгээр л өдөр тутмын ажлаа гүйцэлдүүлж, архадуудыг хүндэтгэн чөлөөт цагаараа сургаалийг нь сонсон, ирээдүйд дахин төрөхдөө өөрсдөө ариун санваартан болно гэдэгтээ найддаг байв. Ягшмал номлол угсаатны зан үйлийн тогтсон бодолд маш бага нөлөөлдгийг бид ч гэсэн бусад жишээн дээрээс мэдэх билээ. Турфан, Карашар, Кучийн архад, наймаачид, цэрэг, газар эзэмшигчид зэрэг нь буддизмыг зөвхөн өнгө будаг болгосон нэгдмэл системийг бүрдүүлж байжээ.
Гэхдээ юмсын өнгө будаг нь өөрийн, заримдаа чухал үүргийг гүйцэтгэдэг байв. Хинаяна нийтлэг нь XY зуун хүртэл амьдарч, Яркенд болон Хотокт дахь бүдэг бараг, олон шинжит, нарийн нийлмэл махаяна шашин нь XI зуун гэхэд исламд байр сууриа тохиолдлоор тавьж өгөөгүй нь тодорхой байна.
Турфанд нүүн ирсэн нүүдлийн уйгарууд манихейн шашныг шүтэж, гэхдээ турфанчууд буддизмыг хэлбэрийн төдий шүтэж байсан шиг шүтсэн нь илт байна.
54 Гумилев Л. Н. Древние тюрки. С. 381-386 –д тодорхой бий.
XII зуун гэхэд манихей бие даасан сургаалийн хувьд устаж, гэхдээ манихейн зарим үзэл санаа буддын гүн ухааны зарим чиглэл, болон несториан шашинд орж, XI зуунд бүх Төв Азиар ялалтын марш хийсэн юм. Энэ зуунуудад Турфан, Карашар, Кучийн оршин суугчид өөрсдийгөө уйгурууд хэмээн нэрлэж байлаа.
Уйгар дахь несторианууд өөрсдөд нь шингэсэн тэвчээргүй занг үл харгалзан буддистуудтай хэл амаа ололцсон юм. Христийн ёс нь хинаянагийн бурхангүй хийсвэрлэлээс ангид, шашны хэв маягийн бүхий хүмүүсийн хувьд хүссэн хүмүүс байсан нь ойлгомжтой юм. Буддын сургаал нь алт, мөнгө, эмэгтэйчүүдийг “хүртэх зам”-ыг хориглодог болохоор худалдаачид хүртэл христианчууд болсон аж. Ийм учраас шашин шүтдэг, гэхдээ эдийн засгийн амьдралын идэвхитэй оролцоог хүлээн авсан хүмүүс арга буюу ажиллаж, амьдрахад нь саад болохооргүй тийм л сургаалийг хайх болжээ. Эндээс энэхүү үзэл суртлын хоёр системийн хувьд таарч тохирох экологийн холбоо олдсон гэсэн дүгнэлт хийж болно.
Энэ орны баялаг нь голчлон газар зүйн таатай байршил дээр суурилж байсан: эндээс нэг нь умар, нөгөө нь Тянь-шанаас урдуур чиглэсэн худалдааны хоёр зам гардаг байв. Энэ замаар хятадын торго Прованс руу урсаж, харин Франц, Византийн тансаг эдлэл Хятад руу явдаг байв. Цөл дамнасан хүнд хүчир аяллынхаа дараа жинчид баян бүрдүүдэд амарч, тэмээ, морьдоо тэнхрүүлдэг байжээ. Үүнтэй холбоотойгоор нутгийн эмэгтэйчүүдэд “эртний анхны мэргэжил” маш их хөгжсөн бөгөөд эрчүүд нь ч энэ орлогын заримыг нь халааслахын тулд эхнэрүүддээ үүнийг зөвшөөрдөг байв. Уйгар эмэгтэйчүүд ч энэ ажилд дадсан ба бас монголчуудтай тогтоосон холбооны ачаар Уйгар нь үлгэрийн мэт баяжсан ажгуу. Эндэхийн оршин суугчид монголын хаан Өгөдэйгөөс эхнэрүүд нь аянчдыг зугаацуулахыг хориглохгүй байхыг гуйж байжээ.
55 Книга Марко Поло /Отв. ред. И.П. Магидович. М.; Л., 1956, С. 81-82.
Энэ заншил нь зөвөөр хэлбэл, хэл, шашин, улс төрийн байгуулал болон өөрийн нэрийн дэргэд хавьгүй тогтвортой болсон угсаатны зан үйлийн тогтсон үзлийн элемент болох юм. Зан үйлийн тогтсон үзэл нь дасан зохицох шинж тэмдэг байдлаар, өөрөөр хэлбэл газар зүйн орчинд угсаатны дасан зохицох арга байдлаар бүрэлддэг байна. Энд байгаа угсаатнаасаа олон нэр солигдсон бөгөөд чингэхдээ этноним солигдохыг улс төрийн байдлаар тайлбарлаж байсан байна.
Эдгээр үржил шимт баян бүрдийн баян, олон тоотой хүн ард дайчин нүүдэлчдийг амархаан тэжээж чадах байв, тэр тусмаа эхлээд уйгарууд, дараа нь монголчууд өөрийн албатуудыг харийн дайснаас хамгаалах болжээ. Гурван зуун жилийн хугацаанд уйгарууд унаган хүмүүсийн дунд уусч, гэхдээ тохар хэлийг нь түрэг хэлээр солихыг шахаж амжсан аж. Дашрамд дурдахад энд нэг их хүч ороогүй юм. Учир нь XI зуунд түрэг хэлээр номин давалгаат Гантигийн тэнгис, ойн нуруут Карпатаас эхлээд Бенгалийн шугуй, Хятадын Их цагаан хэрэм хүртэлх бүх ард түмэн ярьдаг байлаа. Түрэг хэлний ийм өргөн тархалт нь энэ хэлийг баян бүрдүүдэд худалдааны үйлдэл хийхэд тохиромжтой болгосон байна. Дундад азийн хоёр талын оршин суугчид адилхан л худалдаа наймаанд дуртай байжээ. Иймээс төрөлх, гэхдээ бага хэрэглэгдэх хэлээ нийтэд ойлгомжтой хэлээр солих нь зөвхөн Таримийн сав газрын зүүн хойд хэсэг төдийгүй, уйгаруудын үүргийг ягма, харлиг зэрэг түрэг овгууд хүлээн авсан баруун өмнө хэсэгт ч бэрхшээлтэй тулгараагүй юм. Гэхдээ тэдий болон уйгаруудын хоорондын ялгаа асар их байлаа. Уйгарууд өөрийн албатуудынхаа ахуй, шашин, соёлын алийг нь ч хөндөөгүй аж. Харин 960 онд исламын шашин авсан харлигууд Кашгар, Яркенд, Хотаны баян бүрдүүдийг Самарканд, Бухарыг дуурайн хувирган өөрчилсөн юм. Ийнхүү газар зүйн хувьд нэг цул муж, өөр хоорондоо огтхон ч нөхөрсөг бус угсаатан соёлын хоёр бүс болон хуваагдсан юм. Гэхдээ хүч тэнцүү, баян бүрдүүдийн хоорондын зай их бөгөөд нэвтэршгүй байв.
Энэ нөхцөл байдал нь яагаад энэ орон нэгдмэл нэргүй үлдсэнийг тайлбарлаж байна. Эрт үед хятадууд түүнийг Сиюй буюу “Алтан хязгаар” гэж, түүний төгсгөл Памир, Алтайг “Сонгинот уул” гэж нэрлэж байжээ. Эллинчүүд энэ орныг “Серика”, харин тэндээс авчирсан үнэт таваарыг “серикум” (торго) хэмээн нэрлэж байжээ. Энэ үгийн гарал үүслийг би тайлбарлахыг оролдохгүй.
Шинэ үед түүнчлэн Кашкар, Дорнод Туркестан, XYIII зуунд манжийн тогтоосон шууд утга бүхий “шинэ хил” буюу Шинжаан гэх мэт нөхцөлт нэрүүдийг хэрэглэж байна. Манай эрин үед энэ бүх нэрүүд нь тохирохгүй юм. Эртний хятадуудын хувьд “Өрнө” гэж байсан тэр зүйл нь XII-XIII зууны дунд болчихжээ. Энэтхэг европчууд амьдардаг, турк хэлээр ойлгож сурч буй энэ орныг “Туркестан” гэж нэрлэх нь тэнэг хэрэг. Кашгар хараахан нийслэл болоогүй, “шинэ хил” нь алс холоос ч үзэгдээгүй байна. Таримийн сав газар хэмээх газар зүйн нөхцөлт нэрийг нь үлдэээх нь дээр бизээ. Гол нь найдвартай баримжаа бөгөөд ямар ч тохиолдолд төвч, удаан байдаг. Түүнээс гадна “Шинжаан” гэсэн нэр томъёо нь Тянь-Шанаас хойгуур байрлах, огт өөр түүхэн хувь заяа бүхий Жунгарыг (мөн л нөхцөлт бөгөөд хожмын нэр) өөртөө багтаадаг.
Уйгарын дорнод хилийг тодорхойлоход маш хүнд. Өнгөрсөн зуунуудад тэр нь ихээхэн өөрчлөгдөж, энэ өөрчлөлтийн олонхийг нь цаг хугацааг тэмдэглээгүй. Уйгаруудад Хамийн баян бүрд, магадгүй буддын урлагийн эрдэнэсийн сан болсон агуйн хот–Дуньхан хамаарагдаж байсан байх. Гэхдээ илүү зүүн тийшлэх газар болох Наньшаны салбар уулсын баян бүрдүүдийг уйгаруудаас тангадууд булаан авсан юм. Хэдийгээр өөрийгөө ингэж нэрлэдэг хүмүүс байдаг боловч Тангад нь уйгаруудтай адил одоо байхгүй ард түмэн юм. Гэхдээ энэ нь зэрэглээ болой. Өөрсдийгөө уйгар гэж нэрлэдэг ферганы туркууд XY-XYIII зуунд дорно зүгээс нүүн гарсан юм. Мөн өөрсдийгөө тангад гэж нэрлэдэг тэр хүмүүс эрт дээр үед тангадуудын заналт дайсан агсан нүүдлийн төвдүүд, үлдэгдэл угсаатан болно.
Ингээд Азид нэрийн утга болон тэдгээрийг хэрэглэх нь дандаа давхцдаггүй гэдгийг түүхэн шүүмжлэл үзүүллээ. Харамсалтай, дээр нь бас дахин дахин хийх алдаанаас зайлсхийхийн тулд Европ, Ази, Америк, Далайн орнууд, Африк, Австрали хаа ч байсан үйлчилдэг тооцооллын систем боловсруулах хэрэгтэй байна. Гэхдээ энэхүү системд утга санааг фонетика буюу авиан зүйд даатгаж, өөрөөр хэлбэл түүний үндсээр нь хэл шинжлэл биш, харин түүхийг авна.
“УГСААТАН” БОЛ С.М.ШИРОКОГОРОВЫН БҮТЭЭЛ МӨН
Угсаатан бие даасан үзэгдэл болох хоёрдагч биш, анхдагч ерөнхий үзэл баримтлал С.М.Широкогоровт хамаарна. Тэрээр угсаатныг “элементүүдийн бүтээх, хөгжих, үхэх үйл явц явагдаж байдаг, зүйлийн хувьд хүнд төрөлхтөнд оршин байх боломж олгодог хэлбэр” гэж үзсэн юм. Мөн угсаатныг “гарал үүсэл, зан заншил, хэл, амьдралын хэвшлээрээ нэгдсэн хүмүүсийн бүлэг” гэж тодорхойлсон.
56 Широкогоров С. М. Этнос: Исследование основных принципов изменения этнических и этнографических явлений //Изв. восточного ф-та Дальневосточн. ун-та (Шанхай). 1923. XVIII. Т. 1. Предисловие. С. 4-6.
57 Там же. С. 28.
58 Там же. С. 122.
Энэ хоёр сэдэв нь ХХ зууны эхэн үеийн шинжлэх ухааны төлөв байдлыг харуулж байна. Газар зүйн хувьд “угсаатан нь орчиндоо дасан зохицож, захирагдаж, энэхүү орчны хэсэг болж, түүгээр уламжлагддаг” гэдгийг хүлээн зөвшөөрсөн. Энэ концепцийг В.Анучин “нэгдмэл газар зүй” нэрийн дор дахин амьдруулж байна, гэхдээ түүнийг нь хүлээн зөвшөөрөөгүй юм. Энд социал бүтцийг биологийн категори мэтээр авч үзэн, угсаатны хүрээллээс үүдэн дасан зохицохуйн шинэ хэлбэр болох хөгжил явагддаг гэж үзсэн. “Угсаатан нь өөрийнхөө хөршөөс түүний хувийн жинг өргөдөг, түүнд эсэргүүцэх чадварын шинжийг мэдэгдэж байдаг гэж хэлж болохоор өөрчлөлтийн дохиог хүлээн авдаг” Энд С.М.Широкогоровын үзэл баримтлал бүтээлч үйлийг орчны “дуудлагад” үзүүлэх хариу үйлдэл гэж үздэг А.Тойнбийн “дуудлага- хариу”–ны тухай үзэл бодолтой хоолой нийлж байна.
59 Там же. С. 124-126.
60 Toynbee А. J. Study of History /Abridgement by D. Somervell. London, New York, Toronto, 1946.

Харин С.М. Широкогоровын “ерөнхий дүгнэлтүүд” хамгийн бага эсэргүүцэлтэй тулгарч байна. “1.Угсаатны хөгжил нь бүх бүрдэл дасан зохицох замаар…чингэхдээ зарим үзэгдлүүд нь нарийсахын зэрэгцээгээр бусад нь хялбарших боломжтой. 2. угсаатан нь орчиндоо өөрөө дасан зохицох ба түүнийг өөртөө зохицуулан өөрчилнө. 3. Угсаатнуудын хөдөлгөөн нь хамгийн бага эсэргүүцлийн шугамаар явагдана.”
61 Широкогоров С. М. Этнос… С. 126-129.
Одоо энэ нь шинэ зүйл биш. Мөн Широкогоровын үзэл бодол хагас зуун жилээр хоцрогдсонд гайхаад байх юу ч үгүй юм. Этнологийн хувьд анхдагч материал болж байдаг түүхэнд амьтан судлалын зүй тогтлыг механикаар шилжүүлэн авчирсан нь үүнээс ч долоон дор болно. Ийм учраас Широкогоровын зарчмуудыг хэрэглэхэд давж болшгүй бэрхшээл учирна. Жишээлбэл, “угсаатны хувьд хэрэв энэ нь зүйлийн хувьд түүний амьдралын зорилго болох, түүнийг оршихуйг хангаж чадаж байвал оршихуйн дурын хэлбэр боломжтой” гэсэн сэдэв нь угаасаа буруу юм. 62 Там же. С. 100.
Ингэвээс Хойт Америкийн индианчууд, Зүүнгарын нүүдэлчид АНУ–ын эрх мэдэл дор амьдарч, Хятад өөрийн ахуйгаас татгалзсаны үнэ цэнээр ингэж бас болох нь. Гэхдээ тэдний аль аль нь амжилт олох найдваргүй ч гэсэн тэнцүү бус тулааныг илүүд үзэх байх. Тун цөөн угсаатан л зөвхөн амьдрахын тулд дайсандаа захирагдахыг зөвшөөрнө байх. Энэ нь ойлгомжтой бөгөөд нэмэлт үндэслэл шаардахгүй юм. “Газар нутгаа өргөтгөх, соёл болон хүн амын тоогоо хөгжүүлэх эрмэлзэл зэрэг нь угсаатан бүрийн хөдөлгөөний үндэс мөн” гэсэн нь буруу бөгөөд эртний угсаатнууд огтхон ч харгис байгаагүй билээ. “Бага соёлтой угсаатан мөхдөг” гэсэн мэдэгдэл нь зарим талаар л үнэн, ихэнхи тохиолдолд угсаатны мөхөл илүү соёлтой хөршүүдээс болдог нь ажиглагдсан зүйл. “Зохион байгуулалт нь нарийн, тусгайлан дасан зохицох хэлбэр нь дээд байх хэрээр зүйлийн (өөрөөр хэлбэл угсаатан) ахуй богиносно” гэсэн үндэслэлийг огт хүлээн зөвшөөрч болохгүй.
63 Там же. 64 Там же. С. 118. 65 Там же. С. 119.
Эсрэгээр угсаатны устан алга болох нь бүтцийг хялбаршуулсантай холбоотой байдаг ба энэ тухай хожим өгүүлнэ. Ямар ч гэсэн Широкогоровын ном өөрийнхөө үеийн хувьд алхам урагш болж, угсаатны зүй (этнографи), угсаатан судлалын (этнологи) хөгжлийн хэтийн төлөвийг өргөтгөжээ. Одоо миний бичиж буй зүйлийг хагас зууны дараа дахин хянах болно, энэ л шинжлэх ухааны хөгжил билээ.
С.М.Широкогоровоос ялгаатай нь бид системийн хандлага, экосистемийн үзэл баримтлал, био хүрээний болон амьд бодисын (биохимийн) тухай сургааль, түүнчлэн даяар хэмжээнд антропогенийн ландшафт үүссэн тухай материалуудтай байна. Энэ бүхэн нь урьд өмнө нь огт боломжгүй байсан асуудлыг илүүтэй шийдвэрлэх санал дэвшүүлэх боломж олгож байна.
“ТӨЛӨВ БАЙДАЛ” БА “ҮЙЛ ЯВЦУУД”
Дээр дурдсан баримтуудын нийлбэр нь формацийн үзэл баримтлалын үндэс болсон категорийн системийг угсаатны нийлэгжилтэд зарчмын хувьд хэрэглэж болохгүйг харуулж байна. Энэ систем нь үйлдвэрлэлийн тодорхой аргаар тодорхойлогдох нийгмийн “төлөв байдлыг” харуулдаг, энэхүү үйлдвэрлэлийн арга нь эргээд үйлдвэрлэх хүчний түвшингөөс, өөрөөр хэлбэл техно хүрээнээс хамаардаг байна. Тооцооллын энэ систем нь материаллаг соёл, төрийн институт, урлагийн арга барилын түүх, товчоор хэлбэл хүний гараар бүтээсэн бүх зүйлийн түүхийг судлахад нэн тохиромжтой юм. Сүүлийн зуун жилд маш их дасал болж, түүнийг угсаатны нийлэгжилтийн шинжилгээнд механикаар шилжүүлэн авчрах болов. Жишээлбэл 1) “угсаатан бол хүмүүсийн социал нийтлэг”, 2) “угсаатан бол социал байгууллага биш, харин ангитай адил овог, овгийн холбоо, төр, сүм хийд, нам зэрэг дурын социал хэлбэрийн нэгийг биш, харин нэгэн зэрэг хэд хэдийг хүлээн авдаг аморф буюу хэлбэржээгүй төлөв байдал” хэмээн тунхаглах болжээ. 66 Козлов В. И. Что такое этнос? //Природа. 1971.№ 2. С. 74.
Түүнээс гадна “угсаатныг арьстан мэтийн социал байгууллагын янз бүрийн төрлүүдтэй холихгүй байх” –ыг сануулах болов.
67 Артамонов М. И. Опять “герои” и “толпа” //Там же. С. 77.

Хэрэв нэгдэх тодорхойлолт нь дээр авсан жишээнүүдийг шууд л бут цохиж байгаа бол хоёр дахийг нь ухамсарлаагүй үзэл бодол байсан ч гэсэн энэ суурин дээр эзэнт гүрнүүд үүсч, задарч байсан нь мэдээжийн хэрэг түүнд захирагдаж байсан ард түмний хувь заяанд тусч байсныг нарийн хямбай авч үзэхэд нийцэж байна.
“Төлөв байдал” гэсэн ойлголт нь байгальд ч, нийгэмд ч байдаг. Байгальд хатуу, шингэн, хийн болон плазмын гэсэн дөрвөн төлөв байдал буй. Тогтонги бодисын молекул нэг төлөв байдлаас нөгөөд шилжихэд хайлах хаалттай дулаан буюу уур үүсэхтэй тэнцүү нэмэлт эрчим хүч шаарддаг. Энэ шилжилт их биш эрчтэй явагдах ба эргэх үйл явц байдаг. Био хүрээний амьд бодисын хувьд ийм шилжилт организмын үхэлтэй холбогдож, эргэлтгүй байдаг. Энэ нь организмын хувьд амьдрал ба үхэл гэсэн хоёрхон төлөв байдал байдаг гэсэн утга болж чадна, Гэхдээ үхэл нь организм бүхэлдээ устаж байгаа хэрэг бөгөөд шилжилтийн энэ агшныг “төлөв байдал” гэж нэрлэх нь тэнэг хэрэг болно. Организмын амьдралын хувьд гэвэл энэ нь ч гэсэн “төлөв байдал”, харин төрөхөөс эхлээд нас бие гүйцэхийг дамжин үхэх хүртлээ өтлөх үйл явц юм. Жирийн бодис дээр амьдралын үйл явцын адилтгал нь эрдэс бодис талсжиж, дараа нь тэр метамофизаци буюу бүтцийн хувьд эвдэрч эмх цэгцгүй масс болдог үзэгдэл болно.
Бид “төлөв байдал” болон “үйл явцыг” судлахдаа ямагт янз бүрийн арга зүйг хэрэглэдэг. “Төлөв байдлын “ хувьд үзэгдлийг бүхэлд нь харахад тохиромжтой санамсаргүй сонгон авсан дурын зарчмаар ангилж болдог. “Үйл явцын”, ялангуяа хувьсал буюу хэлбэр үүсэхтэй холбогдсон үйл явцын хувьд хэдийгээр янз бүрийн түвшний адилтгашгүй бүлгүүд байдаг ч нэг төстэй зүйлд захируулах гэсэн шаталсан зарчимд үндэслэсэн систем зайлшгүй байдаг. Ч.Дарвины боловсронгуй болгосон Линнейн систем ийм бөлгөө. Органик ертөнцийн системийн шаталсан шинж чанар нь амьдралаас салгашгүй, түүнд зайлшгүй байх хувьслын үйл явцын явц болон шинж чанараар нөхцөлдөж байдаг. Амьдрал дөнгөж зогсмогц орчны их бага хурдаар задлах үйлчлэлийн “төлөв байдал” үүсэх бөгөөд энэхүү орчин нь ч гэсэн буцалтгүй эвдрэлд нэрвэгдсэн бусад үхсэн төлөв байдлаас бүрддэг. Ингэхлээр организм, түүний дотор хүний организмд “төлөв байдалд” орох цорын ганц арга байдаг бөгөөд энэ нь хүний хувьд муми болох, угсаатны хувьд археологийн соёл болох явдал юм.
Техно хүрээ болон түүнтэй холбоотой үйлдвэрлэлийн харилцаа бол өөр хэрэг. Энд “төлөв байдал” байдаг.Трактораас маш амархан хог хаягдал хийж болно, эргээд хог хаягдлаас трактор хийж болно. Зөвхөн зарим (харамсалтай нь бага биш) эрчим хүч зарцуулах хэрэгтэй. Нийгмийн амьдралд ч гэсэн “төлөв байдал” байдаг. Эдүгээ түүнийг гэр бүлээ бүртгүүлсэн иргэний төлөв байдал гэж нэрийддэг. Урьд нь түүнийг язгуур угсаа (etat) гэж нэрлэж байв. Ангийн хамаарлыг шилжүүлсэн утгаар нь мөн л “төлөв байдал” хэмээн нэрлэж болох ба гэхдээ энэ нь үйлдвэрлэлийн харилцаа, үйлдвэрлэх хүчний бүтээгдэхүүн, өөрөөр хэлбэл мөн л техно хүрээний бүтээгдэхүүн гэдгийг санах хэрэгтэй. Олзлогдсон цэрэг боол болж, эргээд оргомогцоо феодал болон хувирч болно. Шаталсан зарчмын хувьд ийм хүний хувь заяанд байр ч, хэрэгцээ ч байхгүй, энд энгийн тэмдэглэгээ л хангалттай. Ингээд социал төлөв байдал халагдаж солигдох нь хэдийгээр адилтгал биш ч гэсэн байгалийн төлөв байдал халагдахтай төстэй, энэ нь эргэлт буцалттай бөгөөд нэгээс нөгөөд шилжихдээ нэмэлт эрчим хүч оруулахыг шаарддаг. Харин угсаатан ийм байх болов уу ? Хүчлэн байж угсаатныхаа хамаарлыг сольж болох болов уу ? Мэдээж, үгүй. Зөвхөн энэ л гэхэд угсаатан “төлөв байдал” биш (тэр тусмаа иргэний), харин үйл явц гэдгийг харуулж байна.
“Төлөв байдал”үзэл баримтлалыг тэтгэдэг аберраци хэмээх гажилт нь ажиглагч хүнд түүхэн алслалт байхгүйтэй холбоотой. Гаднын зөрчилгүйгээр угсаатны нийлэгжилтийн үйл явц бүрэн намсах нь 1200 – 1500 жил болдог, тэхэд эрдэм шинжилгээний ажилтан төлөвлөгөөт сэдэвдээ хоёр, хүч хүрвэл гурван жилийг зориулна. Ийм болохоор түүнд өнгөрсөн үе систем болон зүй тогтолгүй эрээлжилсэн цаг үйлийн хүснэгт лугаа харагдах бөгөөд тэрээр нэг эрин үед хязгаарлагдмал бүс нутагт болсон хэдхэн өөрчлөлтийг тэмдэглэж, өөр хоорондоо огт холбоогүй, ердөө л цаг хугацаа, орон зайд давхцсан “төлөв байдлын” бөөгнөрлийг олж хардаг. Геоморфологи буюу газрын гадаргуугийн шинжлэх ухаан бий болох хүртэл хүмүүс террас буюу тэгш газрын байгааг тэртээ дор хаа нэгтэй урсан буй голын элээгч үйл ажиллагаатай холболгүй, уулсыг хүртэл мөнхийн, бараг л гадаргуугийн анхдагч хэлбэр мэтээр үздэг байсан. Харамсалтай нь шинжлэх ухааны бүх нотолгоо нь шүүмжлэгчийн эрдэм мэдлэгийн тодорхой хэмжээний нөхцөлд л бодитой болдог юм. Коперник-Келлерийн нар төвт систем нь XYII зууны үед одон оронг хангалттай сайн мэддэг байсан тэр хүмүүсийг л итгүүлсэн юм. Мөн Г.Менделийн нээлтийг Де Фризе давтсан бөлгөө.
“Төлөв байдал” үзэл баримтлалын эсрэг хоёр дахь үндэслэл бол угсаатны харилцааны бүсэд угсаатны хоорондын хил бүрхэг байдаг явдал мөн. Хэрэв иргэний (өөрөөр хэлбэл нийгмийн ) төлөв байдал нь жишээлбэл, дээдэс морилон ирэх, цэргүүд гомдол гаргах, боол худалдах, шоронгоос чөлөөлөх гэх мэтээр шууд өөрчлөгдөж болдог бол харин Хуанхэй хөндий, Константинополь буюу Хойт Америкт угсаатны харилцаа ямагт гэнэтийн, тэгэхдээ үргэлж удирдлагагүй байдаг эрлийзжилтийн үр дүн гэсэн утгаар үргэлжийн зовлонтой, удаан, нэн олон хэлбэрийн үйл явц байдаг. Сүүлчийн энэ байдал нь мухар сохроор үйлчилдэг биш, харин угсаатны үйл явцын үр дагаврыг харгалздаг, этнологийн хувьд боловсруулсан онол байхгүйгээр голчлон тайлбарлагдана.

