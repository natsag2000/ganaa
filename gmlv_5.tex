Тавдугаар хэсэг

БИДНИЙ ДОТОРХИ БАЙГАЛ
ЭНД ХҮНИЙ ЯМАР ХЭСЭГ БАЙГАЛЬД ХАМААРАХ, ЯМАР НЬ ҮГҮЙГ БОЛОН ЕРТӨНЦИЙН ЯМАР ХЭСЭГ НЬ ХҮНИЙ БИЕЭС ГАДУУР, БАЙГАЛИЙН ГАДНА ОРШДОГ ТУХАЙ ӨГҮҮЛНЭ. ТҮҮНЧЛЭН ЯАГААД ЭНД ЯРЬСАН БОЛОН ДЭЭР ӨГҮҮЛСЭН ЗҮЙЛС НЬ УГСААТНЫ НИЙЛЭГЖИЛТИЙН АСУУДЛЫГ ШИЙДЭЖ ЧАДАХГҮЙ БАЙГААГ ХАРУУЛНА.
XVIII. Угсаатан ба амьтны бүл
УГСААТАН БОЛ АМЬТНЫ БҮЛ БИШ
Заримдаа энгийн үзэгдэл нь хэтийн хандлагаараа сургуулийн төсөөллөөс халин гардаг шинжлэх ухааны дүгнэлт хийх хөрс болж өгдөг. Угсаатны тухай шинжлэх ухаанд олон зүйлийг дахин бодох, мөн дассан олон зүйлээсээ татгалзах хэрэгтэй болдог.
Гоморхой уншигчид угсаатан нь биднийг цэвэр биологийн функци бүхий организмтай адилтгаж байгаа мэт санагдаж болно. Гэхдээ энд төстэй зүйл нь гадаад шинжтэй, харин ялгаа нь зарчмын шинжтэй юм аа. Угсаатан нь цуваанд орж, заримдаа сарниж, их биеэсээ салсан гар буюу чих шиг мөхдөг. Организм заавал өөртэйгөө төстэй удам төрүүлдэг, харин угсаатан бүр давтагдашгүй байж, уламжлал нь хэт угсаатны бүхэллэгийн хил заагийг давдаггүй юм. Организм эрт орой нэгэн цагт үхнэ, тэгтэл үлдэц–угсаатан гэх мэт байж байдаг.
Ямар ч тохиолдолд угсаатан болон түүний адилтгал мэт авч үздэг (амьтдын дунд) бүлийн хооронд тэнцүүгийн тэмдэг тавьж болохгүй юм. Энд ялгаа нь төстэйгөөсөө хавьгүй гүнзгий байдаг. Бүл нь тодорхой газар нутаг дээр хэд хэдэн үеийн турш оршин амьдардаг, дотор нь чөлөөт үржил явагддаг, үүнийхээ зэрэгцээ өөр бүлээс зарим хэмжээгээр тусгаарлагдсан нэг зүйлийн амьтдын төрлийн нийлбэрийг хэлнэ. Бидний үзсэнээр угсаатан бол төстэй амьтдын нийлбэр биш, харин генетикийн хувьд ч, функцийн хувьд ч олон янзийн хүнээс төдийгүй, техник, антропогенийн ландшафт, соёлын уламжлал зэрэг хүмүүсийн олон үе дамжсан үйл ажиллагааны бүтээгдэхүүнээс бүрддэг. Хөдлөнги угсаатны хувьд бас түүхэн цаг хугацааны мэдрэмж байх бөгөөд үүнийг тооцооллын янз бүрийн систем бүхий цаг тооны бичгээр тэмдэглэдэг юм.
Гэхдээ царцанги шатандаа буй угсаатанд угсаатан сэтгэл зүйн категори болох түүхэн цаг хугацааны мэдрэмж байдаггүй нь тэднийг зөвхөн бүл гэж авч үзэх эрхийг олгохгүй. Тогтонги угсаатан ч гэсэн хэдийгээр тодорхой хүрээнд ч гэсэн газар нутгаа солих, газар зүйн орчин өөрчлөгдсөн үед дассан нөхцөл хайж нүүдэл хийх зэргийг нэлээд чөлөөтэй хийдэг. Угсаатны доторх эвцүүлгийг эсвэл язгуур угсааны харилцаагаар, эсвэл цус холилдохыг хориглосон уламжлалаар, эсвэл эрхийн хэм хэмжээ буюу шашнаар зохицуулдаг. Хааяа тохиолддогчлон эдгээр хориг суларвал энэ нь ямагт угсаатны ойртон ирж буй задралын шинж тэмдэг болно.
Эцэст нь угсаатан хөршөөсөө тусгаарлагдсан шинж чанар нь газар нутагтай холбоогүй. Хэрэв хоёр бүлийн газар нутаг давхцах аваас тэд хурдан нэг болон уусан нэгддэг, харин хоёр буюу түүнээс илүү угсаатан нэг газар нутаг дээр олон зуун жилээр зэрэгцэн орших болбол эсвэл хэт угсаатан, эсвэл угсаатны харилцааны дурын түвшний бүсийг бүрдүүлдэг. Үүний эсрэгээр угсаатны хоорондын тэмцэл нь байнгын үзэгдэл боловч үүнийг оршихуйн төлөө тэмцлийн үүднээс тайлбарлаж болдоггүй, учир нь энэ тэмцэл газар нутаг хэт олон хүнтэй болсноос болдоггүй. Харин бүл хоорондын тэмцлийг дискрет (корпускуляр) систем гэх боломж байхгүй, учир нь бүл дэх амьтны зорилго нь өөрөө амьдрах, үр удмаа үлдээхэд оршдог.
Дээд сүүн тэжээлтний бүлийн оршихуйн хэлбэр болох сүрэг, бүлгүүд нь өнгөц харахад энгийн угсаатантай төстэй. Гэхдээ энэ төстэй байдал нь хуурамч юм. Сүрэг бол моногами, полигами буюу түр зуурын шинжтэй гэр бүлийн үүр юм. Удирдагч-эр амьтан нь сульдаж, үр удамдаа нөлөөгөө алдвал эдгээр нь задардаг. Угсаатан нь нэг хувь заяагаар нэгдсэн хүмүүсийн бүлэг буюу консорциос өсөн гардаг. Хэрэв энэ нь дан эрчүүд байвал тэд гаднаас эхнэр авч, гэр бүлийн харилцаа хоёр гуравдахь үедээ үүсдэг. Гэр бүлийн холбоо үүсэн буй угсаатныг бэхжүүлж байдаг, гэхдээ энэ нь заавал байх зүйл биш, учир нь гарем (мусульманы олон эхнэр) бүрдүүлэх үед онцгой илт байдаг өргөн экзогами (овог дотроосоо суух) байх явдал тохиолддог. 1 Бромлей Ю. В. Этнос и эндогамия //Советская этнография. 1969. № 6.
Ийнхүү угсаатан бол амьтны бүл биш, харин зөвхөн хүнд хэвшмэл байдаг, нийгмийн хэлбэрээр дамжин өөрийгөө илрүүлдэг, тохиолдол бүрт өвөрмөц байдаг системийн үзэгдэл юм. Учир нь улс орны аж ахуй тэжээгч ландшафт, техникийн хөгжлийн түвшин, үйлдвэрлэлийн харилцааны шинж чанар зэрэгтэй ямагт холбоотой байдаг юм. Мэдээж хэрэг энэ нь этнолог хүний бүлийн нийлэгжилтийг үгүйсгэх ёстой гэсэн хэрэг биш, гэхдээ тэрээр энэ нь бидний судлан буй үзэгдлийн зөвхөн нэг, гол биш талыг тусгаж байдгийг санавал зохино. Ийм учраас цаашдын шинжилгээнд тустай өгөгдөхүүнийг түүнээс ялган авахыг оролдъё.
Бүл бүр нь өөртөө маш олон янзийн генийн хэв маяг агуулж байдгийг тэмдэглэх нь нэн чухал юм. Янз бүрийн бүл дэх генийн хэв маягийн бөөгнөрөл янз бүр байдаг, гэхдээ бүлийн бүлэг тус бүр нь тухайн зүйлд тааралдах генийн бараг бүх хослолуудыг агуулж байдаг. 2.Четвериков С. С. О некоторых моментах эволюционного процесса с точки зрения современной генетики. М., 1926; Тимофеев-Ресовский Н. В. Микроэволюция //Ботанический журнал. 1958. № 3.
Гэхдээ тоогоор цөөн бүлүүд генийн аль нэг шинжээ алдаж, үүний улмаас тэдгээрийн өөрчлөгдөх чадварын хэмжээ багасаж, мөн дасан зохицох чадвар нь харгалзан буурдаг. Үүнийг дахин төрөх гэнэ. Бүлийн генетикийн зарчмын ёсоор ихэнхи бүлүүд хөдлөнги тэнцвэрийн төлөв байдалд оршиж, өөр хоорондоо хэмжээ, бүтэц, генетик бүтцээрээ ялгагдана. Тэнцвэр алдагдах нь тооны өөрчлөлт буюу хэт хэлбэлзэл, эсвэл “амьдралын давалгаа”, байгалийн шалгарал буюу тусгаар байдал алдагдах зэрэг хувьслын болон гэнэтийн өөрчлөлтийн хүчин зүйлүүдийн даралтаас болдог юм. Эдгээр өөрчлөлтийн үр дүнд эсвэл түрэмгийлэл үүснэ, эсвэл зарим тохиолдолд зүйлийн өөрчлөлтөд хүргэдэг мутаци (өөрчлөлт) болон флуктацид (хазайлт-Орч) хүргэнэ. 3. Энэ томъёоллыг С. С. Четвериков и Н. В. Тимофеев-Ресовский нар гаргасан.
Угсаатан нь зүйлийн дотор байдаг учраас түүнийг бүрдүүлэхэд маш өчүүхэн (амьтдын зүйлтэй харьцуулахад) өөрчлөлтийн даралт, харьцангуй их биш тусгаар байдал байгаа үед флуктацийн маш бага өөрчлөлт хэрэгтэй. Ийм учраас угсаатан нь зүйлийг бодоход хавьгүй олон үүсдэг ба ихээхэн бага хугацаанд оршин амьдардаг. Үүнийх нь ачаар энэхүү үйл явцыг түүхэнд тэмдэглэдэг.
МОНОМОРФИЗМ БУЮУ НЭГ ХЭЛБЭР
Угсаатны түүхийг ажиглахад хувиран өөрчлөгдөх тасралтгүй мэт харагдаж буй үйл явцад угсаатан аль нэгэн ландшафтдаа дасан зохицохуйн дээд хэмжээнд хүрэхтэй холбогдож тогтвортой байдлын үеүд байдгийг амархан ажиглаж болно. Энэ ажиглалт нь усны амьтан судлагч Ю.П.Алтухов, хүн судлаач Ю.Г.Рычагов нарын бүлийн генетикийн бааз дээр хийсэн дүгнэлттэй давхцаж, “зүйлийн хязгаар дотор дасан зохицох ач холбогдол бүхий удамших өөрчлөлтийн зүйл хоорондын түвшин дэх дасан зохицох чадвар” гэж заасан сэдвийг нэмэн баталж байна. Эндээс “жинхэнэ хөдөлгөөн нь тогтвортой байдалд хувиран өөрчлөгддөг” гэсэн дүгнэлт гарч байгаа бөгөөд үүнийг хязгааргүй удаан тусгаарлагдсан амьтад харуулж байна. 4. Алтухов Ю.П., Рычков Ю.Г. Генетический мономорфизм видов и его возможное биологическое значение //Журнал общей биологии. 1972. N! 3. С. 282.
Гэхдээ байнга үйлчлэхгүй ч гэсэн угтвар үйл явц ажиглагдахгүй бол амьтдын ертөнц дэх зүйл үүсэх үйл явц ч, үлдэцүүдийг шахан шинэ угсаатан үүсэх үйл явц ч боломжгүй байхсан билээ. Үүнд дээрх зохиогчид дараах хариулт өгч байна: “зүйлийн өвөрмөц чанарын өөрчлөлт нь тун ховор тохиолдолд шинэ зүйл төрнө гэсэн хэрэг. Гэхдээ үүнийг энгийн түвшинд тоглож буй мэт байнгын магадлалт үйл явц биш, харин тодорхой амьтны нөхөн үржихүйн тусгаарлалттай хавсарсан ганц л үйл явдал гэж үзэж байж л төсөөлж болно.“ Үр хөврөл тогтох нь эхийн хэвлийд гэрлийн хурдаар болдог шүү дээ. 5. Там же. С. 296.
Хэрэв бид энэ сэдвийг этнологид хэрэглэвэл энэ нь экцесс буюу журмаас гажих үзэл баримтлал болох бөгөөд өөрөөр хэлбэл хэт тогтворгүй орчны онцгой таатай нөхцөлд л бий болох чадвартай түлхэлтийн үр дүн болно. Үүнээс өөр тохиолдолд түлхэлтийн инерци унтрах бөгөөд харин “тодорхой амьтад” нь зүйл нэгтнийхээ гарт үрэгдэх болно. Энд угсаатан энэхүү төсгүй хүмүүсийг өөртөө багтаж байна уу, эсвэл тэрээр персистент буюу үлдэц тайван байдалд байна уу, аль эсвэл угсаатан бий болох урсгал дунд бүхий л янз бүрийн шатуудыг дамжин байна уу гэдэг нь ямар ч ялгаагүй юм. Энэ хоёр тохиолдолд тэр өөрийнхөө хэмжүүрээр гажиг, мангар гэж шударгаар нэрлэх тэр бүхнээ хөнөөх болно. Гэсэн хэдий ч шинэ угсаатнууд бий болсоор л байдаг. Энэ нь “тодорхой амьтныг” зөвхөн амьдрах биш, мөн ялахад хүргэдэг тийм нөхцлүүд байна гэсэн хэрэг болно. Энэ нь ландшафтын ч, зүгээр л хэлэхэд судлан буй хүний хөршүүдийн хоорондын харилцааны шинж чанар гэж ойлгодог угсаатны ч орчны нөхцөл юм. Хэрэв бидэнд нэг овогтон буюу хамтран амьдрагчдынхаа атаархал, тэнэг зан, хорсол заналын улмаас өөрийгөө илрүүлж амжаагүй эртний хүмүүсийн намтрыг судлан мөшгөх нь маш хүндрэлтэй байх ахул үүнээс хэдэн эрэмбээр дээгүүр байгаа систем, өөрөөр хэлбэл угсаатныг судалгаанд шилжиж угсаатны нийлэгжилтийн асаах агшин болох экцессийн буюу гажилтын үзэл баримтлалыг үндэслэхэд биднийг хүргэх, бидэнд зайлшгүй хэрэгтэй өгөгдөхүүнийг олж авч болох юм. Судлан буй системийн эрэмбэ хэчнээн дээш байх тутам алдааны хэмжээ болон зайлшгүй оролтууд төдий чинээ бага байна. Дээр үзсэн бүхнээс үзэхэд угсаатан нь цаг ямагт нийгмийн аль нэгэн бүрхүүлээр хучигдсан биофизикийн бодит байдал гэдэг нь тодорхой байна. Эндээс шинэ угсаатан үүсэхэд биологийн юмуу нийгмийн зүйлийн аль нь анхдагч вэ ? гэсэн маргаан нь өндгөнд уураг юмуу хальсны аль нь анхдагч вэ ? гэсэн маргаантай адил юм. Гэхдээ нэг нь нөгөөгүйгээр боломжгүй гэдэг нь ойлгомжтой учраас энэ сэдвээр цэцэрхэх нь маргалдах зүйлгүй болно.
Үнэн хэрэг дээрээ томоохон хамт олон–угсаатнуудын дотор төдийгүй, газрын ландшафтад шууд нөлөөлдөг, улмаар хийсвэр байдлаар бус, бүрэн бодитой оршигч, нэг хүний биеийн дотор ч материйн хөдөлгөөний бүх хэлбэрүүд байнга хавсран ажиглагддаг. 6. Калесник С. В. Общие географические закономерности Земли. М., 1970; Гумилев Л. Н. Место исторической географии в востоковедных исследованиях // Народы Азии и Африки. 1970. № 1.
Хэрэв тэр ч байтугай хүний зан үйлийн бүх хэсгийг түүний социал хүрээлэл хөдөлгөдөг гэж үзсэн ч гэсэн хүний үр хөврөлийн генетик код нь биологийн үзэгдэл, адреналин бага ялгарах нь химийн үзэгдэл болно. Гэхдээ аль аль нь хүний үйл ажиллагааны шинж чанарт нийгмийн хүчин зүйлийн зэрэгцээгээр маш их нөлөөлдөг.
Хүний байгал орчинтой харьцах тухай ярихдаа түүхийг үгүйсгэгч дурын өнгөц ажиглагч хялбаршуулах зарчимд үнэнч үлддэг. Хүн амын өсөлт, хөдөлмөрийн бүтээмж хурдан өсөхөд нөлөөлөгч таатай нөхцөлтэй, хүний нийгмийн дэвшил хурдан явагддаг газрууд, ийм зүйл байдаггүй, удаан явагддаг газрууд нь мэдээжийн юм шиг санагдана. Тэгвэл ямар нөхцлийг таатай гэж үзэх вэ ? Андалусийн цаг агаар Англи болон Кастилийнхаас зөөлөн, гэхдээ кастильчууд 1492 онд Гранадаг эзэлсэн. Харин Англи далайд 500 жил хаанчилсан. Норвегийн нөхцөл 2 мянган жил өөрчлөгдөөгүй, гэтэл викингүүд зөвхөн IX–XII зуунуудад далайн давалгаагаар тэнүүчилсэн. Норвеги зогсонги байдалд байсан, дараа нь Кальмарын унийн үеэс эхлэн данийн түрэмгийллийн золиос болсон. Яагаад ?
Энэ бүх дүгнэлтийг зөвхөн ганцхан зорилгоор л хийсэн бөгөөд угсаатны гяслхийх нь хөгжил буюу зогсолтод байгаа ард түмнүүдийн ахуй болон соёлтой холбоогүй, тэдний арьстны бүрэлдэхүүн ч биш, эдийн засаг болон техникийн түвшин ч биш, угсаатны экологийг өөрчлөгч цаг агаарыг хэлбэлзэл ч биш, харин орон зай, цаг хугацааны тодорхой нөхцөлтэй холбоотой байдгийг л үзүүлэх гэсэн юм. Ландшафт нь өөрөө шинэ угсаатныг төрүүлдэггүй, яагаад гэвэл тэдгээр нь байж байдаг, хаа нэгтээ, тэр ч бүү хэл хамгийн тохиромжтой газарт ч гэсэн бүтэн мянган жилээр үүсдэггүй. Угсаатны нийлэгжилтийн бүсүүд нь байнга өөрчлөгдөж байдаг. Яг энд л бидний сонирхож буй үйл явц эхэлж байгаа ба энэ нь түүнийг бидний аль хэдийн тооцож үзсэн тэрхүү газрын хүч биш, харин бидний олбол зохих ямар нэг өөр юм өдөөж байна.
ДЭВСГЭР БА ХҮЧИН ЗҮЙЛ
Угсаатныг бие даасан үзэгдлийнх нь хувьд ландшафттай харилцах харилцааны шинжилгээ: тэдгээр нь харилцан холбоотой, гэхдээ угсаатан ч ландшафт өөрчлөгч байнгын хүчин зүйл болдоггүй, ландшафт ч гаднын үйлчлэлгүйгээр угсаатны нийлэгжилтийн шалтгаан болж чаддаггүйг нотолж байна. Угсаатны болон нийгмийн зүй тогтлын харьцаа эсрэг холбоог үгүйсгэж байна. Учир нь Дэлхийн угсаатны хүрээ нийгмийн хөгжлийн хувьд зөвхөн дэвсгэр (фон) болж, хүчин зүйл болдоггүй байна.
Зөвхөн үйлдвэрлэх хүчин, техник төдийгүй, бас шинжлэх ухаан, урлаг зэрэг салбарууд өөрийн зүй тогтолтой байдаг аж. Эдгээрийн эх нь маш эртнээс улбаатай бөгөөд тэдгээрийн уламжлагдах шинж нь манай өнөөдрийг хүрч, Дэлхий дээр хүн байсан цагт тасалддаггүй юм байна.
Тэр ч байтугай уламжлалын эсрэг эсэргүүцэл үүслээ ч гэсэн энэхүү эсэргүүцэл нь мөн л уламжлалт шинжтэй байна, яагаад гэвэл эртний эрин үеүд нь шинэ, өвөрмөц зүйлүүд хайх үеүдээр солигдож байжээ. Гэхдээ хэрэв энэ зарчим тогтвортой байвал урлагийн бүтээл, шинжлэх ухааны нээлт, шинийг бүтээх зэрэг тусгай туурвил бүхэн нь яг нарийн хуулбарыг нь хийх боломжгүй шинэ зүйл болно. Соёлын хөгжлийн мөнх байх зарчмыг хүлээн зөвшөөрөх нь угсаатны үйлдлийг шалтгаацуулж байж магадгүй бус уу ? Үгүй юм.
Польшийн зохиолч, гүн ухаантан С.Лем тусгай бүтээлдээ: “Соёл нь физик, биологи, социал болон техник–эдийн засгийн шинж чанартай хүчин зүйлүүдээр тодорхойлогдоно. Харин эдгээр бүх шалтгаацуулагчийг хэмжээгээр илэрхийлбэл энэ нь “цэвэр соёлын хувилбарууд”-ын орон зай тэгтэй тэнцүү болох уу ? эсвэл ямар гэсэн эрх чөлөөний ямар нэгэн зурвас үлдэх үү ? Антропологийн харьцуулсан шинжлэх ухаан (компаративистика) ийм орон зай оршин байдаг, үүн дотор хэлбэр болон утга санааны цэвэр соёлын өөрчлөгдөх байдал нэгэнт явагдаж байдаг” гэж бичсэн байна. 7. Лем С. Модель культуры //Вопросы философии. 1969. № 8. С. 51.
Гэхдээ “эрх чөлөөний зурвасыг” юу дүүргэдэг вэ ? Шийдвэр гаргах эрх болон боломж бүхий хүний үйлдэл болох нь илт байна. Ийм байдаг байж, гэхдээ л физик утгаараа ажил болох энэхүү үйлдэлд өөрт нь тухайн хүний сэтгэл зүй, физиологоор дамжин хугарсан эрчим хүч хэрэгтэй. Ингээд нийгмийн болон биологийн талуудыг мөнгөний сүлд болон тооны талуудтай адилтган үзвэл энэхүү эрчим хүч, түүний илрэл нь эдгээрийн аль алийг нь цутгасан тэрхүү металл болох юм.
Бид шууд ажиглах үед тэдгээрийн дүүргэлтээр (жишээлбэл, мөнгийг цутгасан хайлшуудын элементүүдийн процентийн харьцаа) халхлагдсан мөн чанарыг мэдээж тусгаагүй ил байх дүрснүүд л харагдана. Гүний үзэгдлийг танин мэдэхүйд ч гэсэн зөвхөн “эмпирик нэгтгэн дүгнэх” замаар л хүрч болно. Ийм учраас сүлд, тооны аль нь чухал вэ гэсэн маргаан (манай тохиолдолд “нэгдмэл газар зүй” юу, бүхнийг хамарсан социологи уу) утгагүй юм. Түүнээс гадна энэ нь нухацтай зүйл биш бөгөөд энэхүү хоёр тохиолдолд эрдэмтдийн өмнө тавигдсан зорилтыг хялбаршуулах ухамсаргүй эрмэлзэл бий болж, өөрөөр хэлбэл судалгаа нь өөрөө утга санаагаа алдаж, үр дүн нь анхнаасаа бүрэн биш, ингэхлээр буруу болдог зарим профанаци буюу гажилт үүсдэг. Гэхдээ энэ нь шинжилгээний ойлголтод хүрэхэд зайлшгүй юм. Ийм учраас бид хүчээр тааруулж урьдчилан авсан үзэл санааг үзэгдлийн бүрдэл хэсгүүдээс ялган хаяж, харин тэдгээрийн тус бүрийн байр суурь, үүргийг тайлбарлах замаар явах нь эцсийн эцэст судалгааны маань зорилгод буюу синтезэд хүргэх юм, ингэсний дараа нийгэм, биологи, газар зүйн хандлагуудын хоорондын зөрчил хуурамч гэдэг нь тодорхой болно.
Маш энгийн хувилбар болох нэг хүнийг авч үзье. Энэ хүний анатоми, физиологи, сэтгэл зүй нь бие биенээсээ хамаарч, нягт сүлжилдсэн байдаг, хүний ахуйн эдгээр талуудыг ялган салгаж шинжлэх шаардлага бидэнд байхгүй юм. Хүн нийгмийн амьтан болох нь тодорхой бөгөөд тэр хувь хүн түүний өвөг дээдсийн гараар бүтээгдсэн (техник) эд юмс болон бусад хүмүүстэй шууд харилцан байж төлөвшдөг. Тэгвэл сперматозоид яахав? Энэ “хүн” бол сээр нуруутны хувьслын хуулиар хөгждөг цэвэр биологийн зүйл юм. Гэсэн ч гэсэн хувь хүн өөрийнхөө үр хөврөлтэй холбоотой нь эргэлзээгүй, улмаар дээд мэдрэлийн үйл ажиллагааг (психик) оролцуулаад хүний бие өөрөө материйн хөдөлгөөний нийгмийн болон байгалийн хэлбэрүүд хослон байдаг лаборатори болж байдаг.
Хөврөлийн үеэсээ гарч, нийгмийн орчинд бүрэн орлоо ч гэсэн хүн байгалийн зарим зүй тогтолд захирагддаг. Бэлгийн бойжилт болон хөгшрөх үеүд нь нийгмийн хөгжлийн шатнаас үл хамаарна, харин зүйлийн дотоод хувьслын үйл явцад боловсрогдсон удамшлын шинж тэмдгүүдээс хамаардаг. Жишээлбэл: халуун бүсийн ард түмнүүдийн бэлгийн бойжил нь хойд зүгийнхнээс эрт болдог, негржүү хүний хариу үйлдлийн хурд нь европжуу болон монголжуу хүмүүсийнхээс өндөр байдаг, зарим өвчнийг жишээлбэл, улаан бурхныг эсэргүүцэх чадвар полинезчуудад европынхноос бага байдаг гэх мэт. Эдгээр онцлогууд нь нийгмийн хөгжилд ямар ч хамаагүй, гэхдээ янз бүрийн ард түмний хүмүүсийн зан үйлд ул мөрөө үлдээдэг. Эдгээр ялгааны үүсэл нь маргаангүй бөгөөд өнгөрсөн болон одоо амьдарч буй угсаатнуудын үүсэл, газар зүйн янз бүрийн нөхцөл дэх аль нэгэн бүлийн өвөг дээдэс дасан зохицохтой холбоотой юм. Чухамхүү дасан зохицохуйн урт удаан үйл явцын үр дүнд үүссэн шинж тэмдгүүдийн хуримтлал л хүн төрөлхтөн хөгжлийн адил үе шат – нийгэм–эдийн засгийн формацийг дамжих үед угсаатны олон янз байдлыг бий болгодог байна. Гэхдээ нийгмийн хэлбэрүүдээр л угсаатны нийлэгжилтийн асуудлын нарийн нийлмэл байдал шавхагдахгүй. Хэрэв ийм байвал угсаатны зүй нь социологийн жирийн нэг салбар байж, нэг формацийн нийгэмд, тухайлбал боол эзэмшлийн нийгэмд амьдрагч хүмүүсийн зан үйл адилхан байх байсан. Гэвч хятадын эртний үе нь эллинээс төдийгүй, мөн япон, энэтхэг, египетийнхээс ялгаатай байдаг. Нийгмийн адил төстэй байдал нь угсаатны өвөрмөц байдлыг устгадаггүй байна.
КОМПЛИМЕНТАР БУЮУ НАЙРСАГ ШИНЖ
Тэгвэл угсаатныг биологийн хэмжигдэхүүн болдгийг зөвшөөрсөн үзэл санаа байж болохгүй юм уу ? Үгүй, энэ ч гэсэн шийдвэр биш, учир нь угсаатны үйл явцууд нь зүйлийн моноформ буюу нэг хэвийн нөхцөлд явагдана.
Тийм байлаа ч гэсэн хүний биологийн зарим онцлогууд нь тодорхой үүрэг гүйцэтгэдэг нь илт юм. Ухаандаа угсаатны нийлэгжилтийг даяар үзэгдэл, ердөө л ерөнхий хувьслын тухайлсан тохиолдол гэж үзлээ гэхэд энэхүү “тухайлсан” гэдэг нь маш чухал бөгөөд холимог гарал үүсэлтэй, соёлын янз бүрийн түвшинтэй, янз бүрийн онцлогтой амьтдаас (хүнээс) угсаатны бүхэллэг анхлан үүсэх асуудлыг дэвшүүлэн тавьж байна. Иймээс бид: тэднийг бие бие рүү нь юу ингэж татдаг вэ ? гэж асуух эрхтэй юм. Энд ухамсартай тооцоо, ашиг тусыг эрхэмлэх зарчим байхгүй байгаа нь илт байна, учир нь анхны үеийнхэн хэвшиж тогтсон харилцааг эвдэж оронд нь хэрэгцээндээ нийцүүлсэн шинэ харилцаа тогтоохын зайлшгүйтэй холбоотой асар их бэрхшээлтэй тулгардаг. Энэ нь ямагт эрсдэлтэй ажил байдаг учраас үүнийг эхлэгч нь ялалтынхаа үр шимийг бараг хүртдэггүй. Түүнчлэн нийгмийн ойртолтын зарчим энд бас л тохирохгүй, учир нь шинэ угсаатан хуучны институтыг устгадаг. Улмаар хүн шинэ угсаатанд яг бүрэлдэх агшинд нь орохын тулд дасаж сурсан хуучин зүйлээсээ бүрэн салах ёстой. Бидний сайн мэдэх Долоон толгод дээрх квирит чонын овгууд римчүүд болсон, Шотланд, Исланд, Норманд болон Английг эзлэн суусан викингуудын нөхөрлөл, түүнчлэн XIII зуунд феодалуудаа зайлуулсан монголчууд зэрэг нь чухамхүү ингэж төрсөн юм. Амьтны ухамсаргүйгээр харилцан найрсахтай холбогдсон комплиментар буюу найрсах шинж гэсэн өөр зарчмыг хэрэглэх нь тохиромжтой. Энэ зарчим дээр хайр дурлалаараа гэр бүл болдог, гэхдээ найрсах шинжийг зөвхөн сексийн хүрээгээр хязгаарлаж болохгүй бөгөөд энэ нь дээрх зарчим илрэх хувилбар төдий юм. Угсаатны үр хөврөл–анхдагч хамт олон бүрдэхэд тодорхой зан араншинтай хүмүүс бие биедээ ухамсаргүйгээр татагдах явдал гол үүрэг гүйцэтгэдэг. Ингэж татагдах нь үргэлж байдаг, гэхдээ тэр нь хүчтэй болбол угсаатны уламжлал үүсэх зайлшгүй урьдач нөхцөл бий болдог. Харин үүний дараа нийгмийн институт үүсдэг.
Ингээд угсаатны дурын уламжлал, түүнтэй хавсарсан нийгмийн институт төрөхийн өмнө бие биедээ найрсаг хандах зарим тооны хүмүүсийн нэгдэл болсон үр хөврөл бий болдог. Тэд үйл ажиллагаагаа эхэлмэгцээ сонгон авсан зорилго, түүхэн хувь заяагаараа гагнагдсан түүхэн үйл явцад оролцож эхэлдэг. Юу ч болсон гэсэн тэдний хувь заяа нэгдэж, “түүнгүйгээр байж болохгүй нөхцөл” бий болдог. Ийм бүлэг нь флибустьеруудын дээрмийн бүлэг, мормоны шашны сект, тамплиерийн одон (феодалын үеийн цэргийн бүлэглэл –Орч ), буддын лам нарын нийгэмлэг, импрессионистуудын сургууль гэх мэт байж болно, гэхдээ энд хаалтнаас гаргаж болох нийтлэг байгаа зүйл нь гэвэл өөр хоорондоо маргаан явуулахын тулд байсан ч гэсэн энэ бол ухамсаргүйгээр харилцан татагдах явдал юм. Ийм учраас бид ийм “үр хөврөлийн” нэгдлийг дээр консорци буюу түүхэн хувь заяагаар нэгдсэн гэж нэрлэсэн. Эдгээр нь бүгдээрээ амьдарч чаддаггүй: ихэнхи нь үндэслэгчийнхээ амьд ахуйд задардаг, харин тэсэж үлдсэн тэр хэсэг нь нийгмийн түүхэнд орж, голдуу уламжлал бүтээдэг нийгмийн хэлбэрт шууд л орж өсөн дэвждэг. Гаднын цохилтод хувь заяа нь тасалдаагүй үлдсэн тэр цөөн хэсэг нь өндөр идэвхээ жам ёсоор алдах хүртлээ амьдардаг бөгөөд бие биендээ найрсаг байх инерци нь нийтлэг заншил, ертөнцийн мэдрэмж, сонирхол гэх мэтээр илэрхийлэгдэж байдаг. Найрсан нэгдэх энэхүү шатыг бид конвиксия буюу ахуй байдлаараа нэгдсэн гэж нэрлэдэг. Энэ нь нэгэнт хүрээлэлдээ үйлчлэх хүчгүй болж, социологийн биш, эдгээр бүлгийг ахуй нь нэгтгэдэг учраас угсаатны судлаачдын мэдлийн асуудал болдог. Таатай нөхцөлд бол конвикси нь тогтвортой, гэхдээ тэдний орчныг эсэргүүцэх чадвар тэг рүү тэмүүлж байдгаас тэдгээр нь хүрээлэн буй консорцийн дунд тархан алга болдог.
Найрсаг шинжийн зарчим угсаатны түвшинд ч нэн бодитойгоор үйлчилдэг. Энд түүнийг эх оронч үзэл гэж нэрлэх бөгөөд түүхийн мэдэлд байдаг. Учир нь өвөг дээдсийг нь хүндлэлгүйгээр ард түмнийг хайрлаж болдоггүй юм. Угсаатны дотоод дахь найрсаг шинж нь хүчирхэг хамгаалах хүч болж угсаатанд ямагт ашигтай байдаг. Гэхдээ заримдаа тэр нь харийн бүх юмыг үзэн ядах гаж, сөрөг хэлбэртэй болдог бөгөөд энэ үед нь түүнийг шовинизм гэж нэрлэдэг.
Хэт угсаатны түвшинд бол найрсаг шинж нь зөвхөн оюун дүгнэлт байж болно. Энэ нь их зангаар илрэх бөгөөд өөрсөдтэйгээ адилгүй бүх хүнийг, харийн бүх хүнийг “зэрлэгүүд” гэж нэрлэдэг.
Найрсаг шинжийн зарчим нь нийгмийн үзэгдлийн тоонд орохгүй. Тэр нь зэрлэг амьтдад ч ажиглагддаг бөгөөд харин гэрийн амьтдад бол эерэг ( морь, нохой эзэндээ үнэнч байх ) болон сөрөг хэлбэртэй байдаг.
Энэ зарчим нь тэргүүлэх үүргээ хамтын ахуйн нийгмийн хэлбэр байхгүй үед л гүйцэтгэж, харин нийгмийн тогтвортой олдцын үед захирагдмал байдлаа хадгалж байгааг бид харж байна. Энэхүү нөхцөл байдал нь биднийг аз болоход хангалттай сайн боловсруулагдсан хүний биологид хандахад хүргэж байна.
СУДАЛГААНЫ БИОЛОГИЙН ШУГАМУУД
Нэр томъёогоо тохиръё. Биологийн шинжлэх ухаануудад анатоми, генетик зэргээс гадна хүрээлэлтэй холбогдон организм бий болохыг судладаг рефлексологи, экологи, биоценологи болон этологи (зан араншингийн тухай) зэрэг шинжлэх ухаанууд багтдаг. Бид организмын үйл ажиллагаатай холбогдоогүй бүх зүйлийг мөн чанараараа нийгмийн ухаан гэж үзэж байна. Хүнээс гадна зэрлэг амьтад, шувуу зэрэг нь үр төлөө хүмүүжүүлж, сургаж байдаг. Бүхий л сүргийн амьтад сүрэг дотроо бэлгийн харилцаагаа зохицуулдаг, дайснаас хамгаалах үед нас хүйсний хуваарьтай, дохиоллын системтэй байна. Эрэгчин нь эмэгчин болон үр төлөө хамгаална. Энэ төрлийн харилцааг материйн хөдөлгөөний нийгмийн хэлбэр гэдэг утгаар нь нийгмийн харилцаа гэж нэрлэж болно гэж үү ? Зөвлөлтийн шинжлэх ухаанд хүлээн зөвшөөрсөн үг хэрэглэх ёсонд ийм зүйл байхгүй. Учир нь нийгмийн хөгжил нь эдийн засаг дээр суурилж, үйлдвэрлэх хүчний хөгжилтэй холбоотой байдаг. Ийм маягаар нийгмийн харилцаа нь ямагт аль нэгэн формацитай хавсарч байдаг. Энэ бол зөвлөлтийн шинжлэх ухаанд хүлээн зөвшөөрсөн нэр томъёо бөгөөд түүнийг өөрчлөх нь өөрийгөө болон уншигчдыг төөрөгдүүлж буй хэрэг болно. Гэхдээ зүйлийн ахуйчлалын хамтын хэлбэр нь бидний алс холын өвөг дээдэст хэвшмэл байсан юм. Хүн нийгмийн амьтан болох хүртлээ сүргийн амьтан байсан бөгөөд энэ нь хүний давуу талыг огтхон ч үгүйсгэхгүй.
Амьтны физиологид хамт олны нөлөөллийг өнөөдөр хангалттай судалсан. Хэрэв түүнийг уурлуулах юм бол хулганад ч гэсэн даралт ихдэх өвчин үүсгэж болно, гэхдээ хулгануудаас бидний нэгэнт хүлээн зөвшөөрсөн үгийн утгаар нийгмийн гэж нэрлэж болох лаборант болон туршигч гарна гэж үү ? Хилийн чанадад биологийн хуулиудыг нийгмийн амьдралд дэлгэрүүлдэг “социал–дарвинизм” хэмээх урсгал үнэхээр байдаг юм, гэхдээ дасаагүй, хэрэггүй нэр томъёоноос зайлсхийх нь сайн бөгөөд энэ нь манай уншигчдад ач холбогдолгүй билээ.
Манай ажилд биологи юугаар тус болох вэ? Нэгдэн амьдрах хамтын хэлбэр шоргоолжны үүр, туурайтны сүрэг, бүлэг гэх мэтээр газрын амьтдын олонхи зүйлүүдэд тархсан байдаг, гэхдээ зүйл бүр хамт олон бүрдүүлэх өөрийн гэсэн шинж чанартай байна. Харин Homo sapiens зүйлийн хувьд ийм хэлбэр нь угсаатан болно, гэхдээ энэ нь ямар ч тохиолдолд тэрээр шоргоолж буюу сүрэгтэй адил байна гэсэн хэрэг биш . Бусад сээр нуруутнаас хүн хэрхэн ялгагддаг вэ гэвэл тэр нь эрс ялгаатай байдаг. Иймээс угсаатан нь бусад амьтдын хамт олонтой адилгүй байдаг.
Амьтны хамтлаг, угсаатнуудын хоорондын ялгаа маш олон, гэхдээ шинжилгээний зорилгын тулд соёлын уламжлалын үүргийн асуудлыг боловсруулахад бидэнд хэрэгтэй энгийн бүдүүвчээр хязгаарлая. Нэг өвгөөс гаралтай, хатуу тогтоосон газар нутаг дээр амьдардаг, ахуй, заншил, шашин, хийдэг ажлынхаа төрлөөр хөршүүдээсээ нарийн ялгагддаг нэгэн овгийн төсөөлөөд үзье. Энэ тохиолдолд гэрлэлтийг голдуу тухайн угсаатны төлөөлөгчдийн хооронд хийнэ, учир нь гэр бүлээ хангалуун тэжээхэд шаардлагатай ахуй болон хөдөлмөрийн дадлагагүй хүнийг хамт олонд хүлэээн авах нь зохимжгүй байдаг. Өөр нөхцөлтэй холбогдсон өөр дадал туршлагыг энд албаар хэрэглээгүй. Ингэж тусгаарлагдсан угсаатны соёлын дүр төрх нь гаднын хүчний (дайн) хүчтэй оролцоогүйгээр харьцангуй тогтвортой байна. Учир нь шинэ үе болгон нь өмнөх үеийнхнийхээ амьдралын орчлыг давтахыг эрмэлзэх бөгөөд энэ нь тухайн угсаатны соёлын уламжлал болно.
Уламжлалыг ямар ч тохиолдолд биологид хамааруулж болмооргүй мэт санагдана, гэтэл үе хоорондын харилцамжийн механизмыг профессор М.Е.Лобашев чухамхүү амьтдыг судлах үндсэн дээр нээсэн бөгөөд тэрээр зүгээр л удамшлын өөр нэр болох “дохиоллын удамшлын” үйл явцыг нээжээ. 8. Лобашев М. Е. Сигнальная наследственность //Исследования по генетике / Под ред. М. Б.Лобашева. Л., 1961.
Амьтдын ертөнцөд хувиараа дасан зохицохуй нь уг амьтанд өөрийгөө хамгаалах, амьдрах зохистой нөхцлийг идэвхитэйгээр сонгохыг нь хангаж өгдөг нөхцөлт рефлексийн механизмын тусламжтайгаар явагддаг байна. Эдгээр нөхцөлт рефлексүүд нь эцэг эхээс үр төлд нь, мөн сүргийн ахмад гишүүдээс багачуудад дасан зохицохуйн дээд хэлбэр болсон зан үйлийн тогтсон үзлийн ачаар дамждаг аж. Хүнд бол энэ үзэгдлийг “дохионы дохио” болох хэлээр хангаж байдаг соёл иргэншлийн залгамжлагдах шинж гэж нэрлэдэг. Энэхүү залгамжлагдах шинжид хамгийн шилдгээр дасан зохицохуйг хангагч, дохиоллын удамшлын замаар дамжиж байдаг ахуйн дадал, сэтгэлгээний арга барил, урлагийн зүйлсийг хүлээн авах чадвар, ахмадууддаа харьцах, гол хүмүүсийн харьцаа зэрэг ордог. Эндогами буюу зөвхөн овгоосоо гэрлэхтэй, өөрөөр хэлбэл генийн сангийн бүрэлдэхүүнийг тогтворжуулагч хөршөөс нь тусгаарлахтай хослуулахад уламжлал нь угсаатны хамт олны тогтвортой байдлыг бүтээгч хүчин зүйл болж байна.
Эцэст нь Homo sapiens зүйлийн бүлийг биологийн цаг хугацаанд нь, өөрөөр хэлбэл үе солигдохтой холбон авч үздэг антропогенетика болон антропологийн шинжлэх ухаанууд багагүй ач холбогдолтой юм. Угсаатны амьдрал бол биологийн цаг хугацааг түүхэн цаг хугацаанд тохоож буй явдал, үе солигдох нь шалтгаант дэс дараалалд буй үйл явдлын хэлхээсүүд болно. Ингэж тохоох нь генетик ой түүхэн уламжлалтай хослосны ачаар шалтгаант (каузаль) зүй тогтлын тасралтгүйгээр хэрэгждэг, үүний улмаас угсаатан нь бүхэллэг байдлаар оршин байдаг.
Гэхдээ бидний зүйрлэл байдлаар “Х хүчин зүйл” гэж нэрлэсэн шинж тэмдэг бүл үүсэхэд бүр ч илүү чухал юм. Чухам үүний үрээр хожим нь унтардаг угсаатны нийлэгжилтийн үйл явц эхэлдэг юм. Энэ шинж тэмдгийг илрүүлбэл бид тавьсан асуудлаа шийдвэрлэх бөгөөд гэхдээ түүнийг олох нь түвэгтэй, хайхад дэс дараа хэрэгтэй.
XIX. Хэлний нийлэгжилт үү, угсаатны нийлэгжилт үү ?
ДЭВШИЛ БА ХҮНИЙ ХУВЬСАЛ
Нийтээр зөвшөөрсөн хувьслын онолын ёсоор Homo төрөл нь дөрөвдөгч галавын эхэнд магадгүй нэг нэгнээ дараалсан, эсвэл зэрэгцэн оршиж байсан байж ч магадгүй янз бүрийн хэлбэрийн хэд хэдэн гоминид буюу хүн төст мичнээс бий болжээ. Таамаглаж буй өвөг дээдэс австралопитектэйгээ адилаар гоминидууд нь өөрсдийгөө идэх нь харш биш, том махчин амьтан байж, улмаар биоценоз дахь дээд экологийн хөндийг (орон зай-Орч) эзэлж байжээ. Сүүлчийн мөстлөгийн төгсгөлд энэхүү төрлийн бүх мөчрүүд мөхөж, зөвхөн орь ганц зүйл Ноmо sapiens буюу орчин үеийн хүн үлджээ. Гэхдээ энэхүү зүйл гаригийн бүх хуурай газраар тархаж, дараа нь түүхэн үед усан хүрээний гадаргууг эзэмшиж, эдүгээ Дэлхийн ландшафтын бүх бүрхүүлийг антропогенийн хэмээн шударгаар нэрлэж буй тийм их өөрчлөлтийг Дэлхий дээр хийсэн байна. Туйлын мөсийг эс тооцвол чулуун болон төмрийн зуунуудын археологийн дурсгал байхгүй тийм газар байхгүй юм. Бид чулуун зэвсгийн сууринг өнөөгийн цөлөөс, ширэнгээс, шинэ зэвсгийнхнийг орчин үеийн тундр, тайгаас олсоор байна. Энэ нь бүс нутгуудад эрт үед хүмүүс оршин сууж, дараа нь орхин явж, өнөөдөр бид машин, техник ашиглан дахин эзэмшиж байгааг харуулж байна. Мэдээжийн хэрэг өнгөрсөн 17–20 мянган жилд янз бүрийн нутгуудын цаг агаарын нөхцөл өөрчлөгдсөн, гэхдээ Ноmо sapiens зүйл бусад сээр нуруутны зүйлүүдээс ялгаатай нь тодорхой талбараар хязгаарлагдахгүй, байгалийн янз бүрийн нөхцөлд дасан зохицож чадсан баримт нь түүнийг сээр нуруутны экологид онцгой байранд зүй ёсоор тавихад хүргэж байна.
XIX болон ХХ зууны эхээр техникийн ололт амжилт нь байгалийн баялгийн нөөцийг хөнөөлтэйгээр устгахад хүргэж байна, энэ нь дэвшлийн замаар явсан мэт санагдана. Одоо л гэхэд аж үйлдвэрийн хэрэгцээнд цэвэр ус нэгэнт хүрэхгүй байна, ургамлын аймаг доройтож, АНУ дахь тоосон шуурга нь прерийн биоценозийг устгасны төлөө өшөөгөө авч байна, томоохон хотуудын агаар хүчилтөрөгчөөр дутагдаж, сүүлийн 300 жилд Дэлхийн өврөөс 110 зүйлийн сээр нуруутан арчигдан алга болж, ахиад 600 зүйл сүйрлийн ирмэгт байна. Дөнгөж саявтар энэ үйл явцыг ноо хүрээ, байгалийг ялсан ялалт гэж нэрлэж байв. Гэвч бид огт өөр эрэмбийн (нийгмийн биш) үзэгдэл: Дэлхий гаригийн био хүрээний бүрдлүүдийн нэг Ноmо sapiens зүйлийн хэт дасан зохицох чадвар, харгис шинж зэргийг харж байгаа нь одоо нэгэнт тодорхой байна.
Энд анхны асуулт гарч байна. Бидний тэмдэглэсэн үзэгдлүүдийн аль нь сээр нуруутны хувьслын хүрээнд, аль нь Ноmо sapiens–д өөрт ногдох бол? Үүнээс дутахааргүй хоёр дахь асуулт бол: Хүн зэвсэг бүтээж, гал гаргаж сурсныхаа дараа биоценозын бүрэлдэхүүнд дээд, төгсгөлийн мөчир байдлаа үргэлжлүүлэх болов уу? Аль эсвэл тэрээр байгальтай харилцах ямар нэг өөр хүрээнд шилжиж, түүндээ гэршүүлсэн амьтад, тэжээмэл ургамлаа татан оруулах болов уу? Хувьслын буцалтгүйн тухай хуулиар хүний үйлчлэлээр танигдахын аргагүй болтлоо өөрчлөгдсөн амьтан, ургамал бие даасан амьдралдаа эргэн орж чадахгүй, цөөн тохиолдлыг эс тооцвол тэд зэрлэг хэлбэрүүдтэй хийх өрсөлдөөнийг тэсэн гарч чадахгүй нь бүр ч илүү мөн чанарын шинжтэй юм. 9. Быстрое А. П. Прошлое, настоящее, будущее человека. Л., 1957. С. 300.
Ийм маягаар био хүрээний дотоодод онцгой давхарга үүсчээ. Энд байгалийн шалгарлын зарчим үйлчлэх болов уу ?
Ч.Дарвинийг оролцуулаад хувьслын онолын талынхны олонхи нь орчин үеийн хүн өмнө түүний өвөг дээдэст үйлчилж байсан тийм байгалийн шалгаралд өртөх нь үргэлжилж байна гэж үздэг.
10. Энд бид дарвиний эсрэг концепцуудыг (А.Додерлейны инерцийн хүч, Т.Эймерийн ортогенез, А.С.Бергийн номогенез, Х.Осборны аристогенез гэх мэт) зэргийг авч үзээгүй, учир нь ерөнхий шинж чанар бүхий байгалийн зүй тогтлыг харьцангуй хэсэгчилсэн шинжтэй нийт фаунд механикаар шилжүүлэн үзэх тохиолдолд нь зүгээр л эдгээрийн хэмжээ дамжааг (масштаб) жишиж огт болохгүйгээс алдаанд хүргэх юм. Хувьслыг бүхэлд нь судлах үед ач холбогдолтой байсан зүйлс хугацааны хязгаарлагдмал хэрчим дэх нэг зүйлийг судлах үед эсвэл маш чухал болж болно, эсвэл тухайн зүйлд огт ач холбогдолгүй байж болно.Манай тохиолдолд сүүлийн 5 мянган жилд байгаа хүн төрөлхтөнд энэ нь ач холбогдолгүй юм.
Өөр зарим нь үүнд эргэлзэж “Оршихуйн төлөө тэмцэл аажмаар суларсан нь биоценозын бүрэлдэхүүнээс хүнийг гарцаагүй гарахад хүргэсэн. Энэхүү урт удаан үргэлжилсэн үйл явц нь хүний хувьд байгалийн шалгарал эхлээд суларч, дараа нь бүрмөсөн зогсоход хүргэсэн…Гэхдээ байгалийн шалгарал байхгүй болсон нь хувьслын хүчин зүйлүүдийн нэгнийх нь үйчлэлийг зогсоосонтой адил …ингээд хүний биологийн хувьсал зогсох ёстой байсан. Энэ нь кроманьон хүн бүрэлдсэн 50 мянга орчим жилийн өмнө болжээ” гэсэн үндэслэл гаргадаг. 11. Быстрое А. П. Указ. соч. С. 299. – Хронология А. П. Быстрова требует уточнения. По новым данным С14 кроманьонский человек в Европе имеет давность около 20 тыс. лет, a Homo sapiens в Северной Америке – около 37 тыс. лет (см.: Мочанов Ю. А. К вопросу о начальных эпохах заселения Нового Света // Доклады по этнографии ВГО. Вып. 4. Л., 1966. С. 34).
Я.Я.Рогинский, М.Г. Левин нар 1955 онд биологийн хувьслын үйл явц нь орчин үеийн хүний дүр төрхийн хувьсал унтрахад хүргэсэн зүйлийн тийм шинж чанарыг эзэмшигч болгон бүтээжээ гэж бичиж байжээ. 12. Рогинский Я. Я., Левин М. Г. Основы антропологии.
Эндээс хүний хувьсах хөгжил аль эрт зогссон гэдэгт эргэлзэхгүй байж болмоор мэт санагдана. Гэхдээ зүйлийн дотоод дахь янз бүр байдал үргэлжилж, энэхүү судлах зүйл, асуудлыг ингэж тавих явдал шавхагдаагүй байна. Гэхдээ судалгааг үргэлжлүүлэхийн тулд шинэ аспект, шинэ арга зүй хэрэгтэй байна, учир нь зөвхөн үзэгдлийнхээ онцлогийг дүрслээд авбал өөр аль нэгэн үзэл бодолд нийлж болох юм. 13. История полемики до 1957 г. см.: Быстрое А. П. Указ. соч. С. 277.
БҮС НУТГИЙН МУТАЦИ БУЮУ ӨӨРЧЛӨЛТ
А.П.Быстровын нэг сэдэвт бүтээл гарснаас дөрвөн жилийн дараа Г.Ф.Дебец шуугиан дэгдээсэн дүгнэлт бүхий бүтээл хэвлүүлсэн юм. Эрт дээр үед нүсэр байсан хүний гавлын яс нимгэрч (грацилизаци), чингэхдээ аажмаар биш, гэнэт гэнэтээр, даяар хэмжээнд биш, харин өргөргийн бүсүүдээр ингэж өөрчлөгдөж байжээ. 14. Дебец Г. Ф. О некоторых направлениях изменений в строении человека современного вида //Советская этнография. 1961. № 2. С. 9-23.
Ингээд субтропикийн бүсэд гавлын ясны нимгэрэл НТӨ YI мянган жилд, харин зөөлөн уур амьсгалтай ойн бүсэд НТӨ I мянган жилд болсон байна. Энэ он цагийг Г.Ф.Дебец ангийн аж ахуйгаас газар тариаланд шилжсэн он цагтай харьцуулж үзээд “газар тариаланд шилжсэн нь тархины бүтцийн өөрчлөлтөд хүргэсэн гэсэн үнэлэлт боломжтой юм” гэж заасан байна. Дашрамд дурдахад өөрчлөгдсөн хүн шинэ хэрэг ажил олж авсан нь адил хэмжээгээр боломжтой юм. Гэхдээ түүний: “харьцуулсан анатоми ч, угсаатны судлал ч гэсэн Homo sapiens зүйлийн хүрээнд тархины нимгэрэх (грациль) хэлбэр нь илүү төгс гэж үзэх эрхийг бидэнд олгохгүй байна” гэсэн дүгнэлт нь бас л бүрэн шударга болсон юм. Тун зүйтэй. Гэхдээ нэг шинж тэмдгийн өөрчлөлт нь хүний анатоми төдийгүй, мөн түүний зан үйлд нөлөөлдөг нь нэн тодорхой юм. Г.Ф.Дебец “биологийн мөн чанар бүхий өөрчлөлтийн тухайд ажил хэрэг явагдаж байна” гэсэн гаргалгаанд хүрчээ. Ингээд хүний хамтын нийгэмлэг дэх түүхэн ахуйн нөхцөлд араг ясны өөрчлөлтийг хүртэл тэтгэдэг биологийн үйл явц явагдах нь үргэлжилж байна. Хэрэв ийм бол физиологи болон зан үйлд тусгагдаж байдаг бага хэлбэлзэлтэй хувилбарууд байх ёстой юм. Тэдгээрийг нээх нь хүндрэлтэй, гэхдээ тэдний байгаа тухай таамаглал нь одоо ч хүчинтэй байгаа нь биднийг илтэд тодорхой нийгмийн хүчин зүйлийн зэрэгцээ үйлчилж буй хүний үйл ажиллагааны хүчин зүйлийг эрж эхлэхийг шаардаж байна. Магадгүй энэ нь нийгмийн эхлэлийн нөлөөний дор анхдагч талбарынхаа хязгаараас зүйл гарч тархах-өвөрмөц хэлбэр болсон зүйлийн дотоод хувьсал байж болох бус уу ? Эсвэл бүр дахин судлавал зохих ямар нэг шинэ зүйл байж болох бус уу ? Харан байж болъё.
Хувьслын онолын үндсэн материалыг палеонтологи өгдөг, гэхдээ түүний цаг тооллын бичиг бүрэн биш, зүйлүүдийн гарал үүсэл болон мөхсөн тухай асуудлууд нь одоо болтол маргааны сэдэв болж байдгийг санавал зохино. Онцгой хүндрэлийг цаг хугацааны тойм байдал, чингэхдээ зүйлүүд бий болох буюу устах тооллын үед боломж нь заримдаа сая жилээс даван гардаг. Ийм бэрхшээлтэй бид Homo sapiens зүйлийн зарим соматик буюу биеийн салбарыг судлах, чухамхүү нэгдүгээр эрэмбийн: европжуу, монголжуу, австралжуу, негржүү арьстан бүрдэх үед тулгардаг. Эндээс асуудалд цэвэр биологийн хандлагыг цаг хугацааны хязгаарлалт дотор хэрэглэсэн ч бидэнд ямар ч давуу тал өгөхгүй юм. Түүнээс гадна арьстны хамаарал нь хүн гаригийн нүүр царайг өөрчлөхөд хүргэдэг дасан зохицох тэрхүү өндөр чадвартай ямар ч холбоогүй юм. Эцэст нь томоохон арьстнууд нь ихээхэн тодорхойгүй нийтлэг байдаг болохоор антропологид тэдгээрийг арьсны пигмент, гавлын бүтэц гэх мэтийн зарим гадаад шинж тэмдгээр нь янз бүрээр ангилсан байдаг. Хамгийн гол зүйл нь янз бүрийн арьстны төлөөлөгчдийн өвөг болгон авсан амьтдын дийлэнхи олонхи нь хэрэв нэг биш юмаа гэхэд хоёрдугаар эрэмбийнх байдаг. Эндээс бодитой оршиж, шууд ажиглагдан буй хүмүүсийн нийтлэг нь ямагт гетерогенийн буюу бүрэлдэхүүн нь нэг төрлийн биш байдаг юм. Бидэнд аймаг, угсаатан нь тодорхой байгаа чухамхүү тэд л Homo sapiens зүйлийн оршихуйн хамтын хэлбэр болж байдаг ба оршин буй газар нутгийнхаа ландшафттай харилцаж, өөрөөр хэлбэл энгийн экологийн зүйл дотоодын эрэмбэт бүлэг болж байдаг.
БИОЦЕНОЗ ДАХЬ КОНВЕРСИ БА СУКЦЕССИ ( УДАМШИН СОЛИГДОХ)
Үлдэц угсаатан хөршүүдийнхээ хувьд ч, ландшафтынхаа хувьд ч аюул учруулдаггүйг бид дээр дурдсан билээ. Одоо үүний шалтгааныг нэлээд тодорхой тайлбарлахыг оролдъё. Хэрэв угсаатан нь геобиоценозын дээд, төгсгөлийн салбар юм бол тэр нь түүний конверсийн хэмнэлд орох ёстой. Конверси гэх энэ ойлголтыг дээд тайлбарласан.
Английн биолог Т.Гексли “Конверсийн хэмнэл бол нэг амьдрах газар дахь амьтад болон ургамлын дунд эрчим хүчний эргэлтийг хангаж байдаг механизм юм, өөрөөр хэлбэл тухайн амьдрах газар хэвшмэл байх экологийн хамтын нийгэмлэг дэх бодисийн солилцоо болно. Амьдрах газрыг хадгалахын тулд эрчим хүчний эргэлтийг дэмжиж, хүчтэй болгож байх хэрэгтэй” гэсэн сэдвийг томъёолжээ. 15. Цит. по: Дорст К. До того как умрет природа. С. 350.
Сүүлчийнх нь бидэнд онцгой чухал юм. Өнгөрсөн үед үлдэц угсаатны жам ёсны өсөлт голдуу хүүхдийн үхлийн өндөр түвшингөөр хязгаарлагдаж байсан, харин ахмад насандаа гэрлэсэн хосуудын хамгийн их хуримтлал нь угсаатныг орчинтой нь тэнцвэрийг дэмжихэд л хангалттай, мөн тахал, байгалийн гамшиг мэтийн экзогенийн үйлдлийн эсрэг зарим баталгаа болж өгдөг. Байнга үүсдэг эдгээр бэрхшээлийг даван туулахад энэхүү тусгаар нийгмийн хэвийн хүчдэл зарцуулагдана. Энэ нь ямагт харгис шинжээсээ хагацсан, бөгөөд байгалийг өөрчлөх чадвар байдаггүй. Ийм угсаатан нь тэдний буй бүс нутгийн байгалийг сүйтгэх сүйрлийн шалтгаан болж чадахгүй нь илт байна.
Гэхдээ өөр, бүр эсрэг зөрчилдөөн ямагт үүсч байдаг. 1948 онд Ф.Осборн: “Өнгөрсөн зууны үндэстний (америкийн) түүх байгалийн баялгийг ашиглах үүднээсээ бол жишээлэх зүйлгүй юм…үнэндээ энэ бол утга учиргүй, хяналтгүй эрчим хүчний түүх мөн” гэж бичжээ. 16. Там же. С. 45.
Угсаатан хоорондын зөрчлийн үүднээс ч гэсэн энэ нь бас л ийм байсан. Индианчуудыг хүйс тэмтэрсэн, боолын худалдаа, 1836–1848 онуудад Техасыг эзлэн авсан, 1885 онд франц- индианы эрлийзүүдийг хядсан, Калифорни болон Аляскийн алт хайгчдыг залгисан зэрэг энэ бүх үйл явдлууд нь зохион байгуулалтгүй, хяналтгүй явагдсан юм. АНУ, Канадын засгийн газар зөвхөн дараа нь л нэгэнт байсан баримтаар арга хэмжээ авсан бөгөөд тэдгээрээс ашиг гаргасан юм.
Мөн яг ийм зарчмаар Дорнод Африкт арабууд нэвтэрсэн, голландын шилжин суугчид Капын газар, цаашаагаа Зүржийн гол руу хөдөлсөн. Мөн ийм аргаар оросын газар хайгчид Сибирийг, хятадууд Хөх мөрнөөс хойшхи газрыг булаан эзэлсэн юм. Дээрх үзэгдлүүдээс Газрын дундад тэнгисийг эллинчүүд колоничлосон, викингийн аян дайнууд нь ялгагдахгүй билээ. Харсаар байхад кельтүүдийн аян дайн, хойт Энэтхэгийг арийчууд эзэлсэн зэрэг нь мөн л ийм шинж чанартай байсан аж. Ингээд бид угсаатан буюу түүний хэсэг хөдлөнги төлөв байдалд шилжих байнга дахин давтагдах үзэгдэлтэй тулгарч байна. Ингэхэд тэдний харгис шинж асар их хэмжээгээр өсч, энэ болтол дасаагүй оршихуйн шинэ нөхцөлд хэрэглэхэд хүргэдэг дасан зохицох чадварыг нь нэмэгдүүлдэг аж.
Энд дурдсан болон эдгээртэй адилтгах үйлдлүүд нь тэдгээрт оролцогчдоос булчингийн, оюун санааны, сэтгэл хөдлөлийн аль нь ч бай асар их ажил (физик утгаар) шаардана. Дурын ажлыг хийхийн тулд хаа нэгтэйгээс шавхах ёстой харгалзах эрчим хүч зарцуулахыг шаардана. Илтэд цахилгаан биш, механик биш, дулааны биш, гравитацийн биш энэхүү эрчим хүч нь юу байх вэ ? Үхлийн эрсдэл рүү явж буй хүмүүс түүнийг хаанаас олдог вэ? Тэдэнд ийм хортой шохоорхол хэрэгтэй юм гэж үү? Гэхдээ тэд ямар ч гэсэн энэ эрчим хүчийг хожихоосоо илүүтэй үхэж зарцуулж байвал дурдсан үзэгдлүүд нь бидний шаргуу хайж буй “Х хүчин зүйл”–д хамаатай байж болох уу ? гэж зүй ёсоор асууж болно. Магадгүй. Гэхдээ эхлээд асуудлаа нарийвчлан тавиад үзье.
ХҮНИЙ УДАМШИН СОЛИГДОХ БУЮУ АНТРОПОСУКЦЕССИ
Түүхэн зарим үйл явдлын дээр дурдсан онцлогийг түүний бүх үзэгдэлд тархаах ёсгүй юм. Энэ нь хүний үйл ажиллагааны бүх илрэлийг нийгмийн эхлэлтэй нийлүүлэх гэсэнтэй дүйцэх алдаа болно. “Агуу их дүрэм бол …холих биш, харин ялгах хэрэгтэй, олон янз байдлыг багасгах нь үнэнд хараахан хүргэдэггүй. Харамсалтай нь хэр тааруухан ухаантнууд нэг янз байдалд дуртай байдаг. Нэг янз байдал нь их тохиромжтой. Тэр нь бүх зүйлийг гажуудуулдаг ч наад зах нь бүх асуудлыг зоригтойгоор шийдвэрлэдэг” гэж Опостен Тьерри харамсан бичиж байжээ. 17. Тьерри О. Избр. соч. С. 210.
Тэр ямар үнэн хэлээ вэ? Бөөндөх нь тэнэг хэрэг юм. Долоон жилийн дайн, Пруссийг наполеон булаан эзэлснийг гэнэтийн үйл явц гэх нь ийм гэдгийг хэлье. Ийм эрэмбийн үйл явдлыг зөн билэг биш, харин нийгмийн ухамсраар чиглүүлсэн улс төрийн зүтгэлтнүүдийн ухамсартай тооцоогоор сайхан тайлбарлаж болно. Энэ нь тухайлсан хүний үйлдлийг ухамсартай, ухамсаргүй гэж сэтгэл зүйн хувьд ангилдаг шиг тийм нарийн тодорхой ангиллын шалгуур юм. Энд шийдвэр гаргах үед сонгох эрх чөлөө байгаа нь шалгуур үзүүлэлт болно, энэ нь улмаар өөрийн үйлдлийн төлөөх ёс суртахуун–хууль зүйн хариуцлага болно. Хүмүүсийн практик үйл ажиллагаанд зан үйлийн энэ хоёр шугам хэзээ ч холилддоггүй. Тухайлбал, залуу насандаа дурлахыг жам ёсны гэж шударгаар тооцдог, харин танхайрал, янханчлал зэргийг ухамсартайгаар дураараа аашилсан гэж шийтгэдэг, хөгшин болсон үедээ үс болон шүд унахыг хүний буруу гэдэггүй, харин уучилж өрөвддөг, хэдийгээр ямар нэг хэмжээгээр зөнөг болсноор тайлбарлаж болох авч албаны өдөөн хатгалгад оролцох нь байж л байдаг. Түүхийн янз бүрийн шинж чанартай үзэгдлүүдийг зааглахад ийм хандлагыг шинжлэх ухааны анализаар хэрэгжүүлж болно. Дээр нэг удаа бид Евроазийн нүүдэлчин ард түмнүүдийн шилжих хөдөлгөөн нь талын бүсийн чийглэгийн зэргээс харааралтайгаар олон янзийн шинж чанартай байдгийг тухайлсан жишээгээр харуулсан билээ. Одоо бид иймэрхүү харьцаа бүх Homo sapiens зүйлийн хувьд байдгийг зүгээр л тэмдэглэж байна. 18. Гумилев Л. Н. 1) Истоки ритма кочевой культуры Срединной Азии //Народы Азии и Африки. 1966. № 4. С. 85-94; 2) Роль климатических колебаний в истории народов степной зоны Евразии //История СССР. 1967. № 1. С. 53- 66.
Ердийн нөхцөлд ард түмэн шилжин явах нь өөрийгөө угсаатны системийн хувьд хадгалан үлдэх, тэжээгч ландшафтаа сүйрлээс хамгаалах эрмэлзэл болдог. Антропосукцесси, өөрөөр хэлбэл дандаа болдоггүй, оршин суух нь үнэ цэнэтэй мужид нэвтрэн орох, заримдаа булаан эзэлж болдог үзэгдэл нь сөрөг тэмдэгтэй миграци юм. Хамгийн аймшигтай нь ялагчид нь ялагдагсдаасаа багагүй нэрвэгддэг, учир нь тэд ялалтаа хэрэгжүүлэхийн тулд шинэ нөхцөлд дасан зохицох ёстой бөгөөд энэ нь өөрийн мөн чанарыг угаар нь эвдэнэ гэсэн хэрэг билээ. Ийм сэгсрэлтэд хамгийн уян хатан, лабил буюу тогтворгүй залуу хүмүүс л чадвартай байх нь ойлгомжтой юм.
Үйл явц эхлэх үед (уншигчид сукцесси юмуу түрэмгийлэл гэхийгээ дураараа шийд) эдгээр элементүүд нь зөвхөн захирагдмал үүрэг гүйцэтгэдэг. Тэргүүлэгч хүний хувьд үйл явдлын цусан гинж татах нь утгагүй бөгөөд хүсмээргүй зүйл байдаг. Гэхдээ антропосукцесси нь ямар ч гэсэн болдог бөгөөд тэдгээрийн шалтгаан нь хүний ухамсрын хянадаг хязгаарын чанадад байдаг нь илт байгаа юм. Тэгвэл угсаатны нийлэгжилтийн хөдлөнги болон тогтонги чанар нь адил зүй тогтолтой, тэдгээрт буруу болон хариуцлагын ойлголт байхгүй. Үгүй юм. Энэ сэдэв нь бүхнийг өршөөхийг гаргаж ирэхгүй байна. Мэдээжийн хэрэг тодорхой хүмүүс угсаатны нийлэгжилтийн аль шатанд байгаагаасаа үл хамааран үйлдсэн гэмт хэрэгтээ буруутай. Гэхдээ угсаатны зүй тогтол нь бүтэн эрэмбээр дээр байдаг, түүнд үйлдэл нь эсрэг үйлдэлтэй тэнцүү гэсэн Ньютоны гуравдугаар хууль мэтийн их тооны статистик хуулийг хэрэглэж болно. Ялагчид ялагдагсадтайгаа хамт буюу арай хожим мөхдөг, гэхдээ биет үхлийн утгаар биш, угсаатны өөрчлөн байгуулалтын утгаар үхдэг. Угсаатнууд бол могой биш, тэд арьсаа биш, сэтгэлээ сольж байдаг.
XX. Үхэлгүй байх нь үхлээс аймшигтай болоход
ОВГИЙН НИЙЛЭГЖИЛТ (ФИЛОГЕНЕЗ) УГСААТНЫ НИЙЛЭГЖИЛТЭД ХУВИРДАГ
Хүн араатан уу, бурхан уу ? гэдэг тухай маргаан романтикууд болон үгүйсгэгч үзэлтнүүдийн ухаан санааг зовож байснаа аз болоход эдүгээ ач холбогдлоо алдаад байна. Хүн зөвхөн амьтан төдийгүй, мөн түүний доторх амьтан болох нь тодорхой болоод байна. Энэ нь түүний давуу талыг огтхон ч доромжлоогүй болно. Иймээс тэр хамт олонд-өвөрмөц хамтын нийгэмлэг угсаатан дунд амьдардаг. Харин манай сэдвийн хувьд гэвэл Homo sapiens зүйлийн хязгаар доторхи өвөрмөц үзэгдэл болсон угсаатны байр суурийг тогтоож, угсаатны харьцангуй тогтвортой байдлыг юу тэтгэж байдаг, түүний устах (маш амархан) болон үүсэх (асуултын асуулт) шалтгааныг ойлгох нь чухал юм.
Чухамхүү угсаатны хамт олнууд л аль нэгэн салбар нөхцөлд дасан зохицдог ба харин хөгжлийн шат формаци нь даяар шинжтэй, газар зүйн орчинтой тэдгээрийн холбоо нь зүймэл антропо хүрээ, өөрөөр хэлбэл байгал судлаачийн ажиглалтад хүрэлцээтэй угсаатны хүрээгээр нөхцөлдөж байдаг. Бид олон тооны үзэгдэлтэй тулгарахдаа тэдгээрийг төстэй байдлын зарчмаар, шалтгаацлын дарааллаар нь бүлэглэж болдог, өөрөөр хэлбэл түүхэн материалд байгалийн шинжлэх ухааны арга зүйг хэрэглэж байна. Ингээд бид угсаатнууд үүсч, алга болох нь орчин үеийнхэнд ямар нэг төсөөлөл байгаа эсэхээс үл хамаарна гэсэн хатуу дүгнэлт гарган авч байна. Энэ нь хэдийгээр хүмүүсийн хамтын үйл ажиллагааны хэлбэрүүдтэй туйлаас холбоотой ч гэсэн угсаатнууд нь тодорхой хүмүүсийн нийгмийн өөрийн ухамсрын бүтээгдэхүүн биш гэсэн хэрэг юм. Нийгмийн хөгжил нь хүмүүстэй холбоотой байдаг учраас материйн хөдөлгөөний бусад бүх хэлбэрүүдэд өөрийн ул мөрөө үлдээдэг. Гэхдээ хэн ч гравитаци буюу цахилгаан дамжуулалт, тахал, үхэл буюу удамшлыг нийгмийн талаас нь тайлбарлахыг оролдоогүй ба энэ нь байгал судлалын салбар болно. Дээр дурдсан “түлхэлт”, түүнчлэн иймэрхүү зарим үзэгдлийг бид антропогенийн сукцесси гэж үзэх эрхтэй юм. Гэхдээ энд үүссэн эргэлзээ, тээнэглэлийг тэдгээрийн шалтгаан, өөрөөр хэлбэл мөнөөх оньсого мэт “Х хүчин зүйлийг” тайлбарлах үедээ нэлээд сүүлд шинжлэн үзэх болно. Одоохондоо үзэгдлээ үргэлжлүүлэн судалья.
Сүүлийн 5 мянган жилийн туршид ландшафтын антропогенийн өөрчлөлт нэг бус удаа үүсэн ба гэхдээ ямагт тодорхой бүс нутгуудын хүрээнд янз бүрийн идэвхижилттэйгээр явагдаж байсан. Түүхийг харьцуулан үзэх үед байгалийн антропогенийн өөрчлөлт болон шинэ угсаатан үүсэх эрин үеүдийн хооронд нарийн тодорхой холбоо тогтоогдсон юм.
Угсаатан үүсэх болон ландшафтын өөрчлөлт нь түүний шинэ эрмэлзлийн дагуу хөдөлмөрийн багаж, зэвсэг бүхий ихээхэн тооны хүмүүүсийн миграци бөгөөд энэ нь биет утгаараа эрчим хүчний зардал шаардсан ажил болно. Түүнээс гадна системийн хувьд тогтоон барих нь хүрээллийн байнгын эсэргүүцлийг даван туулах эрчим хүч зарцуулалгүйгээр боломжгүй юм. Тэр ч байтугай угсаатны уналт, өөрөөр хэлбэл түүний хөгжил саарах нь нэмэх хасах хурдатгал өгөгч шалтгаан болсон хүч тавих агшинтай холбоотой байдаг.
Энэ сэдвийг би томъёолж, дараа нь Ю.К.Ефремовт бичсэн зохиолоо өгсөн Ю.В.Бромлей дэмжсэн юм. Харин Бромлей өөрийнх нь чин сэтгэлийн мэдэгдлээр буруугүй байсан юм. Бүр ч илүү гайхалтай нь Ю.В.Бромлей угсаатны үйл явцад “био эрчим хүний үүргийг” хүлээн зөвшөөрч, энэхүү эрчим хүч нь “тэдний (угсаатны нийтлэгийн) оршихуйн түүхэн тодорхой нөхцлөөс хамаарна” гэж үзсэн юм. Энд энерги хадгалагдах хуулийг авч үлдэх шаардлагагүй мэт болж байна, энэ шалтгаанаар маргаан дэгдээх нь зохимжгүй юм. Үйл явцынх нь хувьд угсаатны нийлэгжилтэд шаардлагатай ажил хийхэд эрчим хүчний тодорхой төрөл байдаг нь хүлээн зөвшөөрөгдсөн нь сайн хэрэг болно. 19. Гумилев Л. Н. Этногенез и этносфера //Природа. 1970. № 2. С. 49-60, 20. Ефремов Ю. К. Важное звено в цепи связей человека с природой //Там же. 1971. № 2. С.79. 21. Бромлей Ю. В. Этнос и этнография. М., 1973. С. 163.
Эрчим хүчний энэхүү өвөрмөц хэлбэрүүдийн тодорхойлолт В.И.Вернадскийн гайхамшигтай номонд: “Бүх амьд биет нь өөрөө тасралтгүй өөрчлөгдөгч, хамгийн янз бүрийн маш нягт байдлаар өөр хоорондоо холбогдсон амьд бодисоос бүрдсэн, геологийн хугацааны туршид хувьслын үйл явцад өртсөн организмуудын нийлбэр болно. Энэхүү өвөрмөц хөдлөнги шинжтэй тэнцвэрт байдал нь цаг хугацааны явцаар тэмүүлсээр тогтонги тэнцвэрт байдалд шилждэг…Хэрэв эсрэг тал руу нь үйлчлэх түүнтэй дүйцэх ямар ч үзэгдэл байхгүй бол түүний оршихуй нь хэчнээн удаан байх тусмаа чөлөөт эрчим хүч нь тэгд ойртож очно, өөрөөр хэлбэл нэг талдаа илэрч байгаа амьд бодисын эрчим хүч нь урвуу энтропид (замбараагүй төлөв байдал–Орч) шилжинэ. Учир нь амьд бодисын үйлчлэлээр ажил гүйцэтгэх чадвартай чөлөөт эрчим хүчний хөгжлийг бий болгодог” гэж бичсэн байдаг. 22. Вернадский В. И. Химическое строение биосферы Земли и ее окружения. С. 284-285.
Эндээс угсаатны бүтэц болон зан үйлийн тогтсон үзэл нь хөдлөнги хэмжигдэхүүнүүд байдаг бөгөөд нийгэм болон биологийн аль алинтай нь жигд төсгүй угсаатны дотоодын хувьсал байгаа эсэхээр тодорхойлогдоно.
Энэ дүгнэлтийг этнологийн хэл рүү хөрвүүлж, бүх угсаатны хувь заяа угсаатан-ландшафтын тэнцвэр рүү аажмаар шилжих явдал мөн гэж үзэж болно. Тэнцвэр гэдэгт угсаатны хамт олон, жишээлбэл, овог, аль нэгэн бүс нутгийн биоценозод орж, биохорийн хязгаарлагдмал боломжоос хүн амын өсөлт зогссон үеийн нөхцөл байдлыг ойлгоно. Энд заасан аспектаар угсаатнууд нь геобиохимид өөрийн байраа олдог: угсаатны тогтвортой төлөв байдал нь байгал орчноос хүлээн авч буй бүх эрчим хүчийг системийн дотоод үйл явцыг тэтгэхэд зарцуулах тэр тохиолдол бөгөөд түүний гаралт нь тэгтэй ойрхон байдаг. Хөдлөнги төлөв байдал гэдэг нь их эрчим хүч эзлэн авах гэнэт үүсэх чадвар бөгөөд түүнийгээ системийн хүрээний хязгаараас гадуур ажил хэлбэрээр өгдөг. Энэ нь угсаатны нийлэгжилтийн шинжүүдээ аажмаар алдахыг нөхцөлдүүлдэг. Өөрөөр хэлбэл илүү их эрчим хүчийг татан авах чадвар, угсаатны бүтцийн хялбаршилтыг дагалдуулдаг ажил хэлбэрээр түүнийгээ тараах чадвараа алддаг.
Үлдэц угсаатан бүр (персистент) тэрээр зөвхөн хэзээ нэгэн цагт бүрэлдсэн болохоороо л оршин байдаг, энэ нь хөгжлийнхөө хөдлөнги шатуудыг туулчихсан гэсэн хэрэг юм. Энэ нь нэг талаас, болоод өнгөрсөн үйл явцын талстжсан хэлбэр, нөгөө талаасаа шинэ угсаатан үүсэх суурь болно. Хөдлөнги төлөв байдалд байх хугацаандаа дурын угсаатан өөрсдийнхөө булаан эзэлсэн байгалаас төдийгүй, организм шинэ нөхцөлд дасан зохицох явдлаар илэрхийлэгддэг өөрийнхөө физиологи, зан үйлээс ч гэсэн байнгын зовлонтой эвдрэлийг амсаж байдаг.
Гэхдээ хөдлөнги төлөв байдалд шилжихтэй холбогдсон эвдрэл дандаа гараад байдаггүй. Бидний харснаар бол эдгээр нь зарим харьцангуй ховор эрин үеүдэд ард түмний урсгал шилжих хөдөлгөөний үед болдог, дараа нь урт удаан хугацаанд угсаатны зүйн зурагт тэмдэглэгдэх уламжлалын систем тогтдог.
Ийнхүү Homo sapiens зүйлийн дотоод дахь биологийн хувьсал хадгалагдаж, харин бусад амьтны зүйлүүдэд байдаггүй шинжүүдийг олж авдаг.
ХУВЬСАЛ БА УГСААТНЫ НИЙЛЭГЖИЛТ
Шинэ угсаатнууд зүйлийнхээ хүрээнд үлддэг болохоор мэдээж угсаатны нийлэгжилтийг овгийн нийлэгжилттэй тэнцүүлэх үзэх ёсгүй юм. Бидний авсан адилтгал нь зарчмын хувьд бүрэн биш, үүнийхээ ч ачаар макро болон микро хувьслын хоорондын ялгааг тайлбарладаг. Орчин үеийн хүнд биологийн хувьсал байгааг хүлээн зөвшөөрөхдөө этнолог хүн хүний бүхий л дүр төрхийг өөрчлөх ёстой тархи зорилго чиглэлтэй хөгжих тухай орчин үеийн барууныхны таамаглалыг зөвшөөрч болохгүй.
Ж. Холден гэгч шинэ зүйлийн гоминид–Homo sapientissimus –ын зургийг зурж, ирээдүйн дэвшил, зөвхөн дэвшлийг харахыг хүссэн өөрийнхөө сонсогчдын хүсэлд хүндэтгэл үзүүлсэн бололтой. 23. Быстрое А. П. Указ. соч. С. 292.
Хэрэв яг ийм байх юм бол биднээс өмнө 2–5 мянган жилд амьдарч байсан хүмүүс биднээс соматик буюу биеийн мэдэгдэм ялгаатай байх байсан. Г.Ф.Дебецийн нээсэн грацилизаци буюу гавлын яс нимгэрэх үзэгдлийг санаж болно. Арьстны өөрчлөлтийн талын энэ хүн хүртэл “Тухайлсан “бүдүүлэг” болон “дэвшилтэт” шинж тэмдгүүд бүх арьстанд тааралддаг. Гэхдээ тэдгээрийн нэг нь ч хэрэв эдгээрийг эртнээс ингэж үзэхгүй бол шинж тэмдгийн “бүдүүлэг” болон “дэвшилтэт” бүрдлээрээ ялгагддаггүй. Хэрэв хүн төст сармагчин, ядахдаа неандерталь хүний гавлыг бүдүүлэг байдлын шалгуур болгон авч үзвэл Оросын тэгш газрын неолитийн эрин үеийн эртний европ хэв маяг шинж тэмдгийнхээ нийлбэрээр эртний славян буюу орчин үеийн украин хүний хэв маягаас илүү бүдүүлэг байж чадахгүй” гэж мэдэгджээ. 24. Дебец Г. Ф. О некоторых направлениях… С. 19-20
Үнэхээр ч хүн төрөлхтний хөгжил нь газар нутгаа өргөжүүлэх шугамаар болон зүйлийн дотоод дахь өөрчлөлтийнхөө, өөрөөр хэлбэл угсаатны тоог ихэсгэх замаар явж иржээ. Угсаатны зарим нь хойч үедээ биет болон утга зохиолын дурсгал үлдээн мөхөж, зарим нь үлдэц байдлаар оршиж, зарим нь ул мөргүй алга болсон ч зан үйлийн нэгдмэл тогтсон үзэл бүхий бүлийн ухамсартай үйлдлээр тийм хамт олонд ямар ч нөхцөл бүрдүүлж өглөө гэсэн өөрийнхөө жам байдлыг зорилго чиглэлтэйгээр өөрчлөх гэсэн нэг ч тохиолдол гарч байгаагүй юм.
Заримдаа хүмүүс нэр хүндтэй үхэл, амьдралаа хадгалахын тулд сайн дураараа өөрийгөө гажигтай болгох нь байдаг боловч энэ тохиолдол нь тэдний хувьд аливаа сонирхол татах шинжээ алддаг. Зүйлийн дотоод дахь сэтгэл зүйн тогтсон үзлийн энэхүү онцлог нь угсаатны нийлэгжилтийг салбар үйл явц гэж үзэх боломжийг хязгаарлаж, угсаатны нийлэгжилтийг хувьсалтай адилтгах явдлыг эргэлзээтэй болгож байна.
Энэ дүгнэлт хачирхалтай боловч дэс дараатай бөгөөд үнэн юм. Учир нь угсаатан нийгмийн хэлбэр олж аваад байгалийн биш үзэгдэл болох улс төрийн институтыг байгуулдаг юм. Римчүүд сенат, консулат, трибунат болон эрхийн системийг, франкууд феодализмыг, YI зууны түргүүд овгийн холбоо болон цэргийн нэгдлийг хослуулсан (орд) эль улсыг, инкүүд индиан овгуудыг дарлан мөлжих нарийн байгууламж, өөрийнхөө шатлал зэргийг бүтээсэн юм. Гэхдээ энэ бүх институт нь хүний гараар бүтээгдсэн явдлууд бөгөөд энэ утгаараа колоннад, орд харш, сүх, хувцас зэрэгтэй төстэй болно. Эдгээр нь дээр ярьснаар өөрөө хөгжих боломж байхгүй, зөвхөн цаг хугацааны үйлчлэлд эвдрэн сүйрч чадна.
Хүний оюун ухаан, хөдөлмөрөөр бүтээгдсэн хэлбэрүүд нь юмс аажмаар устах явдлыг сөрөн зогсч байдаг, гэхдээ дурын хангалттай хүчтэй гаднын үйлдэл хэлбэрийг эвдэж, түүнийг агуулгыг задралд хүргэж чадна. Ийм гунигт явдал болсны дараа шуурхай нөхөн сэргээлт дагалдуулахгүй бол угсаатан нь геобиоценозын бүрэлдэхүүн хэсэг, царцанги бүл болон хувирдаг. Зөвхөн угсаатны нийлэгжилтийн шинэ тэсрэлт л түүнийг мухардлаас гаргаж, хөршүүдтэй нь холилдуулан, шинэ угсаатны ноёрхлыг зарлан тунхагладаг. Тэхдээ энэ нь шинэ угсаатан болно.
УРАН БҮТЭЭЛ ҮҮ ?, АМЬДРАЛ УУ ?
Анх харахад энэхүү хатуу дүгнэлт нь гутранги үзлийг гутааж байна, гэхдээ анх харахад юм шүү. Хүмүүст “бурхангүй, сэтгэл татам юмгүй, нулимсгүй, амьдралгүй, хайр дурлалгүй” мөнхөд амь улжин байх хэрэг байна уу ? гэж бодоод үзье.
Хүмүүст уран бүтээл туурвих чадвар илүү сайн юм биш үү ? Энэ нь хүний организмын амьдралын эрчим хүчийг эргэн нөхөлтгүй зарцуулдаг шүү дээ. Хэрэв дээд эрэмбийн систем–угсаатны тухай ярьж байвал энд ч бас тийм зүй тогтол байдаг. Чөлөөлөх буюу булаан эзлэх дайнд хүчтэй дайсныг ялсан ялалт нь олон баатрууд болон тэдэнд буй генийг аван оддог. Гэхдээ ийм золиосыг ичгүүртэй боолчлол гэж тооцох нь юу л бол ? Ландшафт өөрчлөх, шинэ улс орнуудыг нээх, манай үед бол гариг нээх зэрэг нь лаборатори буюу номын сан дахь үүргээрээ биш, сэтгэлээрээ хийж буй шаналгаат ажил мөнөөс гадна хүмүүсийг гэр бүл эсвэл ерөөс түүнийг бүтээхэд саад хийх бүх юмнаас хөндийрүүлдэг. Гэхдээ л бид ажилдаа шатсан Колумба ба Магеллан, Пржевальский ба Ливингсон, Эварист Галуа ба Анри Пуанкаре, Опостен Тьерри ба Дмитрий Иванович Менделеев нарын нэрсийг хүндэтгэдэг. Рембрандт ба Ван Гог, Андрей Рублев ба Михаил Врубель нарын зураачдыг яахав ? Тэгээд бас яруу найрагчид, хөгжмийн зохиолчид, эх орныхоо төлөө тулалдсан баатрууд, энэ бүхнийг тоолж ч баршгүй, энэ нь хүн бүхэн мэдэх жишээнүүд юм. Тэдний олонхи нь генийн санд ул мөрөө үлдээгээгүй, гэхдээ энэ золиосоор эдүгээ хойч үеийнхийг алмайруулж буй соёлын барилга сүндэрлэсэн юм.
Гэхдээ иймэрхүү хүмүүсийн зарим нь гэр бүлтэй байсан, гэтэл тэдний хүүхдүүдэд эцэг эхийнх нь авъяас илэрдэггүй. Энэ нь бидний дүгнэлттэй зөрчилдөхгүй юу ? Учрыг нь олъё.
Чадвар өөрөө хараахан бүх зүйл биш юм. Агуу их үйл хэрэгт хүмүүсийг бодитой буюу хуурамч төгс эрхэмлэлд үхэн хатан зүтгэхэд түлхэгч асаагуур хэрэгтэй. Үзтэл орь ганц чухамхүү энэ асаагуурыг шинж тэмдэг хэмээн үзэж болно, учир нь тэр дандаа дамждаггүй юм. Хэрэв энд авч үзсэн шинжтэй хүнд зуун хүүхэд байвал магадгүй хувийг нь тооцоолон гаргаж болох байсан ба ингэснээр энэ шинж тэмдгийн дамжих магадлалыг олох байсан. Гэвч харамсалтай нь вандуй болон ялаанд тохирох судалгааны аргыг хэрэглэж болохгүй юм. Түүх нь маш нарийн тоолсон янз бүрийн эрин үеүд дэх янз бүрийн угсаатны үйл ажиллагааны шинж чанарыг нэгтгэн дүгнэсэн материалууд байдаг. Угсаатны түүх, янз бүрийн угсаатны нийлэгжилтийн шинжилгээ нь: угсаатны нийлэгжилтийн идэвхи нь ямар ч гэсэн эцэс төгсгөлгүй оршин байж чадахгүй угсаатны системийн оршин үргэлжлэх хугацаатай урвуу хамааралтай гэсэн энэхүү хамаарлыг тогтооход хүргэж байна.
Нэгдүгээрт, гансарсан оршихуйн нэг хэвийн байдал нь хүмүүсийн амьдралын тамирыг бууруулж, үүсэн бий болсон сэтгэл санааныхаа хоосролыг дүүргэхийн тулд хар тамхинд дурлах, бэлгийн гаж явдал зэргийг үүсгэдэг байна. Энэ нь угсаатныг цаг ямагт системийнх нь хувьд сулруулж байдаг. Хоёрдугаарт, тулшрамтгай хэв маягуудыг амьдралаас зайлуулснаар олон янз байдлыг багасган угсаатан хялбаршдаг, энэ нь эргээд нийт угсаатны хамт олны резистент буюу эсэргүүцэх чадварыг бууруулдаг. Тайван нөхцөлд энэ нь бага мэдрэгддэг, харин биологийн орчинтой, ялангуяа хөршүүдтэйгээ тулгарах үед идэвхитэй төрөлжсөн, золиослогдох элемент байхгүй нь туйлын эмзэг мэдрэгдэнэ. С.М.Широкогоров угсаатан нь “өөрийгөө хамгаалах ухамсрыг (буюу инстинкт) удирдлага болгон урагш явсан оюун ухааныг тэгшитгэн намтгаж, хувь хүмүүсийн дундаж түвшин рүү хүргэхэд” тэмүүлдэг гэж үзээд энэ үйл явцыг ухамсартай гэж тооцсон нь үнэн болох нь юу л бол. 25. Широкогоров С. М. Этнос: Исследование основных принципов изменения этнических и этнографических явлений //Изв. восточного ф-та Дальнево-сточн. ун-та (Шанхай). 1923. XVIII. Т. I. С. 130.
Сэтгэдэг, алдар цуутай хүмүүсээ устгах тухай ухамсартай шийдвэрийг нэг ч угсаатан гаргаж байгаагүй, харин тэд оролцогсдын хүсэл зоригт үл хянагдах үйл явдлын логикоор үхсэн байдаг. Ийм зүйл эзэнт гүрний Римд болсон: цэргийн үймээний үеэр тэдний золиос нь хамгийн сахилга баттай центуронууд болсон бөгөөд үүнийн дараа легионерууд бүдүүлэгчүүдийг бутниргэсэн юм. Византид 1204 болон 1453 онуудад хүн ам нь цайзын ханан дээр гарч, өөрийнхөө гэр орныг хамгаалахаас татгалзсанаар баатарлаг хамгаалагчид тусламж авалгүйгээр үхэцгээсэн, XII–XIII зууны Хятадад хүн ам нь ч, засгийн газар нь ч зүрчид болон монголчуудад бууж өгсөн гэх мэт. Гэхдээ энэ нь зөвхөн түүхэн үйл явдлын логик чиглэмжээрээ биологийн доройтол, нийгмийн хямралтай давхцсан уналтын эрин үед л байдаг байна. Иймээс угсаатны нийлэгжилт бүр системийн мөхлөөр төгсдөг бөгөөд энд телеологик буюу бүх юм цаанаасаа зохицуулагдсан гэх зарчим утгагүй болж байна. Өөрийнхөө аймшигт төгсгөл рүү тэмүүлж болно гэж үү? Түүний зайлшгүйг эрэлхэгээр хүлээн зөвшөөрч л болно ! Ингээд хувьслын тухай дарвины ч, дарвины эсрэг ч, шинэ хосолмол үзэл баримтлал ч угсаатны нийлэгжилтийг тайлбарлахад тохирохгүй байна. Энэ нь ч жам ёсны юм, харин этнолог нь биологийн биш, харин газар зүйн шинжлэх ухаан учраас хэдийгээр шинэ организмын зан үйл болон тэдний амьдрагч орчинтой холбоотой ч гэсэн өөрийн гэсэн онцлогтой байдаг.
С. И. КОРЖИНСКИЙН САНАА
Ямар ч гэсэн тусгай сэдвүүдийг нь урьдач үзлээс цэвэрлэж, засвар хийвэл манай судалгаанд тохирох бас нэг үзэл баримтлал бий.
1899 онд Петербург хотноо С.И.Коржинскийн “Гетерогенезис ба хувьсал” хэмээх ном хэвлэгдэн гарсан юм. Түүний бодлоор оршихуйн төлөө тэмцэл болон байгалийн шалгарал нь шинэ хэлбэр бүрдэхийг хязгаарлагч, өөрчлөлтийн хуримтлалыг таслан зогсоогч хүчин зүйл болдог бөгөөд дундаж хэв маяг тэсэн үлдэх, өөрөөр хэлбэл status quo–г тэтгэхэд тусалдаг байна. Шинэ хэлбэр бий болох нь аль нэгэн газар зүйн бүсэд ховорхон “үсрэлтийн өөрчлөлт”–ийн улмаас үүсдэг. Хувьслын үйл явц нь шинэ арьстан болон түүний эхийг тавигчийн хооронд бэлгийн саад (үл эвцүүлэх) үүсэх, шинэ гетерогенийн өөрчлөлт үүсэхэд хүргэдэг” байна. (Гетерогенезис гэдэг нь хоёр буюу түүнээс дээш үе дамжин өсөн үржихүйн арга солигдохыг хэлж байна. Үүнийг удамшсан нийлэгжилт гэж орчуулж болох юм –Орч.) 26. Завадский К. М. Развитие эволюционной теории после Дарвина. Л., 1973. С. 223-225.
Шинэ газар зүйн арьстан бий болохыг: “Нэг ижил нөхцөлд хөгжиж буй, ямар нэгэн зүйл буюу арьстны хэвийн төлөөлөгчдөөс гарсан удмынхны дунд бусдаасаа болон төрөл төрөгсдөөсөө их бага хэмжээгээр зайлсхийдэг тухайлсан индивидуум (орь-Орч) амьтад гэнэт бий болдог. Энэхүү зайлсхийлт нь заримдаа хангалттай ач холбогдолтой байж, маш олон шинж тэмдгүүдээр илэрхийлэгдэн, голдуу цөөн буюу ямар нэг ганц ялгаагаар хязгаарлагдана. Эдгээр шинж тэмдгүүд нь ихээхэн тогтвортой байж, заавал үеэс үед удамшин дамждаг нь гайхамшигтай юм. Ийм маягаар балар эртний үеэс оршин байсан арьстнууд шиг тийм бөх, тийм тогтмол шинэ арьстан шууд үүсдэг” дүрсэлсэн байна.
С.И.Коржинскийн таамаглал нь арьстан буюу зүйлийн үүсэлд хамаарч, харин зүйл бүрэлдэх болон дасан зохицох нийлэгжилтийн (адатациогенез) хоорондын холбооны асуудлыг тавилгүй, зүйд нийцэх асуудлын хажуугаар өнгөрснийг тэмдэглэсэн К.М.Завадскийн зөв байх. Эндээс уг таамаглал нь зүй зохистой шинж тэмдэг бүрэлдэх гэж ойлгогдож буй хувьсалд шууд хамаарах зүйл байхгүй юм. 27. Там же. С. 225
Зүйл үүсэх харьцаанд С.И.Коржинскийн дүгнэлтүүд хэр зэрэг зөв болохыг шүүн тунгаахыг орхиё. Гэхдээ хэрэв хэд хэдэн эрэмбээр дээгүүр байгаа үйл явц, угсаатны нийлэгжилтийг ярих юм бол тэдгээрийг бүхэлд нь хэрэглэж болно. Угсаатан бүрдэх үйл явц бол хувьслын үйл явц биш. Энд л хүний нийлэгжилтээс (антропогенез) угсаатны нийлэгжилт ялгагдана.
УГСААТНЫ НИЙЛЭГЖИЛТ ДЭХ ГАЖИЛТ (ЭКСЦЕСС) БА ИНЕРЦИ
Гетерогенезийн үзэл баримтлал нь угсаатан генетикийн үйл явцын шинж чанарын шалтгаарх бараг бүх эргэлзээг арилгадаг. Байгалийн шалгарал нь угсаатны системийг тогтворжуулна, энэ нь түүнийг гарцаагүй хялбаршихад хүргэнэ. Энэ нөхцөл байдал нь эргээд гажилтын үзэл баримтлалыг, өөрөөр хэлбэл угсаатан үүсэхтэй холбоотой, эрчим хүчний түвшний өөрчлөлтийн жам ёсны явдлыг зөрчиж, үе үе үүсдэг түлхэлт болох микро бүлгийг хүлээн зөвшөөрөхийн зайлшгүйг харуулж байна.
Хэрэв дурдсан үйл явцуудыг мөн тийм хүчтэй, гэхдээ урвуу тэмдэгтэй бусад үйл явц тэнцвэржүүлээгүй байсан бол шинэ угсаатан үүсэхгүй байхсан. Тэгвэл хүн төрөлхтөн бүр палеолитийн үеэс цаг уурын нэг бүсэд амьдардаг, өөр хоорондоо төстэй, хүн төстнүүдийн (антропоид) царцанги масс болон хувирах байхсан. Энэхүү хоёр хөлт махчин маш удаан үржих байсан ба учир нь түүнийг болон бусад араатныг хоол хүнсний хэмжээ хязгаарлах байсан юм. Тэдэнд оюун ухаан ч хэрэггүй, тэд таатай нөхцөлд дасан зохицохуйн дээд хэмжээндээ хүрээд л юмсыг өөрчлөх шаардлага нь алга болох байсан. Товчоор хэлбэл тэд өнөөгийн зожиг – үлдэц угсаатан шиг л бүгдээрээ амьд байх байсан.
Үнэн хэрэг дээрээ хааяа хааяа газар нутгаа өргөтгөх, “хүн төрөлхтөн” хэмээх хэт аварга системийн элементүүдийг хольж хутгах явдлыг дагалдуулдаг угсаатны нийлэгжилтийн тэсрэлт болж байдаг юм. Дээр өгүүлсэнчлэн эдгээр тэсрэлт нь нийгмийн хөгжлөөр тайлбарлагдахгүй, дэвшилд огтхон ч чиглэгдээгүй, формацийн халагдахтай маш цөөн давхцдаг, хэрэв давхцвал тохиолдол гэж үзэх ёстой тийм л зүйл юм. Ингэхдээ Homo sapiens–ийн биологийн хувьслын үзэл баримтлал руу эргэн орох хэрэгтэй боллоо. Дээд палеолитэд жинхэнэ хүн төрөлхтний нийгэм бий болсны дараа “дүрс бүрдүүлэгч хүч болсон шалгарал өнгөрсөн”, мөн хэл, сэтгэлгээний өндөр хөгжилтэй нь харьцуулахад Homo sapiens –ийн бусад онцлогууд мэдээж жаахан ялгаатай авч шийдвэрлэх ач холбогдолтой байгаагүй юм”. Бидэнд сүүлчийн энэ үг л хангалттай. 28. Рогинский Я. Я., Левин М. Г. Указ. соч. С. 314
Хүний анатоми болон физиологийг өөрчилдөггүй, харин зөвхөн зан үйлийн тогтсон үзлийн гажилт үүсгэхийн тулд тэр нь хүчтэй байх ёсгүй. Харин ч урвуугаар зөвхөн сул гажилт л газар зүй, физиологи, нийгмийн дэвсгэрийг хөндөлгүй үлдээх юм Энэ дэвсгэр дээр энэ тохиолдолд сэтгэл зүйн шинэ тохируулгын абрис буюу хүрээ зааг тод харагдана. Ийм гажилт буюу түлхэлтийг өдөөгч нь дээр олон удаа дурдаж байсан зөвхөн “Х хүчин зүйл” л байж чадна.
XXI. Зөрчлүүдийн нийлбэр
ХАРИУЛТ ХАРААХАН ОЛДООГҮЙ
Угсаатны үзэгдлийн мөн чанарын зөрчилгүй тайлбар олох гэж эрмэлзэхдээ бид янз бүрийн шинжлэх ухаанд хандаж, газар сайгүй ямар нэг хариултууд олж, гэхдээ ямагт бүрэн бусыг олж байлаа. Эдгээр хариултууд бидэнд хэрэггүй байсан гэхээсээ илүүтэйгээр тэдгээр нь бидэнд зайлшгүй чухал байлаа. Тэдгээр нь угсаатны нийлэгжилтийн аль нэгэн нөхцлийг нээн харуулж байсан, гэхдээ зорилтын нөхцлөөр хөдөлбөргүй байх ёстой үнэн бус шалтгааныг зааж байсан, өөрөөр хэлбэл ямагт байж, үзэгдэлд нэг л утгаар үйлчилж байсан. Үүнийг ойлгоё.
Экзогамийн замаар буюу уусан нэгдэх замаар угсаатныг арьстан буюу арьстны дотоодод холиход заримдаа шинэ угсаатан төрүүлж, заримдаа анхдагч хэлбэр рүү нь эргүүлэн хаяж, харин заримдаа бүлийг үхэх хүртэл нь доройтуулдаг. Эдгээр үйл явцуудад үр дүнг нь орвонгоор нь хольж буй тооцоолоогүй шинж тэмдэг оролцож байгаа нь илт байна.
Эндогамаар дамжин хэрэгжүүлж буй тусгаарлалт нь угсаатныг голдуу хамгаалдаг, гэхдээ заримдаа тэднийг байгалийн ч, угсаатны ч орчинг эсэргүүцэх чадвараа алдах хүртэл нь сулруулдаг. Ингээд хөршүүдийн шахалт буюу дайралтаас болж угсаатан алга болдог.
Янз бүрийн шинж чанар бүхий ландшафтын нөхцөлд дасан зохицохуй нь заримдаа угсаатны дивергенци буюу салалтад хүргэнэ, заримдаа үгүй, цаг уурын янз бүрийн бүсэд ч гэсэн угсаатан цул хэвээрээ, мэдээж хэрэг зүймэл байдлын өгөгдсөн түвшиндөө үлддэг.
Үүний урвуугаар газар нутгийн ландшафтын төслөг байдал нь тийшээгээ хоёр– гурван угсаатан миграцаар очиход заримдаа харилцан уусан нийлэх үзэгдлийг дагуулдаг, заримдаа угсаатнууд уусан нийлэлгүйгээр зэрэгцэн оршдог. Үүний шалтгаан нь бүс нутгийн байгалд биш, харин угсаатанд өөр нь байх ямар нэг юманд байгаа илт байна. Үүнийг л нээж олох хэрэгтэй юм.
Хоёр буюу түүнээс илүү ландшафтын хослол нь салбар угсаатан генетикийн үйл явц эхлэх зайлшгүй нөхцөл болдог, гэхдээ энэ нь хангалтгүй юм. Дурдсан нөхцөлд угсаатнууд дандаа үүсдэггүй. Иймээс нэмэлт хүчин зүйлийг хайх хэрэгтэй.
Соёлын нэг нэгдмэл хэв маяг, жишээлбэл, шашны системийн тархалт нь заримдаа угсаатныг нийлэхэд хүргэдэг, заримдаа шинээр хандагсдын угсаатны бие даасан байдалд яагаад ч нөлөөлдөггүй. Үүнтэй адилаар материаллаг соёлын төстэй байдал нь эсвэл ард түмнүүдийг ойртуулдаг, эсвэл тэднийг өрсөлдөөнд түлхдэг, эсвэл тэдний харилцаанд ямар ч хамаагүй байдаг. Соёлын хэв маягийн хуваагдлын тухайд мөн л ийм зүйлийг хэлэх хэрэгтэй юм. Шинэ сект буюу түүнийг бишрэн шүтэгч сургааль бий болох үед заримдаа онцгой угсаатан ялгаран гардаг, заримдаа итгэл үнэмшлээ хадгалан хуучнаараа үлддэг. Үл тэвчих зан бол бүх үе, бүх ард түмэнд байгаагүй юм.
Нийгмийн нөхцлийн төстэй байдал нь угсаатны нийлэлтийг дагуулж болно, гэхдээ энэ нь заавал биш юм. Нэг угсаатны зарим хүмүүс нь овгийн ахуйн дадсан нөхцөлдөө амьдарч, зарим нь феодализмд, өөр аль нэг бүлэг нь капиталист харилцаагаар ажиллах явдал бас л байнга тохиолддог. Энэ үзэгдэл тодорхой бөгөөд түүнийг “олон хэвшилт“ гэж нэрлэдэг.
Магадгүй, даяар түүхэн үйл явц аварга том угсаатны эрхэмлэлийг бүтээхээр хөтөлж яваа ч юм уу ? Заримдаа хөтөлдөг, заримдаа угсаатан хоёр гурван хэсэгт хуваагдаж, тэдгээрээс нь эсвэл шилжин явдаг, эсвэл нэг газар нутаг дээр зэрэгцэн оршдог шинэ угсаатнууд төрөн гардаг. Ахиад л боломж гарч байна, гэхдээ энэ бол зүй тогтол биш.
Ингээд энд дурдсан бүх аспектууд нь аль нэг угсаатны хүрээнд угсаатны нийлэгжилт явагдахад ач холбогдолтой бөгөөд хүчин зүйлийг биш, харин үзүүлэлтийг авч тэдгээрийг зөв тооцох хэрэгтэй, учир нь салбар өөрчлөлтүүдийг хасах замаар л угсаатны бүх нийлэгжилтэд нэгдмэл байх жинхэнэ “Х хүчин зүйлийг” илрүүлж чадах бөгөөд түүнийг нээснээрээ дээр дурдсан бүх эргэлзээг тайлж болно.
УГСААТНЫ НИЙЛЭГЖИЛТ БА ЭРЧИМ ХҮЧ
Дурын угсаатны хувьд нийтлэг шинж нь: 1) бусад угсаатнуудад өөрийгөө сөргүүлэн тавьдаг, улмаар өөрийгөө баталсан; 2) зүймэл шинжтэй, нарийвчилбал төсгөлгүйгээр хуваагддаг, системийн холбоогоор гагнагдсан; 3) ажиллаж эхлэх үеэсээ эхлээд оргил шатнаас сарних буюу үлдэц болон хувирах хүртлээ хөгжлийн нэгдмэл үйл явц явагддаг. Угсаатан бол “царцанги төлөв байдал” биш, “нийгмийн категори” биш, “хэл, эдийн засаг, газар нутаг, сэтгэл зүйн хэв шинж зэргийн нийтлэг бүрдэл” биш, харин манай зорилтыг шийдвэрлэх түлхүүр нь чухамхүү заавал байх гуравдахь онцлогт нь буй угсаатны нийлэгжилтийн үйл явцын шат юм.
Өөрөө гарч ирж байгаа дүгнэлтийг хийе. Асаах агшны хувьд ч, оргил үед хүрэхийн хувьд ч, мөн адил дахин сэргэхийн хувьд эсвэл байгалийг өөрчлөхөөр илэрдэг, эсвэл нүүдлээр илэрдэг гэх мэт үүссэн бүлээс хэт хүчдэл гаргах чадварыг шаардана. Энэ нь бидний хайж буй “Х хүчин зүйл” юм ! Бидэнд тодорхой байгаа бараг бүх угсаатан хэт угсаатны бүхэллэг гэсэн өвөрмөц маягийн хийц болон бүлэглэгдсэн байдаг. Угсаатны тархалт нь тэдний үүссэн газар, миграци буюу их нүүдэл, байгалийн гамшиг болон хөршүүдтэйгээ хийх тэмцэлд ялах, ялагдах зэрэгтэй холбоотой, харин үхэхгүйн тулд ердийн хүчдэл хангалтгүй байна. Орчны дурын нийлбэр төлөв байдал нь идэвхигүй байдаг ба түүнийг зөрчихийн тулд хайлуулах буюу уур гаргах үеийн “далд дулаан” гардагтай адилтгам эрчим хүчний нэмэлт зардал шаардагдана. Гэхдээ хэт хүчдэл гаргасны дараа инерцийн үйл явц эхлэж, энэ нь орчны эсэргүүцлийн улмаас л унтарна.
Угсаатны “төлөв байдал” хоёр байдаг нь бидэнд тодорхой. Энэ нь гомеостатик (хөдөлгөөнгүй)–амьдралын орчил нь үеэс үед дамждаг: динамик (хөдлөнги) угсаатан өөрөө дээр дурдсан хөгжлийн шатуудыг гомеостазын хүрээнд туулдаг төлөв байдлууд болно. Хоёуланд нь хөдөлгөөн ажиглагдах бөгөөд гэхдээ эхнийхийг нь зүйрлэл маягаар эргэлдэх, удаахыг нь идэвхийг нь давтамжаар хэмжиж болох хэлбэлзэх хөдөлгөөн гэж нэрлэж болно. Нийгмийн дэвшил бол давших хөдөлгөөн бөгөөд тэр нь угсаатны нийлэгжилтээс ялгаатайг бид нэгэнт үзүүлсэн билээ.
Юу хөдөлгөдөг вэ ? гэсэн асуултад Дэлхийн био хүрээний бүтцэд байгаа угсаатны систем гэж хариулъя, хаашаа хөдөлдөг вэ ? гэсэн асуултад хэлбэлзэх хөдөлгөөний үед “урагш”, “хойш” гэсэн ойлголтыг хэрэглэдэггүй учраас хаашаа ч биш гэж хариулъя. Угсаатны хувьд шинжилгээг үлэмж хөнгөлөх ёстой математик илэрхийлэл олж чадах болов уу? гэсэн асуултад нэг үгээр хариулж болохгүй юм. Тодорхой тайлбарлахыг оролдоод үзье.
29. Угсаатны хэлбэлзэх хөдөлгөөн гэдэг бол угсаатан нэг тэнцвэрт байдлаас нөгөөд шилжих явдал болно. Хөдөлгөөний энэ хэлбэрийг эртний Хятадад мэддэг байсан бөгөөд түүнийг “хувьслын хууль” (превратность) гэж нэрлэж байжээ. VI зуунд Чэнгийн гэрийн хатан хаан Да И азгүй хувь заяаныхаа тухай гунихралт дуулалдаа ингэж бичиж байжээ.
Хугацаа биш, хундага дарсанд мансуурнам бид
Хуурын утас нэг дуугарч, нэг зогсдог ажгуу
Худлаа гэвэл эргэн тойрноо хараач, Та нар
Хувирал хувьсал энэ дэлхийг захирдаг ажгуу
Хуучин цагт дуурссан дуунууд одоо ч ялгаагүй
Хувьгүй адлагдагсдыг үүрд түгшүүлнэм…
(Л.Н.Гумилевын орчуулга)
Этнологийн асуудлуудын хувьд ямар нэг зорилт тавихдаа бид далд хэлбэрээр томъёологдсон ямар нэг техникийн зорилтыг орчин үеийн тооцоолох хэрэгслээр шийдвэрлэхийг оролдож буйтай адил тийм хүндрэлтэй тулгарч байдаг. Тэнд ч, энд ч тоон арга тохирохгүй юм. Гэхдээ аль аль тохиолдолд нь загварчлалын тодорхой аргуудыг хэрэглэн шийдэл гарган авч болох юм. Энэ үйл явцыг үзэх бидний үзэл бодлын нийлбэрийг тусгасан үйл явцын загвар гаргаж, баталгаатай баримтуудаар засвар оруулна. Дараа нь энэ загвараа үлдсэн баримт болон үйл явдлуудын олонлигтой жишин адилтгахад хэрэглэнэ, мөн үйл явцын өнгөрсөн төлөв байдал дахь ирээдүйн буюу бидэнд тодорхой бус шинж чанаруудын талаар таамаглалт дүгнэлт гаргахад ашиглана. Сэтгэн олох үнэлгээний үндсэн дээр бидний зөв гэж хүлээн зөвшөөрсөн шийдвэр бүрийг зорилго чиглэлтэйгээр олсон шинэ хүчин зүйлүүдээр (үнэнтэй төстэй шийдвэр болох нь тодорхой) баталсны үр дүнд загвараа нарийвчилж, хөгжүүлнэ.
Эцэст нь бид эдүгээ оршин буй бүх угсаатнууд нь харьцангуй саяхан бий болсон, эртний угсаатнуудаас цөөхөн үлдэц бүтэн үлдсэн, хүй нэгдлийн үеийнхээс нэг нь ч үлдээгүй гэдгийг мэднэ. Энэ нь угсаатны нийлэгжилт нь хэдийгээр хатуу хэв маягийн системийг төрүүлэгч социогенезтэй хавсран үйлчилдэг ч байгалийн бусад үзэгдлийн адил байнга урагшлагч үйл явц гэдгийг харуулж байна.
Аль нэгэн хэв шинжийн дурын системийг өөрчлөн байгуулахад ажил хийх, өөрөөр хэлбэл эрчим хүчний зарцуулалт шаарддаг гэдгийг бид нэгэнт ярьсан билээ. Мэдээж энэхүү эрчим хүч нь цахилгаан соронзон, дулаан, гравитаци, механик эрчим хүчний аль нь ч биш. Энэхүү эрчим хүчний тэсрэлт, түлхэлтийг байгал орчны эсэргүүцлийн улмаас унтарч байдаг сукцесси буюу хүний удамшин солигдох явц нөхцөлдүүлдэг. 30. Гумилев Л. Н. Этнос как явление //Доклады отделений и комиссий ВГО. Вып. 3. Л., 1967. С. 106
УГСААТНЫ ТҮҮХИЙН ТАСРАЛТАТ ШИНЖ (ДИСКРЕТ)
Түүхийн үйл явцын тасралтат шинжийг бүр эртний түүхчид тэмдэглэсэн байдаг. Сыма Цянь энэ хуулийг: “Гурван улсын зам дуусч, ахиад л эхэлж байлаа” гэж маш гайхамшигтай томъёолсон байдаг. 31. Цит. по: Конрад Н. И. Запад и Восток. С. 76.
Энэ үзэл санааг Ибн Халдуна болон Жамбаттист Вико-оос авахуулаад О. Шпенглер болон А. Тойнби хүртэл дэмжиж байсан. Хэрэв түүнийг нийгмийн түүхэнд нийлүүлбэл буруу болно, хэрэв тодорхой улсын түүхийг боловсруулах үед хэрэглэвэл яг таарахгүй, гэхдээ мэдээж мөн чанарын засвартайгаар угсаатны нийлэгжилтийн үйл явцыг судлахад хэрэглэж болно.
Нэгдүгээрт “төгсгөл” нь “эхлэл” бий болохыг дандаа тунхагладаггүй. Угсаатнууд болон хэт угсаатны соёлууд нь хөгжлийн өмнөх орчил дуусангуут үүсдэггүй, харин заримдаа түүний төгсгөлийн дараагаар ихээхэн урт хугацааны завсарт үүсдэг. Түүхийн чанд хэмнэл олох эрмэлзэл нь баримтаар батлагдаагүй. Тухайлбал, визант угсаатан эллин-римийн цэцэглэлтийн эрин үед үүсч, тэдгээр нь хэд хэдэн зуунаар зэрэгцэн оршсон. Мусульманы хэт угсаатан византийн болон роман-германы угсаатныг шахагдахад хүргэж, үүнтэйгээ нэгэн зэрэг дундад персийн (Сасанидын Иран болон Согд) угсаатны залгисан. Харин хүн болон түрэг, түрэг болон монголчуудын хооронд тал нутагт үлдэц угсаатнууд л оршин байсан цаг хугацаагүй эрин байсан юм. Байдал нөхцөл нарийн байгаа нь илт байгаа бөгөөд угсаатны нийлэгжилтийн шалтгаан нь угсаатны түүхийн хэмнэлд байдаггүй юм байна.
Хоёрдугаарт, сэргэлт, цэцэглэлт, уналт хэмээн шатуудыг ердийн байдлаар гурав хуваан үзэх нь юунд сэргэж, юунд унаж байгаа юм бэ ? гэсэн жирийн асуултад хариу өгөхгүй юм. Амьдралын түвшин эдгээр шатуудаас үл хамааран хэлбэлзэж байдаг, соёлын цэцэглэлт нь эдийн засгийн буюу улс төрийн таатай нөхцөл байдалтай давхцдаггүй, төрийн хүчин чадал нь хөнгөн амьдралын үзүүлэлт биш: жишээлбэл, Наполеоны үед сахар, кофе, ноосон бөс ч байхгүй маш хүнд байсан. Товчоор яривал чанарын үзүүлэлтүүд нь гарцаагүй субъектив шинжтэй байдаг бөгөөд угсаатны нийлэгжилтэд хамаарах байгалийн үзэгдлийг тайлбарлах үеийг тооцоонд авч чаддаггүй. Эцэст нь тодорхой хүнд болон нийгмийн хамт олон доторх нийгмийн болон биологийн зүйлүүдийн хоорондын зааг хаана байдаг вэ ? Тэр нь нэг талаас хүний биеийн дотор байдаг, нөгөө талаас түүний хязгаараас хол оршдог. Анатоми, физиологи, рефлексологи, генетик код энэ бүхэн нь нийгмийн зүйл биш, харин биологи, биохими, биофизикийн үзэгдлүүд юм. Үүний урвуугаар төрийн харилцааны хөгжлийн шинж чанар, улс төрийн эрэлтүүд, ёс зүйн болон гоо зүйн төгс эрхэмлэлүүд зэрэг нь биологи болон газар зүйн хүчин зүйлүүдэд ордоггүй, харин нийгмийн хөгжлийн үр шим болдог. Хөгжлийн энэ хоёр шугамыг судлах явдлыг хослуулах нь хэрэв эдгээрт ландшафтын түүх, соёлын түүх зэргийг нэмж нийлүүлэх юм бол тодорхой угсаатны түүхийг нөхөн сэргээхэд хүргэх бөгөөд энэ нь л харин угсаатны түүх болно.
” ИКС ФАКТОР ” ХААНА БАЙНА ?
Одоо угсаатны нийлэгжилтийн үзэгдлийг янз бүрийн аспектаар тайлбарласан үед: эдгээр инерцийн үйл явц үүсэх шалтгаан нь юу вэ ? гэсэн асуулт тавьж болно. Хүч гаргалгүйгээр ямар ч үйлдэл болж чадахгүй болохоор хүний зан үйлд шууд нөлөөлөгч эрчим хүчний тэр төрлийг, мөн хүний сэтгэхүйд илрүүлж болох энэхүү эрчим хүчний үр нөлөөг хайх ёстой юм. Энэ бол дурын организмд хэвшмэл байх хувийн, тэр ч байтугай зүйлийн өөрийгөө хадгалах, өөрөөр хэлбэл энэ нь өөр ямар ч зүйлийн амьтанд ажиглагддаггүй, өөрийнхөө удмыг ч цэвэрлэдэг золиос хийх чадвараар илэрдэг инстинктийг даван туулахуйц хангалттай хүчтэй лугшилт байх ёстой. Өөрөө хөгжих институт, материйн хөдөлгөөний нийгмийн хэлбэр байхгүй хамтын нийгэмлэг амьтанд байдаг ч угсаатанд байдаггүй. Иймээс бидний сонирхон буй “Х хүчин зүйл” хүний сэтгэхүйн хүрээнд тусч байдаг байна. 32. Гумилев Л. Н. Этнос и категории времени //Доклады ВГО. Выл 15. Л., 1970. С. 143-157.
Угсаатныг төрүүлж, эвдлэгч хүчин зүйлсийн хайхдаа тэр нь: 1) өөрчлөгдөн буй газар зүйн орчин, 2) нийгмийн хөгжлийн хувьслын үйл явц, 3) түүхийн перитети буюу гэнэтийн явдал, 4) соёлын өсөлт буюу уналтын дэвсгэр дээр үйлчилдгийг санах хэрэгтэй. Мэдээжийн хэрэг дэвсгэрийн тоонд буй дурдсан зүйлүүдээс дурынхыг судлахад угсаатны нийлэгжилт орж ирнэ. Ингээд шинжлэх ухааны нийлбэр биш, харин дэвшүүлсэн зорилтоор тодорхойлогдох тэдгээрийн систем л дэвшүүлсэн дурын асуудлыг шийдвэрлэх, өөрөөр хэлбэл шинжлэх ухааны анализын түлхүүр болно. Чухам ийм учраас л гол санаагаа гаргах гэж угсаатны үзэгдэл, түүний алс өнгөрснөөс өвлөн авсан байгаль нийгэм, соёлын уламжлал зэрэгтэй харьцах харьцааг нь ихээхэн нуршин байж урьдчилан тоймлосон билээ.
Тодорхой хүмүүсийн зан үйлийг шинжлэх замаар энэхүү “Х хүчин зүйл”- ийг олох гэсэн бүхий л оролдлогууд бүтэлгүйдсэн нь бүрнээ ойлгомжтой юм. Юуны өмнө нэгж тохиолдлоор бид ерөнхий зүйлийн болон зүй тогтолт зүйлсээс хэсгийн болон тохиолдлын зүйлүүдийг ялгаж чадахгүй. Харин их тооны статистик хууль ажиллаж эхэлмэгц л зүй тогтлоос гажсан жижиг хазайлтууд харилцан устаж, боломжийн нэмэх хасах хазайлттай, дүр зургийг огтхон ч гажуудуулдаггүй холбооны системүүд илрэн гарч ирнэ. Гэхдээ тодорхой жишээнүүд нь зарчмыг тайлбарлахад зайлшгүй ил тод шинжтэй байдаг учраас бид тэдгээрийг үл тоомсорлож болохгүй юм. Гэхдээ үзүүлэн таниулах зүйл хэчнээн хэрэгтэй байлаа ч хэзээ ч утга санааг орлодоггүй гэдгийг санавал зохино.
ГАЗРЫН БУРХНЫ (САТУРН) ЭСРЭГ ТҮҮХИЙН БУРХАН (КЛИО)
Энэ талаар ярих юм байгаа учраас одоо түүхийн тухай ярилцъя. Үгүйсгэгч XIX зуунд төдийгүй пропан буюу юу ч үл мэдэгсэд түүхийг хоосон зугаа, сонирхолтой уншлага, баян, ажилгүйчүүдийн адайр зан, суртал нэвтрүүлгийн арга, тэр ч байтугай “өнгөрсөнд хандах улс төр” хэмээн нэрлэж байв. Саяхан түүхийг явц дундаа эрчим хүч ялгаруулдаг, агуу их болон хааяа багавтар үйлсэд хэрэглэгддэг цаг хугацааны функци мэтээр ойлгох оролдлого хийсэн байна. 33. Козырев Н. А. Причинная механика и возможность экспериментального исследования свойств времени (рукопись).
Гэхдээ энэ үзэл баримтлал нь үндэслэлгүй юм, учир нь үнэхээр цаг хугацааны дагууд явдаг түүхэн үйл явцууд нь энтропи (замбараагүй) болон инерцийн (аяараа явах) шинжтэй. Иймээс ч тэр өөрийнхөө хүүхдүүдийг залгидаг Хроносын (цаг хугацааны бурхан ) ачаар биш, харин түүний эсрэгээр үүсдэг.
Хэрэв ийм бол түүхийн шинжлэх ухаан нь цаг хугацаатай хийх тэмцэл бөгөөд үүнийг эллинчүүд өөрийн эцэг, аянга цахилгаан эзэмшигч Ураныг (тэнгэрийн бурхан) доромжилсноор Зевсийн түлхэн унагасан Сатурнаар (газар тариалангийн бурхан) төлөөлүүлэн аймшигтайгаар бурханчлан үзүүлсэн байдаг. Харин энэ цахилгаан бол манай хэлээр бол энтропийн эсрэг лугшилт болон эрчим хүч бөгөөд энэ нь үүсэхдээ Ертөнцийн энтропи болох үхлийн үйл явцыг зөрчдөг. Харин Хүч нь бол Огторгуйг Хаос (замбараагүй) болон хувирахаас авардаг хурдатгалыг өдөөгч шалтгаан юм. Энэ хүчний нэрийг Амьдрал гэдэг.
Гэхдээ Сатурны зарц анхдагч гамшгийн мөнхийн дайнд аваргууд юмуу асурууд (санскр ) юу ч алдсангүй, учир тэдэнд алдах юм юу ч байсангүй. Хронос секунд бүр тэдний дүр төрхийг хувирган өөрчилж, ингэснээрээ тэдний хувийн чанар болон шинжүүдийг алга болгож байна. Гэхдээ Сансрын үнэнч цэргүүд (паладин) – өөрийн мөн чанараараа ийм л байдаг цэгцэрсэн Орчлон хэлбэр олж, улмаар тохиолдол бүрт үл давтагдах хувь хүмүүсийг бий болгожээ. Хаостай хийх тэмцэлд тэд үхэлтэйгээ учирдаг ба үүнийг В.И. Вернадский цаг хугацаанаас орон зай тасрах гэж авч үзжээ. 34. Вернадский В. И. Химическое строение биосферы Земли и ее окружения. С.135
Микроб буюу баобаб, хүн буюу үр хөврөл аль нь бай үхсэн болгоны хувьд цаг хугацаа алга болдог, гэхдээ био хүрээний бүр организмууд өөр хоорондоо холбоотой байдаг. Нэг хүн одох нь олон хүнд гарз болно, яагаад гэвэл энэ нь амьдралын мөнхийн дайсан Хроносын ялалт юм. Гарз хохиролтой эвлэрнэ гэдэг нь байр сууриа алдана гэсэн хэрэг. Үхлийн эсрэг Ой дурдатгал босч ирдэг бөгөөд энэ нь энпропийн саад ахуй нэгэнт биш, харин ухамсар болно. Чухамхүү ой дурдатгал л цаг хугацааг өнгөрсөн, одоо, ирээдүй хэмээн хуваах бөгөөд тэдний дундаас зөвхөн өнгөрсөн нь л бодитой байдаг.
Үнэн хэрэг дээрээ одоо бол зөвхөн агшин зуур өнгөрсөн болж буй агшин юм. Ирээдүй бол байхгүй, учир нь аль нэгэн үр дагаврыг нь тодорхойлогч үйлдэл хийгдээгүй, мөн тэр нь болох эсэх нь тодорхойгүй байдаг. Нүүрлэн буй үйл явдал гэдэгт практик үнэ цэнэгүй, зөвхөн статистик оролтыг л ойлгож болно. Харин өнгөрсөн бол оршин байна, дурын ямар ч үйл хэрэг өнгөрсөн болдог учраас оршин буй бүх юм өнгөрсөн юмс болой. Ийм болохоор л түүхийн шинжлэх ухаан бидний гадна оршдог, бас бидний хажуугаар оршдог цорын ганц бодит байдлыг судалдаг.
35. Үүний эсрэг үзэл санааг Жованни Жентме: ”Өнгөрсөн үеүдэд хүмүүс төрж, бодож сэтгэж, хөдөлмөрлөж байсан…гэхдээ одоо тэд гоо үзэсгэлэнг нь бахдаж байсан цэцэг лугаа, хавар нүдэн дээр нь цэцэглэж, намар болоход шарлан унадаг навч лугаа бүгд мөхжээ. Тэдний тухай дурдатгал амьд үлддэг бөгөөд гэхдээ дурдатгал бол мөрөөдлийн ертөнц лүгээ адил юу ч биш, дурдатгал бол хүсэл мөрөөдлөөс илүү юм биш” гэж хэлсэн байдаг. (цит. по: Кон И. С. Философский идеализм и кризис буржуазной мысли. М., 1959. С. 155). Хэдийгээр тун сайхан хэлсэн боловч харамсалтай нь түүхэнд оролцож буй хүмүүс бол цэцэг, навч биш юм. Хүмүүст хүнд хэцүү байдаг, гэлээ ч гэсэн тэд илүү баялаг бөгөөд ухаантай болой.
Өнгөрсний мэдлэг нь бидний практик амьдралд ашиггүй гэж зөвхөн юу ч үл мэдэгсэд яриагүй байна. Дээр үед тийм хүмүүс домч болон астрологуудад очиж, ирээдүйн тухай мэргэлүүлдэг байжээ. Тэд нар нь ч гэсэн тэр дор нь гайхалтай үнэн таадаг байв. Яагаад үзмэрчид амжилт олсон юм бол ? Өнгөрснийг судлан, боломжит хувилбаруудыг шалган, таамаглалын нарийвчлаад ирэхлээр тухайн нөхцөл дэх хувилбаруудын тоо ямагт хязгаарлагдмал болдог. Сайн шатарчин өөрийг нь төрөхөөс бүр өмнө тоглосон олон зуун өрөг, этюдийг судлахад хүч хайрлалгүй байсныхаа ачаар өргийг олон нүүдлийн өмнөөс тооцож чаддаг. Шатрын тоглолтын түүх нь түүнд эхлээд хамгийн магадлалт, дараа нь практикийн хувьд зөв таамаглал зохиоход тусалж, үүнийхээ дараа л тэмцээн, уралдаанд хождог. Мэдэж буй өнгөрсөн нь одоод, өөрөөр хэлбэл амжилтанд шингэдэг.
Физикч буюу химичийн туршлага бүр, геологи буюу ургамал судлаачийн ажиглалтууд, онолчийн сэтгэлгээ, эдийн засагчийн тооцоо зэрэг нь бичигдмэгцээ түүхэн сурвалж буюу тэмдэглэгдсэн өнгөрсөн болж хувирдаг, эдгээр нь хэрэв ухаалаг хандвал бүдэг бараг ирээдүйд оршин буй зорилгодоо хүрэх зан үйлийн зохистой хувилбаруудыг олоход хүргэнэ.
Эцэст нь өөрийгөө болон, ертөнц дэх өөрийн байр сууриа ойлгох нь зөвхөн мөнгө олох хэрэгсэл гэж үү ? Үгүй ээ. Хүндэтгэл хүлээсэн олон хүний хувьд энэ нь зорилго байдаг. Бидний амьдран буй хотуудыг босгосон, өнөөдөр бид зүгээр л явж буй улс орнуудыг нээсэн, бидний таашаадаг зургуудыг бүтээсэн, бидний суралдцаг номуудыг бичсэн өвөг дээдэсдээ талархах нь хүний мэдрэмжээ алдаагүй байгаа хүн бүрийн үүрэг биш гэж үү ? Хойч үеийнхнийхээ төлөө амьдралаа өргөсөн баатруудаар бахархах нь хоосон бодол гэж үү ? Үгүй ээ. Түүхийн алдар бадартугай.
Гэхдээ түүх бол үнэний эрэл юм. Учир нь эртний сурвалжуудын мэдээллүүд шавхай нялсан мэт худлаар шүршигдсэн байдаг. Өнгөрснийг хий бодлоор солих, эсвэл дутуу дамжуулан гажуудуулж, эсвэл учир утгагүй нарийн ширийний хэрэггүй алтан утсаар хэрэх аваас өнгөрсөн нь бодит зүйл байхаа болих ба ингэхэд түүхэнд үнэн байхгүй, зөвхөн хувийн ойлголт байдаг , түүний үзэгдэл нь өөр хоорондын шалтгаацсан холбоо бүхий үйл явдлын хэлхээ биш , сэдвүүдийг нь зөвхөн үг үсэггүй цээжилбэл зохих, тогтоох боломжгүй, учир утгагүй зүйл , түүх бичигч түүнийгээ өөрийнхөө үеийнхэнд биш, хойч үеийнхэнд зориулж бичсэн мэт аашлах, эцэст нь угсаатны бүх нүүдэл, тэдний сэргэлт болон уналт, тэдний алдар суу болон мөхлийг сарны гэрэлд нуурын усанд загас наадах мэт санахад хүргэдэг. Хэрэв ийм байх юм бол түүхийг юунд судлах юм, байсан юм ой ухаанаас алга болохоор байхгүй зүйл болон хувирдаг, ингээд Огторгуйн байрыг Хаос эзэлнэ.
36. Түүхэн ертөнцийн анхдагч элемент нь субъект орчинтойгоо амьдралын идэвхитэй харилцан үйлчлэлд орж байгаа зовлон (переживание) юм. (цит. по: Кон И.С. Указ. соч. С. 112).
37. Янз бүрийн түвшин, янз бүрийн зай, янз бүрийн зорилго, ашиг сонирхол, янз бүрийн утга санаа, янз бүрийн үзэл бодлын үүднээс бичиж буй түүхчдээр нээлгэхийг хүлээж буй туйлын бодит шалтгаан гэж байдаггүй. цит. по: Кон И.С. Указ. соч. С. 192).
38. Энэ тухай Анатоль Франс “Оцон шувууны арал” зохиолдоо: “Бид түүхийг бичиж байна гэж үү дээ ? Бид ямар нэг сэдэв, баримт бичгээс амьдрал болон үнэний бага ч гэсэн хэлтэрхийг гарган авч чаддаг бил үү ? Бид зүгээр л сэдвүүдийг тараадаг. Бид үсгийг баримтладаг…Ингээд сэтгэлгээ оршдоггүй юм” гэж бичсэн байдаг.
39. Түүхийн үнэт чанар бол тэрээр “бидний өөрийнхөөс нэн онцгой ялгаатай нөхцөл байдалд оршин буй хүмүүсийн тухай, шинжлэх ухааны нарийн задлан шинжилгээг биш, харин нохой сонирхогч нохойныхоо тухай мэддэгтэй төстэйвтэр мэдлэг өгдөг”-т оршино. (цит. по: Кон И. С. Указ. соч. С. 176).
YII зууны Төвдөд Махаяны талын буддын номлогчид хорвоо бол хоосон зүйл, аврал нь–нирван болох явдал, түүнд хүрэх зам нь сайн ч, муу ч үйлийг эс хийх, учир нь “хар үүл ч, цагаан үүл ч биднийг нарнаас адилхан халхална” гэж сургадаг. Үүнд нь төвдийн шен, бумбын шашны мэргэдүүд ард түмэнд хандан “Махаяны чалчаа үгийг бүү сонс, аль нь хар, аль нь цагааныг зүрх сэтгэл чинь хэлж өгнө” гэж номлодог. Илт байдал хийгээд зөн билэг нь шинжлэх ухаан, урлаг хоёрын зааг дээр байдаг аж. Ийм ч болохоор түүхэнд өөрийн гэсэн Клио бурхан байдаг ажгуу.
Энд бид зөн билэг буюу илт зүйлийн заасан болгоныг Дэлхийг тойрон Нар эргэдэг мэтээр ул үндэсгүй, цаг ямагт дэмий балай мэдэгдэл болгох эрхийн тухай яриагүй байна. Худал хуурмаг нь өөрийгөө хуурахад тулгуурлаад ирэхлээрээ л боломжтой болдог. Харин Клио бол өөрийгөө шүтэгчдэд огт өөр, хавьгүй чухал зүйлээр: зөв сэдвийн баталгааг олох, анхдагч мэдээлэл цуглуулахдаа гаргасан алдаагаа илрүүлэн олох, логик байгууламжид гарсан зөрчлийн илрүүлэх зэргээр тусалдаг. Энэ бүгд нь энгийн амархан юм шиг боловч үнэн хэрэг дээрээ түүхэн үнэнд ойртсон бүх зүйл, тэр байтугай бүр өчүүхэн зүйл ч гавъяа юм.
Хамгийн энгийн дүгнэлт гэхэд л асар их сэтгэл санааны сэргэлт, мэдрэхүйн соргог байдлыг шаарддаг ба ингэхэд сэтгэлгээ хайлан цэгцэрч, эхлээд гайхашралд оруулсан, дараа нь үнэнч уншигчдыг итгүүлсэн хэлбэртэй болдог. Сэтгэлгээ ямар явцаар явсан буюу сэдэв баталсан үндэслэлүүдийг хэрхэн сонгодогт хэргийн гол нь байгаа юм биш, энэ бол мэдээж мэдвэл зохих, гэхдээ ганц мэдэх нь багадах шинжлэх ухааны урлалын гал зуух юм. Одоо яагаад заримдаа шинэ сэдвийг олж, бас баталж болдог юм бэ ? гэсэн гол асуулт тавигдаж байна. Уран бүтээлийн сэтгэл зүйн энэхүү нууцлаг байдлыг эртний грекүүд түүхийн бурхан Клиод ногдуулдаг байсан юм.
Гэхдээ нэг удаа яруу найрагч “хэзээ нэгэнтээ хэлчихсэн цаг хугацаагаар гүйж байсан аймшгийг бид яах ёстой юм бэ ?” гэж дуун алдсан байдаг. Сатурны гэзэг хэдийгээр нийгэм-эдийн засгийн формацийн зүй тогтлыг хөнддөггүй ч амьдралын дурын хувийн илрэлийг тас цохидог. Зөвхөн тэд нар л дэвшлийн замаар явна, харин өнгөрсөн зуунуудад байсан үл давтагдах, сайхан бүхэн гарз болдог. Ийм болохоор угсаатны нийлэгжилт нь хувьсал биш, харин Клиогийн оролцоотойгоор багасч байсан жагсаалт алдагдсан. Учир нь үхсэн ч гэсэн ой ухаанд бүтэн үлдсэн юм болгон мөхөшгүй билээ.
Гэхдээ Клио зөвхөн Хуурамчийн үнсэнд будагдсан, Хугацааны нурманд дарагдсан өнгөрсний үлдэглүүдийг хадгалаад байдаг юм биш. Тэрээр эдгээр махчдын олзыг булаан авдаг бөгөөд үүнийгээ бидний нүдэн дээр бидний гараар хийдэг юм. Трой хотын туурийг олсон, Вавилоны цамхгийг ухан гаргасан, Тутанхамоны бунхны эрдэнэсийг аварсан, Иван Грознийн үйлдсэн түүх бичлэгийн хуурамчийг илрүүлсэн, монголчуудын тухай хар домгийг авч хаясан. Хувь хүний биш ч бай, амь оруулах жагсаалт гарсан, гэхдээ агуу их үйл хэргүүдийг эцэс төгсгөлгүй үргэлжлүүлж болно. Тэнд ч, энд ч агуу их болон багавтар нээлтүүд хийгдэж байдаг юм.
Энэ нь Сатурныг ялсан ялалт биш гэж үү ? Энэ нь өвөг угсаатнуудыг дахин амьдруулсан явдал биш гэж үү ?
Одоо бид угсаатнууд яагаад үүсдэг, яагаад тэдний төгсгөл зайлшгүй вэ? гэсэн гол асуултаа тавьж болно.
