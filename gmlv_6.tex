Зургадугаар хэсэг

УГСААТНЫ НИЙЛЭГЖИЛТ ДЭХ ОРГИЛОХ ШИНЖ
ЭНД ТҮҮНГҮЙГЭЭР УГСААТНЫ НИЙЛЭГЖИЛТИЙН ҮЙЛ ЯВЦ ЭХЭЛДЭГГҮЙ, МӨН ЯВАГДДАГГҮЙ ШИНЖ ТЭМДГИЙН ТУХАЙ, ТҮҮНЧЛЭН УГСААТНЫ СИСТЕМ ДЭХ ТҮҮНИЙ АЧ ХОЛБОГДЛЫН ШИНЖ, МӨН ТҮҮНИЙ ГАДААД ҮЙЛЧЛЭЛИЙГ ЭСЭРГҮҮЦЭХ ЧАДВАР БОЛОН ИДЭВХИЙН ХЭМЖЭЭ БОЛОН ДҮҮРДЭГ ТЭДНИЙ ЭРЧИМ ХҮЧНИЙ ЗЭРГИЙН ТУХАЙ ӨГҮҮЛНЭ.
XXII. Угсаатны генийн шинж буюу “икс хүчин зүйл”
“ИКС ХҮЧИН ЗҮЙЛ” ЭНЭ БАЙНА
За одоо уншигчид минь миний хүлцэлийг хүлээн авна уу, Би газар зүй, биологи, угсаатны судлалын сэдвийн “ширэнгэ болон цөлөөр” та бүхэнтэй хамт удаан тэнэхдээ нууц юунд байна гэдгийг шууд хэлээгүй юм. Учир нь уншигч надад шууд итгэхгүй байхсан билээ. Тэрээр: “Энэ чинь бүрэн ойлгомжтой байна. Угсаатан нь хэл, арьс, газар зүйн орчин, нийгмийн харилцаа, өөрийн ухамсар, хувьслын үйл явц буюу тэдгээрийн бүгдийнх нь хослол, бас дээр нь дураараа сонгож болох дээр дурдсан зарим хүчин зүйлүүд юм” гэж хэлэх байсан. Энэ үзэл нь зөвхөн хуурамбуудын төдийгүй, бас олон мэргэжилтнүүдийнх бөгөөд угсаатны нийлэгжилтийн шинжилгээний практикт түүнийг хэрэглэх гэсэн тэдний оролдлого болгон бүтэлгүйдсэн юм.
Харин миний зорилт бол дурдсан хүчин зүйлүүдийн аль нь ч, мөн тэдгээрийн дурын хослол ч таамаглал гаргах боломж өгдөггүй, өөрөөр хэлбэл хэдийгээр нарийн тэмдэглэсэн баримтууд хязгааргүй биш ч гэсэн тухайн хугацаанд тодорхой байгаа угсаатны нийлэгжилтийн бүх баримтуудыг зөрчилгүйгээр тайлбарлах боломжгүйг харуулах явдал байлаа. Эндээс санал болгож буй шийдэл нь төгс биш гэсэн дүгнэлт гарч байна. Мөн түүнчлэн шийдлийн шинэ хайгуул, өөрөөр хэлбэл өвөрмөц таамаглал дэвшүүлэх эрх үүсэн гарч байгаа юм. Дурын таамаглал хүлээн зөвшөөрөгдсөн байхын тулд мэдэгдэж буй бүх баримтыг тайлбарлах ёстой. Гэхдээ таамаглалаа онол болгон хувиргах нь маш нарийн үйл явц бөгөөд энэхүү чанарын шилжилтийн агшинг тогтоох эрх нэг ч эрдэмтэнд байдаггүй. Түүний зорилт бол өөр бөгөөд өөрийнхөө үзэл бодлыг гарган тайлбарлаж, түүнийхээ үндэслэлийг өнөөгийн болон хойч үеийнхний шүүхэд толилуулах явдал мөн.
Өнөөдөр организмын түвшинд сэтгэл зүй гэдгийг хүмүүсийн зан үйлд илэрдэг гормоны үйлчлэлийг харгалзсан дээд мэдрэлийн үйл ажиллагааны физиологи хэмээн ойлгодог.
Хувийн сэтгэл зүй нь нийгмийн болон угсаатны сэтгэл зүй зэрэг дээд эрэмбийн системүүдэд ямагт ойртон нягтарч байдаг, гэхдээ манай асуудлын хувьд системийн хэмжээ нь ажил хэргийг өөрчлөхгүй. Ийм учраас манай шинжилгээний хувьд тухайлсан хүмүүсийн зан үйлийн сэдэл нь хамаагүй байдаг бөгөөд учир нь тэдгээрээс угсаатны зан үйлийн тогтсон үзэл бүрдэж байдаг.
Угсаатны сэтгэл зүй нь бүлийн түвшин дэх системийн зан үйлийн сэдлийг, өөрөөр хэлбэл организмынхаас бүтэн эрэмбээр дээгүүр, маш нарийн систем мөн л тийм системийг судалдгаараа сэтгэл зүйгээс ялгагддаг. Бидний шууд ажиглалтад угсаатны сэтгэл зүй хүрч болшгүй байдаг, гэвч угсаатны зан үйлийг амархан хүлээн авч, мэдэрч болдог.
К.Маркс, Ф.Энгельс нар: ”Өөрийнхөө ямар нэг хэрэгцээний төлөө, энэхүү хэрэгцээний эрхийн төлөө үүнийг хийлгүй л бол хэн ч ямар нэгэн юм хийж чадахгүй” гэж бичсэн байдаг. 1. Маркс К., Энгельс Ф. Соч. 2-е изд. Т. 3. С. 245.
Хүний хэрэгцээг ангилж болдог, түүний тулд санал болгосон хуваагдлын олон түвшинт олон шатат ангилал бидэнд хэрэггүй юм. 2. Симонов П. В. Высшая нервная деятельность человека. Мотивационно-эмоциональные аспекты. М., 1975. С. 27-29
Манай анализын хувьд янз бүрийн утга бүхий хоёр бүлэгт хуваахаар хязгаарлах нь зохистой юм. Нэгдэх нь: “хэрэгцээний хэрэгцээ болсон–зүйл болон хувь биеэ өөрөө хамгаалах явдлыг хангаж байдаг хэрэгцээнүүдийн бүрдэл, хоёрдахь нь “өсөлтийн хэрэгцээ” болсон дотоод зохион байгуулалтын хүндрэл болон танин мэдэгдээгүй зүйлийг оюун ухааны хувьд эзэмшсэний ачаар явагддаг өөр төрлийн сэдэл болно. 3. Там же.
Ф.М.Достоевский “Ах дүү Карамазовынхон”-доо үүнийг “танин мэдэхүйн хэрэгцээ” гэж нэрлэсэн бөгөөд учир нь “хүний ахуйн нууц нь зөвхөн амьдрахын тулд биш, харин юуны тулд амьдрахад байдаг”, чингэхдээ “бүхий л аргаар заавал бүтээхэд” оршдог гэжээ. Ийм учраас бидний угсаатны голлох зүйл гэж нэрлэмээр төгс эрмэлзэл (идеал) нийт хүнд хэрэгтэй байдаг аж. Гэхдээ энэхүү голлох буюу ноёрхох үзэл нь өөрөө аяндаа үүсдэггүй, харин угсаатны нийлэгжилтийн шатууд, өөрөөр хэлбэл хайж буй “икс хүчин зүйл”–ийн өөрчлөлттэй хамт бий болж, өөрчлөгддөг байна. Одоо бид бараг зорилгынхоо дэргэд байна.
Угсаатны нийлэгжилтийн үйл явц эхэлдэг нөхцөл нь нэн олон хувилбартай. Гэхдээ түүний хамт заримдаа гадаад үйлчлэлээр зөрчигдөж байдаг тэдгээрийн цаашид явагдах ямар нэг хэмжээний нэгдмэл байдал ажиглагддаг. Хэрэв бид даяар зүй тогтлыг нээхийг эрмэлзэж байвал үйл явцын байнгын шатлаг бүдүүвчийг ашиглаж, гадаад цэгүүдийг тохиолдлын саад гэж үл тооцох аваас дэлхийн бөмбөрцөг дээрх бүх угсаатнуудын гарал үүслийн нэгдмэл шалтгаан байна гэсэн дүгнэлтэд гарцаагүй хүрэх юм. Энэ бол мөнөөх л “икс хүчин зүйл” бөгөөд хайж буй үл хөдлөх хувилбар гэж үзэн хашилтаас нь гаргах ёстой юм.
Угсаатны нийлэгжилтийн лугшилт болсон чухам тэр хэмжигдэхүүнийг илрүүлсэн гэдэгтээ итгэлтэй байхын тулд бид түүнийг тооцохдоо дээр дурдсан a)“аминч” болон “аминч бус” гэсэн хуваагдлыг тооцсон этнологийн, b)ландшафтад харьцах харьцааг харуулах газар зүйн,c) сэргэлт болон уналтын шатуудыг дамжсан угсаатны хамтын нийгэмлэг зүй тогтлоороо мөхөхийг тодорхойлсон түүхэн гэсэн гурван ангиллыг нэг бүдүүвчид оруулж харуулах ёстой юм. Энэхүү гурван шугамын давхцал нь санал болгож буй үзэл баримтлалын зөв эсэх болон “икс хүчин зүйл” нээгдсэнийг харуулж өгөх болно.
Одоо “эмпирик нэгтгэл”-ийн зам дээр очъё. Угсаатны нийлэгжилт хэчнээн ч олон янз байлаа гэсэн түүний бүх эхлэлд ямар ямар агшнууд оролцдогийг үзье. Шинэ угсаатан төлөвших нь ямагт орчин, нийгэм буюу байгалийн өөрчлөлттэй ямагт холбогддог, зорилго чиглэлтэй үйл ажиллагаанд дотоодын хаалт саадгүй тэмүүлдэг зарим хувь хүн байдагтай холбоотой байдаг, чингэхдээ тухайн субъектын хувьд ямагт хий хоосон буюу хөнөөлтэй тавьсан зорилгодоо хүрэх явдал нь түүнд өөрийнх нь амьдралаас ч илүү төсөөлөгддөг. 4. Гумилев Л. Н. 1) Этногенез и этносфера //Природа. 1970. № 1. С. 46-56; № 2. С. 43-50; 2) Этногенез – природный процесс //Там же. 1971. № 2. С. 80-82.; 3) О соотношении природы и общества согласно данным исторической географии и этнологии //Вестник ЛГУ. 1970. № 24. С. 39-49.
Ийм гарцаагүй ховор тохиолддог үзэгдэл нь зүйлийн зан үйлийн хэм хэмжээнээс гажсан хэрэг бөгөөд учир нь дурдсан лугшилт нь өөрийгөө хамгаалах инстинктийн эсрэг байр сууринд, улмаар сөрөг тэмдэгтэй байна. Энэ нь хэтэрхий болон дундаж чадвартай (авъяас) холбоотой байж болно, энэ нь түүнийг сэтгэл зүйд байдаг зан үйлийн бусад лугшилтын дундаас бие даасан шинжтэй байгааг харуулж байна. Энэ шинж тэмдгийг одоо болтол хэн ч, хэзээ ч дүрсэлж, шинжилж байгаагүй юм. Гэхдээ чухамхүү энэ нь буруугаар ойлгогдсон ч бай хамт олны ашиг сонирхлыг амьдралд тэмүүлэх, удам угсаагаа халамжлахаас нь ч даван гаргадаг аминч бус ёс зүйн үндэс болдог. Энэхүү шинж тэмдгийг эзэмшсэн хүн өөртөө таатай нөхцөлд нийлэн нийлсээр уламжлалын инерцийг эвдэх, шинэ угсаатан үүсгэх үйлдэл хийдэг (бас хийхгүй ч байж болно).
Энэхүү генетик шинж тэмдгээр үүссэн онцлогийг эртнээс харж байсан, түүнээс гадна энэ үзэгдэл тэвчишгүй хүслэн гэдэг нь тодорхой, харин ердийн хэллэгээр бол дурын хүчтэй хүсэл эрмэлзлэлийг, шог байдлаар бол зүгээр л дурын, тэр ч байтугай сулавтар сонирхлыг ч хэлдэг байна. Ийм учраас шинжлэх ухааны анализ хийхийн тулд бид пассионар шинж (латины Passio, ionis, f гэснээс гаралтай) шинэ нэр томъёо санал болгож байна. 5. Энэ нэр томъёоны англи дүйцэл нь “drive” болно. ( үз: Soviet Geography. 1973. Vol. XIV. № 5. P. 322).
Гэхдээ түүнийг хэдийгээр пассионар ч гэсэн сэтгэл мэдрэлийн өвчинтэй адил эх захгүй сэтгэхүйн шинж тэмдэг болсон аминч ёс зүй, олон занг өдөөгч амьтны инстинктийн агуулгаас нь салгаж, мэдээж зүйлийн хэм хэмжээнээс гажих нь огтхон ч гаж өвчин биш гэдгийг тэмдэглэж байна. Цаашдаа бид түүний биет үндсийг зааж, “пассионар” гэсэн ойлголтын агуулгыг нарийвчлан үзэх болно.
(Орчуулагчийн тэмдэглэл: Энэхүү пассионар гэх нэр томъёог Л.Н.Гумилев шинээр зохиосон болно. Энэ нь английн drive буюу урам зориг, сонирхол, шаардлага гэсэн үгүүдтэй, оросын страсть, сильное желание, влечение гэсэн үгүүдтэй харьцангуйгаар дүйж байгаа юм. Бид агуулгыг нь удаан бодсоны эцэст пассионар гэсэн энэхүү нэр томъёог “оргилох шинж” гэсэн монгол үгээр орчуулахаар шийдэв. Яг таг оносон эсэхээ мэдэхгүй хэдий ч миний уншсан мэдсэн хэмжээнд өөр тохирох үг одоохондоо олдохгүй байна. Ямар гэсэн монголын ард түмэн тухайн үедээ түүхийн тавцанд ёстой оргилон гарч ирсэн юм. Иймд цаашдаа “оргилох шинж”гэсэн монгол нэр томъёог хэрэглэх болно.)
Ф.ЭНГЕЛЬС ХҮНИЙ ТЭМҮҮЛЛИЙН ҮҮРГИЙН ТУХАЙ
Энгельс хүний тэмүүллийн хүч, түүний түүхэнд гүйцэтгэх үүргийн тухайд “аль дивангалавийн овгийн нийгэм хамгийн алслагдсан хэмжээнд хүртэл өсч чадахгүй тийм их зүйлийг соёл иргэншил хийсэн юм. Тэрээр хүмүүсийн хамгийн өөрчлөлтгүй сэдэл, тэмүүллийг хөдөлгөөнд оруулж, хүмүүсийн үлдсэн бүх унаган зүйлст хортойгоор түүнийг хөгжүүлэн байж энэ бүхнийг хийсэн юм. Дорой шунаг зан соёл иргэншил үүссэнээс өнөөдрийг хүртэл түүнийг хөдөлгөгч хүчин болж, баялаг, бас дахин баялаг, гурав дахин баялаг, баялаг бол нийгэм биш гэсэн энэ л зүйл энэхүү орь ганц өрөвдөлтэй хувь хүнийг тодорхойлогч цорын ганц зорилго болж байлаа. Хэрэв энэхүү нийгмийн хөрсөнд шинжлэх ухаан улам бүр хөгжиж, урлагийн дээд цэцэглэлтийн үеүд давтагдаж байсан нь зөвхөн баялаг хуримтлуулах салбарт манай үеийн бүхий л ололт амжилт боломжгүй болох байснаас болсон юм.” гэж яруу тод өгүүлсэн байдаг. 6. Маркс К., Энгельс Ф. Соч. Т. 21. С. 176.
Энэхүү санаа нь Энгельсийн “Өрх гэр ,хувийн өмч, төрийн үүсэл” зохиолынх нь гол хэлхээс болсон байдаг. Тэрээр чухамхүү “баялагт шунах эрмэлзэл” нь антагонист анги үүсэхэд хүргэсэн гэдгийг заасан юм. 7. Там же. С. 165.
Нийгэм дэх овгийн байгуулал унах тухай ярихдаа (бидний бодлоор гомеостазийн шатандаа буй угсаатнууд оршин байсан тэр нийгэмд ) Ф.Энгельс: ”Энэхүү хүй нэгдлийн нийтлэгийн эрх мэдэл эвдрэх ёстой байсан–ингээд ч тэр эвдэрсэн. Гэхдээ тэр нь хуучны овгийн нийгмийн ёс суртахууны өндөр түвшинтэй харьцуулахад бидний шууд төсөөлж байгаагаар нүгэлд автсан уналтын тийм нөлөөгөөр эвдэрсэн билээ. Бүдүүлэг шунал, таашаал авах бүдүүлэг тэмүүлэл, бохир харамч зан, нийтийн хүртээлийг дээрэмдэх эрмэлзэл зэрэг хамгийн дорд хар санаа шинэ соёлжсон ангит нийгмийг сэргээн авагч болсон бөгөөд хулгай, хүчирхийлэл, хар санаа, урвалт зэрэг хамгийн жигшүүрт хэрэгслүүд нь хуучны ангигүй овгийн нийгмийг тамирдуулж, түүнийг мөхөлд хүргэсэн юм” гэж бичжээ. 8. Там же. С. 99.
Хүн төрөлхтний дэвшилтэт хөгжлийг Энгельс ингэж үзсэн юм. Шунал бол ухамсаргүйн хүрээнд суурилан байдаг сэтгэл хөдлөл, физиологи болон сэтгэл зүйн зааг дээр байдаг дээд мэдрэлийн үйл ажиллагааны функци билээ. Үүнтэй адил зүйл нь Энгельс дурдсан шунал, таашаалд тэмүүлэх, харамч зан, хомхой сэтгэл зэрэгтэй дүйцэх сэтгэл хөдлөл нь эрх мэдэлд дурлах, алдар гавъяанд шунах, атаархал, ихэмсэг зан зэргүүд болно. Бэртэгчин байр сууринаас энэ нь “тэнэг мэдрэхүй”, харин гүн ухааны байр сууринаас бол “тэнэг” буюу “сайн” гэдэг нь ухамсартай бөгөөд чөлөөтэй сонгосон үйлдлүүдийн зөвхөн сэдэл байж болно, харин сэтгэл хөдлөл нь ямар үйлдлээр төрснөөсөө хамаарч зөвхөн “тааламжтай”, “тааламжгүй” байж болно. Харин үйлдлүүд нь туйлын олон янз, түүний дотор хамт олонд жинхнээсээ ашигтай байж болно. Жишээлбэл, нэр алдарт дурлах нь жүжигчин үзэгчдийн талархал хүлээх, мөн өөрийнхөө авъяасыг төгөлдөржүүлэхэд хүргэдэг. Эрх мэдэлд дурлах нь улс төрийн зүтгэлтний идэвхийг урамшуулж, тэр нь төрийн шийдвэрт хэрэгтэй байж болно. Мөн шунал нь материал баялаг хуримтлуулахад хүргэх гэх мэт. Гэхдээ оргилох чанарын хэвүүд болсон эдгээр бүх мэдрэхүй нь бараг бүх хүнд хэвшмэл байх бөгөөд гэхдээ туйлын олон янзын тунгаар байдаг. Оргилох чанар нь зан араншингийн хамгийн олон янзын шинжээр илэрдэг ба гавъяа болон гэмт хэрэг, бүтээн байгуулалт болон эвдэн сүйтгэлт, сайн болон мууг яг адил амархан төрүүлдэг, гэхдээ зүгээр суух, амгалан тайван байх зэрэгт л байр үлдээдэггүй.
Философийн түүхийн лекцендээ Гегель мөн л ийм эрс тэс: ”Өөрийнхөө үйл ажиллагаагаар оролцоогүй тэрхүү хүмүүсийн ашиг сонирхолгүйгээр ерөөсөө юу ч хэрэгждэггүйг бид баталж байна. Хувь хүн өөрт нь байсан, байж болох бусад ашиг сонирхол, зорилгоо хаяж орхиод өөрийгөө бүхэлд нь түүндээ өгч, өөрийнхөө бүхий л хүч, хэрэгцээг энэхүү зорилгод төвлөрүүлдэг болохоор хэрэгцээг бид тэмүүлэл гэж нэрлэж байна. Дээрээс нь бид энэхүү тэмүүлэлгүйгээр ертөнц дээр ямар ч агуу их зүйл хийгддэггүй гэж шууд хэлэх ёстой”. 9. Гегель Ф. Соч.; В 14 т. Т. 8. М., 1935, С. 23.
Энд иш татсан нийгэм сэтгэл зүйн механизмын дүгнэлт нь түүний бүхий л гоёмсог байдлыг эс харгалзвал багагүй доголдолтой юм. Гегель тэмүүллийг “ашиг сонирхол”-той адилтгаж байна, гэхдээ XIX зуунд энэ үгийг анхнаасаа “өөрийгөө золиослох боломжийг үгүйсгэсэн материал баялаг хуримтлуулах эрмэлзэл” гэж ойлгодог байсан юм. Гегелийн зарим залгамжлагчид түүхэн хүмүүсийн зан үйлийн сэдлээс өөрийнхөө тэмүүлсэн зүйлийн төлөө үнэн сэтгэлээсээ зүтгэх, амь хайргүй золиос болсон зэргийг хасах болсон нь тохиолдллын хэрэг биш юм. Харамсалтай нь нийтлэг төөрөгдөл болсон ийм бүдүүлэгжүүлэлт нь немцийн гүн ухаантны буруу томъёоллоос гарсан билээ.
Гэхдээ марксизмын сонгодгууд энэхүү заагийг давж чадсан юм. Хүмүүсийн бүхий л үйлдлүүдийг зөвхөн явцгүй аминч байдал хэмээн хардаг бэртэгчингүүдийн дайчин улиг болсон үзлийн хариуд тэд хүний сэтгэхүйн янз бүрийн илрэлд байр суурь үлдээсэн дам шалтгаацлын үзэл баримтлалыг дэвшүүлэн тавьсан юм.
Ф.Энгельсээс 1890 оны 9 дүгээр сарын 21–22–ны захидалдаа бичсэн: “Түүхийг ойлгох материалист ойлголтын ёсоор түүхэн үйл явц дахь тодорхойлогч агшин нь эцсийн бүлэгт бодит амьдралын үйлдвэрлэл, нөхөн үйлдвэрлэл байдаг. Би ч тэр, Маркс ч тэр үүнээс илүүг хэзээ ч батлаагүй. Хэрэв хэн нэг нь энэхүү үндэслэлийг эдийн засгийн агшин нь цорын ганц тодорхойлогч зүйл мэт утгаар гуйвуулах аваас тэр нь юу ч өгүүлэхгүй, хийсвэр, утгагүй ишлэл болон хувирна” гэсэн үгийг дахин нэг саная. 10. Маркс К., Энгельс Ф. Соч. Т. 37. С. 394
Үнэхээр үзэл санаа бол шинэ шинэ үйлсэд татан дуудах шөнийн гал мөн, харин хөдөлгөөн болон бүтээл туурвилыг хавчин хяхагч төмөр хүзүүвч биш. Өмнөх үеийнхнээ хүндэтгэх нь тэдний гавъяаг үргэлжлүүлэхийн тулд байдаг болохоос биш, тэд юуны төлөө үүнийг хийснийг мартахад байдаггүй.
XXII. Оргилон гарагчдын дүр төрх
НАПОЛЕОН
Их бууны бага дарга Напалеон Бонапарт залуудаа ядуу байсан ба алба ахихийг мөрөөддөг байжээ. Энэ нь улигт зүйл ч гэсэн ойлгомжтой байв. Огюстен Робеспьертэй тогтоосон хувийн холбооныхоо ачаар тэр ахмад болон дэвшиж, Тулоныг эзлэн авсны дараа үүнийхээ үр дүнд генерал болж, 1795 оны 10 дугаар сард Парижид гарсан роялистуудын бослогыг дарсан байна. Түүний алба тушаал хангагдсан боловч гоо бүсгүй Жозефина Богарнетай гэрлэсэнтэй нь түүнд эд баялаг авчирсангүй. Гэхдээ италийн аян дайн Бонапартыг нэгэнт баян болгов. Ингээд тэр үлдсэн амьдралаа хөдөлмөрлөлгүйгээр дуусгаж болох болов. Гэтэл ямар нэг юм түүнийг Египет рүү татаж, дараа нь брюмерийн 18 оны эрчимтэй эрсдэл рүү түлхжээ. Юу тэр вэ? Эрх мэдэлд дурлав уу гэтэл биш байв. Тэрээр францын эзэн хаан болоод тайвширсан гэж үү? Үгүй ээ. Тэр испанийн дайн, Москва руу хийсэн аян дайн шиг францын хөрөнгөтний жинхэнэ ашиг сонирхлоор огтхон ч нөхцөлдөөгүй дайн, дипломат, хууль тогтоох ажил, тэр ч байтугай үйлдвэрүүд зэрэг хэмжээ хязгааргүй хүнд ачааг өөртөө үүрсэн юм.
Напалеон өөрийнхөө үйлдлийн сэдлийг тухай бүр янз янзаар тайлбарладаг байсан нь мэдээж, гэхдээ түүний жинхэнэ сурвалж нь зөвхөн ажилгүй байж чадахгүй байснаасаа болж намтраа бичиж байсан Ариун Елана арал дээр ч түүнийг орхиогүй санаанд багтамгүй үйл ажиллагааны шунал тачъяал байлаа. Өнөөгийнхний хувьд Напалеоны үйл ажиллагааны өдөөлт нь ойлгомжгүй хэвээр үлдсэн юм. 1814 онд оросын арми Парижид орж ирэхэд парижийн хөрөнгөтнүүд “Бид дайныг хүсэхгүй байна, бид худалдаа хийхийг хүсч байна” гэж хашгиран баяр хүргэж байсан нь талаар хэрэг биш юм.
Цэцэглэн буй францын капитализмын нийгмийн захиалгыг биелүүлж байсан хөрөнгөтөн–хаан Луи Филипп Англитай дайн хийх уламжлалыг таслан зогсоож, өөрийн дайчин албатуудын үйл ажиллагааг Алжирт шилжүүлсэн нь энх тайван, амгаланг хүсч байсан францчуудын олонхийг хөндөлгүй, илүү ашигтай бөгөөд аюулгүй байсан нь үнэн юм. Гэтэл яагаад Наполеон Амьены энх тайвны дараа ингэж үйлдээгүй юм бэ? Учир нь тэр Луи Филипп биш байлаа, тэгээд ч парижийн лангууны худалдаачид түүнд юу ч тушааж чадахгүй байсан юм. Тэд зөвхөн эзэн хаан нь яагаад зөвхөн мөнхөд байлдахыг эрмэлдээд байдгийг гайхаж байсан юм. Үүнтэй яг адилаар булаан эзлэгч-хааны ойрын хамтрагчид гэж нэрлэгдсэн “найз нар” нь ч хүртэл Александр Македонскийг ойлгодоггүй байв.
АЛЕКСАНДР МАКЕДОНСКИЙ
Александр Македонский хоол хүнс, орон гэр, зугаа цэнгэл, тэр ч байтугай Аристотельтэй ярих гээд л хүнд хэрэгтэй юм бүрийг төрснөөсөө авахуулаад эдлэх эрхтэй байлаа. Гэсэн хэдий ч тэр зөвхөн түүний Перстэй хийх дайнд нь туслаагүйн учир Беоти, Иллири, Фраки руу зүтгэж, мөн үүнийхээ зэрэгцээ грекүүд хүртэл аль хэдийнээ мартчихсан байсан грек–персийн дайны үед персүүдийн учруулсан хохирлын төлөө тэр өс авахыг хүссэн аж. 11. Арриан. Поход Александра /Пер. М. Е. Сергиенко. М.; Л., 1962. II. 14.4; III. 18.12 – Далее сноски на это издание даются в тексте.
Тэр персүүдийг ялсныхаа дараа Дундад Ази, Энэтхэг рүү довтолж, энэхүү сүүлчийнх нь учир утгагүй дайнд македончууд өөрсдөө ч хүртэл дургүйцэж байжээ. Порын дэргэд гайхамшигт ялалт байгуулсных нь дараа “дуулгавартай хүмүүс нь зөвхөн үүнд оролцсоныхоо төлөө уйлж, бусад нь Александрыг цаашид дагахгүй гэж хатуу мэдэгджээ” (Арриан Y.26). Эцэст нь Полемократын хүү Кен “Хэчнээн македончууд, эллинчууд чамтай цуг яваад, хэчнээн нь үлдсэнийг чи өөрөө харж байна. Чиний үндэслэсэн хотуудад үлдсэн эллинчүүд, тэнд үлдсэн нь ч сайн дураараа ч үлдээгүй…Зарим нь тулалдаанд үрэгдэж, зарим нь…Азийн хаа нэгтэйгээр сарнижээ. Бүр ч олон хүн өвчнөөр үхэж, цөөхөн хүн үлдлээ, тэдэнд өмнөх хүч нь нэгэнт алга, сэтгэл санаагаар бол тэд бүр ч илүү ядарч байна. Эцэг эхчүүдтэй нэг нь тэднийгээ санаж байна, эхнэр хүүхдээ санаж байна, төрөлх нутгаа санаж байна. Тэдний энэ гансралыг уучилж болно. Тэд ядуугаараа дайнд явж, чамаар өндийлгөсөн тэд нэртэй баян хүмүүс болцгоон тэднийгээ харахаар шунаж байна. Тэдний хүсэл зоригийн эсрэг цэрэг байх гэж үү?“ (Арриан. Y.27). Энэ бол цэргүүдийнхээ сэтгэл санааг тооцож илэрхийлсэн ухаантай, ажил хэрэгч хүний үзэл бодол юм. Бодит улс төрийн бодлогын бүхий л бодлоор Кенийн оюун ухаан биш, тэр өөрөө зөв байсныг зөвшөөрөхгүй байж болохгүй юм, харин Александрын бодит бус зан үйл бидний “эллинизм” гэж нэрлэдэг тэрхүү үзэгдэл үүсэхэд болон мөн Ойрхи Дорнодын угсаатны нийлэгжилтэд чухал үүрэг гүйцэтгэсэн нь ямар ч эргэлзээ төрүүлэхгүй. 12. Эллинизм гэдэгт Александрын аян дайны үр дүнд эллиний элементүүд дорнодынхтой холилдоход үүссэн соёлыг хэлж заншжээ.
Үүнтэй холбоотойгоор хааны өөрийнх нь хэлсэн үг, цэргүүдээ аян дайнаа цаашид үргэлжлүүлэхийг ятгасан шалтгаан нь маш сонирхолтой юм. Өөрийнхөө байлдан дагуулалтыг тоочоод Александр: “Агуу их зорилгын төлөө аюул, хөдөлмөрийг тэвчдэг хүмүүст алдар гавъяагаар амьдрах буюу мөхөшгүй алдрыг үлдээгээд үхэх жаргалтай. Хэрэв бид Македондоо суугаад байсан бол ийм агуу их, сайн сайхныг хийж чадах байсан уу? Өөрийнхөө газрыг хадгалж, бидэнд дайсагнагч хөршүүдээ зөвхөн хөөгөөд л бид тайван амьдарч ханалаа гэж үзэх үү? (Арриан. Y.26 -27) Алдар нэрийг хувийнхаа сайн сайхан байдал, улс орныхоо ашиг сонирхлоос дээгүүр тавьсан хүний хөтөлбөр энэ байна. Чингэхдээ “тэр өөрөө хувийн жаргалаа үгүйсгэж, хувийн цэнгэлдээ зориулсан мөнгөнд маш харамч байсан, гэхдээ сайн үйлд өгөөмөр гараараа асгадаг байв” (Арриан. YII. 29). Таашаал авахын тулд л дайнд явдаггүй шүү дээ. Түүний цэргүүд индусуудтай байлдахыг үнэхээр хүсэхгүй байлаа, тэр тусмаа дээрэмдсэн баялгаа тэр үеийн тээврийн хэрэгслээр гэртээ хүргэх боломжгүй байлаа. Гэхдээ л тэд байлдсан юм, яриангүй сайн байлдсан гээч.
Македоны хааныг аян дайнд түлхсэн шалтгааныг худалдааны хотууддаа зах зээл олж авах буюу финикуудын өрсөлдөөнийг устгахыг эрмэлзсэнээс хайж олох нь юу л бол. Зэвсгийн хүчээр дөнгөж эзлэгдсэн Афин болон Коринф нь Македоны дайсан хэвээрээ үлдэв, дайсныхаа тулд өөрийгөө золиослох нь угаасаа утга учиргүй юм. Иймээс Александрын зан үйлийн сэдлийг түүний хувийн зан араншингаас хайх хэрэгтэй. Александрыг туйлшралд хүргэдэг хоёр чанарыг Арриан, Плутарх нар: нэр төрд дурлах болон бахархал гэж тэмдэглэсэн байдаг. Өөрөөр хэлбэл энэ нь бидний үзэж буй оргилох шинжийн илрэл болно. Энэхүү илүүдэл эрчим хүч нь зөвхөн ялалтад төдийгүй, өөрийн албатуудыг тэдэнд огт хэрэггүй дайн явуулахыг албадахад хангалттай байсан байна.
Мэдээж хэрэг Александрын Пердикк, Клит, Селевк, Птометей болон бусад хамтрагчид оргилох чанартай байсан бөгөөд хааныхаа ажил хэрэгт чин сэтгэлээсээ хамтран оролцож байлаа. Үүний ачаар л жирийн македон, грекчүүдийг аян дайнд татан оруулж чадсан юм. Македоны армийн бүрэлдэхүүнд байсан нэг хүн биш, харин оргилуун хүмүүсийн бүхэл бүтэн бүлгүүд персийн хаант улсыг нурааж, түүний оронд македоны хэд хэдэн хаант улс, тэр ч байтугай сири гэсэн шинэ угсаатан бий болгож чадсан юм. Македончууд болон персүүд нь өөрсдөө шинэ нөхцөлд танигдахгүй болтлоо хувиран өөрчлөгдөж, рим болон парфянчуудын олз болсон билээ.
Чухамхүү Элладыг Дорно дахинтай хослуулах үзэл санаа Александрыг алдар гавъяа руу түлхсэн юм биш байгаа? Үгүй, тэр Аристотелийн гүн ухаанд суралцсан, багш нь түүнд ийм зүйл зааж өгөөгүй. Цаг хугацааны дарааллаар бол энэ үзэл санаа Персийг булаан эзлэхээс өмнө биш, хойно үүссэн, өөрөөр байсан бол Персеполийн ордныг шатаахгүй байх байсан юм. Ялагдсан ард түмний урлагийн сорыг устгаж, зөвшилцөл хайж байгаагүй юм.
Ийнхүү оргилох чанар гэдэг нь орчноо өөрчлөх чадвар болон эрмэлзэл бөгөөд физикийн хэлээр хөрвүүлбэл, орчны агрегат төлөв байдлын инерци алдагдаж буй хэрэг юм. Оргилох чанарын лугшилт нь энэхүү шинжийг агуулагчийг асар хүчтэй болгодог ба оргилуун хүмүүс үйлдлийнхээ үр дагаврыг өөрөө тооцоход хүргэж чаддаггүй. Энэ нь маш чухал байдал бөгөөд оргилох чанар нь ухамсрын биш, харин мэдрэлийн үйл ажиллагааны үндсэн хуулийн өвөрмөц байдлаар илэрхийлэгдэх чухал шинж, ухамсаргүйн зүйл болж байдгийг харуулж байна. Оргилох чанарын зэрэг янз бүр, гэхдээ түүнийг үзэгдэх, түүхэнд тэмдэглэгдэх илрэл болгохын тулд оргилуун хүмүүс олон байх хэрэгтэй. Өөрөөр хэлбэл энэ шинж нь хувийн төдийгүй, бас бүлийн шинжтэй байдаг.
ЛЮЦИЙ КОРНЕЛИЙ СУЛЛА
Илрүүлж олсон энэ шинж тэмдгийнхээ зөв эсэхийг хэд хэдэн өөр хүн дээр шалгаж үзье. Римийн патриций (язгууртан) дээд гаралтан Люций Корнелий Сулла Римд байшинтай, хотын захад харштай, олон боол, захиалагчидтай байсан. Александрын адилаар тэр хоол хүнсээр ч, зугаа цэнгэлээр ч дутагдаж байсангүй. Юу түүний цэргийг үзэн ядаж жигшдэг байсан Мария руу түлхэж, амьдралаа эрсдүүлэн байж тулалдаанд оролцон Югуртыг барин авч, түүнийгээ Римд авчран Мамертиний шоронд өлсөж үхтэл нь зовоох болов оо? Энэ бүх гавъяаныхаа төлөө тэр зөвхөн ганцхан шагнал авсан бөгөөд тэр нь талбайгаар ганхаж, найз нартайгаа дэмий чалчиж, Марийг утгагүй тэнэг, өөрийгөө баатар хэмээн нэрлэж чадах байв. Үүнд олон хүн итгэж байсан боловч бүгдээрээ биш байв. Тэгэхэд нь Сулла ахин зодоонд орж, италид нэвтрэн орсон бүдүүлэгчүүдийн удирдагчтай тулалдан дийлж, түүнийг алаад … улам бүр онгирох болов. Гэвч энэ түүнд багадаж байлаа. Тэр Марияг даван гарсан мэт боловч Александрын тухай дурдатгал үлдсэн байна. Ингээд Сулла Дорно дахиныг номхотгох, өөрийгөө македоны хаанаас илүү алдаршуулахаар шийдвэрлэв. Түүнд “Одоо болоо, бусдад ажиллах боломж олго” гэж хэлцгээжээ. Римийн бүгд найрамдах улсын өмнө байгуулсан түүний гавъяаг хүлээн зөвшөөрсөн, гэр нь бүх юмаар дүүрэн, эргэн тойронд бүгд түүнийг хүндэлж, цэнгэ, жарга хэмээн шагширч байхад Суллагийн сэтгэл ханах ёстой баймаар. Гэхдээ Сулла өөрөөр явжээ. Тэр легионуудын үймүүлж, төрсөн хотоо дайран эзэлж, чингэхдээ хамтран зүтгэгчдээ урамшуулахын тулд хаалтанд дээр дуулгагүй гарч, түүнийг ээлжит нүсэр дайнд явуулахад хүргэж чадсан. Түүнийг юу гээч нь түлхээ вэ? Ашиг олох эрмэлзэл байгаагүй нь ойлгомжтой. Гэхдээ бидний үзлийн үүднээс оргилох шинжийн дотоод дарамт нь өөрийгөө хадгалах зөн билэг, хууль хүндэтгэх, өөрийг нь хүмүүжүүлсэн соёл, зан заншлаас илүү хүчтэй байдаг.
Цаашаагаа бол үйл явдлын логикийн энгийн хөгжил эхлэх бөгөөд А.С.Пушкиний үеэс түүнийг “юмсын хүч” (ихээхэн мартагдсан нэр томъёо) гэж нэрлэж байсан. Энэ нь нэгэнт этнологоор бэхжиж байдаг түүхийн шинжлэх ухаанд бүрэн хамаарна. НТӨ 87 онд Мария эрх чөлөөг амлан боолчууд болон ахмад дайчдын цэргээр Суллын эсрэг дайтав. Түүнийг популяр–италикууд буюу дарлагдсан угсаатнуудыг талдаа татсан консул Цинн дэмжсэн юм. Мария Римийг эзлэн авмагцаа өөрийнхөө хамгийн энэрэнгүй жанжингуудын нэгэнд цаашид тэдгээрт тулгуурлах нь нэр хүндийг нь гутаасан учир боол цэргүүдийг устгахыг тушаажээ. Ингээд 4 мянган хүн нойрон дундаа өөрийн дайчин нөхдүүдэд хэрчүүлжээ. Энэ яргалал нь популярууд ардчилсан зүйлсийг тунхагладаг хэдий ч өөрийн дайсан оптиматуудаас бага ялгарахыг харуулж байна.
Гэхдээ ямар ч гэсэн ялгаа байсан юм. Сулла мөн л 10 мянган боолчууд бүхий өөрийн цэргийг дайчилсан, гэхдээ ялалтын дараа тэднийг хэсэг газар, римийн иргэншлээр шагнасан юм. Мария, Сулла хоёрын ялгаа нь намын хөтөлбөрүүд гэхээсээ тэдний хувийн шинж чанараар илүүтэй тодорхойлогдож байв. Мөн Александраас ялгаатай нь Сулла нэр төрд дуртай, бардам байгаагүй, учир нь тэрээр өөртөө хангалттай гэдгийг мэдэрмэгцээ эрх мэдлээс татгалзсан. Сулла туйлын нэрэлхүү, ихэрхэг, атаархүү зантай байсан ба гэхдээ эдгээр чанарууд нь зөвхөн оргилох шинжийн илрэл болно. Суллын амжилт нь түүний хувийн шинж чанараас төдийгүй, хүрээлэлтэйгээ тогтоосон харилцаанаас хамаарч байсан гэдгийг дахин онцлон дурдаж байна. Түүний офицерүүд Помпей, Лукулл, Красс тэр ч байтугай зарим легионерүүд нь бас л оргилуун хүмүүс байсан ба удирдагчтайгаа нэг дуугаар ажиллаж, мэдэрдэг байсан юм. Үүнээс өөрөөр байсан бол Сулла Римийн дарангуйлагч болохгүй байхсан билээ.
ЯН ГУС, ЖАННА Д’АРК ба ПРОТОПОП АВВАКУМ
Оргилуун хүмүүс нь дотно хүмүүсээ өөрийнхөө тэмүүллийн золиос болгоод байдаггүй, харин тэднийгээ аврахын тулд, мөн үзэл санааны төлөө өөрийгөө золиослох нь ч бас байдаг. Чин сэтгэлээсээ ийн үйлдэхүйн жишээг Прагийн их сургуулийн профессор Ян Гус үзүүлсэн бөгөөд тэрээр Чехийн хаант улсад чехүүд хууль ёсоор “…байгалийн шаардлагаар Францдаа францчууд нь, немцүүд нь өөрийнхөө газарт байгаатай адилаар нэгдүгээр албан тушаалд байх ёстой гэдгийг хэлж байсан, хэлж ч байна” гэж өгүүлсэн юм. Хэрэв Жижка, Прокопын ах дүү нар, Прагийн их сургуулийн оюутнууд, хотынхон, рыцариуд, тариачид, санваартнууд Шинэ хотын (Прагийн дүүрэг) удирдах байгууллагын (ратуша) цонхоор Люксембургын улсуудын урагшгүй хаан IY Вацлавын бургомиср (хотын дарга) болон зөвлөхүүдийг чулуудаагүй бол Констанц дахь Гусын золиос үр ашиггүй байхсан билээ. Тэд немцүүдэд баригдаж, шатаагдсан ректорыгаа шударга бусаар шийтгэсэнд уур хилэн нь дүрэлзэж, өш хонзонгоо авсан юм.
Наполеон, Александр Македонский бас Люции Корнелии Сулла нарын энд авсан жишээнүүдэд “олон түмнийг дагуулагч баатруудыг” олж харах маш их сонирхол бий, гэхдээ үйл явдлын адилтгам хослолын үед гол асуудал хувийн “баатарлаг” явдалд биш, харин системийн оргилох шинжийг зохион байгуулж, түүнийг тавьсан зорилгодоо чиглүүлдэг угсаатны ноёрхох зүйлийг бүтээхэд байдаг байна. Баатарлаг, эх оронч жолоодогч харгис дайснаас өөрийгөө болон гэр бүлээ хамгаалахын тулд иргэдээ гартаа зэвсэг барин тэмцэхэд сэрээж сэргэж чадаагүй олон тохиолдол байдаг шүү дээ. 1204 онд загалмайтнуудын эсрэг Константинополийн хананд тулалдсан Алексей Мурзуфлаг санахад л хангалттай. Алексейн дэргэд гаднын хөлсний цэргийн ганц баг, хэдхэн зуун сайн дурынхан байсан бөгөөд тэд бүгдээрээ үрэгдсэн юм. Харин Константинополийн 400 мянган иргэн загалмайтнууд хотыг нь шатааж, дээрэмдэхийг харж л байлаа. Жолоодогчийн үүрэг болон оргилох шинжийн түвшнээр тодорхойлогдох угсаатны боломж хоёрын ялгаа энд л байгаа юм.
41 онд Римд болсон үйл явдал бүр ч илүү сургамжтай юм. Августын тогтоосон дэглэм бүгд найрамдах улсын бүх хуулийг хан хүүгийн дурын авирыг халхалсан хуурамч зүйл, хоосон үзүүлэн болгон хувиргасан билээ. Принципат (хаан) Тиберийн үед, ялангуяа Калигулийн (энэ нь Гай хаан бөгөөд байнга өмсөж явдаг цэрэг гутлынхаа нэрээр энэхүү хочоор алдаршжээ-Орч) үед баян хүмүүсийг хэрцгийлэн дарлах явдал моодонд орж, эзэн хааны санг дүүргэдэг байв. Түүнээс гадна Калигула хий харах унадаг өвчинд нэрвэгдэж, энэ үедээ нүдэнд нь өртсөн буюу тохиолдлоор санаанд нь орсон дурын хүнийг удаанаар алахыг тушаадаг байжээ. Бүгд найрамдах улсын үед ийм юмыг сэтгэж ч зүрхлэх хүнгүй байсан, иргэний дайнууд нь маш олон оргилуун хүмүүсийг авч одсон учраас сенаторууд болон морьтонгууд нь зөвхөн чичирч, үхлээ хүлээдэг байв. Гэхдээ Кассий Херей, Корнелий Сабин гэгч хоёр зоригтон олдож нөгөөх хортонг алжээ. Хууль ёсоор ногдсон эрх мэдлээ сенат авч чадах байлаа. Гэвч сенаторуудын ихэнхи хэсэг гэр гэр лүүгээ зугтаж, ард түмэн талбай дээр бужигнаж, харин дараа нь тархан явжээ. Эзэн хааны бие хамгаалагч германчууд түүнийг үхснийг хараад таран явцгаажээ. Гэсэн ч эргэлт болсонгүй.
Хэн нэгэн цэрэг Калигулийн айж сандарсан өвөө Клавдияг олж, түүнийг найз нар луугаа авчирсан байна. Тэгтэл Клавдия легионер бүрт 15 мянган сестерций (мөнгө) төлснөөр эзэн хаанаар зарлагджээ. Бүх когортууд (цэргийн жижиг бүлэг–Орч) Клавдийд нийлэх хүртэл сенатад “санал зөрөлдөөн” болжээ. Ингээд хуйвалдагч-бүгд найрамдахчууд цаазлагдан, дарангуй засаг дахин сэргэжээ. 13. Гай Светоний Транквилл. Жизнь двенадцати цезарей, М., 1964. С. 132, 135
“Баатар” болсон удирдагчид, олуулаа байсан “олон түмэн”–ий байж байгаа царай нь энэ байна. Римийн угсаатны систем нь римийн ард түмнийг бүх хөршүүдээ ялагч болгож байсан, Рим хотыг дэлхийн хагасын нийслэл болгож байсан оргилох эрчмийн дүүргэлтээсээ хагацсан байна. Тэдний легионерууд нь ямар ч эсэргүүцэлтэй тулгараагүйгээсээ болж ялж ч чаддаггүй байсан.
Одоо Прагийн их сургуулийнхаа ректороо алдсан чехүүд рүү эргэж оръё. Чехүүд Приципатын үеийн римчүүдтэй биш, харин Мария, Сулла нарын үеийн римчүүдтэй төстэй байлаа. Мэдээж хэрэг Гус сайн профеесор, чех оюутнуудын хаа дунд ихээхэн нэр хүндтэй байсан, гэхдээ түүний нөлөөлөл чех угсаатны бүх давхаргад тамалгаат үхлийнх нь дараагаас санаанд оромгүйгээр өссөн юм. “Баатар” биш, харин түүний сүүдэр угсаатны өөрийгөө бататгасны бэлэг тэмдэг болон чехүүдийг босгож, тэд немцүүд рүү дайран орсон юм. Герман болон Унгарын рыцар цэргүүд чехийн партизаны отрядаас санд мэнд зугтдаг байв. Прагийн профессорын үзэл санаа чехүүдийг зоригжуулж байсан гэж хэлж болохгүй юм. Гус английн санваартан Викликийн сургаалийг хамгаалж байсан ба харин түүнийг дагалдагчид … зарим нь бурхны хишиг хүртэх ёслол хийх, өөрөөр хэлбэл үнэн алдрын шашин руу эргэн орохыг шаардаж, зарим нь папын ёстой салалгүйгээр үндэсний сүм хийдээ шаардаж, зарим нь шатлалын зайлшгүйг үгүйсгэж байсан ба дөрөв дэх нь өөрсдийгөө “адамит” гэж зарлан, чармай шалдлан гүйж, бүхнийг үгүйсгэж байсан юм. (эдгээр ухаангүйчүүдийг чехүүд өөрсдөө устгасан юм).
Эерэг хөтөлбөр биш, “немцүүдийг цохь” гэсэн угсаатны сөрөг голлох үзэл нь католик байсан, тайж байсан, хамгаалалтгүй болсон тариачид байсан, баян бюргерүүд байсных нь төлөө, ер нь бол дурын юмны төлөө шийтгэж байсан бөгөөд гай болоход чехүүдэд хорин жилийн дайныг (1415–1436 он) өгсөн юм. Ямар үнээр гээч? Чехүүд хүн амынхаа тал орчмыг алдаж, Саксон, Бавари, Австри зэрэг нь бас ойролцоогоор талыг нь, Унгар, Померан, Бранденбург нэлээд бага боловч бас л чамгүй олон хүнээ алдсан юм.
Чехүүд эрх чөлөө, соёлоо хамгаалж чадсан бөгөөд гэхдээ зөвхөн дотоодын тэмцлийн замаар олж авсан байдаг. Липаны дэргэд утраквист- чашникууд таборист–протестантуудыг бут ниргэж, тэднийг хайр найргүй сүйдэлсэн юм. Үүний дараа л немцүүдтэй энх тайван тогтоох боломж үүссэн юм. Ядралын суурин дээрх тэвчээрийн бодлогыг Георгий Подебрад (1458-1471) хаан хэрэгжүүлсэн юм.
Хэдийгээр энэхүү товч тойм нь оргилох шинж нь байгалийн үзэгдэл гэдгийг харуулж байна, гэлээ ч гэсэн тэр нь аль нэг угсаатны голлох үзлийг зохион байгуулж магадгүй юм. Угсаатны доминант буюу голлох үзэл гэдэгт бид угсаатны нийлэгжилтийн анхдагч үйл явцын угсаатан соёлын олон янз байдлыг зорилгод чиглэсэн нэгдмэл байдалд шилжихийг тодорхойлдог шашин, үзэл суртал, цэрэг, ахуйн үзэгдэл буюу бүлэг үзэгдлийг нэрлэж байна. Гэхдээ энэ нь нэгдмэл урсгалд нийлэлгүй цалгиж болох бөгөөд чухам ийм зүйл XY зууны Чехэд болжээ.
Төстэй боловч арай өөр явдал Английн хаан IY Генрих болон герцогууд нь Валуа нэрийг авч байсан хэдий ч францаас салахыг эрмэлзэж байсан түүний холбоотон–бургундчуудын эрх мэдлээс Францийг чөлөөлөх жилүүдэд болжээ. Немц аялгаар францаар ярьдаг, лотарины бүсгүй Жанна д’ Аркыг зөвхөн ордны титэм залгамжлагчид болон түүний шадар эмэгтэй Агнессы Сорель нарын ордны хог шаарууд хүрээлж, харин хэдийгээр үүнээс өмнө “англичууд болчихгүйн” тулд эдгээр титэм залгамжлагчдын төлөө тулалдаж байсан ч угсаатны гол үзлийн томъёолол болсон “Үзэсгэлэнт Франц” гэсэн хоёрхон үг сонсоод ялалт хүртэл юуны төлөө тэмцэхээ ойлгосон Ля Гиртэй хамт байсан Дюнуа, маршал Буссак, ахмад Поитон де Сантрайла, зоригт хуягт цэргүүд, чадварлаг харваачид нар байгаагүй бол тэр бүсгүй Орлеаныг ч, хаанаа ч, эх орноо ч хэзээ ч аварч чадахгүй байхсан билээ. 14. Райцес В. И. Процесс Жанны д’Арк. M., Л., 1954. С. 12. См. также: Райцес В. И. Жанна д’Арк. Л., 1982.
Мөн Аввакум ч гэсэн ганцаараа байгаагүй, уншсаныхаа төлөө өөрийгөө шатаахад бэлэн хүмүүс түүний хувийн намтрын галт хуудсыг уншиж байсан ба дахин дахин уншиж байв. Ахиад л “хуучин ёслолтны” хүч оюун ухааны шалтгаанд биш байлаа, никонианчуудтай тайван маргаан хийх хүртэл тэдний тэдний ажил хэрэг нэг ч удаа хөдөлж байгаагүй юм. Аввакум эртний үнэн алдартныг биш, түүнтэй огтхон ч адилгүй ердийн үнэн алдартны ёсыг хамгаалж байв. Ах дүү Альд нарын хэвлэж байсан патристик (сүм хийдийн эцгүүдийн эх бүтээл)–аар хийсэн грек хэл дээрх шилдэг сэдвүүдийг Венец хотоос Никон гэгч хуулан явуулж байв. Аввакум үүнийг XIII–XIY зууны орос орчуулгаар засахыг сүмийн албатуудаас шаардаж байв. Эдгээр орчуулгууд нь маш сайхан байсан, гэхдээ IY–Y зууны эхээсээ дутахааргүй нарийн байжээ. Хуучин ёслолтнуудын икон дээрх хурц тод будгийг эсэргүүцэж байсан нь цаг хугацааны эрхээр харлан бүдгэрсэн дүр төрхөд дассан зуршилд үндэслэгдэж байв. Андрей Рублев, Феофан Грек нар хурц будгаар зурж байсан ба энэ нь XYII зуунд мартагдсан юм.
Товчоор хэлбэл, маргааны санаа нь тохиолдлын байсан ч маргаан нь өөрөө зүй тогтлын шинжтэй байв. Энэ маргаан их оросын үндэстний хожим нь “хуучин ёслолтон” гэх дэд угсаатан салах хоёрдмол байдлыг илэрхийлж байлаа. “Хуучин ёслолтноос” догматик нэгдлийн сүүдэр ч үлдээгүй бөгөөд харин “половчууд” болон “половчууд биш” гэсэн урсгалууд үүсч, дараа нь олон тооны “тайлбарууд” бий болсон юм. Гэхдээ угсаатан задарсангүй. Белорусс руу шведүүд довтлох үеэр дүрвэгч-хуучин ёслолтнууд партизаны отряд байгуулан, Леснойн дэргэд ялалт байгуулахад нь Меньшиковт ихээхэн тусалсан юм.
Ингэхлээр агуу их үйл хэргийг тусгаар оргилуун хүмүүс хийдэггүй, харин оргилуун шинжийн хүчдэлийн түвшин гэж нэрлэж болох нийтлэг сэтгэл санаа бүтээдэг байна. Энэ үзэгдлийн механизмыг Каролингад Гуго Капетийн ялалт байгуулж, үр дүнд нь франц угсаатны цөм бүрэлдсэн явдлыг шинжлэх үедээ Опостен Тьерри нүдэнд харагдтал бичсэн билээ. “Ард олон хөдөлгөөнд орох үедээ тэдгээрийг түлхэн буй тэрхүү хүчинд өөрийгөө зориулдаггүй. Тэд зөн билэгтээ автан хөдөлж, түүнийгээ яг таг тодорхойлохыг оролдолгүйгээр зорилго руугаа явдаг. Хэрэв өнгөц дүгнэх юм бол тэд зөвхөн нэр нь л түүхэнд үлдэх ямар нэгэн жолоодогчийн хувийн ашиг сонирхлыг мухар сохроор дагаж байна гэж бодогдох болно. Эдгээр нэр нь зөвхөн асар олон тооны хүмүүсийг татах төв болж, тэдгээрийг дуудахдаа энэ нь юуг тэмдэглэх ёстойг мэддэг, тухайн агшинд илүү нарийнаар илэрхийлэх шаардлага байдаггүй болохоороо энэ нэрүүд нь нийтэд тодорхой болдог”. 15. Тьерри О. Избр. соч. М., 1937. С. 255.
Тиймээ, энэ нь бидний авч үзэж буй үзэгдлүүд үндэс сууриндаа, нарийн яривал гүндээ угсаатны дүүргэлттэй байна гэсэн хэрэг юм. Александр, Сулла, Ян Гус, Аввакум нарын аль алийг нь янз бүрийн шат, бүс нутгууд дахь янз бүрийн угсаатны нийлэгжилтэд оролцогчид хэмээн авч үзэх ёстой. Ийнхүү хувийн сэтгэл зүйн хүрээг ялган гаргаснаар бид зан үйлийн лугшилтын илрэх хүрээ болсон угсаатны сэтгэл зүй рүү хүрч ирж байна.
Угсаатан судлалын хуримтлуулсан асар их материал нь нэгтгэн дүгнэхийг шаргуу шаардаж байна. Бүхий л даяар материалыг сэтгэн эрэгцүүлж болох зарчмын хайгуулд зөвлөлтийн олон угсаатны судлаачид ажиллаж байна. 16. Жишээлбэл, Бромлей Ю. В. 1) К вопросу о сущности этноса //Природа. 1970. № 2. С. 51-55, 2) Этнос и этнография. М., 1973.
Энэ зарчим шинэ байх ёстой нь ойлгомжтой юм, өөрөөр байсан бол түүнийг аль хэдийн хэрэглэж, тэр түгээмэл байх байсан билээ. Энэ шаардлагад угсаатны хамтын нийгэмлэгийн зан үйлд байгалийн нөлөөллийн үр дүн болсон бодитой оршин буй оргилох шинжийн үзэгдэл нийцэж байгаа юм. Гэхдээ энэ нь угсаатан бол “нийгмийн төлөв байдал” гэх зуршсан үзэл баримтлалтай зөрчилдөж байгаа юм. 17. Козлов В. И. Динамика численности народов. М., 1969. С. 56.
Хуучирсан, буруу үзлийг баримтлах нь индуктив аргын тодорхой логик алдаа-метафор буюу хувирлын гажилтыг дагалдуулдаг юм. Шинэ сэтгэлгээ, сэтгэгдэл гэх мэттэй тулгармагцаа тархи тодорхой болон шинэ тодорхойгүйг хүлээн авахын хооронд гүүр бүтээдэг, адилтган үзэх хамгаалалтын үйл явцад амралт хайсаар хуучин хувцсаа өмсдөг. Энэ зам бидний сонирхлыг ч татаж байна. Бид дараах алхмыг хийхийг хүсэж байна. Үүнээсээ өмнө нэгэнт хийсэн дүгнэлтүүдээ товч томъёолъё, учир нь тэдгээр нь андагч байдал руугаа хувирч байна.
ХУРИМТЛАЛ УУ, ЗАРЦУУЛАЛТ УУ ?
Амьд бодисийн биохимийн эрчим хүчийг В.И. Вернадский царцааны сүргийг орд газарт буй хүдрийн масстай харьцуулж байхдаа нээснийг санацгаая. Царцаанууд олон байж, тэд үхлийг угтан нисэцгээж байжээ. Тэднийг юу ингэж түлхэв ээ? Үүнд хариулт хайх үеэр энтропийн (замбараагүй байдал) эсрэг шинж бүхий Дэлхийн бүрхэвч био хүрээний тухай сургааль бүтээгдсэн юм. Гэхдээ хүмүүс ч гэсэн адилхан биохүрээний хэсэг болно. Үүнээс үзэхэд амьд бодисын эрчим хүч нь бидний болон бидний өвөг дээдсийн биеийг нэвчиж, бас бидний хойч үеийнхний биеийг ирээдүйд нэвчиж, янз бүрийн угсаатны нийлэгжилтийг урамшуулдаг байна. Одоо манай зорилт бол бидний нээж, дүрсэлсэн үзэгдэл маань дээр тавьсан угсаатны нийлэгжилт болон угсаатны түүхийн асуудлуудыг шийдэж чадах эсэхийг харуулахад оршиж байна.
Тасралтат үйл явц болох угсаатны нийлэгжилтийн дээр дурдсан бүдүүвч нь оргилуун угсаатнуудын бүлэг аль нэгэн газар нутагт гэнэт үүсч, түүний удаах тархалт нь уг нутгийн заагаас гарч, угсаатны системийн нийлмэл шинжээ алдан эсвэл түүнийг бүрдүүлэгч хүмүүс сарниж, эсвэл тэд үлдэц болон хувирдаг болохыг харуулж байна. Салбар хувилбарын олонлигууд байгаа ч гэсэн энэхүү бүдүүвч нь хаа сайгүй мөрдөгдөж байна, иймээс ядахдаа харьцуулах замаар ч гэсэн түүнийг тайлбарлах хэрэгцээ үүсч байна.
Гэнэтийн түлхэлт авсан бөмбөлгийг бодоод үзье. Түлхэлтийн эрчим хүч нь эхлээд тайван байдлын инерцийг давахад зарцуулагдана, дараа нь орчны эсэргүүцлийн улмаас аажмаар зогсох бөмбөлгийг хөдөлгөнө. Энэхүү бөмбөлөг нь тэгш газраар өнхөрч байгаа, эсвэл саад тулгарах, эсвэл нүхэнд унах зэргээс түүний зам хамаарна, энэ болтол бөмбөлөг зогсохгүй. Энэ үйлдлийг бид хэчнээн ч давтлаа гэсэн хөдөлгөөний зарчим нь ганц–түлхэлтийн инерци, өөрөөр хэлбэл лугшилтаас авсан эрчим хүчний зарцуулалт л байдаг.
Био хүрээнд ийм эрэмбийн үзэгдлийг сукцесси буюу удамшин солигдох гэж нэрлэдэг. Сукцесс нь үргэлжлэх хугацаа, шинж чанар, үр дагавар зэргээрээ нэн олон янз, гэхдээ тэдгээр нь бүгдээрээ тэмдэглэгдсэн төстэй шинжтэй байдаг. Тухайлбал энэ нь хүнд оргилуун лугшилтын зарцуулалтаар илэрдэг инерцлэг шинж болно. Энэ нь хүн төрөлхтнийг био хүрээний бусад үзэгдлүүдтэй төрөлсүүлдэг бөгөөд харин зөвхөн хүнд хэвшмэл байх нийгмийн болон соёлын бүтцүүд нь хөдөлгөөний өөр шинж чанартай байдаг. Угсаатны үзэгдэл нь хөдөлгөөний хоёр хэлбэрийн зааг дээр байдаг юм.
XXIV. Оргилох шинжийн хүчдэл
ОРГИЛОХ ШИНЖИЙН БИОХИМИЙН АСУУДЛУУД
Хүн бүр, хүмүүсийн хамт олон бүр био хүрээний хэсэг, нийгмийг бүрдүүлэгч элементүүд болдог нь эргэлзээгүй бөгөөд харин материйн хөдөлгөөний эдгээр хэлбэрүүдийн харилцан үйлчлэлийн шинж чанар нь тодотгол шаарддаг. Соёл иргэншлийг тээгч болсон хүн байгал орчинтой харьцах асуудлаар тавьсан зоригодоо хүрэх, зорилтоо шийдвэрлэхийн тулд өөрийгөө бусад адилтгах бүх хамт олонд сөргүүлэн тавьсан, дотоод бүтэц бүхий, тохиолдол бүрт өвөрмөц, зан үйлийн хөдлөнги тогтсон үзэл бүхий хүмүүсийн тогтвортой хамт олныг тэмдэглэх гэж “угсаатан” гэсэн ойлголтыг гарган ирсэн билээ. Чухамхүү угсаатны хамт олноор дамжин байгал орчинтой хүн төрөлхтний тогтоосон холбооны өвөрмөц хувилбарууд хэрэгждэг юм. Гэхдээ энд байгалийн болон нийгмийн хоорондын харьцаа, хил заагийн асуудал босч ирж байна. Техно хүрээний чанадад байгал ноёрхож буй явдал нь илэрхий авч тэр нь хүмүүсийн биед оршдог байна. Физиологи (түүний дотор гажгийн физиологи) нь организмын мэдрэлийн болон гормоны үйл ажиллагааны бүтээгдэхүүн болох сэтгэл зүйтэй нягт холбоотой байдаг. Йодын дутагдал кретинизм өвчнийг өдөөдөг, адреналин ялгарах нь айдас болон уур хүргэдэг, бэлгийн булчирхайн гормонууд хайр дурлалын яруусал, уянгын адал явдлыг өдөөдөг, допингийн химийн нэгдлүүд тамирчдын биеийн төдийгүй, сэтгэл зүйн төлөв байдалд нөлөөлдөг, хар тамхичид бүхэл бүтэн ард түмнийг доройтуулдаг гэх мэт. Хүнд буй материйн хөдөлгөөний нийгмийн хэлбэрүүдийн зүй тогтол нь биологи, биохими, биофизикийн зүй тогтлуудтай ингэж сүлжилддэг, иймээс тэдгээрийг нарийн зааглах нь зайлшгүй болсон нь тодорхой байна.
Хэрэв судалгааны объектоор нэг хүнийг авч үүнийг хийх нь туйлын хүндрэлтэй ахул дээд эрэмбийн системийн нэгжээр шинжилгээний үед зайлшгүй гарах алдааг харилцан нөхөж болдог угсаатныг авах нь хавьгүй амархан болно. 18. Угсаатны оргилох хүчдэл гэдэг нь тухайн угсаатныг бүрдүүлж буй хүмүүст хуваагдсан, угсаатны системд байж байгаа оргилох шинжийн тоо хэмжээ юм.
Мэдээж хэрэг үүнийг дүрслэх, тэр тусмаа өнгөрсөн үеүдийн хүмүүсийн оргилох шинжийг тооцоолох нь хүндрэлтэй. Гэхдээ сэтгэлгээний урвуу нүүдэл гэж бий. Угсаатны хамт олны гүйцэтгэж буй ажил нь оргилох хүчдэлийн түвшинтэй шууд хамааралтай. Эндээс бид маш их алдаатай ч гэсэн угсаатны түүхэнд болсон үйл явдлын тоог тооцож гаргах үндсэн дээр эрчим хүчний анхдагч цэнэг, өөрөөр хэлбэл оргилох шинжийн түвшингийн тухайд шүүн хэлэлцэж болох юм.
Оргилох шинжээр өдөөгдсөн үйлсийг хувийн болон зүйлийн өөрийгөө хадгалах нийт хүн төрөлхтөнд байх зөн билэг байгаагийн улмаас үйлдэж буй энгийн үйлдлүүдээс хялбархан ялгаж болно. Гадаад өдөөгч, жишээлбэл өөр овгийнхны довтолгооноос өдөөгдсөн хариу үйлдлүүдээс тэдгээр нь багагүй ялгагдана. Хариу үйлдэл нь ямагт богино хугацаатай, иймээсээ ч үр дүнгүй байдаг. Харин оргилуун хүмүүсийн хувьд заримдаа бүх амьдралынхаа туршид тэмүүлэн буй аль нэгэн зорилгодоо өөрийгөө зориулах явдал хэвшмэл байдаг. Энэ нь аль нэгэн эрин үеийг оргилох шинжийн талаас нь тодорхойлох боломжийг өгдөг юм. Судлан буй угсаатны угсаатны нийлэгжилтийн янз бүрийн шатуудыг энэ талаас нь тодорхойлбол бид янз бүрийн угсаатнууд дээр иймэрхүү тооцоо байгаа үед боломжит ойртолттойгоор оргилох хүчдэлийн муруйг гаргах өгөгдөхүүнүүдийг гарган авч болох ба энэ нь угсаатны нийлэгжилтийн ерөнхий зүй тогтлыг барин авч болох хэт угсаатан байвал бүр ч сайн болно. Энэхүү зүй тогтлыг олохын тулд түүхэн үйл явдлуудыг сайн мэдэх хэрэгтэй гэсэн хэрэг бөгөөд учир нь түүх нийгмийн харилцааны тухай шинжлэх ухаан учраас үүнийг биш , харин огт өөр материйн хөдөлгөөний нийгмийн хэлбэрт хэвшмэл байх энгийн хөгжлийн зүй тогтлыг судалдаг юм.
Материйн хөдөлгөөний нийгмийн хэлбэрийн өөрөө хөгжих үзэл санааг Homo sapiens зүйлийн дотоод хувилбар болох угсаатны нийлэгжилтэд хэвшмэл байх байнга унтрагч инерцийн хөдөлгөөн бүхий үзэл баримтлалтай сөргүүлэн тавих нь зүйтэй эсэхэд эргэлзээ төрж болох юм. Хүний биологийн өөрчлөлт нь био хүрээний амьд бодисын эрчим хүчний флуктаци буюу хазайлт, мөн оргилох хүчдэлийн нөлөөгүйгээр явагдаж болох мэт санагддаг. Гэхдээ энэ нөхцөлд аль нэгэн нөхцөлд дасан зохицохуйн зохистой хэмжээ нь дурын хэв маягийн хөгжилд мухардал болох байсан. Энэ тохиолдолд түүний төгсгөл нь зөвхөн зүйлийн бүрэн үхэл л байх юм. Зүйл (буюу угсаатан) физиологи болон экологийн хувьд дахин өөрчлөгдөхийн тулд боловсорсон эрхтнээсээ (буюу дадал зуршил) татгалзах, өөрөөр хэлбэл шинэ зам олохын тулд мухарлаас гарах алхам хийх ёстой. Үүний урвуугаар орчны нөхцөлд огтхон ч хамаагүй өөрчлөгдсөн зүйл үүсэх нь, өөр үгээр бол гажиг бүлд албадан үйлчлэх нь түүний өөрчлөгддөггүй хэсгийг бүр Овидий “алтан үе” гэж төсөөлж байсан гомеостаз хэмээх алдагдсан диваажинд хүрэх замаа хайхад албадна.
БҮДҮҮВЧ ДЭХ УГСААТНЫ СИСТЕМИЙН ОЛОН ЧИГЛЭЛТ ШИНЖ
Оргилох хүчдэлгүйгээр угсаатны нийлэгжилт байхгүй, байж ч чадахгүй болохоор оргилох шинжийг дүрсэлж яривал дотор нь аль нэгэн угсаатны салбар шинжүүдийг үлдээгээд хашилтнаас нь гаргаж болох угсаатны нийлэгжилтийн зайлшгүй элемент гэж үзэж болно. Зүй тогтлыг ялган гаргахын тулд чухамхүү бүх үйл явцуудад нийтлэг байх энэ зураас тун чухал юм.
Гэхдээ оргилох шинжийг үзэгдлийнх нь хувьд хэн ч, хэзээ ч хараагүй, харах ч үгүй. Эндээс бид түүнийг зөвхөн илрэлээр нь л тодорхойлж болно. Гэхдээ хамгийн хүнд нь энд байгаа юм биш, харин угсаатны оргилох шинжээр төрөн гарсан янз бүрийн чиглэлтэй голлох зүйлийг (доминант) тооцож, ойлгоход л байгаа юм. Хэрэв угсаатны хэд хэдэн хүч зэрэг үйлчилж буй физик биеттэй адилтгавал (зураг 1) эдгээр хүчнүүдийн нийлбэр нь F = F + F1 + F2 + F3 + F4 + F5 0 гэсэн вектор болно.
Ажиглаж болох хөдөлгөөний бодит үр дүн нь эдгээр хүчнүүдийн арифметик биш, харин вектор нийлбэрээр тодорхойлогдож, өөрөөр хэлбэл бие баруун дээш өнцгөөр хөдлөх болно. Хэрэв F2, F3, F4 F5 гэсэн дөрвөн бүрдлийг нь авчих юм бол бие F1 чиглэлд илүү хурдатгал авна, өөрөөр хэлбэл түүний үр дүн нь F1 –ээс ч, өмнөх F –ээс ч илүү байх болно. Энэ нь тухайн тохиолдолд хурдатгал нь хүчний хэсгийг алдсанаас биш, мөн тэдгээрийг ихэсгэснээс үүсдэггүй байна, учир нь үр дүн гаргах хүч их байхад түүнийгээ дагаад үр нөлөө нь илүү тод байна гэсэн хэрэг юм.
Зураг 1. Физик биетэд хүч үйлчлэх нь ( нэгтгэсэн бүдүүвч)

Жишээн дээр тайлбарлая. НТӨ YIII–Y зуунуудад Эллада оргилох шинжээр буцалж байлаа. Усан хөлгийн гурван том цуваа газрын дундад тэнгис, Хар далайгаар тэнэж, эллинчүүдийн цуваа Кавказаас Испани хүртэл сунаж, харин Иония болон агуу их Грек (Итали дахь) олон хүнтэй дагуул хот болсон байв. Гэхдээ эллиний хот улсууд хүчээ зохицуулж чаддаггүй, тус бүр нь өөрийнхөө бие даасан байдлыг амьдралаасаа илүүд үзэж, захирагдах гэдгийг боолчлолд орсонтой адилтгадаг байсан юм. Ксерксын аян дайны үеэр үхлийн аюул тулгарсан үед ч беотийчууд болон фессалчууд эллинчууд гэдгээ огтхон ч марталгүйгээр персүүдийн төлөө тулалдаж байсан юм. Үүнийхээ төлөө л тэд харгислалд нэрвэгдсэн бөгөөд учир нь Платейн дэргэдэх тулалдааны дараа афинчууд болон грекчүүд олзлогдсон грек–персофилчүүдийг цаазалж, харин персүүдийг өршөөсөн юм.
Пелопоннессийн болон фиваны дайнууд Элладыг үгүйрүүлмэгц л хүчээ зохицуулах боломжтой болж, Персэд хийх Александрын аян дайн эхэлсэн юм. Эллинизмын талбар нь эллинч ёсны талбараас хавьгүй өргөн байсан бөгөөд эдгээр амжилтууд нь Македонтой зэрэгцэн нэгдүгээр рольд соёл болон эдийн засгийн салбарт хамгийн хөгжилгүй Этолия, болон Ахайчууд өрсөлдөх болсон үед Элладын оргилох түвшин нийтдээ буурснаас олдсон билээ. Гэхдээ тэд хүчтэй болсонгүй, харин Афин, Фиви, Спартыг сулруулсан байна. Өөр үгээр бол системийн хувьд Элладын нийлбэр хүчин чадал багасаж, ингээд л тэд Римийн хялбархан олз болсон байдаг. Ийм байлаа ч гэсэн эллинчуудын хуучны чадавхийн боломжит инерци нь римийн язгууртнуудыг өөрийнхөө соёлд оруулахад хангалтай байлаа. Эллинчүүдийн үлдэгдэл НТ I зууны оргилох түлхэлтээр бүрэн өөрчлөгдөж, византийн грекүүдийн цөм болон хувирах хүртэл энэхүү сулрал үргэлжилсээр л байсан билээ. Гэхдээ энэ нь өөр үйл явц болно.
Ингээд засвар хийгээгүй энгийн ажиглалт дийлэнхи олонхи тохиолдолд хуурамч дүгнэлтэд хүргэдэг. Оргилох хүчдэлийн уналт нь өгсөлт гэж зөвшөөрөх ёстой, дээрх хоёр тохиолдолд олон тооны үйл явдал болж, харин нам буюу өндөр түвшний оргилох шинжийн түвшинд адилхан байх “агуу их үйлс” цөөн болж байна. Яагаад гэвэл хоёр дахь тохиолдолд тал тал тийш чиглэсэн хүчнүүд тэнцвэржиж, түр зуурын тогтворжилт болох боломжтой юм. Угсаатны амьдралын тодорхой агшинг биш, харин үйл явцыг бүхлээр нь судлах хэрэгтэй юм. Тэгэхэд л оргилох чанар нэмэгдэх буюу буурч буй нь тодорхой болно.
Олон хүчин зүйлт агшнуудын улмаас тухайлсан хүн хамгийн нямбай судлаад ялгах боломжгүй тэр зүйлийг хүн төрөлхтний түүхэнд бий болсон том хамт олны үйл ажиллагааг статистикийн хувьд судлах үед илрүүлж болдог. Нэгдүгээрт, чухал биш хүчин зүйлүүд харилцан нөхөгддөг, хоёрдугаарт, түүхэн үйл явцууд нь туйлын цаг хугацаагаар хэмжигддэг, харин биологийн буюу геологийн цаг нь харьцангуй юм. Ийм учраас зөвхөн түүхийн шинжлэх ухаан байгалийн ухаанд туйлын он дарааллыг бэлэглэж, оронд нь эмпирик нэгтгэн дүгнэх аргуудыг авч чадна. Үүний дараа байгалийн шинжлэх ухааны аргаар хүмүүнлэгийн материалуудыг боловсруулдаг шинжлэх ухаан–этнологи бий болно.
Оргилох шинжийн үндэс нь цорын ганц ген байх уу, эсвэл генүүдийн хослолууд байх уу, энэ нь зонхилох шинж үү, эсвэл доройтох шинж үү, тэр нь организмын мэдрэлийн буюу гормоны үйл ажиллагаатай холбоотой юу гэдгийг бид шүүн хэлэлцэхгүй. Энэ асуултад бусад шинжлэх ухааны төлөөлөгчид хариулаг. Бидний зорилт бол этнологийнх бөгөөд биелэгдсэн. Бид нийгмийнхтэй зэрэгцүүлэн антропо хүрээний хөгжлийн био газар зүйн хөгжлийг илрүүлж, түүнийг өдөөгч шалтгааныг олсон. Оргилох шинж үзэгдлийн мөн чанарыг болон био хүрээний бусад элементүүдтэй түүний холбоог бид дор авч үзэх болно.
ОРГИЛОХ ШИНЖИЙН НЭГТГЭН ДҮГНЭЛТ
Оргилох шинж нь халдварлах гэсэн нэгэн чухал чанартай байдаг. Энэ нь зохиролт хүмүүс (цочмог хүмүүс бол бүр ч их хэмжээгээр) оргилуун хүмүүстэй шууд ойр байвал яг оргилуун хүмүүс шиг өөрсдийгөө авч явах болдог. Гэхдээ оргилуун хүмүүсээс хангалттай хол зайнд салмагцаа тэд өөрийнхөө байгалийн сэтгэл зүй угсаатны зан үйлийн дүр төрхөө олж авдаг. Энэхүү нөхцөл байдлыг тусгайлан судлалгүйгээр голдуу цэргийн хэрэгт өргөн ашиглаж, харгалздаг нь илэрхий юм. Тэнд оргилуун хүмүүсийг зөнгөөрөө мэдэж сонгон аваад тэдгээрээс сонгомол, цохилтын хэсгийг бүрдүүлэн “дайчин санааг” өргөхийн тулд дайчлагдсан олон хүмүүсийн дунд тараадаг. Энэхүү хоёр дахь тохиолдолд хоёр буюу гурван оргилуун хүн бүхэл ротын дайчин чадварыг дээшлүүлж чаддаг. Үнэхээр ч ийм байдаг.
Ф.Энгельс “Морин цэрэг” хэмээх өгүүлэлдээ морин цэргийн хоёр хэсэг эгц тулалдаанд маш ховор ордог гэж бичжээ. Голдуу аль нэг нь яг тулалдахаас өмнө дайсны ар тал руу эргэж, өөрөөр хэлбэл “ёс суртахууны хүчин зүйл эр зориг нь энд материаллаг хүч болон хувирч байна”, энд шийдвэрлэх агшин нь “тэсгэл алдах” (dash) явдал болж, ингээд цэргүүд ялалтыг (туйлын зорилго) амьдралаасаа ч илүү үнэлдэг. 19. Маркс К.. Энгельс Ф. Соч. Т. 14. С. 318.
Хороонд буй морин цэргүүд нь сэтгэхүйн шинжээрээ бие биетэйгээ огт төсгүй гэсэн хэдий боловч хороо нь тулалдаанд нэгэн бүхэл болон явдаг, ямар нэг хэмжээгээр оргилох шинжтэй байдаг нь ойлгомжтой юм. Хорооны оргилох шинж нь ялалтыг амьдралаас илүү үнэлэхэд оршдог. Хачирхалтай нь оргилох шинж багатай цэргийн хэсэг сөнөдөг, учир нь морин цэрэг гүйж яваа хүмүүсийг амархан хяргадаг. Зөвхөн индукцийн замаар л хэдэн зуун хүнийг “цахилгаанжуулах” өөрөөр хэлбэл хүн тус бүрт өөр хүний оргилох шинжийн цэнэгээр үйлчлэх хоёр адилхан гэдгийг харгалзая. Энэхүү адилтгалын логик үргэлжлэл нь тус тусдаа байгаа хүмүүсийн хувийн сэтгэл зүйтэй харьцуулахад бүлийн сэтгэл зүйд огт өөрөөр үйлчлэх чанар бүхий оргилох шинжийн талбай (цахилгаан соронзон талбайн адил) бий болох таамаглал болно.
Энд “баатар болон олон түмэн”–ий онолоос ялгаатай нь асуудлын мөн чанар баатар хүн цэргийн ангийг удирддагт биш, харин цэргүүдийн дунд бусдаараа юугаараа ч ялгарахгүй боловч оргилох шинж бүхий хэд хэдэн цэрэг байсны ачаар цэргийн анги нь өөрөө Энгельсийн тэмдэглэсэн тэсгэл алдах шинжийг олж, тэр дороо л ямар ч авъяасгүй жанжныг хөтлөөд явчихдагт оршдог. Жишээлбэл, Бенигсен, Витгенштейн, Блюхер нарын авъяасыг Напалеоны авъяастай харьцуулахыг хэн ч оролдоогүй юм. Гэхдээ 1813–1814 оны орос, англи, пруссийн цэргийн дайралт нь бараг хүүхэд гэмээр францын шинэ татагдсан цэргүүдийг бодоход хавьгүй хүчтэй байсан.
Хамгийн гол зүйл нь иймэрхүү эгзэгтэй тохиолдлуудад ухамсарт, өөрөөр хэлбэл хүмүүсийн бодол санаанд үйлчлэх нь ямагт үр ашиггүй байдаг хамгийн гол зүйл оршиж байгаа шүү. Ялалтын босгон дээр тэнцүү биш тулалдаанд гацсан Ганнибалын гунигт явдлыг санагаая. Каннын дэргэдэх тулалдааны дараа түүнд Римийг эзлэх, үүгээрээ Карфагенийг аврахын тулд бага зэргийн дэмжлэг, явган цэргийн отряд хэрэгтэй байлаа. Ганнибалын элч нар болон Варка нэртний талынхны карфеганчуудын ахлагчдын зөвлөгөөн дээр явуулга хийсэн шалтгууд нь өө сэвгүй байсан юм. Сонсохыг хүсэхгүй нь сонсохгүй л байг, ойлгохыг хүсэхгүй нь ойлгохгүй л байг. Карфагены ахлагчдын зөвлөл жанжинд “Чи ялж л байна шүү дээ, ахиад цэрэг чамд юунд хэрэгтэй юм бэ? “ гэсэн хариу өгч, ач гуч нараа үхлээс аварсан байна. Карфагений эзэд тэнэг юмуу хулчгар байгаагүйг ярихын ч хэрэггүй юм. Гэхдээ эзгүй байсан хүний нөлөөлөл тэдний дунд тархсангүй. Харин ялагдсан Ганнибал төрөлх хотдоо эргэж ирэхэд түүний нэр хүнд хүчирхэг өрсөлдөгчид нь хүртэл мэхийн ёслох хүртэл тийм агуу их болсон байлаа. Зөвхөн Римийн сенатын тулган шаардах бичиг л Ганнибалын эх орноо орхиход хүргэсэн юм. Өөрийгөө золиослох шийдвэрийг Ганнибал өөрөө гарсан бөгөөд учир нь эсэргүүцэх оролдлого бүтэлгүйднэ гэдгийг тэр ойлгосонд байгаа юм.
Энэ удаад утга зохиолын түүхээс нэг жишээ авъя. 1880 оны 7 дугаар сарын 8–нд Ф.М.Достоевский Оросын хэл зохиол сонирхогчдын нийгэмлэгийн хуралдаан дээр Пушкиний тухай үг хэлжээ. Үзсэн хүмүүсийн яриагаар бол амжилт нь нэн гайхамшигтай байжээ. Гэхдээ энэ үгийг унших үед онцгой сэтгэгдэл төрүүлдэггүй аж. Тэр нь хэзээ ч “Ах дүү Карамазовынхон”–ны бүлгүүдтэй нэг эгнээнд явдаггүй. Достоевский өөрөө оролцсон нь үзэгчдэд хүргэх түүний үгийн үйлчлэлийг жинтэй болгоход багагүй үүрэг гүйцэтгэсэн нь илт байна.
Оргилох шинжийн индукц хаа сайгүй илэрдэг. Теарт, хөгжим сонирхогчид Консерватори буюу Москвагийн уран сайхны академик театрын орцуудыг бүсэлсэн байдаг манай үед ч гэсэн нэн тодорхой харагддаг. Тэд радио буюу телевизээр дамжуулсан тэрхүү жүжгээс төрөх сэтгэгдлийг театрын танхимд сонссонтой харьцуулшгүй юм гэдгийг дэндүү сайн ойлгодог юм. Энэхүү жишээ нь угсаатны нийлэгжилтийн үзэгдлүүдтэй харьцуулахад тун өчүүхэн хэдий ч аль аль нь нэг зүй тогтолтой билээ.
Оргилох шинжийн тархах хурц жишээ нь 1796 онд Аркольскийн гүүрэн дээр болсон тулалдаан юм. Австри болон Францын армийг гүүр тавьсан гүнзгий биш боловч зуурамтгай гол зааглаж байв. Францчууд гурван удаа атакт орсон боловч австрийн хорголжин суман мөндөрт гэдрэг хаягджээ. Эцэс нь цэргүүдийг шинэ дайралтад босгох нэгэнт боломжгүй болсон мэт санагдах үед генерал Напалеон Бонапарт тугаа шүүрэн урагш давшмагц түүний араас төмрийн үртэс соронзонд татагдах мэт гренадеруудын бүх цуваа гүүр рүү цутгажээ. Эхний эгнээдийг ахиад л хорголжин сумаар сийчсэн хэдий ч дараачийнх нь эгнээ автрийн их буунууд хүртэл гүйж, их буучдыг нь жадалцгаажээ, үүний дараа францын арми бүтнээрээ гол гарч, тулалдаанд ялсан байна. Напалеоныг үсрэх үед нь гүүрнээс түлхэж гол руу унагасан учраас л тэр амьд үлджээ.
Энд авсан жишээг манай үзлийн үүднээс шинжлээд үзье. Италид явуулсан арми бол тэр үед фронтод байлдаж байсан францын бүх армиудаас хамгийн муу нь байв. Энэ армийг парижийнхан нэг бус удаа үгүйрүүлж, дэвсэж байсан Францын өмнө зүгийн тариачдаас бүрдүүлсэн байсан бөгөөд муу сургасан, түүнээс ч дор хангалттай байсан аж. 20. XIII зуунд альбигойны дайн, XYI зуунд гугенотуудын дайн, XYII зуунд камизаруудын бослогын үед.
Энэ бол цэргийн мэргэжлийн дадлага байхгүй зөнгөөрөө хүмүүс байв. Энэ арми дахь дарга нар нь дүүрчихсэн луйварчид байсан учраас тэдний ихэнхийг нь аян дайн эхлэхээс бүр өмнө Бонапарт хулгай хийснийх нь төлөө буудан алсан байна. Ингэхлээр оргилуун хүмүүсийн хувь тун өчүүхэн болж, тэдний өөдөөс Габсбургийн хаант улсын шилдэг хороод сөрөн зогсч байжээ. Гэлээ ч гэсэн дөрвөн том тулалдаанд (Лоди, Кастильон, Арколе, Риволи) францчууд ялж, түүний өрсөлдөгч генерал Альбинци хийж чадахгүй байхад Напалеон шийдвэрлэх үед оргилох шинжийг бадрааж (нарийн яривал оруулж, өөрөөр хэлбэл тархааж) чаджээ. Хэсэг хугацааны дараа тархсан оргилох шинж алга болж, харин Суворов гурван тулалдаанаар (Адда дээр, Треббий, 1799 онд Новийн дэргэд) Италид францчуудын байгуулсан гавъяаг үгүй хийжээ. Үүнд Журден, Макдональд, ялангуяа Моро зэрэг францын генералуудыг буруутгаж огтхон ч болохгүй юм. Тэд ажил хэргээ сайн мэддэг байсан, хүчдэл гаргасан, гэхдээ хэт хүчдэл гаргаагүй юм. Гэтэл Суворов Бонапартын адил өөрийнхөө илүүдэл оргилох шинжийг зөвхөн оросуудад төдийгүй, гаднын цэргүүдэд ч дамжуулж чадаж байлаа. Гэхдээ Венад хуралдаж байсан учраас Суворов дээрх байдлаар үйлчилж чадаагүй билээ. Оргилох шинжийн тархалт тодорхой ойртолт шаардаг байна, Хэдэн зуун километрийн цаанаас түүнийг нэгэнт мэдэрч болдоггүй аж.
Баатарлаг боловч швейцарийн кампанид ялагдсаныхаа дараа Суворов Венад ирж, театрт орон тэнд оролцогчдыг адислан ерөөхөд хэн ч үүнийг нь инээдэмтэй юмуу зохисгүй зүйл гэж үзээгүй юм. Хэдийгээр хагас жилийн өмнөх түүний үйлдлийг хавчих нь хавьгүй ашигтай байсан хэдий ч бүр урвуугаар Суворовт эзэн хааны хүндэтгэлийг үзүүлсэн байна.
Эдгээртэй адилтгах бусад үй олон тохиолдлыг дурдахгүй байхын тулд бид энэ жишээнүүдийг тодорхой авч үзсэн юм. Мөн чанартаа бол хөгжин буй угсаатнуудын цэргийн болон улс төрийн бүх түүх нь оргилох шинжийн тархалтын аль нэг хувилбаруудаас тогтдог ба ийм замаар л зохиролт хүмүүс олон түмнийг хөдөлгөөнд оруулдаг байна.
Гэхдээ эдгээр хувилбарууд нь олон янз, чингэхдээ шийдвэрлэх агшин нь угсаатны ойртолтын зэрэг байдаг. Суворов орос цэргүүдийн сэтгэл санааг эх оронч үзлийн цөмөөр дамжуулан унгар, тироль, хорват, чех болон түүний командлалд байсан бусад цэргүүдийг бодоход илүү хэмжээгээр өргөж чадсан. Напалеон вестфальц, саксон, голланд, неаполитанчуудыг бодоход францчуудад хавьгүй хүчтэй нөлөөлдөг байсныг 1812–1813 оны кампаниуд харуулсан юм. Оргилох өдөөлтийн нөлөө нь бусад тэнцүү тохиолдолд зохиролт хүмүүс болон оргилуун угсаатнууд хэчнээн хол байхын хэрээр төчнөөн бага байдаг байна. Энэхүү нөхцөл байдал нь оргилох шинжийг асуудлыг угсаатны бат нэгдмэл байдлын мөн чанарын асуудалтай шинж тэмдгийн хувьд дахин ойртуулж байна. Нөлөө нь тархалтын адил эрчим хүчний ойлголт болно. Энэ нь угсаатанд хэр зэрэг хэрэглэгдэх бол ?
Бид эндээс угсаатны нийлэгжилтийн дурын үйл явц нь хүмүүсийн жижиг бүлгийн (консорци) баатарлаг, гэхдээ золиосын үйлдлээр эхэлдэг ба үүнд нь тэднийг хүрээлэн буй олон түмэн нэгдэж, чингэхдээ бүрэн чин сэтгэлээсээ ханддаг. Мэдээж хэрэг, хэн нэг хүн үл итгэх буюу зүгээр л аминч байдлаар хандаж байж болно, гэхдээ тэр нүдэн дээр нь үүсэн буй системд орлоо л бол түүний дээрх зан байдал ач холбогдол багатай болдог. Энэхүү нийтэд тодорхой үзэгдлийг бидний тэмдэглэсэн оргилох шинжийн тархалт болон нөлөөлөл тайлбарлаж өгдөг. Эдгээр нь оргилох шинжээр халдварласан тэр хүмүүсийг “тогтоон баригч” болох биет оргилуун хүмүүсийн ач холбогдлыг ойлгоход тусална. Эхнийх нь байхгүй бол хоёрдахь нь хямралдаж сарнина, оргилох шинжийг тархаах үүсгэвэр алга болмогц нөлөөллийн инерци шавхагдана. Харин энэ нь голдуу маш хурдан болдог.
ОРГИЛОХ ШИНЖЭЭ АЛДАХ АРГУУД
Ингээд, дурын угсаатны нийлэгжилт бол системийнхээ оргилох шинжээ их бага хэмжээгээр алддаг, өөр үгээр бол оргилуун хүмүүс, тэдгээрийн ген мөхдөг, ялангуяа энэ нь хүнд хэцүү дайны үед илэрдэг, учир нь өөрийнхөө чанаруудыг хойч үедээ дамжуулах бүрэн боломжоо ашиглалгүйгээр оргилуун дайчид ихээхэн хэмжээгээр залуу насандаа үрэгддэг билээ.
Гэхдээ хамгийн сонирхолтой нь зөвхөн дайны үед оргилох чанарын хүчдэл буурдаггүй байна. Үүнийг өөрийн хамт олны сайн сайхны төлөө амьдралаа хэтэрхий идэвхитэй золиослогч хүмүүсийн үхлээр амархан тайлбарлаж болно. Гэхдээ оргилох шинж нь гүн гүнзгий энх тайвны үед яг ийм байдлаар тасралтгүй унадаг, чингэхдээ хатуу догшин үеүдээс ч илүү хурдан байдаг аж. Угсаатны хувьд хамгийн аймшигтай зүйл нь тайван оршихуйгаас өөр угсаатны дарамтын өмнөөс хамгаалахад шилжих шилжилт байдаг, энэ үед хэрэв мөхөл болдоггүй юмаа гэхэд эвдрэл болох нь гарцаагүй бөгөөд энэ нь хэзээ ч зовлон зүдгүүргүйгээр өнгөрдөггүй юм. Энэ үзэгдлийг нийгмийн шалтгаан буюу хүчин зүйлээр тайбарлах боломжгүй агаад гэхдээ хэрэв нэмэгдсэн оргилох шинжийг уламжлагдах шинж тэмдэг гэж үзэх юм бол бүх зүйл ойлгомжтой болно.
Үрэгдэхээсээ өмнө ямагт хууль ёсны гэр бүлд биш ч гэсэн үр удмаа үлдээж амждагийнх нь улмаас дайнд явж буй баатруудыг эмэгтэйчүүд үнэлж байдаг. Хүүхдүүд том болж өсөөд эцгүүдээ мэдэхгүй ч гэсэн тэдний гол хуулиар зааж өгсөн үйлдлүүдийг үргэлжлүүлдэг. Үүний урвуугаар дөлгөөн эрин үеүдэд төгс эрхэмлэл нь төвч, нямбай гэр бүлсэг хүн болж, харин оргилуун хүмүүс амьдрах байр суурьгүй болдог. Яг ийм нөхцөл байдлыг И.А.Гончаров “Тасралт” зохиолдоо нэгэн бүсгүй хувьсгалч, жүжигчин, баян газрын эзнийг зэрэг сонгож байгаагаар харуулсан байдаг.
Мөн ийм зүй тогтлыг полигами (олон нөхөр буюу эхнэртэй байх ёс) гэр бүлтэй, эмэгтэй хүн эрхгүй мэт байдаг тэр газар ажиглаж болно. Халифатын үеийн арабууд, мөн турк-османуудын хурдан үржил нь полигамаас болж явагдсан юм. Эхнэрүүдийн гэрийн үйлчлэгчдийг байлдаанд олзлон авч, тэднийг дайны олз буюу эзлэгдсэн орнуудын орлогоор тэжээдэг байжээ. Эх орон нэгт бүсгүйтэй гэрлэх нь хүртэл маш үнэтэй байсан учраас калым буюу эхнэр худалдагч бэлэвсэрсэн тохиолдолд ч гэр бүлээ хангах ёстой байв. Ийм учраас ядуу нүүдэлчин бедуинчууд ганц эхнэрээр сэтгэлээ хангаж, гэхдээ салах эрхтэй байжээ. Учир нь энд гэр бүл бол христианы Европ шиг нууцлаг зүйл биш, харин иргэний байдал байжээ. Ийнхүү мусульманы шариат хууль нь эмэгтэй хүнд өөрийн үзэмжээр нөхрөө сонгоход нь саад болдоггүй бөгөөд харин үзэмж нь эсвэл олз олдог зоригтон, эсвэл гэр бүлээ хангалуун байлгадаг тооцоотой эздийн моодонд нийцдэг байна. Ямар ч тохиолдолд Өрнөдөд ч тэр, Дорнод ч тэр оргилуун хүмүүс хэрэггүй болж заримдаа нийгэмдээ адлагдан хууль ёсны удам хойчгүй үхдэг байна.
Бүлээс тэдний алга болох нь гаднын цохилт угсаатныг донсолгох хүртэл мэдэгдэлгүй өнгөрдөг ба ийм зүйл болмогц л алдагдал эргэн нөхөгдөшгүй нь илт болдог. Чухам энэ үед л хөгшрөлтийн шат, өөрөөр хэлбэл үхлийн өмнөх үе эхэлдэг. Ингэхлээр бид угсаатны үйл явц нь хэдийгээр тэдгээртэй байнга харилцан үйлчилж, түүхэн газар зүйн олон янз байдлыг бүрдүүлж, аль аль нь төвлөрөн хавсардаг ч гэсэн нийгмийн үйл явц нэгэн төрөл биш гэдгийг батлах эрхтэй юм.
Ийнхүү оргилох шинж нь зүгээр л нэг “тэнэг авъяас” биш, угсаатны угсаатны хольцын шинэ хослолуудыг амьдралд өдөөгч, тэднийг шинэ хэт угсаатны систем болгон өөрчилдөг удамшлын нэн чухал шинж тэмдэг юм. Одоо бид экологи болон тусгай хүмүүсийн ухамсартай үйл ажиллагаанд ногдох түүний шалтгааныг хаанаас хайхаа мэдэж байна. Одоо хувийн биш, хамтын ухамсаргүйн өргөн салбар үлдээд байгаа бөгөөд учир нь оргилох түлхэлтийн инерцийн үйлчлэлийн үргэлжлэх хугацаа нь зуун зуунаар тоологддог. Улмаар оргилох шинж бол биологийн шинж тэмдэг, мөн тайван байдлын инерцийг зөрчигч анхдагч түлхэлт–энэ бол оргилох шинж бүхий зарим тооны хүмүүсийг оролцуулсан үе бий болох явдал юм.
Тэд өөрийнхөө оршин буйн энэ баримтаараа л дадсан нөхцөл байдлыг зөрчдөг бөгөөд учир нь тэд өөрсдийнхөө сонирхож буй зорилгогүйгээр өдөр тутмын ажил үйлсээр амьдарч чаддаггүй юм. Хүрээллийг эсэргүүцэх зайлшгүй байдал нь тэднийг нэгдэж, зохицон үйлдэхийг албаддаг, ингэж тухайн эрин үеийн нийгмийн хөгжлийн түвшингөөр өгөгддөг аль нэгэн социал хэлбэрийг маш хурдан олж авдаг анхдагч консорци үүсдэг. Оргилох хүчдэлээр төрөн гарсан идэвхи нь нөхцөл байдлын таатай урсгалын үед энэхүү консорцийг хамгийн ашигтай байдалд тавьдаг, харин тэгэхэд хэсэг бусаг оргилуун хүмүүсийг эрт үед ч гэсэн “эсвэл овгоос хөөж, эсвэл зүгээр л алдаг” байсан юм. 21. Козлов В. И. Что такое этнос //Природа. 1971. № 2. С. 72.
Ангит нийгэмд ч хэрэг явдал бас л нэг иймэрхүү байдаг. Пушкин үүнийг анзаарч “дорд байдал л бидэнд таарна, гайхалтай нь биш дээ” гэж бичсэн байдаг ( “Евгений Онегин” 8 бүлэг, IX).
Зөв хэлжээ! Оргилуун хүмүүс балрах жамтай юм. Гэхдээ хэрэв тэд юу ч хийж амжилгүйгээр дандаа үхээд байсан юм бол бид одоо болтол нялхсаар тахил өргөж, хөгшдийг алж, алсан дайсныхаа махыг шарж, найз нөхөд төрснөө хөнөөх гэж ид шидээр оролдох байсан юм. Пирамид ч, Патеон ч, Америкийг “нээсэн” ч, таталцлын хуулийн томъёолол ч, сансрын нислэг ч байхгүй байхсан. Гэхдээ энэ бүхэн нь бүр чулуун зэвсгийн үед байж, хуримтлагдаж эхэлсэн юм. Ингээд өнөөдөр Дэлхий дээр орчин үеийн францчууд, англичууд, оросууд гэх мэтчилэн хүмүүс биш, харин шумерууд, пиктууд болон нэр нь аль эрт мартагдсан бусад овгууд л байх байлаа.
Угсаатны нийлэгжилтийн сүүлчийн шатанд оргилуун хүмүүс хамгийн харамсалтайгаар үрэгддэг, тэдний тоо цөөн болж, хоорондоо болон эгэл олонтой харилцан ойлголцол нь алдагдаг. 1203 онд Византид ийм зүйл болсон билээ. Ердөө л 20 мянган хүнтэй загалмайтнууд түлхэгдэн унасан эзэн хааны хүүг ширээнд нь суулгах гэж Контантинополийн ханын дэргэд цугларсан юм. Грекүүд 70 мянган цэрэг гаргаж чадах байсан ч ханан дээр гарсан хөлсний цэргийн баг болон бусад чин зоригтнуудад туслалгүй, ямар ч эсэргүүцэл үзүүлээгүй юм. 1203 оны 6 дугаар сарын 18, 1204 оны 4 дүгээр сарын 12–нд хоёронтоо эзлэн авсан юм. Сүүлчийн удаад нь хот аймшигтай эвдэрч, дээрэмдүүлсэн билээ. Загалмайтнууд дайрах үедээ ганцхан… рыцараа алдсан байна !
Ингээд л оргилуун хүмүүс нь тулалдаанд үхэж, үлдсэнүүд нь шатаагдсан байшиндаа бас л үрэгдсэн юм. Тэд амьд үлдэх нь байтугай ялж ч болох байжээ. Дайнд нэг л муж ороход Константинополь ч ялалт амсаж, чөлөөлөгдсөн бөгөөд 1453 онд мөн яг ийм нөхцөл байдалд дахин унасан юм. Ингээд ялагчдад өөрийгөө алуулахаар тайван өгдөг олон хүн ахиад л үлдсэн билээ. Энэ хүмүүс гэдэг чинь юу юм бэ?
XXV. Дэд оргилуун хүмүүс
ЗОХИРОЛТ ХҮМҮҮС
Угсаатны нийлэгжилтэд оргилуун хүмүүсийн үүрэг хэчнээн ч их байлаа гэсэн угсаатны бүрэлдэхүүн дэх тэдний тоо ямагт өчүүхэн байдаг. Оргилох шинжтэй хүмүүс гэдэгт бид үгийн бүрэн утгаар хувь хүний хувьд ч, зүйлийн хувьд ч өөрийгөө хамгаалах зөн билгээс нь энэхүү лугшилт хүчтэй хөгжсөн хүмүүсийг нэрлэж байна. Жирийн хүмүүсийн дийлэнхи олонхид нь энэ хоёр лугшилт тэнцвэржиж байдаг ба энэ нь оюун ухааны хувьд төгс, ажиллах чадвартай, хүмүүстэй нийлэмжтэй, зохиролт, гэхдээ хэт идэвхитэй биш хувь хүмүүсийг бүтээдэг. Түүнээс гадна ийм хүмүүст өөрийгөө золиослон байж авчрах оргилох хүчгүйгээр байж чадахгүй, тэсгэлгүй шатаж буй бусад хүмүүс ангид бөгөөд дургүйг нь хүргэдэг. Үүний дээрээс хөгжиж буй усаатнуудад ихэнхи хүмүүс нь үлдэц угсаатан шиг нэн сул оргилох шинжтэй байдгийг нэмж хэлэх хэрэгтэй. ялгаа нь бол хөдлөнги угсаатанд өөрийнхөө илүүдэл эрчим хүчээ системийнхээ хөгжилд оруулдаг оргилуун хүмүүс байж, ажиллаж байдагт оршино.
Гэхдээ хөгжлийн идэвхи нь дандаа л угсаатны талд яваад байдаггүйг тэмдэглэх нь зүйтэй. Оргилох шинж ухаалаг зүйд нийцэх байдлын хяналтаас гарч, бүтээн байгуулахаас эвдэн сүйтгэгч хүч болон хувирах “халалт”–ууд гарах боломжтой. Тэгэхэд зохиролт хүмүүс бас л тодорхой хязгаар хүртэл угсаатныг аврагч болж болно.
Энэ шинжийн хүмүүс угсаатны биед буй туйлын чухал элементүүд байдаг. Угсаатнаа нөхөн сэргээж, оргилох шинжийн тэсрэлтийг дарсхийж, нэгэнт бүтээгдсэн хэв маягаар материаллаг үнэт зүйлээ арвижуулж байдаг. Тэд гаднын дайсан бий болох хүртэл оргилуун хүмүүсгүйгээр бүрэн болгоод байж чаддаг. Тухайлбал, Исландад викингийн удмынхан оргилох шинжээ аажмаар алдсаар байсан. XII зуун гэхэд тэд далайн чанадын аян дайнаа зогсоосон, XIII зуунд гэр бүлийн хоорондын цуст яргаллаа дуусгасан, ингээд 1627 онд алжирийн далайн дээрэмчид арал дээр буухдаа ямар ч эсэргүүцэлтэй тулгараагүй юм. Исландчууд гэр орноо шатаалгаж, эхнэрүүдээ хүчиндүүлж, хүүхдүүдээ боол болгон булаалгаж байсан, гэвч тэд зэвсгээ өргөж зориг нь хүрээгүй юм. 22. Стеблин-Каменский М. И. Культура Исландии. С. II.
Энэхүү тодорхой тохиолдолд өөр тайлбарлал олохыг оролдоод үзье. Алжирчууд мэргэжлийн яргачид байсан, магадгүй тэд гэнэтийн агшинг ашиглаж сандралд оруулсан байж болно, исландчууд энэ үед Гучин жилийн дайнд татагдан орж, ялагдал хүлээж байсан хараат улс Даниас тусламж авах ямар ч боломжгүй байсан…Эцэст нь бидний үзэл бодлоор исландчуудын оргилох хүчдэл цаашид ч буурах ёстой байсан. Үнэхээр ийм байсан уу? Хоёр зууны дараах Исландыг аваад үзье.
1809 онд Рейкьявикт гучин цэрэг, ахмад, губернатораас бүрдсэн данийн гарнизон байж байлаа. Губернатор нь үзэсгэлэнт охинтой байсан байна. Энэхүү оны 6 дугаар сард зогсоол дээр хар тугтай бриг буюу хоёр шурагт усан онгоц бий болж, хотыг бууж өгөхийг шаардав. Данийн офицер гал нээсэн боловч онгоцны их бууны суманд шархаджээ, ингээд цэргүүд зэвсгээ буулгажээ. Далайн дээрэмчид арал дээр бууж, тэдний ахлагчаар өмнө нь цагчин Юрген Юргенсон гэж алдаршсан, одоо далайн дээрэмчин болсон исланд хүн байсан байна. Энэ новш губернаторын охинд дурласан нь ил болж, түүнийг авахыг шаардаж, өөрийнхөө дээрэмчдэд оршин суугчдыг дээрэмдэхийг зөвшөөрөн өөрийгөө Исландын эзэн хаанаар зарлажээ. Аз болоход бүсгүй хүндээр өвдөж амжсан байна. Гэхдээ исландчууд үнэхээр сайн байлаа! Өчүүхэн дээрэмдэд ямар ч эсэргүүцэл үзүүлсэнгүй. “Далай хөөсрүүлэгчид”–ийн удам, Англи, Норманди, Винландыг булаан эзлэгчид хэдхэн арван дээрэмчдийн бузар авирыг хүлцэнгүй тэвчиж, эсэргүүцэх нь битгий хэл, зугтахыг ч оролдохгүй байлаа. Тэдний эсрэг Испани, Францын хааны флотуудтай өрсөлдөгч догшин маврууд байгаагүй, ердөө л Хойд далайн зогсоолоос ирсэн хэдхэн новшнуудын хэлтэрхий байлаа. Үүнийг оргилох шинж нь унаагүй гэх үү ?
Гэхдээ олонхи зүйлийг бүх зүйлтэй адилтгах ёсгүй юм. Зарим хүмүүс өөрийгөө эзэмдэх чанараа алдаагүй байлаа. Хэдийгээр тэд бүх нийтийн хулчгар зан, арчаагүй байдлыг ганхуулах хүчгүй ч гэсэн өөрсдийгөө аварч чадсан билээ. Тэдний тоонд гоо үзэсгэлэнт дани бүсгүйн сүйт хархүү байсан бөгөөд тэрээр загасчны завиар зугтан английн усан онгоцтой таарч, тусламж хүсчээ. Англичууд маш худран Рейкьявикт дөхөн ирж, их буугаараа сүрдүүлэн дээрэмчдийг буулган авч, тэднийг гинжинд холбон губернатор болон түүний охиныг чөлөөлжээ. Дээрэмчдийн толгойлогчийг английн шүүхээр шүүсэн бөгөөд Их Британийн албатуудын ашиг сонирхлыг хөндөөгүй учраас цагаатгасан байна. Харин дээрэмчин–хааны эрх мэдэл дор зургаан долоо хоног байсныхаа дараа исландчууд ажил ажлаа эргэн хийж эхэлжээ. 23. Томский А. Kopoль иcлaндcкий//Пpиpoдa и люди.1912. № 13. С. 608-670. № 44. С. 684-686.
Өөрөөсөө бусад бүх хүмүүст зохиролт, соёлжсон, гэм халгүй харагдах хүмүүсийн хийж чадах юм нь зөвхөн энэ л байна. Хэтэрхий хамгаалалтгүй чанар угсаатны цэцэглэлтэд ямагт тус болоод байдаггүй аж.
“ШААРАН АМЬТАД”, “ТЭНҮҮЛЧИД”, “ТЭНҮҮЛЧ – ЦЭРГҮҮД’
Эцэст нь угсаатны бүрэлдэхүүнд “сөрөг” оргилох шинжтэй хүмүүс цаг ямагт байдаг юм. Өөр үгээр хэлбэл, тэдний үйлдлийн чиглэмжийг оргилох хүчдэлийн эсрэг чиглэсэн лугшилт удирдаж байдаг ажээ.
Исландчууд ядахдаа гэр бүлээ тэжээхийн тулд ажиллах чадвараа алдахгүй байж болох байсан, мөн түүнчлэн амьдралынх нь эх булаг болсон май загас барих газар, нугасны сөд цуглуулдаг байсан нугасны цуваа, дэлэнт малаа тэжээхэд хэрэгтэй хад асган дундах нуга зэргээ хамгаалж болох байсан. Гэхдээ хотожсон нийлмэл газрууд дахь дэд угсаатны бүрдэл нь эрт үеэс эхлэн хамгийн адгийн хувилбар байсан юм. I зууны үед газрын хэсгээ (парцель) алдаж, задарсан римийн иргэдийн удмынхан Римд бөөгнөрчээ. Тэд таван давхар байшигуудын өрөөнд тохилж, хэвтрийнхээ “шаарны” өмхий самхайгаар амьсгалж, энэ бүхнээ тэр чигээр нь Тибр рүү урсгаж, хортой хар тугалган савнаас дарс ууж, засгийн газраас “талх болон үзэх юм” шаргуу бөгөөд бүдүүлгээр шаарддаг болсон байна. Ингээд түүнийг нь өгөхөд хүрчээ, энэхүү дэд оргилох шинжийн хүмүүс хэрэв тэдэнд талх нэмэн өгч, циркэд илүү гоё тоглолт үзүүлнэ гэж амлах юм бол эргэлт хийхийг хүссэн дурын оргилуун түрэмгийлэгчийг дэмжиж чадах байлаа. Тэд дайснаас өөрийгөө хамгаалж чадахгүй, чадахыг хүсэхгүй байсан ба учир нь цэргийн хэрэгт суралцах нь хүнд ажил байсан юм. Дэд оргилуун хүмүүс өөрийнхөө няцалтгүй логикоор ирээдүйг хэн ч урьдчилан харж чадахгүй хэмээн төсөөлж, иймээс ч боодол талх хүлээн авагч, циркийн тоглолт үзэгч болсон бөгөөд магадлалын үндсэн дээр таамаглал хийж чаддаггүй байв. Ийм учраас тэд хүлээн авсан мэдээллээ тааламжтай, тааламжгүй гэж хоёр хуваадаг. Тааламжгүй мэдээдэл тээгчийг хувийн дайсан шигээ үзэж, тохиромжтой тохиолдолд болгонд тэднийг хэрцгийлдэг байжээ.
Үүний үр дүнд Италийг ярьдаггүй юмаа гэхэд Рим хотын хүрээнд ч байлдах чадвар, цэргийн үүрэггүй шахам байсан Аларих (410 он) Римийг эзлэн авсан юм. Энэ шившиг хүртэл римчүүдийг юунд ч сургасангүй. Готууд зөөлөн ялагдсан болж, явцгаажээ. Энэ нь ч гэсэн ээлжит өөрийгөө тайвшруулахын шалтаг болжээ. Гэхдээ зэрлэг Гензерих Римийг (455 он) өөрийгөө Карфагений сүйрлийн өш авагч гэж зарлажээ, тэрээр дэд оргилуун хүмүүсийг хядаж амархан гэгч нь номхруулсан байна. Гэхдээ зохиролт, гэм халгүй исландчуудаас ялгаатай нь хэн ч тэднийг аврахыг хүсээгүй юм. Зэрлэгээр сүйдэлсний дараа Рим нэгэнт эдгэрээгүй билээ. Яагаад ч юм түүнд харамсах сэтгэл төрөхгүй л байна.
Үүнтэй адилтгам явдал Багдадад болсон бөгөөд хотыг гаднаас ирсэн бүдүүлгүүд биш, харин халифын худалдан авсан түрэг боолчууд болох гулямууд эзэлжээ. IX зуунд арабын дайчид хядагдсан байлаа. Тэдний удмынхан жижиг худалдаа эрхэлж, зах дээр чалчихыг илүүд үзэх болов. Халифийн биеийг болон заримдаа халифатын хилийг хамгаалахын тулд мэргэжлийн цэргүүд шаардагдаж байсан юм. Тэгээд л тэднийг Дундад азийн тал нутаг, Нубийн цөлөөс худалдан авсан байна. Тэд Багдад дахь цорын ганц бодит хүчин болж, халифийг өөрийн үзэмжээр будлиулах болжээ. Харин аварга том хотын хүн ам уйлж, харааж, шоолж байсан, гэхдээ зөвхөн хамгаалуулахгүй байхын тулд ажил хийж амьдралгүй, өвдөг сөгдөн үхэхийг илүүд үзсэн байдаг.
Ийм үр дагавар, мөн харгалзсан төгс эрхэмлэл (идеал) солигдох нь системийн оргилох хүчдэлийн алдагдлыг өгдөг байна. “Өөрийнхөө төлөө амьдар” гэсэн лоозон бол хар үхэл рүү явах дөт зам мөн.
Тухайлсан хүний оргилох шинж нь их, дунд, бага гэсэн дурын чадвартай нийцдэг, гадаад үйлчлэлээс хамаардаггүй, хүний гол хууль нь болж байдаг, мөн тэр нь угсаатны хэм хэмжээнд хамаагүй, харин алдар гавъяа болон гэмт хэрэг, уран бүтээл болон эвдрэл сүйтгэл, сайн болон мууг адилхан амархан төрүүлэх ба зөвхөн амгалан байдлыг л үгүйсгэдэг юм. Тэр нь хүнийг “олон түмэн”-ийг хөтлөгч “баатар” болгодоггүй, Олонхи оргилуун хүмүүс чухамхүү “олон түмний” бүрэлдэхүүнд байдаг бөгөөд аль нэгэн агшинд тэдний идэвхийн чадвар, зэрэг хэмжээг тодорхойлж байдаг.
Түүхэн дэх дэд оргилуун хүмүүсийн бүлгийг “тэнүүлчид”, мэргэжлийн хөлсний сенатчид (ландскнехтууд) хамгийн хурц тод төлөөлдөг. Тэд түүнийг өөрчилдөггүй, мөн хадгалдаггүй, харин түүний ачаар л оршин байдаг. Тэд өөрийнхөө хөдөлгөөнт чанарын ачаар оргилуун хүмүүстэй хамт булаан эзлэлт, эргэлт үйлдэн угсаатны хувь заяанд чухал үүрэг гүйцэтгэдэг. Гэхдээ оргилуун хүмүүс дэд оргилуун хүмүүсгүйгээр өөрийгөө илэрхийлж чаддаг бол нөгөөдүүл нь оргилуун хүмүүсгүйгээр юу ч биш юм. Тэд гуйланчлал буюу дээрэмдэх чадвартай бөгөөд тэг оргилуун шинжийг тээгчид, өөрөөр хэлбэл хүн амын үндсэн хэсгийг золиосоо болгодог. Гэхдээ энэ тохиолдолд “тэнэмэлүүд” сүйрэх тавилантай, тэднийг мөрдөн хөөж устгадаг юм. Гэхдээ л тэд үе болгонд гарч ирдэг болно.
ОРГИЛОХ ШИНЖИЙН ТӨРЛҮҮД ( ГРАДАЦИ)
Оргилуун хүмүүсийг “олон түмнийг” хөтлөгч “баатруудтай” адилтгах, харин “тэнэмэл–цэргүүдийг” “дагалдагч” гэж нэрлэх шохоорхол байдаг, гэхдээ үнэн хэрэг дээрээ түүхэн үйлдлийн механизм нь тийм энгийн байдаггүй. Гүрнийг үндэслэгчдээс бусад Испаний Габсбургууд, францын Бурбонууд нь түрэмгийлэгч-сайд Фуке болон Жон Ло буюу Мануэль Годой мэт нь үе үе дунд нь орж ирдэг байсан тэдний ордныхны ихэнхи хэсгийн адил эгэл жирийн хүмүүс байсан юм. Гэхдээ идальго болон шевалье, негоциантууд болон корсарууд, миссионарууд болон конкистадорууд, гуманистууд болон зураачид зэрэг нь бүгдээрээ XYI зууны Испани, XYI–XYII зууны Францыг угсаатны нийлэгжилтийн үйл явцыг бүрдүүлэгч хэмээн зүйрлэвэл эдгээр угсаатны өндөр оргилох шинжийг тусгасан улс төрийн дотоод хүчдэлийг бий болгогчид байжээ.
Ийм учраас оргилуун хүмүүс нь ямагт ард түмний хөдөлгөөнийг толгойлдог хэдий ч тэднийг “хөтлөгч” биш, харин “түлхэгч” гэж нэрлэх нь зөв юм. Учир нь ор сураггүй үхэж үрэгддэг хангалттай тооны оргилуун хүмүүсгүйгээр уламжлал, өөрөөр хэлбэл олон түмний инерцийг эвдэх боломжгүй байхсан. Энэ тухай эртний испанийн дууллын мөрүүд өгүүлнэм билээ.
Алдарт Оливьерээ ч, Роланаа ч дуулж байна
Ахмад Сурракиний зоригийг л үл дуулж байна.
Эд нар Роланаа ч, Оливьероо ч алдаршуулжээ
Эрэлхэг цэрэг Сурракинаа л харин мартжээ
Ийнхүү бид буурах оргилуун шинжийн гурван төрлийг авч үзлээ. Шаардлага гарвал энэ хуваалтыг илүү жижиглэж болох юм. Ийм учраас гуравдахь шинж чанарлаг хэв маягийг “дэд оргилуун шинж” гэж нэрлэх нь зөв. Гэхдээ хамгийн гол нь дурдсан хэв маягийг ангийн буюу язгуур угсааны ангиллын хэв маягуудтай хольж болохгүй. Тэдгээрийн аль нь өөртөө гурван хэв маягийг агуулж байдаг бөгөөд гэхдээ янз бүрийн харьцаагаар, янз бүрийн голлох шинжээр агуулж байдаг. Угсаатны дотор тэдгээрийн харьцааг тооны хувьд ч, векторын хувьд ч ондоошуулах нь угсаатны нийлэгжилтийн үйл явцыг тодорхойлдог.
Энэ сэдэв нь оргилох шинж гэсэн ойлголтыг нэмэн тайлбарлаж, үзүүлэхдээ нэгэнт түүхийн биш, харин асуудалд илүү дөхөж очсон А.С.Пушкиний бүтээлүүдийн утга зохиолын тодорхойлолтыг ашиглаж байгаагаараа нэн чухал юм.
Ердийн оргилуун хүмүүс нь “баатар” болон “жолоодогчид” огтхон ч биш, харин шуналдаа идэгдсэн Тэнэг рыцарь, ялахын тулд хайр дурлалын ялалт руу тэмүүлдэг Дон Жуан, атаархлаасаа болж Моцартыг алсан Сальери, хардлагаасаа болж Земфираг хөнөөсөн Алеко нар болно. Пушкиний зохиолд Мазела болон Пугачев (нэн алсын түүхэн хэв маягууд) баатар байгаагүй ч оргилуун хүмүүс, жолоодогчид байсан юм. Харин үүргийнхээ төлөө амьдралаа эрсдүүлсэн Гриев болон Машенька Миронова нар нь оргилуун хүмүүс байгаагүй ч баатрууд байсан билээ. Хаан хэдий ч “жолоодогч” биш байсан XII Карл нь оргилуун хүн, баатрын үлгэр жишээ болно. Тэр “дуулга руугаа цэцгэн хэлхээ чулуудуулахын төлөө дайчин алдарт дурлагч”, өөрөөр хэлбэл өөрийнхөө алдар нэрийн төлөө улс орныхоо ашиг сонирхлыг золиосолсон хүн юм. Үүний эсрэг XII Карлыг бодоход хавьгүй илүү хүчтэй, өөрийнхөө олон аашийг дагасан, Оросын өмнө хүлээсэн үүргээ биелүүлсэн зохиролт хувь хүн I Петр тавигддаг. Пушкиний тайлбарлалаар бол энэ нь хөөрүү зан, хүүхдийн гэмээр харгис зан зэрэг хувийн шинжүүдийг нь эс тооцвол бодит байдалтай тун ойрхон аж. Петр өөрийнхөө эцэгтэй адилхан байсан, өөрөөр хэлбэл XYII зуунд Михаил Федоровчийн үеэс эхэлж үүссэн Европтой ойртох гэсэн оросын соёлын уламжлалын нэг шугамыг үргэлжлүүлэгч, өөрийнхөө цаг үеийн хүн байлаа. Гэхдээ энэ үед нь Петрийг жишээлбэл, Меньшиков, Ромодановский, Кикин зэрэг оргилуун хүмүүс хүрээлж байсан, гэхдээ тэд нар нь жолоодогч ч, баатар ч байгаагүй юм. Ийм учраас оргилуун хүмүүсийг удирдагчидтай харьцуулах нь зан үйлийн шинж тэмдгүүдийн аль нэгийг нь аль эрт хаягдсан бүдүүлэг онолд нийцүүлэн тайлбарлах зорилго бүхий хоосон санаа мөн.
Хамгийн янз бүрийн хүмүүсийн бүхий л сэдлийг ашиг хонжоо олох эрмэлзэлд тохдог, чингэхдээ үүндээ зөвхөн мөнгө болон түүнтэй адилтгах үнэт зүйлсийг ойлгодог дээрхийн эсрэг үзэл бодол нь түүнээсээ ч дутахааргүй утгагүй зүйл байдаг. Энэхүү бүдүүлэг байр суурь нь дэд оргилуун шинжтэй бэртэгчнийг ямагт үнэн хэрэг дээрээ нийтлэг зүйл юу ч үгүй материализм гэж харагдуулдаг. Бэртэгчнүүд цаг ямагт сэтгэн бодох чадваргүй байдаг юм.
Тэрээр өөр эрмэлзлээр хөдөлдөг, мөнгө гэхээсээ өөр зорилгод тэмүүлдэг, тэдэнтэй адилгүй хүмүүс байдаг гэдгийг төсөөлж ч, хүсч ч чаддаггүй юм. Шууд ашиг хонжооны үзэл баримтлалыг хэзээ ч нарийвчлан томъёолоогүй, учир нь ингэх л юм бол түүний утга учиргүй нь ил болох юм. Гэхдээ тэр нь хар аяндаа ойлгомжтой юм шигээр шүүн хэлэлцэх дурын шалтгаанд, тэр ч байтугай шинжлэх ухааны байгууламжид байлцсаар л байна. Ингээд л өөртөө анхаарал шаардаад байдаг юм.
ГАННИБАЛ БА КАРФАГЕН
Одоо манай үзлийн үүднээс Пунийн хоёрдугаар дайны үед Ганнибалийн зан үйлийг авч үзье. Барк гэсэн овог бол Карфагены толгой баячуудын нэг байсан юм. Ганибалын эцэг Гамилькар Нумидий болон Испанийг захиран баялгаа үржүүлсэн бөгөөд хүү Ганнибал нь хэрэг дээрээ хаан байлаа. Римтэй хийсэн дайн нь Ганнибалд ямар ч хожоо амлаагүй юм. Харин ч урвуугаар эрсдэл нь хэтэрхий их байлаа. Ганнибалын хувийн ашиг сонирхлын үүднээс бол дайн түүнд хэрэггүй, гэхдээ тэр нь түүний эх орон–Карфагены хувьд гарцаагүй байв. Хэрэв тэнэсэн сум карфагены жанжны цээжинд тусаагүй бол түүний үхлийг байлдааны ямар ч олз нөхөж чадахгүй байсан, тэр тусмаа түүнд мөнгө хэрэггүй байв. Гэхдээ тэр өөрийнхөө иргэдийн хүсэл зоригийг биелүүлсэн байж магадгүй л юм ? ! Үгүй ажээ. Тэд түүнээс байлдахыг гуйгаагүй, шийдвэрлэх үед нэмэлт хүч илгээхээс татгалзаж байсан бөгөөд өөртөө биш, нийтийн хэргийн тулд ямар нэг юм хийх хэрэгтэй гэдгийг мэдэрсэн бэртэгчний бүхий л тэсэн ядалтаар түүнийг үзэн ядаж байсан юм.
Энэ тохиолдолд дэд оргилуун хүмүүс өөрийг нь ямар ч үүргээс зайлхийхэд хүргэж болох шалтгуудыг шууд хайж эхэлсэн юм. Мэдээж хэрэг энэ нь огтхон ч алсын хараагүй явдал байсан, ер нь хүмүүс дандаа хашир байдаггүй бөгөөд энэ нь тэднийг мөхлийн үр дагавар руу хүргэдэг. Товчоор хэлбэл хувийнхаа ашиг тусын тулд Ганнибал Гадестаа сууж, зугаалан цэнгэх ёстой байв. Мөн карфагений ахлагчдын зөвлөл бүхий л хүчээрээ өөрийнхөө жанжинг дэмжих ёстой байсан. Нумидийн морин цэргүүд үзэн ядагч финикийн колоничлогдод алуулахгүйн тулд цэргээс оргон зутах ёстой байсан. Испаний дүүгүүрчид босч, эрх чөлөөгөө эргүүлэн авах байсан. Гэтэл бүх зүйл урвуугаар болох нь тэр. Эдгээр тохиолдлоос болж Карфагены баялаг пунийн утга зохиол алга болов. Энэ улс оронд сая хүнтэй Римийг талхаар хангах үүрэг ногдсон учраас Атласын хавцлын хөндийг хагалж, дараа нь орхин хаяв. Эрх чөлөөнд дуртай берберүүд догшин римчүүдээс дүрвэн өмнө зүг нүүж, тэдний мал сүрэг нь Баруун Сахарын бүр ногооноороо байсан тэгш газрыг талхалж, чулуулаг цөл болгон хувиргав. Ганнибалын үед хойд Сахарт гол ус урсаж, заан тэнэж, адуу бэлчдэг байсан юм. Харин рим болон арабын булаан эзлэгчдийн антропогенийн нөлөөллийн хоёр мянган жилийн дараа энэхүү баян тансаг бүх амьтны аймгийг ганцхан тэмээ л орлох болсон билээ.
Хэрэв бид угсаатны зүй болон физик газар зүйд болсон ийм асар том өөрчлөлтийн шалтгааныг олохыг хүсвэл Барк овгийн оргилох шинжийг карфагений бэртэгчдийн дэд оргилох шинж хүнд ачаагаараа даран унагасан нь ойлгомжтой. Чухамхүү энэ л эхлээд тэдгээрийг дайнд ялагдуулж, дараа нь бүслэгдсэн Карфагений хананд үхүүлж, дараа нь эдгээрийн үр дагавар болж Нумидийн ноёрхлыг дагуулж, удаалаад ландшафт устахад хүргэсэн билээ.
Өөрөөр байж болох байсан уу? Мэдээжийн хэрэг. Ганнибалд цаг тухайд нь тусалсан бол Рим сүйрч, самнит болон цизальпиннийн галлууд чөлөөлөгдөн, хэт нүсэрдсэн, зохиомол хотжилт зогсож, Апеннины царсан болон бусад ой, Капуй болон Тарентийн орчны усан үзмийн тариалан, Арно хөндий дэх эслэг жимсний талбайнууд үлдэх байсан юм. Галлийн баялгууд, Элладын урлагийн дээж сорыг удаан хугацаагаар аварч болох байсан. Гэхдээ Аппиений зам ч, Каракаллын терм ч, ирж буй эрин үеийн сургуулиуд дахь латин хэл ч байхгүй байхсан. Гэхдээ энэ тохиолдолд үйлдвэрлэлийн харилцааны хөгжил өөрийнхөө замаар явах байсан биз ээ. Үеэ өнгөрөөсөн эртний боол эзэмшлийн оронд арай эрт юмуу хойно ч гэсэн феодализм ирэх байсан. Оргилох шинжийн өсөлт ба уналт нь хэрэв нийгэм – эдийн засгийн формаци халагдах гэсэн утгаар нь ойлговол хүн төрөлхтний нийгмийн хөгжилд нөлөөлдөггүй. Сэтгэл хөдлөл нь ухамсрын дадсан байдал–оюун ухааны ямар нэг юмыг яаж өөрчилж чадаж байна аа? Яагаад ийм байдгийг бид одоо үзье.
XXVI. Оргилох шинжийн унтралт
ОЧ ҮСРЭХ БА ҮНС НУРАМ
Одоо бол угсаатны нийлэгжилтийн “асаах агшин” нь бүлд хэсэг тооны оргилуун болон дэд оргилуун хүмүүс гэнэт үүсэх явдал юм. Өгсөлтийн шатанд эсвэл үржих, эсвэл нэгдэх замаар оргилуун хүмүүсийн тоо хурдан нэмэгдэнэ, оргил шатанд оргилуун хүмүүсийн тоо дээд хэмжээндээ хүрнэ, хугаралтын шатанд оргилуун хүмүүсийн тоо эрс багасаж, тэдгээрийг дэд оргилуун хүмүүс шахан гаргана, инерцийн шатанд оргилуун шинжтэй хүмүүсийн тоо удаан багасна, хөгшрөлтийн шатанд оргилуун хүмүүсийг өөрийнхөө зан байдлын онцлогоос болж дэд оргилуун хүмүүс бараг бүрэн халдаг бөгөөд эсвэл тэд угсаатнаа бүхэлд нь хөнөөнө, эсвэл гаднын харь овгийн довтолгоон хүртэл хөнөөж амждаггүй. Угсаатан үлдсэн тохиолдолд амьдран буй газар нутгийнхаа биоценозод дээд, төгсгөлийн мөчир болон ордог, зохиролт хүмүүсээс бүрдсэн үлдэц угсаатан хоцордог.
Угсаатны дотоод хувьслыг бүх угсаатнууд туулсан. Зөвхөн тэдгээр угсаатнуудын бичигдээгүй түүх эрин зуунуудын харанхуйд живсэн учраас л бид тэднийг бүдүүлэг гэж тооцдог. Мөн л ийм дүр зургийг бид түүхээс, чингэхдээ дэд угсаатны бүхэллэг дээр, жишээлбэл, сибирийн казакуудын жишээн дээр нэн тодорхой хардаг юм. .
XIY зуунд оросжсон хазаруудын удмынхан “хэсүүлчид” гэсэн орос нэрээ “казак” гэсэн түрэг нэрээр сольсон юм. XY–XYI зуунд тэд талын ногайчуудын аянга болж, Сибирьт дайныг шилжүүлэн тэдний сүүлчийн хаан Кучумийг барьж байжээ. Москвагийн засгийн газрын дэмжлэг аваад тэд ганцхан зуунд Сибирийг Номхон далай хүртэл туулсан юм. Нэмэлт хүч хэрэг болоход тэд отряддаа их оросуудаас дуртайяа авдаг байсан боловч тэднийг өөрсдөөсөө ямагт ялгаж байсан юм. Тэд нарыг бүгдийг нь хамтад нь газар нээгчид гэж нэрлэж заншсан билээ.
XYII зууны Оросын газар нээгчид зөрүүд, эрс тэс, буулт хийдэггүй хүмүүс байжээ. Тэд ахлагч нараасаа ч, умрын ширүүн байгалаас ч айдаггүй байв. 1632 оноос зуутын дарга Петр Бекетов Лена мөрөн дээр өвлийн суурин байгуулж, 1650 он хүртэл, өөрөөр хэлбэл казак Семен Моторын Анадырийн аялал хүртэл байж байгаад тэр хоёр зүүн хойд Сибирийг бүхэлд нь туулж, америкийн алт конкистадоруудад өгснөөс ч багагүй булган татвар олж авчээ. “Казакийн булаан эзлэгчид цуцалтгүй эр зоригтой, угийн санаачлагатай байлаа. Тэд овог овгоор нь татварлаж, явсаар усан онгоцны сүйрэлд зориулсан мэт модны үндсээр хавтгай модыг бүдүүлгээр цавчин зүйсэн хонуурт Хойт мөсөн далайг олжээ. Гэхдээ XYII зууны төгсгөлд тэдний зан ааш өөрчлөгдөж, алсын аян хийхийн оронд “Манай усан онгоцнууд сул, далбаа жижиг байна, Өмнөх шигээ том усан онгоцыг бид хийж чадахгүй байна“ гэсэн тэмдэглэл л бий болов. XYIII зууны хойд Сибирийн оросын хүн ам талстжсан мэт болжээ. Санаачлага, идэвхи ул мөргүй алга болж, хамгийн эр зоригтой зан нь хулчгар зангаар солигджээ”. 24. Богораз В. Г. Новые задачи российской этнографии в полярных областях // Труды Северной научно-промысловой экспедиции. Вып. 9. Пг., 1921. С. 20-21.
Эцэст нь XIX зуунд казакуудын удмынхан чукча нараас цохилт амсаж, төрийн хамжлага, албан тушаалынхаа буруу үйлдлээсээ болж шийтгэгдэн өмнөөс умар руу илгээгдсэн түшмэл бүрийн эрх байхгүй боолчууд болсон байна. Мөн яг ийм, бас мөн ийм цаг хугацаанд испаний конкистадорууд, Канад дахь францын колонистууд (индианчуудтай холилдсон тэр хэсгээс бусад), Энэтхэгийн далайн сав газрын португал болон арабын худалдаачдын удам угсааныхан оргилох шинжээ алдсан юм. Харин өнгөрсөн зуунуудад мөн ийм хувь заяаг викингууд болон эллинчүүд амссан юм. Энэ нь дурдсан үйл явц зүй тогтолтой болохыг харуулж байна. Оргилох шинжийн зарцуулсан эрчим хүч нь өөрийнхөө оч үсэрсэн газар дээр эхлээд халуун, дараа нь хүйтэн, саарал үнс үлдээдэг юм.
Конкистадоруудын шунахай зан, Александр Македонскийн бахархал, Суллын нэр алдарт дурлах, Гусын тэвчишгүй итгэл үнэмшил зэрэг үзэгдлүүд нь өөр хоорондоо төсгүй мэт санагдана. Энэ нь гаднаасаа бол үнэхээр тийм, гэвч тэдний болон тэдэнтэй төстэй олон үзэгдлүүдийн цаад үндэс нь нэг л бөгөөд энэ бол оргилох шинж юм. Ийм учиртай юм. Бидний авсан бүх жишээнүүдэд оргилох шинжийн буюу хэт идэвхийн лугшилтын шинж тэмдэг нь хүнд төдийгүй, мөн бүлд хэвшмэл байдаг гэдгийг цохон дурдаж байсан билээ. Бид шинж тэмдгийг нь аль болохоор тодруун харуулахын тулд тодорхой хүнд бүтээцийн (композици) зорилгоор л анхаарлаа хандуулдаг юм. Бодит байдал дээр үйл явцууд нь илүү нарийн, Гэхдээ системийг хэрэглэж, түүнийгээ дэс дараалалтайгаар мөрдвөл тэдгээрийг шинжлэх нь хүндрэлд учрахуйц тийм хэмжээнд хүрдэггүй.
Эхэндээ хувь хүн буюу системийн оргилох шинж нь хэчнээн өндөр байх тутам нийгмийн бүлгийн бүтээлч амьдрал төдий чинээ баян, угсаатны соёл төдий чинээ арвин тансаг болох мэт санагдаж болно. Итали дахь Дахин сэргэлтийн эрин үе нь авъяасаар бялхаж байсан болохоор түүнийг угсаатны нийлэгжилтийн оргил шат гэж авч үзэж болно. Гэхдээ XY зуунд италийн угсаатан хамгийн хүнд хугарлын шатаа туулж байсан юм. Миланд Висконти болон Сфорцын кондотьерүүд, Флоренцэд Медичи нар батжиж, Римд пап непотизм (ашиг орлоготой албан тушаал, цолыг тараах) болон симонийг (хулгай, худалдагдах) ил тод хэрэглэж байлаа. Неаполь болон Сицилид гуманизмд хавьтахгүй, бүдүүлэг, дайчин испаничууд эрх барьж байв. Хэзээ нэгэн цагт италичуудыг немцийн эзэн хааны харгис эрх мэдлээс чөлөөлөгдөхөд нь тус болж байсан бүгд найрамдах улсын хотын уламжлал, эх оронч үзэл, алдар суу хаа сайгүй алга болсон байна. Энэхүү нийтлэг ялзрал дээр зураач Беато Анджелико болон Боттичелли, энэрэнгүй үзэлтэн Жовани Понтано, Лоренцо Балла, Марсилио Фичино болон Пико делла Мирандола нарын уран бүтээл болох урлаг, шинжлэх ухааны цэцгүүд ургасан билээ.
Гэхдээ Леонардо да Винчи, Рафаэль Санти, Микаланжело Буанорроти, Тициан, Ариосто, Макиавелли нарын нэрээр алдаршсан Өндөр Дахин Сэргэлт нь (XYI зууны эхэн хагас) Испани, Францын цуврал дайны дэвсгэр дээр явагдсан бөгөөд Итали дайнд оролцоогүй байсан ч тэмцэлдэгч махчингуудын талбар болоод байлаа. Эдгээр дайнууд нь францчууд Итали руу дайран орсноор эхэлж 1494 -өөс 1525 онуудад явагдсан билээ. Франц улс итали дахь эрх мэдлийн төлөө өрсөлдөж байв. Павийн дэргэд францчуудыг ялсны дараа ялагч эзэн хаан Y Карл италичуудын эсэргүүцлийг дарахад цэргээ хаяхад хүрсэн бөгөөд энэ нь 1527 онд Римийг бүдүүлгээр ниргэснээр хэрэгжсэн юм.
Үгүй, Италичууд өөрсдийнхөө дарангуйлагчдаас салахын тулд заримдаа гадаадын цэрэг бий болгохыг ашигласан оролдлого хийгээгүй гэж хэлж болохгүй юм. 1494 онд Флоренцэд францчууд ойртон ирэх үед тэнд медичийн овгийнхон нуран унаж, эрх мэдэл доминиканы лам Савонаролед шилжив. 1498 онд Савонароле нас барсны дараа ч гэсэн амар болсонгүй. Шинээр байгуулагдсан бүгд найрамдах улс бүрэн хүчгүйгээ харуулж, 1512 онд Медичийн гэр бүлийнхний эрх мэдэл сэргэн тогтов. Бүгд найрамдах улсыг дахин байгуулах хоёрдахь оролдлого нэрт зураач Микеланджелогийн оролцоотойгоор хийгдсэн боловч 1530 онд эзэнт гүрний цэрэгт дарагдсан юм. “Флоренцийн баатарлаг үе дуусч, түүнтэй хамт италийн Дахин Сэргэлтийн соёл дууссан юм”
XYI зууны хоёрдахь хагаст Итали нь Испанийн нөлөөний хүрээнд орсон юм. 1563 онд Тридентийн Чуулганаас гаргасан Эсрэг шинэчлэлийн зарчмууд нь мөн чанараараа шинэ католицизм байсан бөгөөд Италид ард түмний эсэргүүцэлтэй биш, харин сэхээтнүүдийн хэсэг бусаг дургүйцэлтэй тулгарсан юм. Тэднийг католик хариу үйлдэл амархан дөнгөсөн юм. 25. Гуковский М. А. Итальянские войны и Высокое Возрождение XVI в. (до 1559 г.)// Очерки истории Италии /Отв.ред. М.А. Гуковский. М., 1959. С. 125-152.
Жордано Бруног шатааж, Кампанеллаг хорьж, Галилейг няцаасны дараагаас 150 орчим жил үргэлжилсэн бүрэн уналт болсон билээ. 26. Ролова А. Д. Период испанского владычества (вторая половина XVI в. – XVIII в.) //Там же. С. 153-191.
Ингэж Италийн оргилох шинж дуусчээ. Уран бүтээлийн болон оргилох шинжийн “цэцэглэлт” хоёр давхцахгүй байгааг хэрхэн тайлбарлах вэ ?
СУЛ ОРГИЛОХ ШИНЖ, ГЭХДЭЭ ҮЙЛЧИЛДЭГ
Харсаар байтал бидний авсан хурц тод жишээнүүдээс гадна оргилуун хүмүүс нь түүдэг гал буюу гудамны бэхлэлт рүү яваад байдаггүй (Гус ба Сулла), гэхдээ л зорилгынхоо төлөө олон зүйлээ золиослодог сул илэрхийлэгдэх хувилбарууд оршин байх ёстой. Гоголь болон Достоевскийн уран бүтээлийн шаталт, Ньютоны сайн дурын даяан, Врубель болон Мусоргскийн зовнил–энэ бүхэн нь мөн л оргилох шинжийн илрэлийн жишээ болно, учир нь шинжлэх ухаан, урлагийн гавъяа нь “шууд үйлдлийн” гавъяа шиг золиос, өргөлийг шаарддаг юм. 27. “Action directe” нэр томъёог бараг орчуулах боломжгүй юм. Энэ нь шууд түлхэлт илрэхийг, манай тохиолдолд оргилох эрчмийн түлхэлтийг хэлж байна.
Улс төрийн түүхийн зүтгэлтнүүдийг бодоход өөр ч гэсэн угсаатны нийлэгжилтийн үйл явцад эрдэмтэд, жүжигчид мөн л чухал үүрэг гүйцэтгэдэг. Тэд өөрийнхөө угсаатанд өвөрмөц өнгө аяс олгож, ийм маягаар эсвэл түүнийг бусад адилхан угсаатнуудаас ялган гаргаж, эсвэл угсаатан хоорондын харилцаанд дэм болдог бөгөөд үүний ачаар хэт угсаатны бүхэллэг буюу соёл үүсдэг. Хэдийгээр бага хүчдэлтэй боловч хүнд мэргэжлээ дотоод тэмүүллээрээ сонгон явсан готын сүмүүдийг барьсан нэргүй барилгачид, оросын уран барилгачид, үлгэр зохиогчид зэргийг оргилох шинж бүхий хүмүүст хамааруулж болно. Бидний гаргасан ангиллаар энэ бүлэгт багтдаг авъяаслаг түүх бичигчид ч тэдэнд хамаарах нь ойлгомжтой билээ.
Харьцангуй сул, гэхдээ системийн оргилох шинжийн бүтээлч зэрэгт анхаарлаа хандуулъя. Энэ нь хоёр байдаг: нэг нь өгсөлт дээр, бидний “оргил шат“ гэж нэрлэдэг системийн “халалт” хүртэл, хоёр дахь нь бидний инерци гэж нэрлэдэг шатад шилжихийг харуулагч хугарлын шатанд байдаг. Дүрслэн яривал, бидний сонирхож буй энэхүү хоёр агшин нь уналтын үед ч гэсэн хүчдэлээ бүрэн алдах хүртэл хамаагүй хол байгаа угсаатны системийн оргилох шинжийн өсөлтийн муруйн (нэмэх–хасах) нугаралт байдаг. Оргилох шинжийн харьцангуй нам түвшний үед хүний зан үйлийн тогтсон үзэл болон нийгмийн захирах шинж нь түүний өөрийнх нь сонгосон төгс буюу хуурмаг зорилгын төлөө сайн дураараа үхэхэд өөрт нь мэдэгдэлгүйгээр түлхэхүйц тийм хэмжээнд байж чаддаггүй. Гэхдээ энэ үед угсаатанд байгаа оргилох хүчдэл нь өөр зорилгод тэмүүлэх, ядахдаа хүрээлэн буй бодит байдлаа бага ч гэсэн өөрчлөхөд нь хангалттай юм. Ингээд хэрэв хүнд зохих ёсны чадвар байвал өөрийн үеийнхнээ итгүүлж, сэтгэлийг нь татахын тулд түүнийгээ шинжлэх ухаан, урлагт дамжуулдаг.
Шүлэг ч тэр, зураг ч тэр, театрын тоглолт ч тэр–энэ бүхэн нь хүлээн авагч хүмүүст үйлчилж, тэдгээрийг өөрчилж байдаг. Чингэх үед бид энэ нь сайн юм руу, аль эсвэв муу юм руу хөтөлж байгаа юу гэсэн асуулт тавьдаггүй. Хэрэв эдгээр чадвар байгаагүй бол хүн эд баялаг, алба тушаал ахих гэх мэтийг хийнэ. Оргилох шинжийн энэхүү түвшин ноёрхож байсан түүхэн үеүд нь соёлын цэцэглэлт мэт харагддаг авч, тэдгээрийн цаана боломжит догшин хоёр үеийн нэг нь ямагт байдаг, эсвэл оргилох шинж өсөх үед дээр нэгэнт дурдсан “халалт” болно, эсвэл түүнийг удаан буурах үед уналт болно. Ингэж Дахин Сэргэлтийг (XIY–XY зуун) Шинэчлэл (XYI-XII) халсан, Гучин жилийн дайны аймшиг, гугенотын дайн, дкагонад буюу хэт харгис хууль, түүнчлэн Энгельсийн нэрлэснээр Робеспьер, Напалеон хоёрыг нэгэн биед нийлүүлсэн “тойрог толгойт” Кромвелийн догшин зан зэргийн дараа XYIII зуунд харьцангуй тайван үе ирсэн ба энэ нь Дахин сэргэлттэй оргилох шинжийн түвшингөөрөө төстэй боловч түүний чиглэмжтэй (вектор) адилгүй байсан юм. Эхлээд түвшин нэмэгдэж яваад, дараа нь сүйрлийн дараа бууралт явагдсан юм. Энэ нь оргилуун хүмүүсийн хувь буурч, тэдний оронд эрсдлээс аюулгүй байдлыг, хурдан амжилтаас хуримтлалыг, адал явдлаас тайван цатгалан амьдралыг илүүд үздэг хүмүүс иржээ. Тэд оргилуун хүмүүсээс илүү ч байгаагүй, дутуу ч байгаагүй, зүгээр л ондоо хүмүүс байлаа.
Энэ үйл явдлыг сурвалжид хэн ч, хэзээ ч тэмдэглээгүй юм. Яагаад гэвэл энэ нь зөвхөн эрин үеүд болон улс орнуудын шинж чанарыг өргөнөөр харьцуулах үед л ил болдог жамтай. Ийм учраас зөвхөн угсаатны түүхийн хүрээнд, этнологийн хэрэгслээр л үүнийг хийж болно.
Бага халалттай оргилуун хүмүүс–зураач, яруу найрагч, эрдэмтэд, хичээнгүй хүмүүс гэх мэт нь угсаатны нийлэгжилтэд үүрэг гүйцэтгэдэггүй буюу эсвэл энэ үүрэг нь жанжин, конкистадорууд, буруу номтнууд болон хуурмаг номлогчдоос бага байдаг юм уу? гэж хэлж болох юм. Үгүй ээ, энэ нь бага байдаг, гэхдээ өөр юм. Бид оргилох ихээхэн хүчдэлтэй хүмүүс ч гэсэн хэрэв тэр нь эх орон нэгтнүүдийнхээ талархал дэмжлэгийг олохгүй бол юу ч хийж чаддаггүйг үзүүлсэн билээ. Чухамхүү урлаг л зохих ёсны сэтгэл санааны хэрэгсэл болж болох бөгөөд энэ нь сэтгэл зүрхийг нэг болгож чаддаг юм. Ийм учраас Данте, Микаланжело нар нь итали угсаатныг нэгтгэхэд Цезарь, Боржиа болон Макиавеллийгээс яагаад ч багагүй зүйл хийсэн. Эллинчууд Гомер, Гесиода нарыг Ликург болон Солоктой адилхан уншдаг, эртний персүүд Заратустрийг I Дарий Гистадугаас илүүд үздэг зэрэг нь талаар зүйл биш. Оргилох шинж угсаатанд янз бүрийн тунгаар нэвчих үед бүтээлч үйлсээр илэрхийлэгдэх хөгжил явагдана, учир нь уншигчгүй яруу найрагч, багш болон сурагчгүй эрдэмтэн, сонсогчгүй номлогч, офицер болон цэрэггүй жанжин чадахгүй, хөгжлийн механизм нь аль нэгэн хүнд биш, харин оргилох хүчдэлийн ямар нэгэн зэргийг эзэмшсэн угсаатны бүхэллэгт байдаг байна.
Үлдэц угсаатны гишүүд олон давуу талтай байдаг, “шинэ нээгдсэн” индиан, полинез, эскимос, тангад, эвенк, айн нарыг хөршүүд болон жуулчид шагшин магтаж, маш их үнэлж, үүнийгээ ямагт онцлон дурдаж байдаг. Анатоми, физиологийн хувьд тэд өөрийнхөө талбар ландшафтад бүрнээ дасан зохицсон үнэ цэнэтэй хүмүүс, харин тэдний оргилох хүчдэл нь угсаатны хөгжлийн үйл явцыг унтартал бага болсон байдаг. Тэдний дунд оргилуун хүн төрлөө ч гэсэн тэд эх орондоо биш, хөршүүддээ өөрийнхөө хэрэглэгдэхийг хайж байдаг. Жишээлбэл, XY–XYIII зуунд албаничууд эсвэл Венецид, эсвэл Стамбулд алба тушаал хашдаг байсан. Орчин үеийн бушмен, ведди, гонци болон Юкатаны майячуудын удмынхны оргилох шинж бүр ч бага байдаг. Хэрэв ахиад доошоо болбол апати буюу үхлийн өмнөх байдал, өөрөөр хэлбэл доройтол, үхэлд хүрнэ, гэхдээ энэ нь зөвхөн онолын үргэлжлүүлсэн (экстраполяци) тайлбар юм. Практик дээр бол сульдсан угсаатныг мөхөхөөс нь өмнө хөршүүд нь учрыг нь олж амждаг.
Энд үзсэнээр угсаатны амьдралын хамгийн хүнд үе бол оргилох шинжийн халалтын оргил шатнаас инерцээ ухаалгаар аж ахуйчлан хөтлөхөд шилжих хугаралтын шат байдаг, харин дараа нь бодлогогүй тайван зогсолтын шатанд орно. Зохиролт болон дэд оргилуун хүмүүсийн хувь асч, бүтээлч, эх оронч хүмүүсийн хүчдэл юу ч үгүй болон багасаж, тэднийг “фанатикууд” хэмээн нэрлэж эхэлдэг. Чухамхүү “өөрийн” хүмүүстээ үзүүлэх дотоодын дэмжлэг байхгүй явдал нь цөөн тоотой, гэхдээ оргилох шинжтэй дайснуудаасаа болж угсаатан мөхөхийг тодорхойлдог. ХХ зууны нэгэн зохиолч үхлийнхээ өмнө “Хайхрамжгүй хүмүүсээс айгтун” гэж хэлсэн байдаг юм аа.
Бут цохигдох замаар ч тэр, уусан нэгдэх замаар ч тэр угсаатны мөхлийн өмнө түүний дотоод бүтэц хялбаршиж, зан үйлийн тогтсон үзэл явцуурдаг гэдгийг дээр өгүүлсэн билээ. Өөрийнхөө орчин дахь туйлшрагч хүмүүсийг устгадаг ядмаг байдал нь уул буюу цөл үлдэц–зожиг угсаатны сүүлчийн хоргодох байр болсон тэрхүү цөөн тохиолдлыг эс тооцвол хамт олонд хэрэгтэй эсэргүүцэх чанарыг үгүй хийдэг. Филогенез буюу удам судлал болон угсаатны нийлэгжилтийн хооронд бүрэн биш ч, тодорхой адилтгал байдаг боловч нийгмийн дэвшилтэт хөгжил нь бүрэн шавхсан түүхэн материализмын онолд судалснаас огт өөр зүй тогтолд захирагддаг.
ЭРЛИЙЗҮҮД (БАСТАРД)
Хэрэв онцгой нөхцлийн шинж тэмдэг болох оргилох шинжээ алдах үйл явц нь нийгмийн нөхцлөөс гадуур явагдвал энэ нь хурдан, илтийн, практик үр дүнгүй байх байсан. Гэхдээ нийгэм-эдийн засгийн үйл явцуудтай байнга харилцан үйлчлэх үед угсаатны түүхийн нарийн зөрчилдөөн дунд оргилох шинжийн алдагдлын үүрэг болон ач холбогдол зарим талаар унтардаг. Ийм учраас материалын бүрэн бишэд тулгуурласан эргэлзээнээс зайлсхийхийн тулд сайтар судалсан эрин үеүдээс жишээ авч, түүх рүүгээ эргэн оръё.
Баруун Европын оргилуун хүмүүсийг колонийн чичрэг өвчин хөнөөж, Ост болон Вест Энэтхэгээс тун цөөн хүн эргэн ирсэн ба гажиг үр удам үлдээгч тэмбүү өвчин хөнөөсөн юм. Тэмбүү нь хүмүүсийг сонголттойгоор иддэг байв. Түүгээр бүхнээс илүү усан цэрэг цэргүүд өвдөж байсан ба тэд нар нь тэр үедээ эсвэл сайн дурынхан, өөрөөр хэлбэл оргилуун хүмүүс, эсвэл “тэнүүлч цэргүүд”, өөрөөр хэлбэл дэд оргилуун хүмүүс байсан билээ. Хот болон тосгодын хүн амын зөнгөөрөө хэсэг энэ хоёр лайнд бага нэрвэгдэж, ийнхүү системийн оргилох хүчдэл багассан юм. Гэхдээ энэ үйл явц бодож байснаас ч удаан явсан юм. Оргилох шинжийн бууралтад саад бологч нөхцөл байдал байжээ.
Оргилуун хүмүүс дайнд үрэгдэхээсээ өмнө тохиолдлын холбооны замаар өөрийнхөө генийн санг бүлд цацаж амжсан байдаг. Цуст алалдаанд залууг түлхэж буй үйлдлийн цангаа нь үе тэнгийн бүсгүйчүүдийн сэтгэлийг дуудаж, тэд үүнийгээ залуусыг ойртуулах аргаар илэрхийлж байв. Оргилох өндөр хүчдэлтэй эрин үеүдэд ийм бүсгүйчүүдийг хэт чанд шүүж байгаагүй, харин хуурмаг зан оргилох шинж царцахад л хамт ирдэг ажээ. Дундад зууны үед “бастард” хэмээх үг нь доромж утгатай байгаагүй билээ. YII Карлын үед Францын цэргийн томоохон зүтгэлтэн Дюнуа бастард–жонон гэж нэрлэгдэж байлаа. Түүнтэй адил хүмүүс олон байсан юм. Зуун жилийн дайны үед гэр бүлээс гадуур төрсөн ихэс дээдсийн хөвгүүд, борчуудын угсааны охидууд тэнэмэл цэргүүдийн, өөрөөр хэлбэл дэд оргилуун хүмүүсийн манлайлагч болж, “цагаан отряд” гэгчийг дүүргэж, өөрсөддөө рыцарийн хүндлэл, нэрүүдийг булаан авсан байна. Энэ отрядууд нь “эх орондоо ч, түүний чанадад ч зөвхөн ганцхан хувийн ашиг хонжоо хайсан, ядуу, гэхдээ мохошгүй, хүчтэй хүмүүсээс бүрдэж байлаа”. 28. Архив К. Маркса и Ф. Энгельса. Т. VI. М., 1939. С. 347.
1431 онд Лотарингийн төлөө хийсэн дайнд бургундын герцог Сайн санаат Филипп нь де Юмьер, де Бримен, де Невилль зэрэг бастардууд, мөн Шиндерганнесийн овгийн бастард шаламгай Робиныг албандаа авсан юм. Энэ хүмүүс Филиппийн ялалтыг хангажээ. 29. Там же. С. 348.
Дорно дахинд бүр ч хялбар байв. Полигами буюу олон нөхөр, эхнэртэй байх ёстой байсан араб, түрэг, монголчууд нь өөрийнхөө бүр хүүхдүүдийг, тэр ч байтугай олзлогдсон хүүхнүүдийн хүүхдийг ч “хууль ёсны” гэж үздэг байв. Анхын эхнэрийн болон татвар эмийн хүүхдүүдийн ялгааг зөвхөн ширээ залгамжлахад л харгалздаг байсан, гэхдээ хүн амын олонхид энэ бол чухал зүйл биш байсан юм. Эрчүүд нь хэчнээн оргилуун байх хэрээр эмэгтэйчүүдийн ген дамжуулах чадвар нь мөн тийм байсан. Ийм учраас эхнэрүүдийн гэрт (гарем) анхдагч генийн сан алдагдах нь Европыг бодоход зовиур багатайгаар угсаатны нийгмийн оргилох шинжийн түвшний хувилбаруудыг бий болгодог байжээ.
Зан үйлийн ийм тогтсон үзэл нь оргилох шинжийн тэмдгүүдийг төөрөлдөхөд хүргэдэг байна. Энэ нь оргилох шинж ямар нэгэн ангид байдаг гэх үзлийг үгүй болгож байна. Хэрэв нөхцөл байдлын тохиолдлын урсгалаар ийм давхцал гарч болох ч дараагийнхаа үед энэ нь зөрчигдөж (үйлчилж буй ёс заншил байгаа үед ч гэсэн), аль нэгэн бүлгийн бүрэлдэхүүнд “хууль бус” гэж нэрлэгдсэн хүүхдүүд бий болох ба эдгээр нь хууль ёсны өвөг дээдсээсээ биш, бодитойгоор уламжилж авсан өөрийнхөө оргилох шинжийн түвшингийн дагуу биеэ авч явдаг.
Жишээлбэл, XYII зуун хүртэл Францад язгууртнууд нь хаалттай каст буюу бүлэг байгаагүй юм. Хааны дэргэдэх албанд буй дурын эрчимтэй хүн жинхэнэ язгууртан болж чаддаг байв. Энэ байдалд Ришелье–ийн тушаалаар гугенотын дайны дараагаар өөрийгөө язгууртан гэж зарласан хүн шалгах үед язгууртан-өвөг дээдсийнхээ хэд хэдэн үеийг нэрлэх ёстой гэсэн зарим хязгаарлалт оруулсан билээ. Ингээд яагаав? XIY Людовикийн үед бараг бүх сайд, цэргийн олон алдартнууд, бүх агуу зохиолчид (Фенелон, Ларошфуко, хатагтай Севинье–гээс бусад) нь язгууртан биш, харин хөрөнгөтнүүд байжээ. 30. Тьерри О. Избр. соч. С. 193.
Тэд феодалын хаант улсад харсаар байтал “хууль ёсны” өвөг дээдэс нь эдэлж байгаагүй, өөрийнхөө ажил хэрэгч чанарын ачаар тэргүүлэх үүрэг гүйцэтгэж байжээ. Өөрөөр байсан бол хааны албанд язгуур угсааны хязгаарлалт үнэндээ байгаагүй Гуа Филипп, Ухаант Карлын үед дэвшиж болох л байсан. Үнэн хэрэг дээрээ хэрэв оргилуун хүмүүс нийгмийн нэг бүлэгт бөөгнөрсөн байсан бол анхны цус урсгасан дайн нь оргилуун хүмүүсийн бүх бүлийг устгаж, угсаатны нийлэгжилтийн эхлэн буй үйл явц анхныхаа шатанд тасалдах байсан биз ээ. Ийм зүйл болоогүйг бид харж байна.
Угсаатны дахин сэргэх (регенераци), өөрөөр хэлбэл донсолгооны дараа угсаатны бүтэц дахин тогтох тийм үзэгдэл байнга ажиглагддаг юм. Энэ үед эх орноо аврагчид нь түүнийг үндэслэгчидтэй төстэй авч, хууль ёсны өвөг дээдэст нь байсныг бодоход хэмжээлшгүй их оргилох шинж гаргадаг. Хэдийгээр тэдний бий болохыг сурвалжуудад тун цөөн тэмдэглэсэн байдаг боловч бастардууд бүх эрин үед, бүх ард түмнүүдэд байсан юм. Гэхдээ энэ нь тэднийг чухал биш гэж үзэх үндэслэл болохгүй.
Угсаатны дахин сэргэх механизм ийм юм. Угсаатныг бүрдүүлж буй дэд угсаатнуудын дотор голдуу нэг нь илүү санаачлагатай, улмаар хөтлөгч нь болж байдаг. Энд хүний оргилох шинж нь эрчимтэйгээр үйл болон хувирдаг, учир нь оргилох шинжийн зарцуулалтын үйл явц маш хурдан явагддаг. Энэ нь бусад дэд угсаатнуудаар дүүргэгддэг, гэхдээ энд эргэх гүйдэл байдаг. Энэ нь оргилох шинжийн удмын сан гэрлэлтийн гадуурх холбоогоор дамжин бүхий л бүлээр сарниж, хүүхэд нь өөрийнхөө дэд угсаатны орчин, нарийн яривал эхийнхээ гэр бүлд үлдэх явдал болно. Ийм учраас системийн оргилох шинжийн зарцуулалт удааширдаг юм.
Хөтлөгч дэд угсаатан боломжоо шавхан дуусаад сүйрэлд орвол захын дэд угсаатнуудын аль нэг нь тасрахад ч, үргэлжлэхэд ч бэлхэн байж угсаатны нийлэгжилтийн үйл явцын халааг залган авдаг. Энэ нь зохицуулж өгсөн гэрлэлтийн холбооны үед явагдаж чадахгүй, учир нь эцэг эхчүүд нь хүүхдээ тэдний балрах тавиланг хуваалцах ёстой, хүний тэмүүллийн халуун дулаанд барьж байх ёстой байдаг. Ингээд овгийн (угсаатны) түүхээ алдах үнэ цэнээр тэр амьдралаа хадгалан үлдэж байна.
Угсаатны дахин сэргэлт нь соёлын хөгжлийн хөдөлгөөнийг тухайн системийн хүрээнд дагалдуулдаг нь ойлгомжтой юм. Үүний ачаар угсаатан арчаагүй оршихуйг биш, идэвхитэй уран бүтээлч амьдралын хугацааг сунгадаг юм. Энэ нь ч гэсэн зан үйлийн зохист хэм хэмжээг эвдлэгч зөн билгийн хослолыг адислахад хангалттай юм. Байгал хүний бодол санаанаас хүчтэй бөлгөө.
УГСААТНЫГ ЮУ ГАГНАДАГ ВЭ ?
Угсаатны нийлэгжилт буюу угсаатан бүрэлдэх хөдлөнги шинжийн чанарын тухай асуултад хариулсаар байгаад бид: угсаатны тогтвортой байдлын шалтгаан юунд байдаг вэ ? гэсэн багагүй чухал асуудалд дөхөн ирлээ. Олон угсаатнууд бараг тэгтэй тэнцүү гэж тооцмоор тийм сул оргилох шинжтэй мөртлөө үлдэц төлөв байдалдаа оршсоор л байдаг. Бушменууд, австраличууд, Деканы негржүү төрлийн одой–пигмей, палеоаазийнхан зэрэг нь ийм юм. Орчин үеийн бага үндэстний хэлбэрээр оршин байдаг, оргилох шинж нь ердөө л амьдралын хэв маягаа тэтгэх болтлоо суларсан, өвөг дээдэстэйгээ адилхан оргилуун хүмүүсийг ховор төрүүлдэг бүрнээ хэвийн угсаатнуудын жишээ бүр ч олон байдаг. Жишээлбэл, оргилох шинжийн энэ түвшинд скандинавын орнууд болон Нидерландын оруулж болно.
Бүтээсэн материаллаг бааз, удирдлагын туршлага болон нийгэм – техникийн бусад хүчин зүйлүүд уналтын хандлагыг сөрөн зогсдог байна. Учир нь угсаатан өөрийнхөө амьдрах хугацаанд ямар нэг хэт угсаатны системийн хүрээнд үйлдэн амьдарч байдаг, ингээд хэт системийн элементүүдтэй “эрчим хүчний” солилцоонд ордог байна. Энэ нь оргилох шинжийн түвшний хэлбэлзлийг нөхцөлдүүлдэг. Угсаатны холбооны гадаад системийн үйл ажиллагаа нь хөгжил хурдсахад ч, уналт, тэр ч байтугай мөхөлд учрахад ч хүргэж болно. Хэрэв солилцооны хэмжээ зарим хязгаарын утгыг давбал угсаатны амьдралын янз бүрийн агшинд зарчмын хувьд янз бүр байна.
Одоо бид голчлон өөр хоорондоо төсгүй, янз бүрийн хүмүүсийг угсаатан гэж нэрлэгдэх бүхэллэгт чухам юу ингэж гагнадаг вэ ? гэсэн асуулт тавих эрхтэй юм. Нийгмийн буюу тооцооллын өөр системд энэ үүргийг нийгмийн өөдлөн хөгжих чадвар болсон үйлдвэрлэлийн харилцаа гүйцэтгэдэг. Гэхдээ угсаатнуудад тооцооллын өөр систем байдаг, үйл явдлыг тэдгээрийн холбоо, дэс дараалалд нь судалдаг түүхийн шинжлэх ухаан нийгмийн институтууд үүсэх, алга болохыг гайхамшигтай харуулдаг боловч яагаад афинчуудад энхийн худалдаа хийж байсан финикуудаас илүүтэйгээр спартчууд ойрын дайсан болсон бэ ? гэх мэтийн асуултад хариулах чадваргүй байдаг. Тэд сайндаа л афинчууд болон спартчууд нь эллинчүүд байсан, өөрөөр хэлбэл хэдийгээр улс төрийн талаар тархай бутархай байсан боловч нэгдмэл угсаатан байсан гэдгийг л тэмдэглэж чадна. Тэгвэл угсаатан гэж юу ? тэдгээрийн гишүүд нь юугаараа холбоотой ? гэсэн асуултад түүх хариулдаггүй. Ингэхлээр байгалд л хандах хэрэгтэй болж байна.
Угсаатны түүх (байгалийн түүхийн илрэл) болон хүний гар, ухаанаар бүтээгдсэн соёлын түүхийн хоорондын ялгаа хаана байгааг бид нэгэнт мэдэж байна. Амьдрал үүсэн дүрэлзэж үхлээр дуусдаг, үүнийг ч жам ёсны үйл явцын төгсгөл гэж, хэрэв тэр нь цагаа олсон, зовиургүй байвал хүсч болох зүйл ч мэтээр ойлгодог. Био хүрээний бүх үйл явц тасралттай (дискрет) байдгийн учир нь энэ юм. Харин тасралтгүй хөгжилд үхэхийн ч, төрөхийн ч байр байхгүй жамтай.
Харин соёлын түүхэнд бол бүгд урвуугаар байдаг. Ордон болон сүм хийдийг он жилээр бүтээдэг, ландшафт зуун зуунаар нөхөн сэргэдэг, эрдэм шинжилгээний бүтээл, яруу найргийг олон арван жилээр бүтээдэг …бүгдээрээ л үхэшгүй мөнхөд найдаж хийгджээ. Энэ итгэл цагаатгагдсан юм: хүний бүтээлд үхэл өгөгдөөгүй, харин удаан удаан эвдрэл, марталт л бий. Эдгээр бүтээгдсэн зүйлст өөрийн оргилох шинж байхгүй, гэхдээ түүнийг бүтээгчийн амьгүй бодисоор хэлбэр оруулсан, өөрөөр хэлбэл хүмүүсийн мэдрэхүй болон тэмүүллийн шаталтаар бүтээгдсэн талстууд л бий. Харамсалтай нь эдгээр талстууд нь биоценозын конверсаас гарчихсан учраас хөгжих, хувиран өөрчлөгдөх чадваргүй байдаг. Үхэх эрх нь амьд юмсын давуу эрх мөн.
Чухам ийм учраас л угсаатнуудын бүтээсэн, археологичдын судалдаг соёл нь нэгдүгээрт төөрөгдөлд оруулж, хоёрдугаарт бүтээлийг бүтээлчтэй нь адилгах, юмс болон хүмүүсийн хооронд адилтгал хайхад хүргэж зовоодог. Тэр тусмаа ингэж шохоорхох нь бүлээс оргилуун хүмүүс явахын хэрээр тэнд олон хүн, бүр ч олон эд юмс, мөн зарим тооны үзэл санаа үлддэг болохоор илүү аюултай. Соёл нь унтарч буй оддын адилаар ажиглагчийг хуурч, харагдах зүйлийг бодитой зүйл мэтээр харагдуулдаг. Гэхдээ үзэгдлийг дүрслэхээс тайлбарлахад шилжих нь судалгааны өөр аппарат, өөрөөр хэлбэл батлагдаагүй, гэхдээ тодорхой бүх баримтууд, тэдгээрийн харилцан холбоог тайлбарласан үзэл–таамаглал хэрэглэхийг шаарддаг юм. Энд бид байгалийн ухааны салбарт шилжин орж байна.
