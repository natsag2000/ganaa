Дөрөвдүгээр хэсэг

ГАЗАР ЗҮЙ ДЭХ УГСААТАН
ЭНД УГСААТАН НЬ ХҮНИЙГ ХҮРЭЭЛЭН БУЙ БАЙГАЛТАЙ БАЙНГЫН ХАРИЛЦАМЖИД БАЙДГИЙГ ХАРУУЛНА, МӨН ЗОРИЛТЫГ ШИЙДВЭРЛЭХ ГЭЖ ТООЧСОН БҮХЭН ХАНГАЛТГҮЙ БОЛОХ ТАЛААР ГОМДОЛЛОНО.
XIV. Эргүүлэн тавьсан зорилт
УГСААТАН БОЛ БАЙГАЛИЙН ҮЗЭГДЭЛ
Ийнхүү дэвшигдэн буй асуудалд хамаарах одоо болтол бидний авч үзсэн бүхий л шинжлэх ухаанууд тулхтай хариулт өгөөгүй төдийгүй, үнэнийг хайх цаашдын замыг ч зааж чадахгүй байна. Энэ нь хариуцлагаас мултрах ёстой гэсэн хэрэг бус уу ? Үгүй ээ, энэ нь маш энгийн юм. Бид тавигдсан асуудлыг шийдвэрлэхэд тохиромжтой судалгааны шинэ арга олж болно шүү дээ. Үүний эхлэлийг эрт тавьсан: Соёл иргэншлийг тээгч хүний байгал орчинтой харьцах асуудалд “угсаатан” гэсэн ойлголт оруулж, түүнийг: өөрийгөө бусад адилтгах хамт олонд сөргүүлэн тавьсан, дотоод бүтэц бүхий, тохиолдол бүрт өвөрмөц байдаг зай үйлийн хөдлөнги тогтсон үзэлтэй хүмүүсийн тогтвортой хамт олон гэж тодорхойлсон. Чухамхүү угсаатны хамт олноор дамжин хүний байгаль орчинтой тогтоосон холбоо хэрэгждэг юм. Иймээс угсаатан нь байгалийн үзэгдэл мөн.
Угсаатан яг байдлаараа өөрийг нь багтаан буй газар зүйн орчны өөрчлөлтийн улмаас үүсэн, хөгжиж, алга болж байх ёстой мэт санагддаг. Энэхүү орчин нь маш хөдөлгөөнтэй байдаг. Янз бүрийн бүсүүдэд удаан хугацааны ган буюу эсвэл урвуугаар хэт чийглэг ажиглагддаг бөгөөд чингэхдээ ландшафтын өөрчлөлт, мөн тэдний өөр хоорондын харьцааг тодорхойлогч цаг агаарын өөрчлөлтийн идэвхи нь Дэлхийн янз бүрийн бүс нутагт янз бүр байна. Түүхэн үйл явдлуудыг цаг агаарын хэлбэлзэлтэй шууд холбоо тогтоох эрмэлзэл нь угаасаа азгүйддэг гэдгийг Франц болон түүний захын орнуудад анхаарлаа төвлөрүүлж байсан Э.Леруа Ладюри нотолсон байдаг. Гэхдээ бидний нэгэнт санал болгосон арга зүйг хэрэглэж, францын түүхчийн хэт шүүмжлэлт үзлээс зайлсхийж холбогч , нарийн холбоог тогтоож болно. 1. Леруа Ладюри Э. История климата с 1000 года.
Европын зөөлөн уур амьсгалд ландшафтын ялгаа нэлээд халхлагддаг, харин эх газрын уур амьсгалын нөхцөл, өргөн орон зайд эрс өөрчлөгддөг. Энд бид шалгуур үзүүлэлт болгон газар зүйн янз бүрийн талбарт нүүдлийн хүн амын бүлгүүд дэх улс төрийн байгууллын шинж чанарыг ашиглаж болох юм. Үүнийг бид аль хэдийн ландшафт бүрдэлтийн цаг агаарын үйл явцын хөдлөнги шинжийг тайлбарлах гэж нэг удаа хийсэн билээ. 2. Гумилев Д. Н. Открытие Хазарии. М., 1966.
Одоо бид түүх–газар зүйн бүсчлэлд, өөрөөр хэлбэл Евроазид нутагшигчдын улс төрийн системийг тэндэх угсаатнуудын ахуйчлан орших хэлбэр болохынх нь хувьд ангилан үзэхэд анхааръя.
Ард түмний улс төрийн систем нь эдгээр орны ард түмний нутаглаж буй аж ахуйн системтэй нягт холбоотой байдаг. Яг энд анхны бэрхшээл үүснэ: НТӨ IX зуунаас эхлэн НТ XYIII зуун хүртэл евроазийн тал газарт нүүдлийн мал аж ахуй гэсэн үйлдвэрлэлийн ганцхан арга оршиж байв. Хэрэв ерөнхий зүй тогтлыг засваргүйгээр хэрэглэх аваас бид нүүдлийн бүх нийгэм нэг л маягаар байгуулагдсан бөгөөд аливаа дэвшлээс асар хол хөндий, тэдгээрийг нийлүүлэн тодорхойлбол овгийн ялгааг л нарийвч хэсэгт хамааруулж болно гэж үзэх ёстой. Ийм үзэл бодлыг XIX болон ХХ зууны эхэнд үнэхээр аксиом гэж үзэж байсан, харин баримтат материалын хуримтлал нь түүнийг няцаахад хүргэсэн юм. 3. Гумилев Л. Н. Древние тюрки. С. 87-100.
Бэлчээрийн талбай, малын тоо толгой, хүн амын тооны хоорондын тогтвортой харьцаа байсан хэдий ч евроазийн тал газарт нийгэм–улс төрийн системийн нэгдмэл байдлын сүүдэр ч байгаагүй юм. Гэлээ гэхдээ нүүдлийн соёл нь өөрийн 3 мянган жилийн оршихуйн хугацаандаа Газрын дундад тэнгис болон Алс Дорнодын орнуудаас дутахааргүй хурц, гоёмсог бүтээлч хувьслыг туулсан билээ. Гэхдээ л нутгийн нөхцөл нь нүүдэлчдийн түүхэнд огт өөр өнгө аяс өгсөн юм. Бидний зорилт бол нүүдлийн болон газар тариалангийн нийгмийн системийн хооронд төстэй элементүүдийн зэрэгцээгээр ялгаатайг нь олон тогтоож, тэдгээрийн боломжит шалтгааныг заахад оршино.
Юуны өмнө газар зүй (мэдээж эдийн засгийнхаас бусад), улмаар түүнд багтан ордог этнологи нь байгалийн шинжлэх ухаан, харин түүх бол хүмүүнлэгийн шинжлэх ухаан гэдгийг тэмдэглэвэл зохино. Энэ нь угсаатны нийлэгжилтийг (угсаатан үүсэх болон устах) био хүрээнд (Дэлхий гарагийн бүрхэвчийн нэг) болж буй байгалийн үйл явц гэж үзнэ гэсэн хэрэг бөгөөд судлаач хүн бүс нутгийн угсаатны түүхийг бүртгэхдээ газар зүйн аргуудыг хэрэглэдэг. Түүхийн шинжлэх ухааны уламжлалт аргуудыг хэрэглэхдээ түүн дээр газар зүйн, мэдээж хэрэг сургуулийн биш, харин антропогенийн биоценоз, зөвхөн хүний зан үйлийн шинжийг өөрчилдөг бичил бүл (мутаци), нүүдлийн үйл явцтай холбогдсон сукцесси буюу солигдон өөрчлөгдөх зэргийн салбар онцлогийн тухай асуудлыг дэвшүүлэн тавьдаг орчин үеийн, шинжлэх ухааны өгөгдөхүүнүүдийг л нэмэх ёстой. Хэрэв угсаатныг “нийгмийн категори” гэж үзэх юм бол энэ нь угсаатны хөгжилд газар зүйн хүчин зүйлүүд “ач холбогдолтой байж чадахгүй” гэсэн хэрэг болно. Энэ сэдвийн утгагүй нь “тэд тусгай угсаатны хөгжлийг хүчтэй хурдасгах буюу урвуугаар хурдасгаж чадах байсан” гэж бичсэн зохиогчид өөрт нь хүртэл илэрхий байна. Хэрэв үүнийг сүүлчийн, үнэн дүгнэлт гэж үзье, тэгвэл урьдчилсан нөхцлийн ёсоор угсаатан социал нийтлэг биш аж. 4. Козлов В. И. О биолого-географической концепции этнической истории // Вопросы истории. 1974. № 12. С. 72. 5. Там же. С. 80.
Ф.Энгельс 1890 оны есдүгээр сарын 21 – 22–нд И.Блоход бичсэн захидалдаа: “Түүхийг ойлгох материалист үзлийн ёсоор түүхэн үйл явцад тодорхойлох зүйл нь эцсийн эцэст бодит амьдралын үйлдвэрлэл байдаг. Би ч тэр, Маркс ч тэр үүнээс илүүг батлаагүй. Хэрэв хэн нэгэн нь энэхүү үндэслэлийг эдийн засаг л цорын ганц тодорхойлогч зүйл болно гэсэн утгаар гажуудуулбал энэ нь юуг ч өгүүлээгүй, хийсвэр, учир утгагүй үг болон хувирна” гэж хэлснийг сануулъя. 6. Маркс К., Энгельс Ф. Соч. 2-е изд. Т. 37. С. 394.
Энэхүү сэдвийн дагуу бид угсаатны нийлэгжилтийн дурын шууд ажиглагдаж буй үйл явц нь нийгмийн зэрэгцээгээр байгалийн талтай байдаг гэж үзэж болно.
БИОЦЕНОЗ ДАХЬ ХҮН
Сээр нуруутны бүх төрлүүдэд: өсөн үржих, үр зулзгаа халамжлахаар илэрдэг хувийн болон зүйлийн өөрийгөө хамгаалах инстинкт, боломжит их талбайд тархах, орчиндоо дасан зохицох (адаптаци) эрмэлзэл хэвшмэл байдаг. Гэхдээ энэхүү дасан зохицохуй нь хязгааргүй байдаг. Амьтнууд голчлон өвгүүд нь дасан зохицсон газрын гадаргуугийн тодорхой хэсэг дээр амьдардаг. Баавгай цөл рүү явдаггүй, минж өндөр ууланд гардаггүй, туулай загас хөөж гол руу харайдаггүй. Гэвч бүр ч илүү хязгаарлалтыг янз бүрийн бүсүүдийн цаг агаарын ялгаа, бүслүүрүүдийн шинж тавьж байдаг юм. Халуун бүсийн зүйлүүд туйлын бүслүүрт оршиж чадахгүй, урвуугаар ч мөн адил. Улирлын шинжтэй нүүдэл боллоо ч гэсэн тэр байгалийн нөхцлийн шинж чанартай холбоотой тодорхой чиглэлээр явагдана.
Энэ талаараа хүн онцгой билээ. Тэрээр нэгдмэл зүйлд хамаардаг ба манай гаригийн бүх хуурай газраар тархсан юм. Энэ нь түүний дасан зохицох нэн өндөр чадвартайг харуулж байна. Энд анхны бэрхшээл үүсч байна: хэрэв хүй нэгдлийн хүн ухаандаа дулаан бүслүүрийн ойн бүсэд дасан зохицсон байлаа гэхэд чухам юуны тулд дассан хоол хүнс, таатай нөхцөл нь байхгүй цөл юмуу халуун бүсийн ширэнгэд татагддаг байна аа, амьтан болгон өөрийнхөө геобиоценезэд (үгчилбэл амьдралын аж ахуй), өөрөөр хэлбэл “оршихуйн нөхцлийн нэгэн бүхэл нийтлэгт түүх, экологи, физиологийн хэлбэрүүдийн зүй тогтолт бүрдэл”–дээ байдаг бус уу ? Дүрслэн хэлбэл биоценоз нь амьтдын байшин юм, тэгвэл төрөлх гэрээсээ яагаад явдаг юм бол ? 7. Калесник С. В. Основы общего землеведения. С. 359.
Биогеоценез нь маш нарийн систем бөгөөд тэр нь “хоол тэжээлийн гинж”- ээр өөр хоорондоо холбогдсон ургамал болон амьтад, нэг зүйл нь нөгөөгөө тэжээж байдаг, харин дээд, төгсгөлийн салбар болох том махчин буюу хүн үхэхдээ өөрийн сэгийг түүнийг тэжээгч ургамалд өгч байдаг бусад зүйлүүдийн үйл ажиллагаанаас бүрдэнэ. Тухайн биоценезэд өндөр хэмжээгээр дасан зохицохын хэрээр зүйлүүд нь хувьслын эргэлт буцалтгүйн тухай хуулийн дагуу зайлсхийж болдоггүй олон шинж тэмдгийг хуримтлуулдаг. Энэ бүхэн нь хүнд ч гэсэн хамаарах авч тэрээр яаж ийгээд эдгээр бэрхшээлийг даван туулж, Дэлхийгээр нэг тархжээ. Хүнийг бусад зүйлүүдтэй харьцуулахад дасан зохицох бага эрэмбийнхээ улмаас ихээхэн уян хатан чанартай болсон гэж хэлж болохгүй юм. Тэр нь хүнд асар их билээ.
Үгүй, томоохон биоценез бүрт хүн хатуу байр суурь эзэлж, шинэ нутагт суурьшихдаа өөрийнхөө организмын анатоми буюу физиологийг өөрчилдөггүй, харин зан үйлийн тогтсон үзлээ л өөрчилдөг байна. Тэгэхлээр энэ нь хүн шинэ угсаатан бүтээж байна гэсэн хэрэг юм. Тийм байж, тэгвэл энэ нь түүнд ямар хэрэгтэй юм бэ? Эсвэл бүр тодорхойгоор юу түүнийг ингэж түлхдэг байна аа? Хэрэв энэ асуултад зүгээр л хариулж болох байсан бол бидний зорилт шийдэгдчихсэн байхсан. Гэвч бид сөрөг хариултаар хязгаарлахад арга буюу хүрч байгаа ба үүний утга нь асуудлыг хязгаарлахгүйн тулд юм.
Энд биологийн, нарийн яривал зоологийн шалтгаанууд үйлчлэхгүй, учир нь хэрэв эдгээр нь үйлчилж байсан бол бусад амьтад ч гэсэн ийн үйлдэх байсан. Өөрийн мөн чанарын өөрчлөлтийн тухайд ухамсартайгаар шийдвэрлэх нь утга учиргүй хэрэг болно. Үүний шалтгааныг нийгмийн шинжтэй болгоод үзсэн ч гэсэн тэр нь үйлдвэрлэлийн аргын өөрчлөлттэй, өөрөөр хэлбэл нийгмийн формаци халагдахтай заавал холбоотой байх ёстой, гэтэл энэ нь байдаггүй. Түүнээс гадна дасаж, төвхнөсөн “багтаагч” ландшафтдаа заавал дасан зохицохуйг К.Маркс “Албадуулсан дүрвэлт” өгүүллэгтээ тэмдэглэсэн байдаг. Тухайлбал, нүүдэлчдийн талаар тэнд: “Бүдүүлэг байхаа үргэлжлүүлэхийн тулд тэд цөөн тоотой үлдсэн байх ёстой. Энэ нь одоо ч гэсэн (XIX зууны дунд үе–Л.Г.) Хойд Америкийн индиан овгуудад байдаг шиг мал аж ахуй, ан агнуур, дайн байлдаанаар амьдардаг, тэдгээрийн үйлдвэрлэлийн арга нь овог аймгуудаас овгийнхоо тусгай гишүүн нэг бүрт уудам орон зай шаарддаг аймгууд юм. Эдгээр овгийн хүн амын өсөлт нь тэд бие биенийхээ үйлдвэрлэлд зайлшгүй газар нутгийг багасгахад хүргэдэг” хэмээн бичжээ. 8. Маркс К., Энгельс Ф. Соч. Т. 8. С. 568.
Энгельс Марксын санааг хөгжүүлж, янз бүрийн овгийн хөгжлийн түвшинг хоол хүнстэй шууд холбоотойг заасан байдаг. Түүний бодлоор : “Арийчууд болон семитуудын махан болон сүүн арвин хүнс тэжээл, ялангуяа хүүхдийн хөгжилд түүнийг үзүүлэх таатай нөлөөллийг энэхүү хоёр арьстан илүү амжилттай хөгжсөнд хамааруулж болох юм. Үнэхээр бараг зөвхөн ургамлын хүнсээр арга буюу хооллож буй Шинэ Мексикийн пуэбло овгийн индиануудын тархи бүдүүлэг байдлын доод түвшинд оршиж, загас, махаар илүү хооллодог индианчуудынхаас бага байдаг” 9. Там же. Т. 21. С. 32.
ФОРМАЦ СОЛИГДОХОД ГАЗАР ЗҮЙН ОРЧИН НӨЛӨӨЛДӨГГҮЙ
Ингээд угсаатанд ландшафт шууд болон дам нөлөө үзүүлдэг нь эргэлзээгүй болж байна, гэхдээ даяар өөрийн хөгжил–материйн хөдөлгөөний нийгмийн хэлбэрт тэр нь шийдвэрлэх нөлөө үзүүлдэггүй. Тэглээ ч гэсэн угсаатны үйл явцад ландшафт нь албадмал нөлөө үзүүлнэ. 10. Берг Л. С. Номогенез. Пг., 1922. С. 180-181.
Италид суурьшсан бүх ард түмэн: этруск, латин, галл, грек, сири, лангобард, араб, норманн, шваб, францууд нь хоёр буюу гурван үеийн туршид аажмаар өмнөх дүр төрхөө алдаж, түүхэн цаг хугацаанд хувьсан өөрчлөгдсөн бүтэцтэй, зан авир, зан үйлийн өвөрмөц шинжүүд бүхий зүймэл угсаатан болох өвөрмөц италийн олон түмэнд (масса) уусан нэгдсэн юм. Энэ байдал нь судалсан хэмжээтэй шууд хамааралтайгаар их бага байдлаар газар сайгүй байдаг юм. Иймээс бид угсаатныг нийгмийн дэвшлийн функци байдлаар биш, харин бие даасан үзэгдлийнх нь хувьд судлах ёстой.
Хүн төрөлхтөн бүрэлдэхүй нь,“бусад амьтдад байдаг шиг зөвхөн байгалийн үйлчлэлтэй төдийгүй, мөн техник болон социал институтын онцгой оргилсон хөгжилтэй холбоотой” байдаг. 11. Калесник С. В. Проблемы географической среды //Вестник ЛГУ. 1968. № 12. С. 91-96.
Практик дээр бол бид хөгжлийн энэ хоёр шугамын интерференц буюу харилцан шилжилтийг ажигладаг. Ингэхлээр формаци дамждаг нийгэм–эдийн засгийн хөгжил нь газар зүйн орчинд явагддаг, тасралттай үйл явц, угсаатны нийлэгжилттэй адилгүй юм. С.В.Калесник хүмүүсийн нэгэн зэрэг амьдардаг газар зүйн болон техногенийн орчны хоорондын ялгааг тодорхой үзүүлсэн. Газар зүйн орчин нь хүний оролцоогүйгээр үүссэн ба өөрөө хөгжих чадвар бүхий байгалийн элементүүдийг агуулж байдаг. Техногенийн орчин нь хүний хөдөлмөр, хүсэл зоригоор бүтээгддэг. Түүний элементүүд нь унаган байгаль дахь адилтгах зүйл байхгүй бөгөөд өөрөө хөгжих чадваргүй. Тэдгээр нь зөвхөн эвдрэн сүйрч л чаддаг. Техно болон социо хүрээ нь хэдийгээр байнга харилцан үйлчилж байдаг боловч газар зүйн орчинд ерөөсөө хамаардаггүй. Энэ зарчмыг бид судалгааныхаа суурь болгосон юм. 12. Калесник С. В. Проблемы географической среды //Вестник ЛГУ. 1968. № 12. С. 91- 96
БАЙГАЛТАЙ ХИЙСЭН ХҮНИЙ ДАЙН
Хүний дээр дурдсан дасан зохицох чадвар нь түүнийг өвөг дээдэстэй нь харьцуулахад харьцангуй нэмэгдээгүй, харин хүнийг бусад сүүн тэжээлтнээс ялгаж буй онцлогтой холбоотой юм. Хүн зөвхөн ландшафтад дасан зохицдоггүй, харин ландшафтыг өөрийн хэрэгцээ, шаардлагад тохируулан өөрчилдөг. Ингэхлээр янз бүрийн ландшафтуудыг туулах замыг түүнд дасан зохицохуй биш, харин бүтээлч боломж олгосон байна. Энэ нь аяндаа ойлгомтой боловч хүн төрөлхтний, мөн тусгай хүний бүтээлч зүтгэл нь тохиолдлын чанартай, тэр бүр хүссэн үр дүнд хүргэдэггүй, улмаар, ландшафтад хүний үзүүлэх нөлөөлөл сайн үр дүнтэй байгаагүй гэдгийг ямагт мартдаг. НТӨ III мянган жилд Шумерчүүд Тигр, Евфрат хоёр мөрний хөндийг хатааж, суваг татаж байсан, Хятадууд 4 мянган жилийн өмнө Шар мөрний тойронд далан босгож эхэлсэн. Дорнод Иранчууд шинэ эриний зааг дээр услалтанд зориулж хөрсний усыг ашиглаж сурсан. Полинезчууд арал дээрээ Америкаас амтат төмс (кумар) зөөн авчирсан. Европчууд тэндээс төмс, улаан лооль, тамхи, түүнчлэн тэмбүү үүсгэгч цайвар спирохетийг хүлээн авсан. Евроазийн тал газар арслан зааныг чулуун зэвсгийн анчид том өвсөн тэжээлтний хувьд устгасан. 13. Будыко М. И. О причинах вымирания некоторых животных в конце плейстоцена //Изв. АН СССР. Сер. географическая. 1967. № 2. С. 28- 36.
Эскимосчууд Берингийн тэнгис дэх стеллерийн үхрийг үгүй хийсэн, маорийчууд Шинэ Зеландын шувуунуудыг дуусгасан, араб болон персүүд байнгын ан хийх замдаа Өмнөд Азийн арслангуудыг устгасан, америкийн колонистууд ердөө л хагас зууны дотор (1830-1880) бизон болон зэрлэг тагтааг оргүй болгосон, харин автраличууд хэд хэдэн ууттан зүйлийг устгасан. 14. Дорст Ж. До того как умрет природа. М., 1968. С. 54- 55.
XIX–XX зуунд амьтдыг устгах явдал нь амьтан судлаач амьтны газар зүйчдийн бичиж байгаа шиг аль хэдийнээ гамшиг болон хувирсан. Иймээс энэ сэдвийг цааш нь үргэлжлүүлэх шаардлага бидэнд байхгүй юм. Гагцхүү хүн байгальтай зэрлэгээр харьцах явдал нь бүх формацид байж болдог, ингэхлээр үүнийг нийгмийн дэвшлийн онцлогийн үр дүн гэж үзэж болохгүй гэдгийг л тэмдэглэмээр байна. Бүх формацид хүн байгалийг эвдэлж байдаг. Энэ нь түүнд хэвшмэл зүйл бололтой. Хэдийгээр хүн үүнийгээ тухай бүрт нь янз бүрээр хийдэг авч ялгаа нь зөвхөн жижиг сажиг зүйл дээр л гарна уу гэхээс биш, үйл явцын чиглэл дээр огт гардаггүй юм.
Байгал ч гэсэн өөрийнхө өтөлөө зогсч чаддаг юм. Шилбээрээ хад чулууг эвдлэгч, уян хатан байдлаараа асфальтан замыг нэвтлэгч ургамлууд төдийгүй, амьтны зарим зүйл ч антропо хүрээг өөрийн цэцэглэлтэд ашигладаг юм. Тухайлбал, бизоныг устгаж, прерийн (америк-мексикийн хээр тал–Орч) биоценезэд тэдгээрийг хонь, адуугаар (мустанг) сольсон нь өвчтэй бизон, буга, мэрэгчдээр хооллодог байсан том саарал чононы тоог цөөрүүлжээ. Ийм учраас дунд нь тахал дэлгэрсэн бугын тоо толгой цөөрч, бизоны дараа үлдсэн тэжээлийг хоньтой хуваан идэх болсон нь мэрэгчдийн тоог ихэсгэв, энэ нь эргээд мэрэгчдээр хооллодог койот буюу хээрийн чоно үржиж олшрох таатай нөхцлийг бий болгов. Хээр талын байгал нөхөн сэргэсэн, гэхдээ биоценозийн бүтцээ алджээ.
Төмсний ургамлыг дэлгэрүүлсэн нь колорадогийн цох үржихэд түлхэц болж, тэр нь Кордильерээс Атлантикийг хүртэл ялалтын маршаар явж, түүнийг гатлан Европыг яруу алдартайгаар эзлэн авчээ. Английн худалдааны усан онгоцууд Полинезийн арлуудаас харх, түүнээс долоон дор нь шумуул авчирч, далайн салхи ямагт үлээж байдаг элсэн эргийн хүний өөрийнх нь амьдардаг нутгийг хязгаарлажээ. Мөн Автралид молтогчин, Мадейрад ямаа аваачих гэсэн туршлага ямар эмгэнэл авчирсан нь нэгэнт тодорхой. Гэхдээ байгалийн регенераци буюу нөхөн сэргээгдэх баримтууд нь хүн төрөлхтний нийгмийн түүхийн эргэлтийн он цагуудтай давхцдаггүй юм. Тэхээр энэхүү хоёр гинжин хэлхээний хооронд харагдах ч юмуу функциональ хамаарал бий юу? Харсаар байхад үгүй мэт, ландшафт руу хүний хийсэн “үсрэлт”-ийг ердийн утгаар ч (сайныг хийх эрмэлзэл), шинжлэх ухааны утгаар ч (доод хэлбэрээс дээд хэлбэрт хөгжих) “дэвшил” хэмээн нэрлэж болохгүй юм. Хэрэв ийм ахул байгалийн ганц нэг гажилтад янз бүрийн угсаатныг тодорхойлж байдаг хамгийн хөдлөнги тэрхүү тогтсон үзэл буруутай юм. Хэдийгээр бид дөнгөж тэмтэрч буй боловч өөрийнхөө асуудалд ойртох шиг болж байна.
НИЙГМИЙН ХҮН, УЛС ТӨРИЙН ХҮН БА УГСААТАН
Хүн бүр нийгмийн аль нэгэн бүлэгт ордог нь маргаангүй юм. Гэхдээ ямар ч байлаа гэсэн нийгмийн хөгжлийн тодорхой шатанд оршдоггүй, овог, орд, төр, нийтлэг, нөхөрлөл, түүнтэй төстэй нэгдлийн гишүүн биш хүн байдаггүйн адилаар ямар нэгэн угсаатанд хамаардаггүй хүн гэж байдаггүй. Нийгэм, улс төр, угсаатны хамт олны хоорондын харьцааг урт, жин, температурын хоорондын харьцаатай адилтгаж болно. Өөрөөр хэлбэл эдгээр үзэгдлүүд нь зэрэгцсэн, гэхдээ жишин хэмжих боломжгүй юм.
Түүхэн газар зүйн мэдлийн салбар хязгаарлагдмал. Жанжин, шинэчлэгч, дипломатуудын үйлдэлд газар зүйн шалтгааныг эрэн хайх нь үр дүнгүй. Гэхдээ л угсаатны хамт олнууд нь дэвшүүлсэн асуудалд тавигдаж буй шаардлагуудад бүрнээ нийцнэ. Хүн байгалийн харилцамж нь хөгжлийн эртний түвшинд төдийгүй, ХХ зууныг хүртэл тодорхой харагддаг.
Хөгжлийн дээр дурдсан гурван шугамын харьцааг өнгөрсөн үе нь бүрэн дүүрэн тодорхой, сурвалж судлалд тусгайлсан аялал, ном зүйн ширэнгийг гатлах шаардлагагүй Англи, Францын жишээн дээр үзүүлэх нь тун амархан юм. Нийгмийн талаасаа энэ хоёр орон хэд хэдэн формацийг туулсан: овгийн байгуулал–кельтээс римийн булаан эзлэлт хүртэл, боол эзэмшлийн үед хэдийгээр Британи нь Галлигаас гурван зуун жил хоцорсон ч Римийн эзэнт гүрний бүрэлдэхүүнд орсон, тэгээд феодализм, эцэст нь капитализм, чингэхдээ энэ удаад Францаас зуун жилээр хоцорсон үеийг давсан. Улс төрийн талаасаа ХХ зууны хүмүүст Ла–Маншаар хуваагдсан энэ хоёр үндэстэн ямагт ИЙМ байсан, өөрөөр байж чадахгүй, сонгодог угсаатан газар нутгийн бүхэллэг мэт санагдана.
Бидний сонирхон буй газар нутаг нь Францын өмнөд дэх субтропикийн, хойт Франц болон өмнөд Английн ойн, Шотланд болон Нотумберлендийн суббореал бутлаг тал гэсэн ландшафтын гурван бүсийг багтаана. Ландшафт тус бүр нь хүмүүсийг түүнд орох, түүний онцлогт дасан зохицоход хүргэдэг ба ийм маягаар тодорхой нийтлэг үүсдэг. Жишээлбэл, кельтүүд Роны доод хөндийд усан үзэм ургуулж эхэлсэн, тийшээ I–IY зуунд очсон римийн колонистууд, Y зууны дайчин бургундууд, YII зууны арабууд, XI зууны каталончууд мөн л усан үзэм ургуулсан ба хөдөлмөрийн нийтлэгээр тодорхойлогдож буй ахуйн нийтлэг нь хэл болон зан авирыг адилжуу болгов. XII зуунд одоо тарчихсан каталончууд, провансальчууд болон лигурийчуудаас нэгдмэл ард түмэн үүссэн юм. Энэхүү нэгдлийг эвдэхийн тулд Альбигойн хөнөөлт дайн шаардагдсан бөгөөд XIX зуун хүртэл өмнөд францчууд провансаль хэлээр ярьж, цөөн тохиолдлыг эс тооцвол франц хэлийг мэддэггүй байв.
Загасчдын хүүхдүүд, норвегийн викингүүд Нормандад очоод хоёр үеийн дотор газар тариаланч-франц болсон бөгөөд ердөө л антропологийн шинжээ хадгалан үлдсэн юм. Мөн тэр норвегууд Твидийн хөндийд хоньчин шотладчууд– лоулендерууд болсон, гэхдээ тэд кельт болон шотланд–гайлендерүүд овгийн байгууллаа хадгалан үлдсэн хойт Шотландын уулс руу нэвтрэн ороогүй юм. Улс төрийн биш, харин угсаатны хилийн хувьд тодорхойлогч хүчин зүйл нь гадаргууг оролцуулсан ландшафт байдаг юм.
Францын зүрх болсон түүний хойд хэсгийн хувьд дорно болон баруун өмнөөс ирсэн асар олон хүнийг адилсган хөгжүүлэх замаар хувирган өөрчилсөн. Эрт үед бельги, аквитан, кельтүүд, шинэ эрийн эхэнд латин, германчууд, Дундад зууны эхэнд франк, бургунд, алан, бриттууд, Дахин Сэргэлтийн үед англи, итали, испани, голландын дүрвэгсэд зөвхөн угсаатны судлалчид төдийгүй, Бальзак, Золя болон бусад реалист зохиолчдын гайхамшигтай бичиж байсан францын тариачдын нэг төрлийн олон түмэнд нэгдэн суурьшсан билээ.
Төстэй ландшафттай, нийгмийн нэг байгуулалтай, эрт үед бүдүүлэг завиар амархан гаталдаг байсан зөвхөн далайн хоолойгоор зааглагдсан газар нутгийн угсаатнууд аль алинд нь ашигтай байтал яагаад нэгдмэл бүрдэл болон нэгддэггүй юм бэ? гэсэн асуулт гарч ирнэ. Дундад зууны хаадууд үүнийг сайтар ойлгож, нэгтгэх оролдлогыг гурван ч удаа хийсэн юм. 1066 онд францын хааны харъяат Нормандын герцог Гийом Британий англосаксоны хэсгийг эзлэн авч, дараа нь нормандын улсыг таслан хувааж францын өөр феодал Генрих Плантагенетад шилжүүлсэн. Ингээд 1154 онд ахиад л Нормандыг Англитай нэгтгэж, араас нь Пауту, Аквитаний, Овернь нэгдэн орсноор Генрих Плантагенетийн хаант улс үүссэн юм. Угсаатан газар зүйн талаасаа гайхамшигтай энэ хослол францын хаан Филипп II Август английн хаанаас Норманди, Пауту, Турень, Анжуг авсан 1205 он хүртэл тэссэн юм. Дараа нь 1216 онд Английг ахиад эзлэхийг оролдсон боловч ялагдал хүлээсэн юм. Плантагенетүүдийг гасконы баронууд дэмжиж байсан бол Англид Бордо, Байонн хоёр л үлдсэн билээ. Гэвч 1339 онд хоёр орныг нэгтгэхийн төлөө Зуун жилийн дайн эхэлж, энэ удаад санаачлага английн талаас гарсан юм. Удаан дайны дараа 1415 онд Генрих Y Ланкастер францын титмээс титэмлэсэн ба гэхдээ Жанна д’ Арк Англиас хүчтэй болсон юм. Үүнээс хойш хоёр орныг нэгтгэх оролдлогыг дахин хийгээгүй билээ.
Физик газар зүйд дурдсан өөрчлөлтийн тайлбарлалыг хайх нь үр ашиггүй юм. Харин энд аль эртнээс бүх түүхчдийн хийсээр ирсэн эдийн засгийн газар зүйг татан оруулж болно. Улс төрийн бүрдэл, тухайн тохиолдолд төр нь нэгдмэл байдалд биш, харин эдийн засгийн янз бүрийн мужуудын янз бүрийн аж ахуйнууд бие биенээ нөхдөг байсан олон янз байдлын тогтвортой шинж, хөгжилд хэрэгтэй байсан юм. Плантагенетүүд хойд англиас хонины ноос, Кент болон Нормандаас талх, Овернээс дарс, Туркээс бөс бараа ирж байхад бат тогтож байлаа. Эдийн засгийн холбоо нь оргилсон нийтлэгт хүргэж, эрх баригчдыг баяжуулсан, гэхдээ угсаатны уусан нийлэлт болоогүй юм. Яагаад? Үүнд хариулахын тулд гурав дахь буюу угсаатны асуудлыг авч үзье.
АРД ТҮМНҮҮДЭД ЭХ ОРОН БИЙ !
Римийн эрх мэдэл унав. Францад оршин суусан овог аймгууд Рейн болон Бискайн булангийн хооронд газар нутаг дээр бий болох үедээ хэл, зан авир, уламжлалаараа асар их ялгаатай байсан юм. Огюстен Тьерри орчин үеийн Францыг нэгтгэх овгийн үзэл баримтлалыг санал болгосон нь зөв байлаа. Тэрээр: “Y–XYII зуун хүртэл Францын түүх нь үнэхээр нэг гарал үүсэлтэй, нэг зан авиртай, нэг хэлтэй, нэг иргэний болон улс төрийн ашиг сонирхолтой нэг л ард түмний түүх байсан уу? Огт тийм биш. “Франц” нэрийг нөхөн хэрэглэхдээ би Рейны чанад дахь овгуудыг ч биш, тэр байтугай улсын анхны үеийг ч яриагүй байна, ингэвэл жинхэнэ завхарсан зүйл үүснэ” гэж бичээд өөрийнхөө санааг “Бретончуудын хувьд үндэсний түүх нь өвөг дээдэс нь франкуудтай бие даасан ард түмэн мэтээр гэрээ хийж байсан Хлодвиг буюу Агуу Их Карлын удмынхны намтар болж чадна гэж үү? YI–X зуун хүртэл, түүнээс хойш ч Хойд Францын баатрууд Өмнөдийн хувьд ташуур болж байсан” жишээгээр тайлбарласан байна. 15. Тьерри О. Избр. соч. М., 1937. С. 207-208.
Дөнгөж XIY зуунд л францчууд Дофин, Бургунд, Провансыг нэгтгэж, Ариун Римийн эзэнт гүрний хамсаатан байсан германуудыг Францын хаант улсад нийлүүлэв. Гэхдээ Бордо, Байонн болон Бискайн булангийн эргийн зурвас газар тусгаар байдлаа хадгалан, Плантагенетүүдийн улсын английн хааны хараат улс болов. Энэ нь Англи Гасконыг ноёрхсон хэрэг биш, харин гаскончууд францын булаан эзлэлтээс өөрийгөө хамгаалах арга байсан билээ.
1339 онд Франц, Английн хооронд дэгдсэн Зуун жилийн дайнд хүчний ихээхэн тэнцвэргүй байдалтай байсан хэдий ч (1327–1418 онд Францад 18 сая , Англид болон Шотландын ар талд 3 сая хүн үхсэн ) гаскон, бретон, Наваррын хаант улсын идэвхитэй дэмжлэгээр л Английн талд амжилттайгаар дууссан юм. 16. Урланис Б. Ц. Рост населения в Европе. М., 1941. С. 40. 17. Там же. С. 57.
Сайн санаат Ионн нас барсны дараа түүний хүү Карл хаан болж, нөгөө хүү Филипп нь бургундийн герцог болсон юм. Ах дүү нар учраа ололцох ёстой мэт санагдавч тэд өөрсдөөсөө гэхээсээ баронуудаасаа илүүтэй хамааралтай байжээ. Бургундын Валуа-гийн улс Францын дорнод мужуудын тэргүүнд гарч, Артуа, Фландрий, Франшконте зэргийг өөртөө нэгтгээд парижчуудын таашаалыг ашиглан Францад ноёрхохын төлөө өрсөлдөх болов. Бургундчуудын эсрэг гүн Арманьякийн удирдсан францын баруун болон өмнөдийн оршин суугчид тэмцэв. Тэдний хоорондын дайн нь англичуудад зам нээж, тэд өмнөд болон Бретонийн уугуул “арманьяк” нарыг “францын хаант улсад хамаарагддаггүй”, өөрөөр хэлбэл франц биш гэж үзэж байсан бургунд болон парижийнхантай холбоо тогтоожээ. Францыг немц аялгатай францаар ярьдаг байсан Жанна д’ Арк аварсан юм. Тусгаарлагдсан Бургундыг швейцарчууд бут ниргэж, ахиад л францчууд Бретань болон бусад хязгааруудад ялалт хийхэд хүрчээ. Түүний удаан эсэргүүцлийн шалтгааныг сүүлчийн герцог–Зоригт Карл “Бид өөр португалчууд” гэж тайлбарласан байдаг. Энэ нь бургунд болон францчуудын хоорондын ялгааг португал болон испаничуудын хоорондын ялгаатай адилтгасан хэрэг байлаа. Тэр өөрөө Валуа гэсэн овогтой, франц гаралтай байсан нь түүнд саад болоогүй юм. 18. Архив К. Маркса и Ф. Энгельса. Т. IV. М., 1939. С. 320. 19. Grousset R. Вilап de l’histoire. P. 33.
Ямар ч гэсэн угсаатны олон янз байдал нь өөрийн хаан IY Генрихэд сайд Сюлли-гийн бичиж өгсөн “Агуу их бодол санаа”–нд томъёолсон “жам ёсны хил”-ийн онолд байраа тавьж өгчээ. Францын “жам ёсны хил”-ээр Пиреней, Альп, болон Рейн, өөрөөр хэлбэл хаан, сайд хоёрын Францын өмнө үеийн улс гэж зориудаар зарласан эртний кельтийн газар нутгийг зарлажээ. Шинжлэх ухааны хувьд нэн хэврэг энэхүү үндсэн дээр Бурбонууд Францад хуучин алдар сууг нь буцааж өгөх, өөрөөр хэлбэл IY Генрихийн “Би испаниар ярьдаг, испанийн хаан захирдаг тэр газрууд, немцээр ярьдаг, австрийн хаан захирдаг тэр газруудын эсрэг юу ч байхгүй, харин францаар ярьдаг тэр газруудыг би захирах ёстой” гэсэн мэдэгдлийг үл харгалзан баск, итали, немцүүдийн оршин суух газруудыг хүчээр нэгтгэхийг эрмэлзэх болов. Франц нь энэ зарчмаа үл харгалзан Наварра, Савойя болон Эльзасыг түрэмгийлсэн бөгөөд ингээд газар зүй нь хэл судлалыг ялжээ. 20. История дипломатии: В 5 т. /Под ред. В. А. Зорина, А. А. Громыко. Т. 1. М., 1959. С. 269- 271.
Францын феодалуудын зарим нь Улаан болон Цагаан сарнайн дайнуудад үхэж үрэгдэн, зарим нь англосаксоны язгууртнуудтай нийлж байсан Англид ч мөн ийм үйл явц болсон бөгөөд дараа нь хаант улс XYIII зуунд жам ёсны хил болох арлынхаа эрэг хүртэл тэлжээ. Англи нь англосаксуудын нутагладаг газар тариалангийн Кельт, мал аж ахуйн Шотландаас бүрдэж, викингийн удмын кельт, скандинавчуудын нутагладаг Уэльс болон Нортумберлендыг өөрийн хэлээр ярьдаг, өөрийн ахуйтай, өөрийн аж ахуйн системтэй ард түмнүүд оршин байсан Прованс, Бретань болон Гасконуудыг Францчууд нэгтгэсэнтэй адилаар өөртөө оруулсан юм.
Энд авч үзсэн үйл явцыг “угсаатны ойртон нягтрал” гэж нэрлэж болох уу? Магадгүй л юм, гэхдээ хоёр тохиолдолд бүхий л боломжоороо харгислал үйлдсэн, шууд булаан эзлэлт байсан ба харин эзлэгдсэн угсаатнууд нь манай үеийг хүртэл хадгалагдан үлдсэн юм. Гэхдээ орчин үеийн Англи, Франц хоёр нь физик газар зүйн бүс нутгууд мөн үү? Энэ нь гарцаагүй, өөрөөр байсан бол тэд оршин байсан угсаатны эрээн мяраанаас болж задрах байсан билээ. Энэ нь газар зүйн болон угсаатны категориуд давхцдаггүй, улмаар ландшафт болон угсаатны холбоо нь геобиоценозод дасан зохицсон, ландшафтаа эзэмшсэн угсаатны түүхээр зангидагддаг гэсэн хэрэг юм.
Энэ үзэгдлийг сукцесси буюу солигдолт, манай тохиолдолд антропогенийн солигдолт гэж нэрлэдэг. Шинэ нөхцөлд дасан зохицох нь угсаатны нийлэгжилтийн газар зүйн тал бөгөөд харилцан нийлэх, ялгаа арилах үзэгдэл болдоггүй, харин ялсан нь хэт угсаатны байр сууринд очдог угсаатны системийн бүхэллэг үүсдэг. Гэхдээ булаан эзлэгч угсаатантай хөршлөн байсан он жилүүд талаар өнгөрөөгүй юм. Бретанийн кельтүүд францчуудтай, Уэльсийн кельтүүд англичуудтай нөхөрлөсөн билээ. Гэхдээ эдгээр ард түмний өнөөдрийн найрамдал нь саяхны дайсагналыг халсан, цаашаа юу болохыг угсаатны түүх харуулах бөгөөд энэ асуудалд газар зүйн шинжлэх ухаан буухиаг дамжуулна гэдгийг этнологи хүн санаж явбал зохино.
О.Тьеррийн түүхийн тасралтат чанарын үзэл баримтлалаас ялгаатай нь Фюстель де Куланж францын тариачдын ахуйд римийн эрин үеийн институтуудын шинж байдгийг авч үзсэн юм. Эхнийх нь миграцийн шинж чанарыг, хоёр дахь нь ландшафтын нөлөөллийг тэмдэглэсэн. Гэхдээ миграцийн шинж чанарыг бүхэлд нь, түүнчлэн дасан зохицохуйн хэмжээг газар зүйн шинжлэх ухаан, этнологи гэж нэрлэгдэх түүний бүлэгт хамаарах үзэгдэл мэт авч үзэж болох бөгөөд ёстой юм. Учир нь чухам энд хүн газар зүйн орчинтой тогтоосон холбоо төвлөрдөг бөгөөд үүний ачаар тэд бие биедээ нөлөөлж байдаг.
Ийнхүү зөвхөн хувь хүнд төдийгүй, мөн угсаатнуудад эх орон байдаг. Угсаатны эх орон нь тэрээр анхлан шинэ системээ бүтээсэн ландшафтын хослол болно. Энэ үзлийн үүднээс хусан төгөл, ойн цоорхой, Волга–Ока хоёр мөрний дөлгөөн голууд нь угро–славян, татаар–славяны эрлийзжилт, Византиас авчирсан сүм хийдийн уран барилга, эрт урьдын домог туульс, шидэт чоно, үнэгний тухай үлгэрүүдтэй яг адил XIII –XIY зууны үед төлөвшсөн их оросын угсаатны элементүүд юм. Орос хүнийг хувь заяа нь хаашаа ч хаялаа гэсэн тэрээр өөрт нь “өөрийн байр”– Эх Орон байдгийг мэддэг юмаа.
Англичуудын тухай Р. Киплинг “Хуучин Англи бол бидний байшин гэж эхчүүд маань сургасан” гэж бичсэн байдаг. Араб, төвд, ирокезүүд бүгдээрээ л ландшафтын элементүүдийн давтагдашгүй хослолоор тодорхойлогдох өөрийн гэсэн анхдагч газар нутагтай байдаг. “Эх орон“ гэдэг нь энэ байдлаараа “угсаатан” гэж нэрлэгдсэн системийн бүрдэл хэсгүүдийн нэг болно.
ХӨГЖИХ ГАЗАР
Бидний авсан жишээнүүд Homo sapiens зүйлийн хамт олон болох угсаатны хамтын нийгэмлэгт газар зүйн ландшафт нөлөөлдөг тухай дүгнэлт хийхэд хангалттай юм. Гэхдээ энэ дүгнэлтийг аль хэдийн 1922 онд Л.С.Берг бүх организм, түүний дотор хүний хувьд хийснийг яаран хэлмээр байна. 21. Берг Л. С. Номогенез. С. 180-181.
“Газар зүйн ландшафт нь организмд албадан нөлөөлж, бүх амьтдыг зүйлийнх нь зохион байгуулалт олгох хэрээр тодорхой чиглэлд өөрчлөгдөхийг тулгадаг. Тундр, ой, тал, цөл, уул, усан орчин, арлууд дахь амьдрал гэх мэт энэ бүхэн нь организмд онцгой ул мөрөө үлдээдэг. Дасан зохицох чадваргүй тэдгээр зүйлүүд нь газар зүйн өөр ландшафт руу нүүн шилжих юмуу үхэх ёстой” Харин “ландшафт” гэдэгт “жам ёсны хилээр хүрээлэгдсэн бусад хэсгүүдээс ялгаатай, юмс үзэгдлийн бүхэллэг бөгөөд харилцан нөхцөлдсөн зүй тогтолт нийлбэрийг төлөөлсөн, ихээхэн хэмжээний орон зайгаар голчлон илэрхийлэгддэг, бүх талаараа ландшафтынхаа бүрхэвчтэй салшгүй холбоотой газрын гадаргуугийн хэсгийг хэлнэ” Энэхүү ойлголтыг “төрсөн газар” хэмээх адилтгах ойлголтод дөхүүлж П.Н.Савицкийн яг онож нэрлэсэн “хөгжих газар” гэсэн нэр томъёогоор нэрлэе. 22. Калесник С. В. Основы общего землеведения. С. 455. 23. Савицкий П. Н. Географические особенности России (I). Прага, 1927. С. 30-31.
Зохиогч хүмүүсийг амьтантай харьцуулж эхлээд минерал хүртэл явсанд уншигчид гайхаж, тэр ч байтугай гомдоллож байж магадгүй. Гэхдээ гомдох хэрэггүй юм. Байгалийн ямар ч зүй тогтолд бидний хүн бүр ямар нэг талаараа шүргэлцэж байдаг ба хүний хувь шинж олон талтай учраас гоо зүй, ёс зүй болон одоо үед “мэдээлэл” буюу “ноо хүрээ” гэж нэрлэж буй тэр бүх зүйлд байр үлдэнэ биз ээ. Одоохондоо бид газар дэлхийн хэрэг явдалдаа эргэн оръё, учир нь ландшафтын тухай яриа дуусаагүй байна.
XV. Ландшафтуудын хослолын үүрэг
ЛАНДШАФТУУДЫН НЭГ ХЭВИЙН БОЛОН ОЛОН ЯНЗИЙН ШИНЖ
Бүхий л газар нутаг хөгжих газар болж чаддаггүй юм. Тухайлбал, Онеги нуурын тайгаас Агнуурын тэнгис хүртэл Евроазийн орон зайн битүү ойн бүх бүсэд нэг ч ард түмэн, нэг ч соёл үүсч байгаагүй юм. Тэнд байгаа буюу байсан бүх зүйлийг өмнө юмуу умраас зөөн авчирсан байдаг. Цэвэр, үргэлжилсэн тал нь мөн л хөгжлийн боломж олгодоггүй. Дешт-и-Кып-Чак, өөрөөр хэлбэл, Алтайгаас Карпат хүртэлх половчуудын тал нь Genius loci байхгүй газар юм. Энэ талуудад бусад газрууд, жишээлбэл, зүсэгдсэн гадаргуу, олон янзийн ландшафтад бүрэлдсэн ард түмнүүд нүүн суурьшиж байв. Хэнтий, Хангайн нуруудад өтгөн ой ургадаг. Туул, Хэрлэнгийн доод хэсгийн ногоон тал нь өмнө зүгтээ Говийн чулуулаг цөл болон шилждэг, энд гурван сар гэхэд цас хайлж, зуны халуун эхлэх хүртэл мал маллах боломж олгодог. Энд харгалзан амьтны аймаг ч олон янз байдаг. Археологийн соёл нь ард түмнүүдийн халагдах солигдохыг тусгаж байдаг ба энэ нь зөвхөн хүн, түрэг, уйгар, монгол, ойрдуудыг судалдаг түүхчдэд л илэрхий байдаггүй.
Үүний урвуугаар Эрчис мөрний дээд хэсгээс Доны доод хэсэг хүртэл, сибирийн тайгийн захаас Балхаш, Арал нуур хүртэлх Их Талын баруун хэсэг нэг төрлийн байдаг, энд суурьшигч ард түмнүүд нь бараг тодорхойгүй. Өнөөдөр казакууд нэг хэвийн талын ландшафт бүхий асар их талбайг эзлэн сууж байна. XIII зуунд монгол–хивчаагийн харгис дайны дараа тал нутаг хүнгүйдэж, гурван Ордын дунд хуваагдсан юм. Алтан буюу Их Орд нь–Ижил мөрөнд, Хөх Орд нь-Арал далай Тюменийн хооронд, Цагаан Орд нь (өөрөөр хэлбэл ахлах) Тарвагатайн нуруу болон Эрчис мөрний дээд хэсэгт байжээ. 24. Грумм-Гржимайло Г. Е. Западная Монголия и Урянхайский край: В 3 т. Т. 2. 1926. С. 502.
Ижил мөрөн дэх ард түмнүүдийн хольцоос татаарууд бүрэлджээ. Хөх орд нь амьдрах чадваргүй болж, XIY зуунд Волжтой нийлсжээ. Харин Обь хүртэлх сибирийн тайгийн хязгаар, Алтайн бэл, сархиаг, тэр үед нарсан төгөл эрээлжлэн байсан Сырдарья–гийн тал зэрэгт тулгуурлаж байсан Цагаан орд бие даасан угсаатан болон хөгжиж, хожим нь Арал хавь, Мангышлак, Рынпескийн талыг эзэмшсэн юм. 25. Грибанов Л. И. Изменение южной границы ареала сосны в Казахстане // Вести, сельскохозяйственной науки (Алма-Ата). 1965. № 6. С. 78-86.
Жинхэнэ хөгжих газар бол хоёр буюу түүнээс илүү ландшафтыг хослуулсан газар нутаг байдаг. Энэ үндэслэл нь зөвхөн Евроазид төдийгүй, бүх дэлхийн бөмбөрцөгт адилхан байдаг. Евроази дахь угсаатны нийлэгжилтийн үндсэн үйл явцууд нь: а) уулын болон талын ландшафтыг хослуулсан дорнод хэсэг, b) ой болон нугыг хослуулсан (Ижил болон Ока голын дундах өвст тал) баруун хэсэг, c) тал болон баян бүрдийг (Крым, Дундад Ази) хослуулсан өмнөд хэсэг, d) ойт тундр болон тундрыг хослуулсан хойт хэсэгт үүсч байсан юм. Гэхдээ хойт хэсгийг би евроазийн хөгжих газраас тусгаарлагдсан “тайгийн тэнгис”, түүнд хэзээ ч нөлөө үзүүлж байгаагүй туйл тойрсон соёлын онцгой салбар болгон ялган үзэхийг санал болгож байна.
Шалгацгаая. Хүнчүүд Иныла–ня уулын ойт бэлд бүрэлдсэн, дараа нь л харин монголын тал нутаг руу шилжин хөдөлсөн. Уйгарууд Наньшан уулын бэлд, Түрэгүүд Алтайн бэлд, Монголчууд Хянган, Хэнтийн бэл ойт Манжуурт нэвтэрсэн талын “хил” дээр, Енисейн киргизүүд Саяны бэл болон Минуссины тал газрын “арал” дээр, Казанийн татаарууд, эртний болгарын удмынхан ой тал хиллэсэн Кам гол дээр, Крымийн татаарууд талархаг Крым, цул баян бүрдийн өмнөд эргийн хил дээр үүссэн. Энэ бол нэгдмэл ард түмэн болон нэгдсэн янз бүрийн гарал үүсэлтэй таслагдсан ливантийчүүд юм. Хазарууд Дагестаны уулын дэвсэгт үүссэн ба тэдний анхны нийслэл нь Терекийн дунд урсгалын хавьд байрлах Семендер байсан юм.
Дурдсан зарчмыг өргөжүүлж, ландшафтын бүсүүдийн хоорондын зааг бүдгэрсэн тэр газарт газар зүйн нэг нөхцлөөс нөгөөд аажмаар шилжих шилжилт ажиглагдах ба угсаатны нийлэгжилтийн үйл явц нь бага эрчимтэй байна гэж үзэж болох юм. Жишээлбэл, дундад азийн голуудын хоорондын баялаг баян бүрдүүдийн бүлэг нь хагас цөл болон талаар хүрээлэгдэж, чингээд баян бүдүүдийг өөр хооронд нь тусгаарладаг. Үнэхээр ч Дундад Азийн угсаатны нийлэгжилт бараг мэдэгдэмгүй тийм удаан явагдаж байсан. Хойт болон баруун өмнөдийн цөлийн бүс нь зэвсэглэсэн дээрэмчдийг амархан нэвтрүүлдэг байсан атлаа амьдралд тохиромжоор тааруу байсан. Гэхдээ Копетдаг, Тянь–Шан, Гисарын уулын бэлээр XI зуунд туркмен–сельжукууд, XY зуунд киркизүүд, YIII–IX зуунд тажикууд, XIY зуунд узбекүүд бүрэлдсэн бөгөөд эртний согдуудын удмынхны газар нутгийг Памир, Гиссарийн уулын нутгуудаар хязгаарлаж, тэднийг тусгаарлагдмал байдалд үлдээсэн юм. 26. Рычков Ю. Г. Антропология и генетика изолированных популяций (Древние изоляты Памира).
Уулын нуруудын систем нь босоо бүслүүртэй ч гэсэн бүслүүрүүд нь хүнтэй харьцах талаасаа нэгдмэл газар зүйн аж ахуйн бүрдлийг үүсгэдэг учраас түүнийг нэг маягийн нутгууд гэж үзэх хэрэгтэй юм. Ийм учраас Баруун Памир, Дартистан, Гидукуш, Гималай, түүнчлэн Кавказ, Пиренейн нурууд нь үлдэгдэл угсаатан–персистентүүдийг хадгалан үлдэхэд тохиромжтой байдаг. Уулын ландшафтыг нэвтрэн гарахад хүндрэлдэйд хэргийг учир байгаа юм биш. Цэргийн ангиуд бүр Кир, Алекмандр Македонскийн үед ч хавцал даваануудыг хялбархан давдаг байсан. Гэхдээ шинэ ард түмнүүд нь уулын нутгуудын дотоодод биш, харин тэдгээрийн хязгаараар үүсдэг байжээ.
Битүү тал, тэр ч байтугай маш баян талуудад амьдардаг ард түмнүүдэд хөгжлийн маш бага боломж илэрдгийг нэгэнт тэмдэглэсэн, жишээлбэл, XI зуунд сельжук нэрийн дор Бага Ази, Азербайжанд очсон тэр хэсгээс бусад саки, печенег, кыпчак, туркмен зэрэг нь угсаатны талаасаа ч, нийгмийн талаасаа ч тогтвортой байна.
Тэнгис, уул, цөл, голын хөндийн хослол болсон Левант буюу Ойрхи Дорнод байна. Энд байгалийн нөхцөл нь зожиг угсаатнуудад тохирох Кавказийн чанадын уулархаг хэсгийг тооцохгүй бол шинэ угсаатны янз бүрийн хослолууд ямагт үүсч байдаг. Ийм зожиг угсаатан нь угсаатныхаа өөрийн ахуйг перс, грек, рим, араб, тэр ч байтугай турк–османуудаас хамгаалж чадсан курдууд байна. Энэ нь бидний зөв болохыг баталж буй тохиолдол болно.
Хятад нь эрт цагаас усанд эзлэгдсэн (эрт үед энэ нь жил бүр гольдрилоо сольдог жижиг нуур, голууд бүхий битүү намаг байсан юм) орон болно. Хятадын ард түмэн гол, уул, ой, талын ландшафтуудын хослол бүхий Шар мөрний эрэгт төвхнөсөн ба харин Хөх мөрнөөс өмнүүрх ширэнгэ ойг зөвхөн нийтийн тооллын нэгдэх мянган жилд эзэмшсэн юм. Гэхдээ өмнөдөд суурьшиж, орон нутгийн хүн амтай холилдсон эртний хятадууд өөрсдийн өвөг дээдэс Шар мөрний хөндийд хүн, сүмбэ нартай холилдсон хойт хятадуудаасаа ялгаатай орчин үеийн хятадын угсаатан болон хувирсан билээ.
Далай болон уулсаар хүрээлэгдсэн Энэтхэгийг хагас тив гэж үзэж болох бөгөөд Европоос ялгаатай нь ландшафтын хувьд ядмаг юм. Деканы ландшафтууд нь хэв маягаараа өөр хоорондоо ойролцоо юм, харин угсаатны нийлэгжилтийн үйл явц, өөрөөр хэлбэл түүхэн хугацаанд шинэ угсаатан бий болох явдал энд сул илэрдэг. Гэхдээ Энэтхэгийн баруун хойд талд YIII зууны хавьцаа ражпут , XYI–XYII зуунд сикхи гэсэн хоёр том ард түмэн бүрэлдсэн юм. 27. Синха Н.К.. Банерджи А.Ч. История Индии. М., 1954. С. 113-II 4.
Ражистан болон Синдийн цөл нь баялаг, ойгоор бүрхэгдсэн Ганга мөрний хөндийгөөс хүний хувьд хамаагүй таагүй мэт мэт санагдана. Гэхдээ хэдийгээр ургамал нь дотоод Энэтхэгт ургадаг ч Индийн хөндийд цөл болон халуун бүсийн ургамлын хослол тод илэрдэг, энд шинэ ард түмний бүрэлдэх нь хязгаарын мужуудтай холбоотой байдаг.
Түүнчлэн хойд Энэтхэгийн ширэнгэ ой Декан–Махараштагийн өвслөг тал газартай нийлдэг доод Нарбадын сав газарт ард түмэн үүсэх үйл явц идэвхитэй явагдсан юм. Энд YI зуунд Ражпутанаас нүүн ирсэн байх ёстой дайчин кшатрийчуудын улс Чалукья, XYII зуунд Энэтхэг дэх Агуу Их Моголын ноёрхлыг сөрсөн, кастын системийн зарим шахалтаас татгалзсан маратхи нар идэвхижиж байсан. Энэтхэгийн бүх түүхчид маратхууд индусийн ерөнхий олон түмнээс (масс) ялгаатайг тэмдэглэдэг. 28. Там же. С. 106-107.
Маратхуудын газар нутаг нь Баруун Гхат болон тэнгисийн хоорондох эрэг орчмын зурвас, Гхатаас зүүн тийшх уулархаг орон, толгодын цувуугаар хязгаарлагдсан хар хөрст хөндий гэсэн физик газар зүйн гурван бүсийн хослол болно. Ийм маягаар Бенгалийн соёл нь харьцуулшгүй өндөр хэдий ч энэ мужийг бидний хөгжих газар гэж нэрлэдэг тэр ангилалд оруулан тооцох бүх үндэслэл бий боллоо. 29. Там же. С. 256.
Хойт америк дахь зах хязгааргүй ой, хээр тал нь угсаатны нийлэгжилтэд таатай нөхцөл бүрдүүлж чадаагүй юм. Гэхдээ тэнд индиан овгууд түүхчдийн нүдэн дээр ард түмэн болон бүрэлдсэн нутгууд байдаг. XY зуунд Их Нууруудын эргийн хэрчигдсэн шугамууд дээр таван овгоос бүрдсэн ирокезийн холбоо үүссэн. Энэ бол өмнөхтэйгээ үл давхцах угсаатны шинэ бүрдэл байсан юм, учир нь түүний бүрэлдэхүүнд цус, хэлээрээ төрөл гуронууд ороогүй билээ.
Хадат арлууд нь далайн гахай, далайн хавны хэвтэш болсон, далай нь эргийн оршин суугчдыг тэжээж байдаг Аляскаас өмнүүрх Номхон Далайн эргээр хөрш анчин овгуудаасаа хэл, заншлаараа эрс ялгаатай тэлэнгидууд боол эзэмшлийн нийгэм бүтээсэн юм.
Кордильлерт ихэнхи хэсэгтээ гэнэт хээр тал болон тасалддаг, уулын ландшафтууд нь зэрэгцэн оршдог атлаа талтай хосолдоггүй юм. Гэхдээ эдгээр ландшафтуудын хооронд аажим шилжилт бүхий Нью-Мехико штатаас өмнө зүгт эрт үед “пуэбло”–ийн соёл үүссэн, харин энд XII зууны үед алдарт ацтек овогт хамаарагдах нагуа бүлэг бүрэлдсэн байдаг. Тивийн ихэнхи, түүнчлэн индианчуудын нутагшсан хэсэг нь хөгжих газартаа бүрэлдсэн ард түмэн нүүн явах юмуу тархсан өвөрмөц маягийн “Hinterland” нутаг байлаа. Жишээлбэл, алконкиний бүлгийн хар хөлтнүүд болон бусад олон овгууд ийм байсан.
Өмнөд Америкийн жишээн дээр энэхүү зүй тогтол бүр ч тодорхой харагдана. Уул болон талын хослол болсон Андын нуруу нь янз бүрийн эрин үед олон ард түмний бүтээсэн соёлын дурсгалуудыг хадгалан байдаг, харин Бразилийн ой, Аргентиний тэгш газрууд нь ахмад Фоссетийн найдсаны эсрэгээр ямар ч соёл бүрдүүлээгүй юм. Ингээд олон тооны жишээн дээр бид харж байгаагаар эдгээр орнуудын байгал нь нэг янзийн бөгөөд өөр газарт бүрэлдсэн ард түмнүүд энэ баялгаас ашиглахад нь хэзээ ч саад болж байгаагүй ба саад болдоггүй гэдгийг нэгтгэж чадахгүй л байна. Патагонид уулын арауканууд, нэвтэрч, XYI зуунд бразилийн ойг инкүүд эзэмшихээр оролдож, харин тэнд XIX зуунд португалийн тариалангийн эзэд үлгэрийн мэт баяжсан билээ.
Мөн ийм зүй тогтлыг бид Африк болон Австралид ч илрүүлж болно, гэхдээ салбар онцлогийг нь гаргахын тулд далайтай холбогдсон угсаатнуудад анхаарлаа төвлөрүүлэх нь зүйтэй юм.
ДАЛАЙ БА МӨСӨН ГОЛЫН ЗАХЫН ЭРГЭЭР
Эргийн шугамынхаа шинж чанараас хамааран далайн үүрэг нь эргийн оршин суугчдын соёл иргэшлийн түвшинд хоёр байдлаар нөлөөлж болдог. Далай нь эзэмшигдээгүй, нэвтэршгүйн зэрэгцээ ландшафтуудыг хязгаарлагч элемент болдог. Америкийн индианчуудын хувьд Атлантын далай, негр болон Австралийн уугуулуудын хувьд Энэтхэгийн далай, тэр ч байтугай печенегүүдийн хувьд Каспийн тэнгис ийм л байсан. Гэхдээ далайгаас хүнс шавхаж, усан замыг эзэмшиж эхэлмэгц далай тэнгис нь хөгжих газрын бүрдүүлэгч элемент болон хувирсан. Ингэж эллинчүүд Эгийн тэнгисийг, викингүүд Хойт мөсөн далайг, арабууд Улаан тэнгисийг, поморын оросууд Цагаан тэнгисийг ашиглах болсон. XIX зуун гэхэд бүх далай, тэнгис Ойкумены бүрэлдэхүүнд орсон бөгөөд гэхдээ энэ нь бүх эрин үед хэвшмэл байгаагүйг харгалзах хэрэгтэй юм. Түүхэн бүхий л үеийн туршид далай нь хөгжих газрын бүрэлдэхүүн хэсэг болсон хоёрхон газар нутаг, тухайлбал Хойт мөсөн далай болон Полинезийн эрэг дагуух туйл тойрсон соёлуудын нутаг тэмдэглэгдсэн байдаг. Энэ талаар олон удаа ярьсан болохоор давтах шаардлагагүй юм. Полинезийн соёл нь европчуудыг ирэхээс өмнө Пасхи арал шиг хуурай газрын тийм тусгаарлагдмал хэсэгт өөр хоорондоо тэмцэлдэж, шинж чанараараа ойрхон хэдий ч өөрийн соёлыг бүтээж байсан янз бүрийн бүрдлүүдийг багтааж байлаа.
Туйл тойрсон ард түмнүүдийн түүх маш бүрхэг. Их дээр үед төстэй соёлуудын хэлхээ өөрсдийг нь тэжээж байсан Хойт мөсөн далайг хүрээлэн байжээ. Эд нар нь үндсэндээ далайн амьтан, загасны төрлийн ангуучид байжээ. Нэгэн түүхэн цаг үед тэдний газар нутгийг угро-өөрсдөөсөө идэгсэд хоёр хэсэгт хэрчин баруун хэсгийгээ бут цохисон байна. Дараа нь тунгусууд Ява болон Индигирке дэх “омок” хэмээх ард түмэн болон эртний азийнхнаас бусад дорнод хэсгийнхнийг устгажээ. “Омок” нь якутуудын довтлох үед устгагдсан аж. Өмнөөс хойт зүгт хийсэн якутуудын хөдөлгөөн нэг талын бөгөөд эргэлтгүй байсан, тэд гол мөрнөөр салаар хөвснөөр урсгал сөрөн эргэж ирж чадахгүй байжээ. 30. Окладников А.П. История Якутской АССР: В 3 т. Т. 1. М.. Л., 1955.
Туйл тойрсон залуу ард түмэн бол НТ I зууны орчим далайгаас тархаж, X зуунд индианчуудыг Канадын өмнөд хил хүртэл хөөсөн, XIII зуунд викингийн удмынхныг Гренландын далайд хаясан эскимосчууд байлаа. Энд ахиад л тэжээгч далай, ойт тундр, мөсөн арлууд гэсэн ландшафтын хослолууд гарч байна. 31. Руденко С. И. Древняя культура Берингова моря и эскимосская проблема. М., Л., 1946. С. 113.
Гэхдээ тэжээж буй тэнгис төдийгүй, мөсөөр бүрхэгдсэн, иймээсээ ч огтын үржил шимгүй мужууд хүртэл НТӨ X мянган жилд Балти орчим болон Скандинавд болсон шиг угсаатан үүсэхэд тус болж болно. Үүний механизм нь тун энгийн.
Мөстлөг өсөхийн тулд далайгаас хангалттай хэмжээний чийг–хүйтэн бороо, нойтон цас авах ёстой. Ингээд мөстлөг дээр ямагт антициклон үүсч, чийглэг агаар түүнийг зах хязгаар руу таслан явуулдаг, Энд бороо цутгаж байдаг. Евроазийн хувьд гэвэл энэ нь Таймыр хүртэл атлантын циклон ирж байдаг баруун хязгаар нь юм. Улмаар мөстлөг нь уулын баруун зүгтээ өсдөг бөгөөд харин зүүн тал нь нарны туяанд хайлж байдаг. Учир нь тэнд үүлэн бүрхэвч байхгүй, инсоляци буюу нарны туяа саадгүй шарж байдаг.
Энд газар зүйн парадокс гарч байна. Абсолют температур нь өндөр, хуурай, салхитай, үүлэрхэг тэр газарт хүн амьтад хүйтэнд нэрвэгддэг, харин температур нь бага чимээгүй, цэлмэг, хуурай тэр газарт хүн, амьтад хүйтэн агаарыг тоолгүйгээр нарын туяанд шууд шардаг. Мөстлөгийн антициклон нь ямагт мөснөөсөө их байдаг учраас мөстлөгийн нутгуудыг бүрхэж, тэдгээрийг хуурай тундр болгодог. Мөстлөгөөс гарсан горхи нь загас сүлжилдэж, уснаа хөвөгч шувууд бүхий цэнгэг нуур голуудыг үүсгэдэг. Тэдгээрийг тойрон үслэг амьтдын орон болсон ойн төглүүд ургаж, цасан бүрхэвч нь бага хуурай тундрт өвсөн тэжээлтний сүрэг бэлчээрлэдэг. Энэ бол хүй нэгдлийн анчид, загасчдын диваажин билээ.
Чухам ийм нөхцөл сүүлчийн мөстлөгийн Помераны үеийн төгсгөлд Дорнод Европод бүрэлджээ. Хөдлөн явж буй мөстлөгтэй хаяа нийлсэн тундрт гол болон нууруудаар хүрээлэгдсэн ховор ойнууд бий болов. Тэр үед Неман, Двиний эрэг дээр дулаанаас нэг хэвийн болсон ландшафтад манай үеийг хүртэл амьдарч буй балтийн бүлгийн эртний угсаатнууд бүрэлдсэн юм. Балтийн топоним, гидроним гэгдэх хуурай болон газрын хэсгүүд нь тэдний өвөг дээдсийг тойрон байсан байгаль орчин нь өөр байсан цагийг санагдуулсан нэн эртний ул мөрийг хадгалж байдаг юм. Угсаатан төдийгүй, ландшафтууд ч түүхтэй байдаг билээ. 32. Сейбутис А. Палеогеография, топоника и этногенез //Изв. АН СССР. Сер. географическая. 1974. № 6. С. 40- 53.
УГСААТНЫ НИЙЛЭГЖИЛТЭД ЛАНДШАФТЫН ШИНЖ ЧАНАРЫН НӨЛӨӨЛӨЛ
Одоо бид хийсэн шинжилгээнээсээ нэг хэвийн ландшафтын газар нь түүнд амьдран буй угсаатныг тогтворжуулдаг, харин янз бүрийн ландшафт нь угсаатны шинэ бүрдэл бий болоход хүргэдэг өөрчлөлтийг тэтгэдэг гэсэн дүгнэлт томъёолж болох байна.
Энд ахиад л асуулт үүснэ. Ландшафтын хослол нь угсаатны нийлэгжилтийн шалтгаан уу, аль эсвэл ердөө л таатай нөхцөл үү? Хэрэв шинэ ард түмэн үүсэх шалтгаан нь газар зүйн нөхцөлд байсан бол тэр нь байнгын үйлчилдэг болохоор ард түмнийг дандаа үүсгээд байх ёстой юм, гэтэл ийм зүйл болсонгүй. Эндээс угсаатны нийлэгжилт нь хэдийгээр газар зүйн байдлаар нөхцөлддөг ч гэсэн бусад шинжлэх ухаанд хандаж баймаажин нээж болох өөр шалтгаанууд бас байна. Эдгээр асуудлуудыг тусгай бүлгүүдэд авч үзэх бөгөөд эцсийн эцэст яагаад угсаатнууд өөр хоорондоо төсгүй байдаг, угсаатны нийлэгжилт нь байгалийн бусад үзэгдлүүдтэй ямар харьцаанд байдаг вэ? гэсэн үндсэн асуултуудад хариу өгөх болно.
Намайг шүүмжлэгчдийн нэг нь шинэ угсаатан үүсэх нь хоёр буюу түүнээс дээш ландшафтын зааг бүсэд байдаг ба харин нэг хэвийн ландшафтад тэдгээрийн хөгжил саадгүй явагдана гэсэн миний сэдэвтэй маргалдсан юм. 33. Козлов В. И., Покшишевский В. В. Этнография и география //Советская этнография. 1973. № 1. С. 9-10.
Тэр энэ хэсэгтээ: “Угсаатны нийлэгжилт нь бага зэргийн онцлогтой ямар нэг (онцлов.Л.Г.) ландшафтуудад салбарлаагүй, харин хэдийгээр зарим тохиолдолд байгалийн нөхцөл нь угсаатны үйл явцыг нэлээд (онцлов-Л.Г.) хурдасгах буюу сааруулж чадаж байсан ч Ойкумены бүх мужуудад бодитой явагдсан юм” гэж бичсэн байдаг. Миний шүүмжлэгч яагаад ч юм шүүмжлэлийг нь арилгаж буй өөрийнхөө зөрчлийг олж харсангүй. Шинэ угсаатан, өөрөөр хэлбэл шинэ системийн бүхэллэг үүсэх явдал нь ямагт шинэ угсаатандаа угсаатны хольц мэт ханддаг хуучин угсаатныг эвдэхтэй холбоотой байдаг. Энэ үйл явц хэрэгжихэд эсвэл угсаатны шинэ үйл явцыг эхлүүлэгч, эсвэл дээр дурдсан “нэлээд” гэсэн үг шийдвэрлэх ач холбогдолтой болж, орчны эсэргүүцлийн улмаас унтардаг, бидний дээр тэмдэглэсэн, мөн дараа тайлбарлах тийм лугшилт хэрэгтэй юм.
Бид хэдийгээр хүний ямар нэг үйлчлэлд өртөөгүй ландшафт Дэлхий дээр байхгүй гэдгийг хатуу мэддэг ч гэсэн унаган байгалийн үзэгдэл болох ландшафтуудын тухай одоо болтол ярьцгаалаа. Асуудлыг ойлгохын тулд бид энэ хялбарчлалыг санаатайгаар оруулсан юм, харин хиймэл, өөрөөр хэлбэл урбанист буюу хотожсон ландшафт нь маш эртнээс тодорхой байсан. Вавилонд нэг сая орчим, Римд бүтэн хагас сая, Константинопольд сая гаруй хүн оршин сууж байсан. Эдгээр аварга хотуудыг бие даасан ландшафтын бүсүүд гэж үзэж болно.Тэд ч өөрсдөө ийн үзэх боломж олгодог: хотын зах болон тосгонуудад голдуу түргэн зуурын, заримдаа тогтвортой, гэхдээ гишүүд нь заавал мөрдөх өвөрмөц, зан үйлийн дахин давтагдашгүй тогтсон үзэлтэй дэд угсаатнууд цаг ямагт үүсдэг байв.
Өөр нэг эмзэг асуудал байдаг юм. Техникийн соёл иргэншлийн эрин үе болсон манай цаг үе орчин үеийг биш, түүхийг судлахад нээгдсэн зүй тогтол хамаарахгүй онцгой эрин үе бус уу? Энэ асуултыг аль хэдийн маш хурцаар тавьж, нарийн өгүүлсэн байгаа. Энэ нь “Эдгээр нэр томъёог ойлгох ландшафтын утгаар нь тал нь талаараа, цөл нь цөлөөрөө үлдсэн үү? Бүхнээс хүчтэй ургамлын аймаг (талд газар тариалан, цөлд газар услаж бэлчээр гаргах) өөрчлөгдсөн, үүний улмаас гадаргуугийн урсгал, хөрсний эвдрэл зэрэг байгалийн шүтэлцээний цаашдын бүх “гинжин” бүрдлүүд өөрчлөгдсөн”. Үнэхээр ч ландшафт бүрдүүлэх хүний хүчин зүйл сүүлийн гурван мянган жилд дэлхий гадаргуугийн нүүр царайнд чухал байр олж авсан ба олсоор ч байна. 34. Саушкин К). Г. По поводу одной полемики //Вестник МГУ. 1965. № 6. С. 79- 82. Ср.: Калесник С. В. Некоторые итоги новой дискуссии о “единой” географии // Изв. ВГО. 1965. № 3. С. 209-211.
Хөдөө аж ахуй амьтан, ургамлын аймгийг өөрчилж, уран барилга нь гадаргуугийн чухал элемент болж, нүүрс болон хийн шаталт нь атмосферийн бүрэлдэхүүнд нөлөөлж байна. Энэ утгаар Парижийг хөгжлийн хурдасгасан хэмнэлтэй ойн ландшафтын бүсэд буй антропогенийн гео чуулбар гэж үзэж болох бөгөөд энэ микро газар нутгийн өнөөгийн дүр төрх нь парижийн гүнгүүдийн дундад зууны цайзын байдлаас ч, римийн Лютецээс ч ялгаатай юм. 35. Геохор буюу чуулбар гэдэг бол экологийн онцлогоороо нэг төрлийн, гэхдээ энэ онцлогоороо хаяа залгалдах хэсгүүдээсээ ялгагдах дэлхийн гадаргуугийн хэсэг болно.
Гэхдээ урсгалгүй нуур шохойжиж, маш хурдан намаг болон хувирахад түүнийг тойрсон ой энэ үед өөрчлөгддөггүй. Антропогенийн болон гидрогенийн (усны –Орч) бүрдлүүдийн хоорондын ялгаа хэчнээн ч агуу их байсан байгал судлалын талаасаа зарчмын шинжтэй биш юм. Гэхдээ бидний тавьсан: хүн хэрхэн, яагаад Дэлхийн нүүр царайг өөрчилдөг юм бэ? гэсэн асуултад ялгаа болон адил талыг иш татах нь хариу өгч чадахгүй юм. Ийм учраас эртний эллинчүүд түүхийн судалгаандаа “үнэнийг хайх” гэж нэрлэж байсан зүйлийг үргэлжлүүлье.
XVI. Антропогенийн ландшафт бий болох нь
НИЙГМИЙН ХӨГЖИЛ БА ЛАНДШАФТЫН ӨӨРЧЛӨЛТ
Янз бүрийн угсаатанд орж буй хүний “зан үйлийн“ тухай ярьж буй болохоор тэд түүхэн хувь заяагаар хаягдсан аль нэгэн байгалийн ландшафтдаа хэрхэн үйлчилж байгаа хамгийн энгийн зүйлд анхаарлаа хандуулах хэрэгтэй юм. Өөрөөр хэлбэл, бид хүн төрөлхтнийг угсаатны хамт олон болгож хуваасан ангиллаа харгалзан ландшафт бүрдүүлэхэд антропогенийн хүчин зүйлийн шинж чанар болон ялгарлыг мөшгөх ёстой болж байна.
Гол асуудал нь хүний үйлдсэн өөрчлөлт хэчнээн их, тэр ч байтугай эдгээр нь үр дагавраараа сайн юмуу муу гэдэгт биш, харин энэ бүхэн хэзээ, яагаад, хэрхэн болдогт оршино.
Аж үйлдвэржсэн дүүргүүдийн ландшафт болон услалтын хиймэл системтэй мужууд нь тал, тайга, халуун бүсийн ой, цөлөөс хавьгүй их өөрчлөгдсөн нь илэрхий юм. Гэхдээ бид эндээс нийгмийн зүй тогтол олохыг оролдох аваас давж баршгүй бэрхшээлтэй тулгарах болно. Юкатан дахь майян газар тариалангийн соёл НТӨ Y зуунд овгийн байгууллын үед бүтээгдэж, ангийн байгуулал үүсэх үед уналтад орж, европын техник оруулж, загалмайтан болсон индианчуудын дэмжлэг байсан хэдий ч Испанийн ноёрхлын үед дахин сэргээгүй юм. Египетийн аж ахуй феодализмын үед удаан, гэхдээ тасралтгүй уналтанд орж байхад мөн энэ үед, мөн ийм нийгмийн харилцааны нөхцөлд Европт худалдааг ярьдаггүй юмаа гэхэд газар тариалан, гар урлалын урьд үзэгдээгүй мандалт болсон юм. Манай судалгааны үүднээс бол энэ нь Египетийн ландшафт энэ үед тогтвортой байж, харин Европт эрчимтэй өөрчлөгдсөн хэрэг юм. XIX зуунд Египетийн гадаргууд антропогенийн зүйлүүдийг оруулж Суэцийн сувгийг татсан нь тийшээ франц, англи зэрэг европын ард түмэн нэвтэрсэнтэй холбоотой болохоос биш, уугуул–феллахуудын үйл ажиллагаатай хамаагүй юм.
XYI зуунд Англид эхлэн буй капитализмын үед “хонь хүн идэх” болсон бол XIII–XIY зууны Монголд хэдийгээр тэнд феодализм хөгжөөгүй байсан хэдий ч хонь нь Их Хянганы хойд, мөн Соён, Хамар Давааны өмнөх бэлд амьдарч байсан анчин–тунгусуудыг “идсэн“ юм. Монгол хоньд зэрлэг салаа туурайтны хүнс, ус болж байсан жижиг булгуудын усыг ууж, өвсийг нь идсэн байна. 36. Грумм-Гржимайло Г. Е. Рост пустынь и гибель пастбищных угодий и культурных земель в Центральной Азии за исторический период //Изв. ВГО. 1933. Т. 65. Вып. 5.
Салаа туурайтны тоо толгой багасахын хэрээр анчин овгууд дасаж дадсан хүнс тэжээлээсээ салж, сульдан доройтож, талын малчдаас хамааралт байдалд орж, Азийн угсаатны зүйн газрын зургаас алга болсон билээ. Өөр жишээнүүд гэвэл: Азорийн арлууд нь Мексик, Нидерландад түйвээж байсан испанийн феодалуудаас болоогүй, овгийн байгуулал нь арай дуусаагүй байсан Астуурийчууд болон баскуудын авчирч тавьсан ямаанаас болж нүцгэн хадан цохио болон хувирсан билээ. Америк дахь бизоныг капитализмын үеийн анчид, харин Шинэ Зеландын мао шувууг ангийн байгууллыг ч мэдэхгүй байсан маорийчууд устгасан. Тэд өөрийнхөө арал дээр америк төмсийг нутагшуулсан, Гэтэл Орост энэ зүйлийг хийхэд II Екатеринагийн цэрэг–хүнд суртлын бүх машин хэрэглэгдсэн юм. Эндээс өөр хавтгайд байх зүй тогтол үйлчилж байна.
Асуудлыг өөрөөр тавиад үзье. Байгаль хүнд хэрхэн нөлөөлж байна вэ ? гэдгийг биш, харин түүнд хөгжлийнхөө янз бүрийн шатанд байгаа янз бүрийн ард түмнүүд хэрхэн нөлөөлдөг вэ? Бид энэхүү дам үйлчлэлийг одоо болтол барьж аваагүй завсрын мөчийг оруулж ирж байна. Ингэхэд шинэ аюул үүснэ: хэрэв ард түмэн бүр, тэр тусмаа оршихуйнхаа эрин үе тус бүрт байгальд өвөрмөцөөр нөлөөлдөг бол бид энэхүү давхцсан явдлыг олж харах боломжгүй, мөн ямар нэгэн дүгнэлт улмаар судлан буй үзэгдлээ эрэгцүүлэх хийх боломжоо алдах эрсдэлд орж байна.
Энд нь харамсалтай нь нийгмийн шинжлэх ухаанд хэрэглэдэггүй, байгалийн шинжлэх ухаан дахь жирийн ангилал, ажиглаж буй үйл явцыг системчлэх аргууд тус болох юм. Иймээс бид угсаатны ландшафтад харилцах тухай ярихдаа хүмүүнлэгийн угсаатны судлалын салбарт оролгүйгээр газар зүйн ард түмэн судлалын суурин дээр үлдэх юм.
Арьстан, нийгмийн болон материаллаг соёл, шашин гэх мэтийн нийгмийн ухаанд хүлээн зөвшөөрсөн угсаатны ангиллын шинж тэмдгүүдээс татгалзсанаараа бид газар зүйн шинжлэх ухаанд буй зарчим, аспектуудыг сонгох ёстой болж байна. Ийм арга нь дээр нэгэнт дурдсан биоценезийн үзэгдлийг бүрдлийг бүрдүүлж буй амьтдын бүх хэлбэрийн тоог зэрэгцүүлэн харьцуулах явдал байж болох юм. Жишээлбэл, тухайн газар дээрх чононы тоо туулай болон хулганы тооноос хамаарч, эдгээр нь өөрийн ээлжинд өвс, усны хэмжээгээр тодорхойлогддог. Энэхүү харьцаа нь голдуу тодорхой хэмжээний дотор хэлбэлзэх ба маш ховор, түр зуур зөрчигддөг.
Энэхүү зураглал нь хүнд ямар ч хамаагүй мэт санагдах боловч гэхдээ дандаа ийм байдаггүй. Биоценез болон биохор буюу био нэгдлийн бүрэлдэхүүнд ордог тооны хувьд ялигүй угсаатны асар олон тооны нэгж байдаг билээ. Энэхүү жижиг ард түмэн буюу заримдаа зүгээр л энгийн овгуудтай харьцуулахад орчин үеийн түүхэн, соёлжсон угсаатнууд нь аварга биет гайхлууд бөгөөд гэхдээ эд тоогоор цөөн, түүхийн үзүүлж байгаагаар мөнх биш байдаг. Энэ үндсэн дээр бид өөрсдийнхөө анхны ангиллыг босгож байна: 1) биоценезэд орж, ландшафтад бүртгэгдэн орсон, ингэснээрээ хөгжлөө хязгаарладаг угсаатнууд. Оршихуйн энэ арга нь амьтдын олон төрөлд хэвшмэл бөгөөд хөгжлөөрөө зогссон мэт байдаг. Амьтан судлалд энэ бүлгийг үлдэц гэж нэрлэх бөгөөд энэ нэр томъёог хөгжлийн тодорхой шатан дээрээ царцсан угсаатанд хэрэглэх ямар ч үндэс байхгүй юм, 2) өөрийн био нэгдлийн хилийг давж нутагладаг, идэвхитэйгээр өсөж үрждэг, өөрийн анхдагч биоценезээ өөрчилдөг угсаатан. Энэхүү хоёр дахь төлөв байдлыг газар зүйд сукцесси буюу дэс дараалан үржих гэж нэрлэдэг.
Эхний бүлгийг бүрдүүлж буй угсаатнууд байгальд хандах талаараа хуучинсаг бөгөөд хэд хэдэн зүй тогтолтой байдаг. Хэд хэдэн жишээ авъя.
ИНДИАНЧУУД, СИБИРИЙН АРД ТҮМНҮҮД, ТЭДГЭЭРИЙН ЛАНДШАФТУУД
Канадын хойт америкийн болон прерий буюу хээр талын индианчуудын ихэнхи нь европчуудыг хүрч ирэхээс өмнө Хойт америкийн биоценезэд амьдарч байсан. Овог дахь хүмүүсийн тоо нь бугын тоогоор тодорхойлогдож байсан ба учир нь ийм нөхцөлд жам ёсны өсөлтийг хязгаарлах нь зайлшгүй байсан ба хамтран амьдрахуйн хэм хэмжээ нь овог хоорондын хөнөөлт дайн байлаа. Эдгээр дайны зорилго нь газар нутаг булаан авах, бусдын өмчийн булаан авах, улс төрийн ноёрхол байсан уу гэвэл үгүй юм. Энэхүү журмын үндэс нь гүн эртнээс улбаатай бөгөөд түүний биологийн зориулалт нь ойлгомжтой юм. Ан олзны тоо хэмжээ хязгаарлагдмал, иймээс өөрийгөө болон хойч үедээ эдгээр амьтдыг алах бодит боломжийг хангаж өгөх нь нэн чухал болдог ба энэ нь өрсөлдөгчдөө зайлуулна гэсэн хэрэг юм. Энэ бол бидний ойлгодог шиг дайн биш бөгөөд харин тодорхой биоценезийг дэмжин тэтгэх тэмцэл юм. Ийм хандлагын үед мэдээж хэрэг, индианчуудын бодлоор төгс байдлынхаа оргилд байгаа байгалийг хүсмээргүйгээр эвдэх, ямар нэг өөрчлөлт оруулах тухай яриа ч байж болохгүй юм.
Газар тариалан эрхлэдэг пуэбло гэгдэх индианчуудын овог яг ингэж амьдардаг байсан ба ялгаа нь гэвэл зэрлэг амьтдын махыг маис хэмээх эртний эрдэнэ шишээр сольдог байлаа. Тэд талбайгаа өргөсгөдөггүй, усалгаа хийх гэж голын усын ашигладаггүй, техникээ боловсронгуй болгодоггүй байв. Тэд сул биетэй хүүхдийг өвчнөөр үхэх боломж олгож, бие сайтайгий нь нямбай хүмүүжүүлдэг байсан ба эдгээр нь хожим навах болон апачи нартай хийх тулгаралтанд амь үрэгддэг байжээ. Ингээд аж ахуйн арга нь өөр мөртлөө байгальд харьцах харьцаа нь огт өөр байна. Гэхдээ нэг л зүйл ойлгомжгүй үлдэж байна. Энэ бол яагаад навахачууд пуэбло индианчуудаас газар тариалангийн дадлага аваагүй, яагаад нөгөөдүүл нь хөршөөсөө хөнөөлт дайн хийх тактик зээлдэн аваагүй юм бол оо ?
Дашрамд дурдахад нагуа бүлэгт багтаж байсан ацтекууд XI–XIY зууны үед Мексикийн уулархаг газар шилжин сууж, түүний ландшафт болон гадаргууг нэн эрчимтэй өөрчилсөн байдаг. Тэд теокалли (гадаргын өөрчлөлт) байгуулж, акведук буюу усан суваг, хиймэл нуур (техногенийн гидрологи) байгуулж, маис, тамхи, улаан лооль, төмс болон бусад олон ашигтай ургамал (ургамлын аймгийн өөрчлөлт) тариалж, гүн улаан өнгийн сайхан будаг өгдөг конишель хэмээх шавьж (амьтны өөрчлөлт) үржүүлдэг байв. Товчоор хэлбэл тэр үед апахи, навакууд байгалийг хамгаалж байхад ацтекууд түүнийг өөрчилж байжээ.
Энд шийдвэрлэх үүргийг хэдийгээр Рио–Грандын эргийн уур амьсгалаас нэг ч их ялгаагүй өмнөд Мексикийн халуун уур амьсгал гүйцэтгэсэн гэж таамаглаж болох юм. 37. Морган. Л. Г. Дома и домашняя жизнь американских туземцев. Л., 1934. С. 146-163.
Гэхдээ Хойд Америкийн яг төвд, Огайон хөндийд индианчуудад өөрт нь ч тодорхойгүй зориулалттай далан–газрын аварга байгууламж илэрсэн байна. Хэзээ нэгэн цагт тэнд байгал болон цаг уурын нөхцлийг өөрчилж байсан, англосакс гаралтай америкчуудад саад болохгүй байгаа шиг тэдэнд ч бас саад болоогүй байсан ард түмэн амьдарч байсан нь магадтай юм.
Үүний зэрэгцээ индиан овгийн нэг болох тлинкит, түүнчлэн алеутууд боол эзэмшлээр ажиллаж, өргөн хэмжээний боолын худалдаа эрхэлдэг байсныг тэмдэглэх хэрэгтэй юм. Боолууд нь баруун хойд Америкийн хүн амын гуравны нэгийг бүрдүүлж, зарим тлинкит баячууд 30-40 боолтой байжээ. Боолчуудыг системтэйгээр зарж, худалдан авч, бохир ажил, оршуулгын үеэр тахилга хийх, өргөлийн ёслолд ашиглаж, боол эмэгтэйчүүдийг эздийн гэрийн боолоор үйлчлүүлдэг байв. Ийм байлаа ч гэсэн тлинкитууд нь үйлдвэрлэгч биш, эзэмшигч шинжийн бүдүүлэг аж ахуйтай жирийн л нэг анчин овог байсан юм. 38. Окладников А. И. Неолит и бронзовый век Прибайкалья. III. (Глазковское время). М.; Л., 1955. С. 238-239
Үүнтэй адилтгам байдал хойт Сибирьт бас байсан юм. Угор, тунгус болон палеоазийн бүлэг нь ахуйн болон аж ахуйн шинж чанараараа биоценезийн дээд хэсэг болох ландшафтынхаа нэгэн хэсэг болсон мэт байлаа. Нарийн яривал тэд ландшафтдаа “элсэн орсон” байв. Үүнээс зарим талаар гажсан нь якутууд байсан бөгөөд тэд хойт зүгт хөдлөхдөө мал аж ахуйн дадлаа авч, адуу, үхрийг энд авчран, өвс хадаж байсан ба ингэснээрээ Лена мөрний хөндийн ландшафт болон биоценезэд өөрчлөлт оруулж байв. Гэхдээ энэхүү антропогенийн дахин үржил нь ердөө л шинэ биоценез бий болоход хүргэсэн ба дараа нь оросын газар хайгчид ирэх хүртэл тогтвортой байдлаа хадгалсаар байлаа.
Евроазийн тал нутаг үүнээс огт өөр дүр зурагтай байлаа. Энд амьдралынх нь үндэс эрчимгүй мал аж ахуйтай, учир түүнчлэн байгалийн өөрчлөлт ч байхгүй байх ёстой мэт санагдана. Үнэн хэрэг дээрээ тал нутаг нь гэрийн тэжээвэр амьтдын сүрэг гадаргууг өөрчлөгч дов толгодоор хучигдсан бөгөөд бүүр эрт дээр үеэс тал нутагт удаан биш ч гэсэн өөхний талбай үүсдэг байжээ. 39. Гумилев Л. Н. Хунну. С. 147.
Хүн, түрэг, уйгарууд бүдүүлэг газар тариалан эрхэлж байсан. Энд байгалийг ариг гамтай хувирган өөрчлөх эрмэлзэл байнга үүсч байсан нь илт байдаг. Мэдээж тоон хэмжээгээрээ энэ нь Хятад, Европ, Египет, Ирантай харьцуулахад өчүүхэн хэдий ч газар тариалангийн ард түмнүүд байгальтай харьцдагаас зарчмын хувьд ялгаатай нь нүүдэлчид түүнийг орвонгоор нь хувирган өөрчлөх биш, харин оршин буй ландшафтыг сайжруулахыг оролддог байсан байна, Ийм учраас ч бид евроазийн нүүдэлчдийг манай ангиллын хоёрдугаар зэрэгт тлинкуудын ангийн харилцаа нь харьцуулшгүй илүү хөгжилтэй байсан хэдий ч тэднийг биш, ацтекуудыг оруулсны адилаар оруулах ёстой юм. Энэ дүгнэлтүүд нь өнгөц харахад хачирхалтай байлаа ч гэсэн судалгааны шинжлэх ухааны дүгнэлт гаргаж авахын тулд бид өөрсдийнхөө ангиллыг хатуу дэс дараатайгаар баримталбал зохино.
Нүүдлийн соёл уналтад орсон дотоод зөрчил нь анхандаа түүний дэвшилтэт хөгжлийг хангаж байсан тэр зүйл буюу нүүдэлчид аридный буюу хуурай бүсийн геобиоценозод орсон явдал байлаа. Нүүдэлчдийн хүн амын тоог хүнс буюу малын тоо толгой тодорхойлж, энэ нь эргээд бэлчээрийн талбайг хязгаарладаг байв. Бидний авч буй үеүдэд тал нутгийн орон зайд хүн ам маш бага хэлбэлзэж байсан: Хүнгийн үед 300 -400 мянга , монгол улсын цэцэглэлтийн үед 1300 мянган хүнтэй байсан ба хожим нь энэ тоо багасаж, XYI – XYII зууны үед хүн амын нарийн тоо байхгүй болсон юм. 40. Hatoun G. Zur Ue-tsi Frage /Ztechr. Disch. Morgenland. Ges. (Liezig). 1937. S. 306. 41. Мункуев Н. Ц. Заметки о древних монголах //Монголо-татары в Азии и Европе / Под ред. CJI. Тихвинского. М., 1970.
42. XX зууны эх гэхэд Гадаад Монголын хүн ам 900 мянга орчим байсан ба харин 3 сая монгол Өвөр Монгол болон хуучны Тангадын хаант улс болон зүрчидийн Чин улсын нутаг дэвсгэр дээр амьдарч байлаа. (Зохиогчийн энэ тайлбар анхаарал татсан тул бусдаас нь онцлов -Орч )
Нийтэд тархсан үзэл бодлын эсрэгээр нүүдэлчид газар тариачланчдыг бодоход шилжин нүүхэд тийм ч их дуртай биш байдаг. Үнэн хэрэг дээрээ газар тариаланч сайн ургац авсан үедээ нэн цомхон хэлбэртэй, хэд хэдэн жилийн хүнс базааж чаддаг. Шуудайдаа гурил чихэж, түүнийгээ тэрэг юмуу завинд ачаалан зэвсэг хэрэгслээ базаахад л хангалттай бөгөөд энэ үед цэргийн хүчнээс бусад юу ч түүнийг зогсоож чадахгүй гэсэн итгэлтэйгээр алсын замд мордож болно. Хойт америкийн скваттерүүд, өмнөд Африкийн бурчууд, ипанийн конкистадорууд, оросын газар хайгчид, хижирийн эхэн үеийн арабын дайчид–Хижас, Йемен, Ираны уугуулууд, Газрын дундад тэнгисийг хэрэн хэссэн эллинчүүд ингэдэг байсан юм.
Нүүдэлчдийн хувьд бол энэ нь хавьгүй илүү хүнд. Тэд амьд байгаа хоол хүнстэй байдаг. Хонь болон үхрүүд удаан хөдөлдөг бөгөөд байнгын дадсан тэжээлтэй байх ёстой. Тэр байтугай энгийн өвсөн тэжээл солиход л мал хорогддог юм. Харин малгүй бол нүүдэлчид шууд өлсөж эхэлнэ. Ялагдсан орныг дээрэмдэх замаар ялгуулсан армийн дайчдыг тэжээж болно, харин тэдний гэр бүлийг тэжээж болохгүй. Ийм учраас алс холын аян дайнд хүн, түрэг, монголчууд эхнэр, хүүхдээ авч явдаггүй байв. Түүнээс гадна хүмүүс хүрээлэн буй байгальдаа дасч, хангалттай үндэслэл байхгүйгээр эх орноо харь нутгаар солихыг эрмэлздэггүй. Харин зайлшгүй шаардлага гарвал тэд орхин явсантайгаа төстэй тийм ландшафтыг сонгож нутагшдаг байна. Ийм ч учраас НТӨ 202 онд арми нь бүрэн эзэлсэн Хятадын газар нутгийг бүрэн авахаас татгалзсан юм. Үүнийхээ шалтгааныг тэд “Хятадын газрыг авлаа ч гэсэн түүн дээр хүнчүүд тэдэнтэй адилхан амьдарч чадахгүй” гэж томъёолж байсан. 43. Гумилев Л. Н. Хунну. С. 66.
Тэд зөвхөн Хятад төдийгүй, хэдийгээр талтай ч гэсэн улирлын чийглэгийн систем нь өөр байсан Семиречье–д НТӨ II зуун хүртэл нутаглаагүй юм. Гэхдээ II–III зуунд тэд эх орноо орхин Шар мөрөн, Или, Эмба, Яик, Доод Ижил мөрний эргийг эзэлсэн юм. Яагаад ?
Хамгийн олон янзийн эх булгаас авсан олон тооны, өөр хоорондоо холбоогүй өгөгдөхүүнүүд нь НТ III зуунд Евроазийн талын бүх бүс нутаг маш их хуурай байжээ гэсэн дүгнэлт хийх үндэслэлийг бидэнд өгч байна. Хойт Хятадад Циньлингийн нурууны сутропикийн ширэнгээс Ордосын тал, Говь хүртэл аажим шилжилт явагддаг. Бут сөөг нь нугаар, нуга нь талаар, тал нь хагас цөлөөр, эцэст нь Бэйшаны манхан, хад цохио зонхилдог. Чийглэг ихэдсэн үед энэ систем нь хойд зүг рүү, чийглэг багатай үед өмнө зүг рүү шилжиж, үүнтэй хамтаар өвсөн тэжээлт амьтад, тэднийг малладаг малчид шилжиж байдаг. 44. Гумилев Л. Н. Хунны в Китае. С. 10-
Ландшафтын чухамхүү энэ хөдөлгөөнийг Дорно дахины хамгийн мэдрэмжтэй түүхч Р.Груссе анзаараагүй юм. Хятадын эсрэг нүүдэлчдийн томоохон дайнуудыг тал нутаг хуурайших үеүдтэй холбох оролдлого хийхдээ тэр хятадын түүхчид энэхүү мөргөлдөөнүүдэд тухай бүрт нь Хятадын дотоод дахь улс төрийн нөхцөл байдлаас гаргасан ухаалаг тайлбар өгдгийг дурдсан байдаг. Түүний бодлоор нүүдэлчдийн түрэмгийллийг Их Талын цаг агаарын хэлбэлзэл гэхээсээ, Цагаан хэрэмний хамгаалалтын шугам муу байснаар тайлбарлах нь хялбархан юм. 45. Grousset R. Bilan de l’histoire. P. 283-
Зарим талаар тэд зөв, цэргийн томоохон үйл ажиллагаа нь ямагт үечилсэн шинжтэй, тэдний амжилт нь олон шалтгаанаас хамаардаг, энгийн натурал аж ахуйг ямагт хараад байх боломжгүй. Суурин газрын тариаланчдад хийдэг нүүдэлчдийн байнгын довтолгоо мөн л үзүүлэн шинжтэй, яагаад гэвэл энэ нь угсаатан хоорондын харилцааны далдлагдсан хэлбэр болно, нүүдэлчин довтлох үедээ зах дээр өөрийн өгөөмөр сэтгэл, заль байхгүйн улмаас алдсан зүйлээ өөртөө эргүүлэн авч байна. Аль алинд нь нүүдлийн их (миграци) ямар ч харилцаа байхгүй.
Гэхдээ үйл явдлыг нарийвчлан судлах үед суурин хөршүүдтэйгээ зөрчилдөхөөс зайлсхийсэн, гэхдээ хараахан хатчихаагүй байгаа горхиос малаа услахыг эрмэлзсэн энгийн хүн амын байнгын шилжих хөдөлгөөнийг амархан харж болно. Иймэрхүү байдал бидний нүдэн дээр Сахельд (Сахараас өмнөх хуурай тал) болж, туагери угсаатан харамсалтайгаар задарч, гэхдээ дайн болоогүй юм. 46. Курьер ЮНЕСКО. 1975. Май.
Үнэхээр ч энд баруун европын капитал туагеруудын аж ахуйг энгийн хэлбэрээс таваарын болгосноор малын бэлчээрийг талхлагдснаас хэрэг явдлыг хүндрүүлсэн юм. Энэ зарчмыг зарим засвартайгаар илүү эртний үеүдэд хэрэглэж болно.
Хятадын хойт хилд болсон үйл явдлыг хангалттай тодорхой судлаад үзэхэд өөрөөр хэлбэл Их Хэрэмийн бүсэд эхлээд хүнчүүд хойд зүгт (НТӨ II–I зуун), дараа нь тэд өмнө зүгт, ялангуяа YI зуунд улам хүчтэй нүүх болсон байна. Тэхэд хүн болон сүмбэ (эртний монголчууд) нар Шэньси болон Шаньси муж, тэр ч байтугай Их Хэрмээс өмнүүр газарт суурьшжээ. Гэхдээ тэд Хунаны чийглэг бүсэд нэвтэрч ороогүй юм.
Өмнө зүг нүүдэлчид анх нэвтрэн орсон нь нүсэр дайнуудтай холбоогүй байсныг тэмдэглэх нь нэн чухал юм. Хятадад булаан эзлэгчид ирээгүй, харин малаа услах боломжтой болох гэж голын эргээр суурьших зөвшөөрөл хүссэн ядуучууд орж ирсэн юм. Хожим нь хойд Хятадыг эзлэн авсан нь голчлон хятадын газар хагалагчид аажмаар, үл мэдэгдэм байдлаар хойд зүг дэх талбайгаа орхиж, бороо хангалттай ордог өмнө зүг рүү нүүснээс болсон юм. Ийнхүү нүүдэлчид хоосорсон талбайг эзэлж, түүнийг малын бэлчээр болгосон юм.
Гэхдээ IY зууны дунд үед урвуу үйл явц ажиглагдсан байна. Уйгарын бусад овгуудын тоонд орж байсан теле (телеутууд) хэмээх томоохон овгийн бүлэг Ганьсугийн баян бүрдээс Зүүнгар болон Халхад нүүн иржээ. Энд мөн ийм замаар эртний түргүүд ирж, талын бүсийн захаар хязгаарлагдсан Агуу их хаант улсыг YI зуунд байгуулсан юм.
Энэ нь юу гэсэн үг вэ? гэхээр Их Талын бүс бэлчээрийн мал аж ахуйд ахин тохиромжтой болсон гэсэн хэрэг юм. Өөрөөр хэлбэл, тэнд цөлийн оронд өвслөг тал сэргэж, өөрөөр хэлбэл бүслүүр хойд зүгт шилжсэн байна. Хэрэв ийм бол хойд Хятадад ч гэсэн хятадуудад тохиромжтой, нүүдэлчдэд хөнөөлтэй чийглэг уур амьсгал дахин сэргэсэн байна. Энэ нь дайны давуу тал өмнөдүүдийн талд байх ёстой гэсэн хэрэг юм. Үнэхээр ч ийм болсон юм. YI зууны эхээр Шар мөрний бүх сав газрыг эзэлж байсан нүүдлийн Тоба гүрэн хятадын Вэй гүрэн болон хувирч, цаазаар авах ялын дор сүмбийн хувцас, зан үйл, тэр ч байтугай хэлийг хоригложээ. Үүний дараа унаган хятадууд эрх баригч эзэнт гүрний гишүүдийг хядаж, бүх харийнханд дайсагнагч, нэн түрэмгий өөрийн Суй эзэнт гүрнийг байгуулжээ.
Энэ үед адилтгам шинж чанартай нүүдэл талын баруун хэсэгт болж байлаа. 155 онд сүмбэчүүдээс сүйрлийн цохилт авсан хойт хүнчүүд баруун зүг явав. Тэдний зарим нь Тарвагатайн уулархаг мужид бэхжиж, хожим нь (талын чийглэг ихсэх үед) Семиречийг эзлэн суух болсон юм. Нөгөө хэсэг нь Ижил мөрний доод эрэгт нүүн ирж, хүчирхэг аланчуудтай тулгарав. Хүнчүүд “алануудыг тасралтгүй тэмцлээр цуцаан, тэднийг эзлэн авав” (Иордан) ингээд 370 онд Доны нутагт хүрч иржээ. Энэ үедээ тэд аймшигт хүчин байсан боловч Y зууны дунд гэхэд баруун талаасаа гепидуудад, зүүн талаасаа болгаруудад цохигдон алга болжээ. Угуул хүмүүс нь нүүн ирэгсдийгээ ялан дийлсэн байна. Нүүдэлчдийн шилжих дараагийн долгион нь X зуунд болсон юм. 47. Гумилев Л. Н. Поиски вымышленного царства.
Тэгэхэд Арал тэнгисийн эргээс нүүн ирсэн печенегүүд хар далайн эрэг орчмын тал нутаг суурьшсан юм. Мөн тийшээ орчин үеийн Казакстанаас түргүүд, Барабиний талаас кипчак–половчууд ирсэн билээ. Энэ нь ахиад л булаан эзлэлт биш байлаа, харин том тулалдаан, аян дайныг мөргөлдөөн, дайралт орлосон жижиг бүлгүүдээр аажмаар нэвтэрсэн билээ.
Мөн ийм байдал тэр үед Ойрхи Дорнодод бүрэлдсэн юм. Зүүнгараас карлукууд Кашкар болон мөсний болон хөрсний усаар тэжээгддэг Хотаны баян бүрдүүд рүү нүүн ирсэн. Туркмен–сельжукууд Кызыл–Кум дахь бэлчээрээ орхиж Хорасанд нэвтэрсэн. Тэнд тэд хүчирхэг хүчин зохион байгуулж, 1040 онд Масуд Газневийн байнгын армийг бут ниргэжээ. Дараа нь тэд Персийг эзэлж, 1071 онд византийн эзэн хаан Роман Диогенийг ялж, бүх Бага ази, Сирийг эзлэв. Тэд суурьших газраараа орхин явсан эх орныхоо ландшафтыг санагдуулсан хуурай тал, уулсын бэлийг сонгосон байдаг нь сонирхолтой юм.
XIII зуунд монголчууд морин цэргээ Аннам болон Бирмийн битүү ширэнгэ, Иордоны хөндий, сувдан Адриатик хүргэх үед бид үүнтэй төстэй юм олж харахгүй. Энэ аян дайн, ялалттай нүүдэл суурьшил огт хамаагүй байсан юм. Монголчууд дайныг багавтар, хөдөлгөөнт, муу зэвсэглэсэн мөртлөө сайтар зохион байгуулсан отрядуудаар явуулсан байдаг. Баруун талын улсуудын эрх баригчдад зарим тооны үнэнч цэрэг өгөх шаардлага гарсан үед ч монголын төв засгийн газар эзлэгдсэн овгуудаас цэрэг ялган өгч байсан юм. Хулагу хаанд найманчуудыг, харин Батад мангууд, зүрчид (хин) нараас хэдхэн мянган хүн гарган өгч байлаа.
Чингисийн үр хүүхэд, ач нарын аян дайныг цаг уурын хэлбэлзэлтэй холбох ямар ч үндэслэл байхгүй юм. Харин ч энэ үед тал нутагт нүүдлийн мал аж ахуйд зохистой нөхцөл бүрэлдсэн байсан гэж бодож болно. Гурван армид морь хангалттай, 1200–1206 оны овог хоорондын хатуу дайны дараа малын тоо толгой сэргэж, хүн ам нь 1300 мянга болтлоо өссөн байна. Гэтэл үүний урвуугаар XYI зууны харьцангуй тайван үед Монгол бие даасан байдлаа алдаж, XYII зуунд тусгаар тогтнолоо алдсан юм.
Тэр үеийн дэлхийд хамгийн хүчтэй гүрний энэхүү сулралын шалтгааныг XYII зууны хятадын газар зүйч: “Бүх монгол хөдөлгөөнд орсон ба монголын овог, аймгууд сайн бэлчээр, ус хайж тархан сарнисан, иймээс тэдний цэрэг аль хэдийн нэгдмэл хүч байхаа больсон” гэж бичиж байжээ. 48. Грумм-Гржимайло Г. Е. Рост пустынь и гибель пастбищных угодий и культурных земель… С. 437-454.
Энэ бол үнэхээр их нүүдэл байлаа, монголын нүүдэлчид хуурайссан эх орноосоо гарч, Төвдийн сүрлэг уулс, их уст Ижил мөрний эрэг, Туркестаны баян бүрдүүд рүү нүүсэн нь даян дэлхийн түүхэн хэмжээнд анзаарагдаагүй юм. Нүүдлийн соёлын сүүлчийн хэлтэрхий Ойрдын холбоо аж ахуй нь Алтай, Тарвагатайн нурууны уулын бэлчээрт суурилсан учраас 1758 он хүртэл тэссэн юм. Гэхдээ тэр манж болон хятадад эзлэгдсэн юм. 49. Гумилев Л. Н. Изменения климата и миграции кочевников //Природа. 1972. № 4. С. 44-52.
Ингээд НТӨ III–XYIII зууны хооронд дахь хоёр мянган жилийн хугацаанд бид тал нутгийн хуурайшлын гурван үеийг ажиглаж байна, энэ бүрт нүүдэлчид Их Талын хязгаар, тэр ч байтугай түүний заагаас даван нүүхтэй холбоотой байсан. Энэ шилжилт нь байлдан эзлэлтийн шинж чанарыг агуулаагүй байсан юм. Нүүдэлчид малынхаа цангаа, өөрийнхөө өлсгөлөнг тайлахаас өөр ямар ч зорилго тавиагүй жижиг бүлгүүдээр нүүн явдаг байв.
Үүний урвуугаар талын бүс чийглэг болох үед нүүдэлчид эцэг өвгөдийнхөө эх оронд эргэн ирж, тэдний дөрвөн хөлт баялаг өсөж үржин, баян тансаг байдалтай холбоотой дайнч бодлого бий болдог байв. Чингэхдээ байлдан эзлэлтийг огтхон ч “амьдралын орон зай” олж авахад биш, төрийн бодлогоор үйлддэг байжээ. Нүүдэлчид ердөө л амь зогоох биш, тэдний зорилго нь давамгайлах явдал болдог.
Халуун бүсийн овог аймгуудыг авч үзэх нь нэгэнт тодорхой буй материалуудтай харьцуулан үзэхэд зарчмын хувьд шинэ зүйл юу ч өгдөггүй юм. Иймээс Египет, Месопотами, Хятад зэрэг байгалийг хувирган өөрчилсөн сонгодог жишээнүүдэд хандах нь зүйтэй юм. Европыг бид түр орхиж байна, учир нь манай зорилт бол зүй тогтлыг хайхад оршдог, түүнийг зөвхөн дууссан үйл явцуудад л ажиглаж болно.
“ЭЛБЭГ ДЭЛБЭГ ХАВИРГАН САРНЫ” ЭРТНИЙ СОЁЛ ИРГЭНШЛҮҮД
Э.Бруксийн судалгаагаар вюрмийн мөстлөгийн үед атлантын циклон Хойд Сахар, Ливан, Месопотами, Иранаар дамжин, Энэтхэг хүрдэг байжээ. 50. Гордон Ч. Древнейший Восток в свете новейших раскопок. М., 1956. С. 44-47.
Тэр үед Сахар нь элбэг уст голууд хаа сайгүй урссан, заан, усны үхэр, зэрлэг бух, гөрөөс, пантер, арслан, баавгай зэрэг зэрлэг амьтдаар элбэг цэцэгт тал байлаа. Одоо болтол Сахар, тэр ч байтугай Аравийн хад асгийг Homo sapiens зүйлийн орчин үеийн хүний төлөөлөгчдийн хийж бүтээсэн эдгээр амьтдын дүрс зураг чимж байдаг юм. Гэтэл циклоны чиглэл өөрчлөгдсөнтэй холбоотойгоор НТӨ YI зууны сүүлчээр Сахар аажмаар хуурайшсан нь Сахаарын эртний оршин суугчдыг зэрлэг өвсний захаар улаан буудай, арвайн “өвөг” ургадаг Нил мөрний намгархаг хөндийд анхаарлаа хандуулахад хүргэжээ. 51. Там же. С. 67.
Чулуун зэвсгийн овгууд газар тариалан эзэмшиж, улмаар хүрлийн эрин үед египетийн өвөг дээдэс Нилийн татмуудад газрыг системтэйгээр боловсруулж эхэлсэн байна. 52. Там же. С. 93.
Энэ үйл явц фараонуудын эрх мэдэл дор Египет нэгдсэнээр дууссан байна. Энэ эрх мэдэл нэгэнт хувирган өөрчилсөн ландшафтын асар их нөөц дээр суурилж байсан ба хожим нь мэдээж бидний үзлийн үүднээс гадаргуугын антропогенийн хэлбэр болсон уран барилгын суваг, далан, пирамид болон сүм хийдийг оруулахгүй бол түүнд зарчмын өөрчлөлт оруулаагүй юм. Гэхдээ бага хэмжээний өөрчлөлт, жишээлбэл, XIII гүрний үед алдарт Файюмийн баян бүрд байгуулах гэх мэт өөрчлөлт XXI гүрний үе хүртэл байсаар байсан ба дараа нь Египет гадаадын түрэмгийллийн талбар болсон юм. Нуби, ливи, ассир, перс, македон болон римчүүд Египетийн баялгийг цөлмөн авсан юм. Харин египетчүүд өөрсдөө болохоор өвөг дээдсийнхээ бүтээсэн биоценезийг шаргуу тэтгэсэн фаллахууд болон хувирсан билээ.
Үүнтэй төстэй дүр зургийг физик газар зүйн зарим нэг ялгааг эс тооцвол Месопотамид бас ажиглаж болно. Персийн булангийн адаг, Тигр, Евфрат мөрний түрцээр бий болсон энэ газар нь үржил шимтэй, салаа, булангууд нь загас, усны шувуудаар бялхаж, финикийн дал мод зэрлэг байдлаараа ургадаг байв. Хүй нэгдлийн энэхүү Эдемийг (Библид байдаг диваажингийн цэцэрлэг–Орч) эзэмшихэд маш их ажил орсон юм. 53. Там же. С. 179-180.
Тариалангийн газрыг “хуурай газраас усыг салгах” замаар бий болгох ёстой болов. Мөн намгийг хатааж, цөлийг усжуулж, гол мөрнийг далангаар хаах хэрэгтэй байлаа. Энэ ажлыг оршихуйн өөр хэрэгсэл байхгүй, жирийн тариаланч-малчид байсан шумеруудын өвөг дээдэс хийсэн юм. Энэ хүмүүс хараахан бичиг үсэг мэдэхгүй, хот байгуулж байгаагүй, сүртэй ангийн хуваагдал хэрэг дээрээ байгаагүй хэдий ч дараа дараачийнхаа үеийн гарын хөдөлмөрөөр ландшафтыг асар их ул суурьтай өөрчилсөн юм. 54. Там же. С. 191-192.
Бүдүүлэг ард түмнүүд соёл иргэншсэн ард түмнүүдийн өмнө байгал хувирган өөрчлөх талаараа давуу талтай байсан гэж бодох хэрэггүй юм. Нилийн хөндий, Евфратын хөндийг Эртний хаант улсын эрин үеийн египетийн олон тосгон цөлийн элсэнд булагдаж, шумер болон аккадын суурингууд лагт даруулах хүртэл дахин дахин хувирган өөрчилж байсан билээ. Евфратын баруун тийшх бэлчээр нутаг бүх Багдадын халифатын үед нарны туяанд хуурайшиж таслтжсан давсаар гялтганадаг байжээ. Эртний ертөнцийн Вавилон хотыг бүр Нийтийн тооллын эхэн гэхэд орон нутгийн нөөцөөр сайн сайхан, цэцэглэл хөгжлийн хорин зууны дараа оршин суугчид нь орхин явсан юм. Хятад дахь хөрс сайжруулалтын түүх бүр ч илүү сургамжтай бөгөөд энэ тухайд тодорхой ярих болно.
ЭРТНИЙ ХЯТАДАД
НТӨ III мянган жилд Хятад нь одоогийн байгаатайгаа төстэй юмаар цөөн байсан юм. Энд гол мөрнөөс тэжээл авсан унаган ой, намаг, шар усны үерээр цутгадаг уудам нуурууд, нарийн зэгс, зөвхөн өндөрлөг тэгш газарт байх нуга, талууд байжээ.
Дорно зүгт голын доод хэсгийн хоорондын бэлчирийн тэгш газруудад хэврэг хөрс үргэлжлэн байх ба Хөх мөрний доод урсгалын хөндийд И болон Хуай голууд цутгадаг байсан. “Вэйхэ голын сав газрыг өтгөн ургамал бүрхэж, тэнд аварга том царс модод сүндэрлэж, газар сайгүй агар, нарсны бүлэг модод үзэгддэг байв. Ой дотор нь бар, ирвэс, шар леопард, баавгай, одос, гахай сүлжилдэж, чоно, цөөврүүд мөнхөд ульдаг байсан” 55. Грумм-Гржимойло Г. Е. Можно ли считать китайцев автохтонами бассейнов среднего и нижнего течения Желтой реки //Изв. ВГО. 1933. Т. 65. Вып. С. 29-30.
Гэхдээ энд хүмүүсийн гол дайсан нь гол мөрнүүд байжээ. Жилийн хуурай үед тэд хүчтэй шохойжиж, гэхдээ голууд нь эргээсээ халин гарсан үед ууланд бороог өнгөрөөх ёстой болдог байв. Голууд цутгах үедээ урсгалынхаа хурдыг алдаж, хагшаас хаядгийг тооцох ёстой байлаа. Шар мөрөн гэхэд л шар усны үерийн үеэр 46 % хүртэл лаг, элс агуулдаг байв. 56. Нестерчук Ф. Я. Водное хозяйство Китая //Из истории науки и техники Китая / Отв. ред. И. В. Кузнецов и др. М., 1955. С. 6.
Бүдүүлэг газар тариаланчид үерээс талбайгаа аврахын тулд далан байгуулахад хүрсэн ч даланг нь үер 2.5 жилд нэг удаа сэтэлдэг байжээ. 57. Зайчиков В. Т. Природные богатства Китая //Изв. АН СССР. Сер. географическая. 1954. № 6.
Хятадын эртний оршин суугчдын зарим хэсэг нь догшин уснаас зугтан ууланд гарч ангаар дагнан амьдрах болсон байна, харин тэднээс ул мөр ч үлдсэнгүй. Өөр нэг хэсэг болох баруун зүгээс Шаньсид ирсэн “хар толгойт зуун гэр бүл” голтой хийх тэмцэлд орсон бөгөөд энэ бол хятадуудын өвөг дээдэс байсан юм. Тэд өмнөх зэрлэг дур зоргоосоо татгалзахад хүрч, харин сахилга бат, хатуу зохион байгуулалт, засаглалын диспот хэлбэрийг сонгон авсан юм, гэтэл байгал тэднийг өгөөмөр шагнаж, идэвхитэй өсөн үржих, өвөрмөц соёл бүтээх хэрэгслийн боломж олгосон юм. 58. Lattimore О. Inner Asian Frontier of China. New York, 1940. P. 275.
Газар тариалангийн ажлын бэрхшээл, усны гамшиг аюулаас халшран ууланд гарсан тэр хэсэг хүмүүс нь жун болон кянь–төвдүүдийн өвөг дээдэс болсон билээ. Тэд ландшафт болон гадаргуугын өөрчлөлтийг шаардахгүй олж авах байгалийн үр өгөөжөөр сэтгэл ханасан ба иймээс ч тэдэнд төрийн байгууллага бүтээх шаардлага үүсээгүй билээ. Тэдний хийх ажил, амьдралын байгууламж, эцэст нь үзэл суртал зэрэг нь хятадуудаас эрс ялгаатай байсан ба үе солигдох тутам хоёр ард түмэн салан холдсоор л байна. Энэ хямралдаан нь эртний Хятад болон түүний хөршүүдийн түүхийн чиглэлийг тодорхойлогч үл эвлэрэх дайсагналаар дууссан билээ.
Одоо ландшафтын антропогенийн өөрчлөлтийн баримтуудыг он цагийн дэс дарааллаар тавиад үзье. Байгалтай хийсэн тэмцлийн эхний шат нь НТӨ 2278 он хавьцаа болсон бөгөөд хятадын анхны Юй гүрний домогт өвөг дээдэс Шар мөрний урсгалыг тохируулах ажлыг явуулсан ба үүний дараа хойт хятадын төв хэсэг (Шаньси болон Шэньсийн хэсэг) газар тариалангийн орон болон хувирсан юм. 59. Нестерук Ф.Я. Водное хозяйство Китая. С. II.
Гол НТӨ 602 он хүртэл буюу арван зургаан зуун жилийн туршид амгалан тайван байлаа. Эртний хятадын соёлын бат цул энэхүү эрин үе нь түүхийн хувьд Ся, Шань, Чжоу гэсэн гурван гүрнийг хамарч, чингэхдээ Хятад нь тэр үеийн соёлын дээд ололт болж байсан дүрс үсгийн бичгээр өөр хоорондоо холбогдсон олон тооны ханлигуудын хамтарсан холбоо буюу конфедераци болж байсан билээ. 60. Гумилев Л. Н. Китайская хронографическая терминология в трудах Н. Я. Бичурина на фоне Всемирной истории (предисловие) //Бичурин Н. Я. Собрание сведений по исторической географии Восточной и Срединной Азии. Чебоксары. 1960. С. 648- 649.
Юй гүрний байгуулсан хиймэл ландшафт энэ бүх үеүдэд зөвхөн тэсэн тогтож байсан ба НТӨ 722 онд “Хавар, намрын” эрин үе (энэ явдал болсон үеийг сурвалж бичигт тэмдэглэсэн гарчгийн нөхцөлт нэр) болоход бүх юм өөрөөр эргэв. Ван (хаан)–гийн удирдлага дор нэгэн бүхэл болон төсөөлөгдөж байсан ханлигуудын холбоо бие даасан 124 улс болон задарч, бие биенээ чармайн идэж эхлэв. Энэ үеэр уулын жун нар болон Шар мөрний ус сөрөг довтолгоонд оржээ. Даланг муу арчилж байснаас болж, 602 онд анх удаа Шар мөрний урсгалд өөрчлөлт гарсан нь тэмдэглэгдсэн байна, түүнээс хойш XYIII зууныг хүртэл гол дээрх үндсэн ажил нь даланг тогтоон барих, сэтэрхийг хаахад орших болжээ. 61. Нестерук Ф. Я. Указ. соч. С. 19. 62. Там же. С. 22-23.
Бидний үзэж буй асуудлаар бол энэхүү үзэгдлийг оршин буй ландшафтыг тэтгэх үйл мэтээр авч үзэх ёстой, өөрөөр хэлбэл бид хятадуудыг алконкин буюу эвенкүүд шиг угсаатны тэр зэрэглэлд оруулах ёстой гэсэн туйлын хачин дүгнэлтэд хүрч байна. Гэхдээ бид анхдагч дүгнэлтээ шалгаад үзье.
НТӨ IY зуунд төмөр нь өргөн хүртэцтэй таваар болон хувирсан бөгөөд түүгээр сэлэм төдийгүй, хүрз хийх болжээ. Техникийн төгөлдөржилтийн ачаар III зуун гэхэд услалтын систем бүтээсэн бөгөөд эдгээрээс хамгийн чухал нь хойд Шэньсид 162 мянган га талбайг усалж байсан Вэйбэйн систем юм. 63. Гумилев Л. Н. Китайская хронографическая терминология. С. 65. 64. Нестерук Ф. Я. Водное хозяйство Китая. С. 51.
Услалтын энэхүү системийн ачаар “Шэньси муж нь элбэг баян, ургац алдахыг мэдэхгүй болсон юм. Тэр үед Цинь Ши Хуанди баян хүчирхэг болж, бусад ноёдыг эрх мэдэлдээ захируулж чадах байв” 65. Там же. С. 52.
Энэ бол Хятадын агуу их нэгдлийн үе байлаа, гэхдээ энэ нь ялагдагсдыг бөөнөөр хядах, амьд үлдэгсдийг нь тамлан зовоох, Хятадын Их Цагаан Хэрэмийг босгох, зөвхөн эрдэмтдийг төдийгүй техникийнхээс бусад бүх ном зохиол (энд төлөгчлөл, анагаах ухаан, агрономийн номуудыг ойлгож байжээ), мөн түүхэн болон гүн ухааны сургаалийг уншигч бүх хүн, түүнчлэн яруу найрагт дурлагсдыг устгах явдлаар төгссөн юм.
Энд бид: ландшафтыг зорилго чиглэлтэйгээр өөрчилсөн нь хүмүүсийг аймшигтайгаар хядсан явдал холбоотой байсан уу, аль эсвэл эдгээр нь зүгээр л цаг хугацааны хувьд давхцсан уу гэсэн асуулт тавьж болж байна. Бүр өөрөөр энэхүү хоёр үзэгдэл нь ерөнхий нэг шалтгаанаас үүдсэн юм бус уу? Энэ асуудлыг шийдвэрлэхийн тулд Хятадын түүх болон Вэйбэйн услалтын сүлжээний түүхийг цааш нь авч үзье.
НТӨ 206 оны бослого нь Цин гүрний дэглэмийг устгасан бөгөөд харин Хань гүрний үед ийм их цус урсгалт болоогүй юм. Улс орон баяжиж, Шар мөрний эрэг дээрх Шаньси дахь өмнөх тарианы сан дээр Вэй болон Цзин голын эрэг дээр шинэ сангууд бий болж байтал байгаль өөрийнхөө үгийг хэлж эхлэв. Усжуулах сүлжээний ус далангаар хаасан Цин голоос ирдэг байв, гэтэл голын гольдрил гүнзгийрч, ус хураах сан нь хуурай газарт гарч иржээ. Иймээс урсгалаас дээгүүр шинэ суваг татаж, шинэ далан барихад хүрсэн бөгөөд дараачийн зуунуудад энэ ажлыг арав дахин давтан хийж, асар их хөдөлмөр зарцуулсан юм. Гэлээ ч гэсэн XYII зуунд Вэйбэйгийн системийг үндсэнд нь орхисон байна. 66. Там же. С. 52-55.
Өнгөрсөн хоёр мянган жилийн туршид Хятадын дундад үеийн–түүний эзэн хаант үеийн түүх ингэж өрнөсөн билээ. Этнологийн үүднээс бол италичуудыг римд, францчуудыг галлд хамааруулдгийн адилаар энэ үеийн хятадуудыг эртний хятадуудад хамааруулж болно. Өөр үгээр хэлбэл Шар мөрний эрэг дээр өнөөдөр бид хуучин нэрээр нь нэрлэж буй шинэ ард түмэн бий болсон юм. Гэхдээ судлах зүйлд бидний нэр томъёо зүйн алдааг оруулан ирж болохгүй юм.
Тэр тусмаа “хятад” гэсэн үг нь XII зуунд жингийн худалдаа хөгжсөний улмаас бий болсон нөхцөлт нэр томъёо болно. Энэ нь тэр үед итали болон оросын худалдаачидтай харьцаж байсан монгол хэлт овгуудыг хэлдэг байжээ. Энэ овгоос Хятад гэсэн нэр нь өөрсдийгөө зүгээр л “Дундад тэгш газрын оршин суугчид” гэж нэрлэж байсан тэдний хөршид шилжин ирсэн байна. Манай шинжилгээний хувьд энэ нь маш чухал бөгөөд нийтэд тодорхой “Хятад” гэсэн үг нь эрэмбийн байрлалаараа “Франц”, “Болгар” гэдэг шиг биш, харин “Европ” буюу Левиант (Ойрхи Дорнод) мэтийн ойлголтуудтай харгалзана. Ингээд Цинь Ши Хуанди эзэн хаан Хятадыг нэгтгэсэн эрин үеэс авахуулаад Хятад бие даасан байдлаа алдах хүртэл Шар болон Хөх мөрний дундах газар нутаг дээр нөхцөлт байдлаар хойт хятад, өмнөд хятад гэж нэрлэдэг хоёр том угсаатан үүсэн бүрэлдэж, хүчээ алдаж байжээ. Өмнөд хятад нь ч гэсэн ландшафтын өөрчлөлттэй холбоотой бөгөөд учир нь эртний хятадууд (эдгээрээс дундад зууны хоёр угсаатан бүрэлдсэн) өргөн урсгалаар Хөх мөрний хөндийд цутган ирж, тэд тэнд ширэнгэ ойн оронд цагаан будааны тариалан байгуулсан юм. Хойд хятадууд ч гэсэн усалгааныхаа системийг тэтгэх эрчим хүч хангалттай байх хүртэл хуурай тал дээр усалгаат тариалан байгуулж, бие даасан ард түмэн гэдгээ баталж, дандаа азтай байгаагүй ч харийхны түрэмгийллийг няцааж байсан билээ. Гэхдээ XYII зуунд услалтын систем оршин байхаа болив, яг энэ зуунд Манжууд хятадыг эзэлсэн байдаг. Эзлэн авахаас өмнө Мин гүрний хүчин чадлыг ганхуулсан тариачдын асар том бослогууд өрнөсөн юм. Хөдөө аж ахуй нь уначихсан үед турсага болсон тариачдыг хатуу ширүүн дайнд оруулах боломжгүй байв. Үнэхээр ч элсээр дүүрсний дараа сувгууд нь битүүрсэн баруун хойд зүгийн баялаг тариалангийхаа газрыг алдсан нь Хятадын эсэргүүцлийг сулруулж, Мин гүрнийг түрэмгийлэгчдийн золиос болгосон юм.
ҮҮСЭЛ БА УНАЛТ
Одоо бид тавьсан асуултдаа хариулж болно. Газар тариалангийн ард түмнүүд хиймэл ландшафт бүтээх эрин үеүд нь харьцангуй богино хугацаатай байна. Эдгээр нь цаг хугацааны хувьд хатуу ширүүн дайнуудтай тохиолдлоор давхцдаг, гэхдээ мэдээж хэрэг газар усжуулах нь цус урсгах шалтгаан болохгүй. Ийм зүйлийг батлах нь газар зүйн шалтгаацлын чиглэлээр Монтескье –ээс ч цааш явна гэсэн үг болно. Гэхдээ энэхүү зэрэгцээ хоёр үзэгдэлд нийтлэг байх сэжим байна, энэ нь угсаатны хамт олны ер бусын хүчдэл гаргах чадвар юм. Энэ хүчдэл юунд чиглэх нь өөр хэрэг, манай асуудалд зорилгыг тооцдоггүй. Хэт хүчдэл гаргах чадвар сулрахад бүтээсэн ландшафт нь зөвхөн тэтгэгдэж байдаг, харин энэ чадвар алга болоход угсаатны ландшафт, өөрөөр хэлбэл тухайн биохорын биоценез сэргэн тогтдог. 67. Шнитников А. В. Изменчивость общей увлажненности материков северного полушария //Записки ГО. Т. XVI. М., Л., 1957. С. 220-221, 262.
Энэ нь хийсэн өөрчлөлтийн цар хэмжээ, үйл ажиллагааны бүтээгч юм уу хөнөөгч шинж чанараас үл хамааран цаг ямагт, хаа сайгүй байдаг. Хэрэв ийм байх аваас бид байгалд үзүүлж буй ард түмнүүдийн байнгын үйчлэлийн үр биш, харин ард түмнүүдийн өөрийнх нь хөгжил дэх түр зуурын төлөв байдлын, өөрөөр хэлбэл угсаатны нийлэгжилтийн сэдэл бологч тэрхүү бүтээлч үйл явцын үр дагавар болох байгалийн өөрчлөлтийн одоо болтол тооцож байгаагүй шинэ үзэгдэлтэй тулгарч байна.
Бидний энэ дүгнэлтийг эртний Европын материалууд дээр шалгаж үзье. НТӨ I болон II мянган жилийн зааг дээр Баруун Европыг кельт, латин, ахей зэрэг төмөр давтаж чаддаг, дайчин ард түмнүүд эзэлж, оршин суух болсон. Тэд газар тариалангийн жижиг жижиг олон нийтлэг байгуулж, унаган хөрсийг боловсруулж, ландшафтыг өөрчилсөн юм. Европт бараг мянган жилийн туршид том төр улс үүсээгүй, учир нь овог бүр нь өөрийгөө чадварлагаар хамгаалан гарч чаддаг, тэднийг эзлэн авах нь хүнд бөгөөд биелэгдэшгүй хэрэг байв, овог бүр захирагдахыг хүлээн зөвшөөрөхөөсөө цохихыг илүүд үздэг байсан. Спарт ч, Афин ч гэсэн Элладыг захирч чадаагүй, харин Римийн латин болон саммит дайчид дараагийн бүхий л булаан эзлэлтээсээ ч илүү хүнд зам туулсныг санахад л хангалттай юм. НТӨ I мянганы эхний хагаст хэсгүүдээ эрчимтэй боловсруулсан жижиг жижиг газар тариалан нь бүтээсэн соёлын ландшафтыг тэтгэгч институт болсон байв. НТӨ I мянганы төгсгөлд энэхүү парцелль хэмээх тархай системүүд нь байгальд хөнөөлтэй ханддаг латифундуудад шахагдаж, үүнтэй зэрэгцэн булаан эзлэх боломж үүссэн байна.
Газрын дундад тэнгис болон Баруун Европ “яагаад ч юм” хүчирхэгжихгүй байсан болохоор нь Рим эзэлсэн гэдэг бодлыг бүгд зөвшөөрдөг. Хэрэв Римийн хүч хэвээрээ байгаад түүнийг тойрсон ард түмний хүч суларсан тэр тохиолдолд ч гэсэн мөн л ийм үр дүн гарах ёстой байсан. Үнэхээр ч ийм болсон бөгөөд Римийн түрэмгийлэлтэй зэрэгцэн тал нь бэлчээр, дараа нь цөл болон хувирч, эцэст нь Y–YI зуунд жам ёсны ландшафт дахин сэргэж, ой, шугуй, бут ургасан юм. Мөн тэр үед хүн амын тоо цөөрч, Римийн эзэнт гүрэн уналтанд орсон. Ландшафт болон угсаатны нийлэгжилтийн хувирал өөрчлөлтийн бүх цикл угсаатан үүсээд тэдний бүрэн алга болох хүртэл ойролцоогоор 1500 жил болжээ.
Хүний үйл ажиллагааны сэргэлт, мөн нэгэн зэрэг дундад зууны угсаатан бүрэлдэх үйл явц IX–X зуунд болсон бөгөөд дуусаагүй байна. Магадгүй энэ үеийн онцлогийг тайлбарлахын тулд шинжлэх ухаан–техникийн урьд байгаагүй хөгжилтэй холбоо бүхий нэмэлт засвар оруулах ёстой байх, гэхдээ энэ асуудлыг онцгойлон судлах ёстой. Одоохондоо бол бидэнд дүрмээс гажсан нь биш, дүрэм нь илүү сонирхоллтой юм.
Одоо бид индианчууд болон Сибирийн ард түмнүүдийг эргэн үзье. Яагаад гэвэл бид дээр тавьсан яагаад газар тариаланчид болон анчид бие биенээсээ ахуй болон хөдөлмөрийн ашиг тустай дадал туршлагыг зээлдэлгүйгээр зэрэгцэн оршоод байдаг юм бэ гэсэн асуултдаа эцсийн эцэст хариулж болох юм. Хариулт нь өөрөө өгөгдөж байгаа юм. Аль алинийх нь өвөг дээдэс эрт үед ландшафтыг эзэмших үеийг туулсан бөгөөд түүнийг янз бүрээр өөрчилсөн, хойч үеийнхэн нь ч гэсэн өвөг дээдсийнхээ бүтээсэн энэ зиндааг хадгалж, өнгөрсөн үеүдийн өв сангаа уламжлал хэлбэрээр ариглан хамгаалж, түүнийг эвдэхийг ч хүсдэггүй, эвддэг ч үгүй амьдардаг. Англосаксуудын түрэмгийллийн үед индианчуудыг биет байдлаар нь устгах аюул байлаа ч гэсэн хэдийгээр түүнийгээ хаяж, колонистуудтай холилдон үхэхгүй байх бүхий л боломж байсаар байтал тэд өөрийнхөө аж төрөх хэв маягаа эрэлхэгээр хамгаалан зогсч байлаа.
Түүний хамт бидний дээр бүтээлч гэж тодорхойлсон төлөв байдалдаа байсан ацтекууд аймшигт ниргэлтийг тэсэн гарснаар барахгүй булаан эзлэгчдийн хэсгүүдтэй холилдох хүчийг өөрөөсөө олсон бөгөөд 300 жилийн дараа испанийн ноёрхлыг түлхэн унагаж, индианы элемент нэгдүгээр үүрэг гүйцэтгэдэг Мексикийн бүгд найрамдах улсыг байгуулсан юм. Мэдээжийн хэрэг Хуаресийн хамтрагчид Монтесумийн зүтгэлтнүүдийн хуулбар биш, гэхдээ Кортесийн цэргүүдээс бүр ч илүү өөр байсан юм. Мексикчүүд бол угсаатны нийлэгжилт нь түүхчдийн нүдэн дээр болсон залуу ард түмэн юм. XYII – XYIII зуунд бүрэлдсэн энэ ард түмэн хүнсний ургамал тариалах, Америкт байхгүй адуу, үхэр зэрэг амьтдыг нутагшуулах замаар ландшафтын шинж чанарыг нэн хүчтэй өөрчилсөн юм.
“Соёлын ландшафтыг” тэтгэдэггүй, харин байгалийн тогтворт байдалд дасан зохицогч угсаатнуудыг “зэрлэг” гэж нэрлэж заншсан нь буруу. Байгалд харилцах тэдний харилцаа идэвхигүй: тэд биоценезэд тэдний дээд, төгсгөлийн салбар байдлаар л орж байдаг. Сүүлчийн энэ бүлгийн угсаатнуудын байгалд харилцах харилцааг тооцооллын анхдагч түвшин болгон авах нь тохиромжтой. Хэрэв ийм угсаатнууд өөр угсаатны суурьшсан газар нутаг дээр ирэх юм бол тэднээр дамжин орших гэсэн тэрхүү зүйлд дасан зохицдог. Тэдний хувьд багтаагч угсаатан нь тэжээгч ландшафтын бүрдэл хэсэг болж өгдөг. Ийм зөрчилдөөн ан хийж, ургамал түүн амьдардаг индианы каражу овгийг олсон Бразильд саяхны үед болсон билээ. Кинокомпани тийш нь экспедици илгээж, зургаа авахуулсных нь төлөө индианчуудад дажгүй хөлс төлжээ. Энэхүү кино реклам олон жуулчдыг татаж, тэдэнд зориулж зочид, буудал, баарууд барьсан байна. Мөн эргэн тойронд нь үйлчилгээний газрууд, цагдаа, эмч нар зэрэг амьдрах болов. Үүний үр дүнд индианчууд үнэгүй хоол авч сураад ойн ан хийх, ургамал түүх дадал заншлаа мартжээ. Тэд өөрсдөд нь тоглоом мэт ханддаг, илүү олон тоотой, баян угсаатнуудын хүчээр амьдардаг шимэгч (паразит)-угсаатан болж хувирчээ. Гэхдээ тэд моодноос гармагц л тэднийг хувь заяаных нь эрхээр хаях бөгөөд эсвэл тэд сул тавьсан гэрийн амьтан үхдэг шиг үхнэ, эсвэл тэд зэрлэг зүйлүүдийн өрсөлдөөнийг тэсэн давж чадахгүй юм. Хувьсал буцалтгүйн тухай хууль этнологид ч бас үйлчилж байна.
ШАТААР ( ФАЗ ) НЬ ҮЕЧЛЭХ НЬ
Одоо бид өөрийнхөө ажиглалтыг нэгтгэж, тэдгээрийгээ угсаатан байгалд, өөрөөр хэлбэл ландшафтын нөхцөлд харьцах бүдүүвч байдлаар харуулж болно. Ямар нэгэн, одоогоор тодорхойгүй шалтгаанаар түүхийн тавцанд бий болсон шинэ угсаатнууд (ихэвчлэн хуучин нэрээрээ) байгалийн нөхцөлд дасан зохицох шинэ аргын тусламжтайгаар ландшафтыг хувирган өөрчилдөг байна. Энэ нь ёс мэт өгсөлтийн шатны далд үед явагдах бөгөөд түүхэн сурвалжуудад (домгоос бусад) тэмдэглэгддэггүй. Түүхэн бөгөөд сурвалжуудад тэмдэглэгдсэн эрин үеүд нь гадаад холилцоо байхгүй үед угсаатны нийлэгжилтийн: 1) өгсөлтийн шатны ил шат, 2) угсаатан дээд зэргийн идэвхитэй, ландшафтад үзүүлэх дарамт нь багассан акматик (хэт халалтын–Орч) шат, 3) антропогенийн буюу хүний хүчин зүйлийн даралт дээд зэрэг бөгөөд эвдлэгч болсон хугаралтын шат, 4) техник хэрэгсэл болон үзэл суртлын эрхэмлэлүүд хуримтлагдаж, энэ үед ландшафт нь анх орсон тэр л төлөв байдлаа тогтоон тэтгэж буй инерцийн шат, 5) соёлын тухайд ч, ландшафтын тухайд ч ямар халамж анхаарал байхгүй болсон обскурацийн буюу хөгшрөх шатуудыг багтааж байна. Үүний дараагаар үхсэн соёлын ландшафтын үлдэгдэл дээр үүссэн хагас ниргэгдсэн угсаатны үлдэгдэл үгүйрч хоосорсон ландшафтын харилцаа явагдах гомеостазийн буюу царцанги шат эхэлнэ. Энд царс модны оронд ургасан лошго, хамхуул дундуур булаан эзлэгчдийн ач нар, дээрэмчдийн хүүхдүүд тоглох болно.
Энэ эрин үед үлдэц–угсаатны байгалд харьцах харьцаа нь нэгэн зэрэг хэрэглээний бөгөөд хамгаалах шинжтэй байдаг. Харамсалтай нь хүсэл зоригийн ухамсартай шийдвэрээр биш, харин уламжлалаар чиглүүлэгдэнэ. Шинэ угсаатан ландшафтыг дахин хувирган өөрчлөх хүртэл ийм хэвээрээ л байна. Угсаатны нийлэгжилт нь нэгдмэл даяар үзэгдэл биш, харин аль нэгэн бүсүүд дэх бие даасан угсаатны нийлэгжилтийн олонлиг болно.
Байгалийн иж бүрдэл бүх үзэгдлүүдэд байгаа шиг угсаатны нийлэгжилт дэх шатуудын хил нь “шугаман бус” буюу яг таг тодорхойгүй байна. Тэдгээр нь ямар нэг хэмжээгээр “арилсан байдалтай” байдаг. Гэхдээ хил заагийн тодорхой бус зарим байдал нь тодорхой угсаатны нийлэгжилтийг шатуудын эхлэл болон төгсгөлийг нь түүхэн үеүдээр тодорхойлон санаж, гэхдээ эдгээр үеийн цаг хугацааг нөхцөлт байдлаар тэмдэглэн, нэг хэвийн эргэлтийн агшин мэтээр тодорхойлох зэргээр цаашид судлахын зайлшгүй шинжийг багасгахгүй юм.
Гэхдээ хэрэв бид угсаатныг ландшафттай харьцуулан үзэхээсээ хөндийрвөл тэдгээрийг түүхэн бүхэллэг мэтээр үзэх болно. Ингэвэл бид шатууд байнга солигдож байх тэр л дүр зургаа зөвхөн тооцооллын өөр систем дээр олж үзэх болно. Энэ нь бид зөв зам дээр байгааг харуулж байна. Ийм учраас урьдчилаад хэлэхэд цаашдаа маш их хэрэг бүхий угсаатны нийлэгжилтийн шатны бүдүүвчийг өгье. Бид одоохондоо “яагаад” биш, “яаж” гэсэн асуултанд хариулж байгаад уншигчид маань бүү дургүйцээрэй. Үзэгдлийг дүрслэн үзүүлэх нь түүний тайлбарлалаас ямагт өмнө нь байдаг. Хэрэв тайлбарлалыг урьдчилж аваагүй бол түүнээс бүх хүчээрээ зайлсхийх хэрэгтэй.
Ийнхүү эхлээд угсаатны төлөвшлийн далд үе болдог бөгөөд энэ нь түүхэнд мэдэгдэм ул мөр голдуу үлдээдэггүй юм. Энэ бол “асаагч механизм” бөгөөд тэр болгон шинэ угсаатан үүсэхэд хүргэдэггүй. Учир нь энэ нь гаднын хүчний үйл явцын гэнэтийн тасралтаар л боломжтой юм. Түүхийн тавцан дээр аль нэгэн үед өөрийн угсаатны нүүр царай, өөрийн ухамсрыг (“бид болон бид биш” буюу “бид болон бусад”) хурдан хөгжүүлж, төлөвшүүлдэг тогтоогдсон (түүхэн ёсоор) бүлэг хүмүүс буюу консорци бий болдог юм. Эцэст нь тэд зохих хугацаанд нийгмийн хэлбэрээр хүрээлүүлж, түүхийн өргөн тавцан дээр гарч ирдэг, үүнийгээ тэд газар нутгийн түрэмгийллээр эхэлдэг. Угсаатан–нийгмийн систем бүрэлдсэнээр сэргэлтийн шатны далд үеийн төгсгөл тунхаглагдана. Шинээр төлөвшсөн угсаатан эсвэл мөхнө, эсвэл, жишээлбэл рим болон византийнхтай адилхан түүхэн оршихуйн гэнэтийн нөхцөл байдлын харьцангуй удаан үеийг даван туулна. Энэ үе нь ландшафтын тохиолдолд байдаг шиг илтийн оргилох өгсөлт, оргил, хугаралт, инерцийн болон хөгшрөлтийн шатуудыг өөртөө агуулж байдаг.
Оргил буюу акматик шат нь нэн голдуу маш эрээн мяраан бөгөөд шинж чанар, угсаатны үйл явц явагдах идэвхижил, доминант буюу голлох шинж чанараараа олон янз байдаг.
Угсаатны систем (хугарах, инерци, бага хэмжээгээр зогсонги болох) хялбарших үйл явцтай холбоотой угсаатны нийлэгжилтийн шатууд нь голдуу угсаатны регенераци буюу дахин сэргэхийн эсрэг үйл явцаар байнга зөрчигдөж байдаг. Энэ тохиолдолд угсаатны хөдлөнги чанарын шинэ хэрэгцээнд нийцсэн, нийгмийн шинэчлэлийн санаачлагыг өмнө нь хөтлөгч дэд угсаатан буюу угсаатан байснаас дарагдаж байсан угсаатны тэрхүү дэд систем олж авдаг байна. Өмнөх манлайлагч байр сууриа цэвэрлэсний дараагаас л угсаатны уналтын үйл явцыг зогсоогч хүчин өөрийгөө илрүүлэн гаргаж чадна.
Он цагийн түүх бичигчдийн ажлын онцлогоос болоод угсаатны нийлэгжилтийн эцсийн, ялангуяа анхдагч шатуудыг судлах нь бүхнээс хэцүү байдаг. Хэрэв түүх бичигчид аль нэгэн хүчирхэг ард түмэн хэрхэн алга болсныг сонирхож, бүрэн төгс биш ч гэсэн өөрийн тайлбарлалаа санал болгохын оронд тэд угсаатны нийлэгжилтийн анхдагч илрэлийг ёс мэт анхаарал үл татах, шал дэмий зүйл гэж үзэн үгүйсгэсэн байдаг юм. Энэ талаар Анатоль Франс “Цагаан чулуун дээр” номынхоо “Прокурор Иудей” гэсэн алдарт өгүүллэгтээ римийн суут ухаантнуудын харилцан ярианд гайхамшигтайгаар үзүүлсэн байдаг.
Нийгмийн оргилсон хөгжилд угсаатны нийлэгжилтийн үйл явц суурь дэвсгэр болдог, эсвэл энэ хоёр хоорондоо хамааралтай байдгийг амархан ажиглаж болно. Түүхийн шинжлэх ухаан чухам энэ байнгын хамааралт байдлыг тэмдэглэдэг ба харин этнологич хүний хувьд эхлээд анализ хийж, өөрөөр хэлбэл байгалийн болон нийгмийн сэдлүүдийг ялган үзэх чухал байдаг. Үүний дараа бидний хүсч буй нэгтгэлтийг хийх боломж нэгэнт бүрдэнэ. Гэхдээ энэ зорилгодоо хүрэхийн өмнө бид ардаа үлдээснээс ч бүр ч илүү хүнд, түвэгтэй өөр нэг саадыг даван туулах хэрэгтэй болдог. Тодорхой орнууд дахь цаг агаарын өөрчлөлт нь хэд хэдэн зуун жилээр тоологдох түүхэн цаг хугацаанд явагддаг, мөн эдгээр орнуудын ландшафт мэдээж хэрэг өөрчлөгдөнө, гэхдээ ямагт тэдний аж ахуйд, тэр тусмаа угсаатны өөрийнх нь амьдралд тусгалаа олж байдаг. Ингээд байгалийн нөхцлийн энэхүү хөдлөнги шинж нь шинэ угсаатан бүрэлдэх шалтгаан биш гэж үү ? Энэхүү шийдэл нь олон нарийн ширийн асуудлуудыг зүгээр л амархан авч хаяж байгаагаараа сонирхол татам байна. Гэхдээ юу л бол ?
Хүн төрөлхтөн өөрийг нь хүрээлэн буй байгалиас, нарийн яривал газар зүйн орчноос хамааралтайн талаар хэзээ ч маргалдаж байгаагүй. Харин энэхүү хамаарлын зэрэг, хэмжээг л олон янзийн эрдэмтэд олон янзаар үнэлдэг. Харин ямар ч тохиолдолд Дэлхийд амьдрагч, амьдруулагч ард түмнүүдийн аж ахуйн амьдрал ландшафт болон оршин буй нутаг газрынхаа цаг агаартай нягт холбоотой байсан.
Ийм байя гэж үзье, гэхдээ л энэ шийдвэрийг бүрэн гэж үзэж болохгүй, учир нь энэ нь “өвчтэй” хоёр асуултад хариу өгөхгүй байна : 1) Хүмүүс “антропогенийн ландшафт” бүтээж, байгалийн нөхцлийг өөрийнхөө хэрэгцээнд тохируулж сурчээ, ингэснээрээ тэд өөрсдийнхөө хувьд хүсмээргүй өөрчлөлтийн эсрэг үйлчлэлд орж байна. Тэгвэл яагаад бидний “соёл иргэншил” гэж нэрлэдэг олон угсаатан аж ахуйн системтэйгээ хамт мөхөөд байдаг юм бэ ? Тэдгээр нь түүхчдийн нүдэн дээр хүртэл мөхөж байна шүү дээ. 2) Цаг агаарын хэлбэлзэл, түүнтэй холбогдсон үйл явцууд нь байгаа тэр зүйл дээр, өөрөөр хэлбэл нэгэнт оршин буй угсаатнууд дээр үйлчилж байж болно. Эдгээр нь бүхэл бүтэн хүй элгэнээр нь, жишээлбэл, НТӨ XXIY зуунд Тигр, Евфратын доод хөндийд болсон шиг сүйтгэж чадна. Байгалийн энэ үзэгдлийг вавилоны “Энума Элиш” хэмээх яруу найраг, эртний еврейн “Ахуйн ном”–д бичсэн агаад тэдгээрийн он цаг нь давхцаж байдаг. Тэдгээр нь XYI–XYII зуунуудад монголчуудад тохиолдсон шиг хүмүүсийг төрөлх нутгаа орхин, харь газар хоргодох газар хайхад хүргэж чадна. 68. Грумм-Гржимайло Г. Е. Рост пустынь и гибель пастбищных угодий и культурных земель… С. 437.
Гэхдээ эдгээр нь хараахан байхгүй байгаа юмны эсрэг хүчгүй юм. Тэд шинэ хиймэл ландшафт бүтээж чадах шинэ угсаатныг бүтээж чадахгүй. Ингээд манай зорилт хэсэгчлэн шийдэгдэж байна. Бид шинэ хөгжих газрыг яаж биш, хэн бүтээдэг гэсэн зүйл рүүгээ эргэн очих ёстой. Ингэснээрээ бид угсаатан үүсэх нууцад ойртох болно.
Энд бидний өмнө ахиад л бэрхшээл үүсч байна. Хэрэв соёл иргэншлийн эхлэл болон төгсгөл нь тодорхой байгаа юм бол угсаатны нийлэгжилтийн анхдагч цэг нь хаана байна вэ ? Далд үе байдаг гэдгийг харгалзаад анхдагчаа байг гэхэд ядаж судлан буй бүх үйл явцад адилхан байдаг, тооцоолол хөтөлж болох зүйл баймаар. Өөрөөр бол янз бүрийн угсаатны нийлэгжилтийг харьцуулах нь утгагүй юм.
Гэхдээ л энэ зорилтыг шийдвэрлэж болно, шинэ угсаатнууд нь хуучин угсаатан задрах замаар биш, харин нэгэнт оршин буй нь нийлэх замаар, өөрөөр хэлбэл угсаатны субстрат буюу сууриас үүсдэг. Эдгээр угсаатны бүлэглэлүүд газар зүйн хатуу зурагдсан бүс нутгуудад, богино хугацаанд үүсдэг бөгөөд харин бүс нутгууд нь тухай бүр өөрчлөгдөн газрын нөхцлийн, өөрөөр хэлбэл Э.Симпсоны тодорхойлсон: “Хүн бол дэлхийн гадаргуугын бүтээгдэхүүн” гэж тодорхойлсон газар зүйн детеминизмийн үйлчлэлийг үгүйсгэдэг байна. Зөвхөн энэ ч биш юм. Гаригийн гадаргууд хааяа л хүрдэг сансрын цацраг, нарны идэвхжил Дэлхийд нөлөөлдөг нь тодорхой бөгөөд мэдэгдсэн зүйл юм. Гэхдээ эргэлзээ тооцохоо больж, үзэгдлээ дүрслэхэд шилжие. 69. Исаченко А. Г. Детерминизм и индетерминизм в зарубежной географии // Вестник ЛГУ. 1971. № 24. С. 90. 70. Ермолаев М. М. О границах и структуре географического пространства //Изв. ВГО. 1969. № 5. С. 401- 427.
XVII. Угсаатны нийлэгжилтийн тэсрэлт
НТ I ЗУУНЫ УГСААТНЫ НИЙЛЭГЖИЛТИЙН ТЭСРЭЛТ
Хэрэв угсаатнууд “нийгмийн категори” байсан бол тэдгээр нь нийгмийн төстэй нөхцөлд үүсэх байлаа. Гэтэл үнэн хэрэг дээрээ одоо үзүүлэх гэж байгаачлан угсаатны нийлэгжилтийн асаах, ажиллуулах механизмуудыг маш нарийн баримтат материалууд дээр хөөгөөд үзэхлээр цаг хугацааны хувьд давхцаж, эсвэл уртраг дагуу, эсвэл өргөрөг дагуу, эсвэл тэдгээр лүү татсан өнцөгт таарах бүс нутгуудад байрлаж, гэхдээ нэлэнхүй шугамын дагууд байжээ. Ландшафтын шинж чанар, тийм бүс дэх хүн амын ажил хэргээс үл хамааран ийм бүсэд тодорхой эрин үед угсаатны өөрчлөн байгуулалт–суурь, өөрөөр хэлбэл хуучин угсаатнаас шинэ угсаатан нэмэгдэх үйл явц болдог байна. Энэ үед хуучин нь эвдрэн задарч, харин шинэ нь нэн идэвхитэй хөгждөг байна.
Харин ийм бүсийн зэргэлдээ хаана ч, юу ч болоогүй мэт тайван байдал оршдог аж. Өөрийгөө тайвшруулсан угсаатнууд тайван бус хөршүүдийнхээ золиос болох нь жам ёсны бөлгөө. Энд угсаатны нийлэгжилтийн эхлэлийн бүсийн байрлалд хаанаас ийм их онцгой шинж гардаг вэ ? Яагаад тухай бүр үйл явц нь шинэ газарт эхэлдэг вэ ? гэсэн өөр асуулт ойлгомжгүй үлдэж байна. Хэн нэг нь дэлхийн бөмбөрцгийг ташуураар ороолгож, сорви руу нь цус юүлж, тэр нь авалцан шатаж байгаа шиг л байна.
Тавьсан асуултад хариулахаасаа өмнө үзэгдлийн тайлбарлалыг түүний дүрслэлтэй нийцүүлэхийн тулд энэ нь хэрхэн болдгийг авч үзье.
I зуунд Римийн болон Парфяны эзэнт гүрнүүд угсаатны ядмаг байдалд байв. Хүн амын тоо цөөрч, сайн үйл мартагдаж, өмнө нь өргөн дэлгэрсэн соёл явцуу мэргэжилтнүүдийн хэрэг болон хувирах болов. Энэ үеэс эхлэн эдийн засаг байгалийн баялагт хөнөөлтэй харьцаанд тулгуурлан тогтож, хагалах газрын талбай багасав. Иргэний дайнуудад хүлээсэн гамшигт алдагдлын дараа чадварлаг түшмэл, офицерүүд хүрэлцэхээ больж, харин люмпен пролетаруудын тоо ихсэв. Архидалт болон завхайрал Римийн ахуйн хэм хэмжээ болов. Дээр дурдсан үзэгдлүүд нь бидний хөгшрөлт (обскураци) гэж зориглон нэрлэж буй угсаатны нийлэгжилтийн шатны элементүүдийн мөн чанар болно.
Хуучны дайчин гавъяагаа алдаж, доош орсон германы болон сарматын овгуудын байдал үүнээс дээрдэх юмгүй байв. Германикууд ямар ч хөдөлмөр гаргалгүй дайсны нутгийг Рейнээс Эльба хөндлөн гарч, Британийг мөн л гайхмаар амархан эзлэн авав. Энэ нь тэр тусмаа гайхалтай бөгөөд НТӨ III зуунд угсаатны түрэмгийллийн санаачлага баруун талд кельтүүдэд, зүүн талд сарматуудад хамаардаг байсан юм. Цезарийн Галлийд, Помпейн Сирид, Марз Антонийн Парпийд, Клавдийн Британид хийсэн байлдааны компанийн ерөнхий явц, нарийн хэсгүүдийг судлаад үзэхэд Римийн бүргэдийн ялалтыг эсэргүүцэл онцгой сул байсан газруудад дагалдаж байсныг бид олж үзнэ. Парфи ядуу орон байсан, Аршакидын улс Иранд нэр төртэй байгаагүй. Иймээс ч түүнийг “туран” гэж нэрлэдэг. Гэсэн хэдий ч тэр нь Евфратын эргийн хилийг барьж байлаа.
Римийн легионерүүд НТӨ 36 онд Таласын дэргэд хятадын харваачидтай тулгарахад хятадууд нэг ч цэргээ алдалгүй римчүүдийг ялсан байна. Ийм учраас бүдүүлэгчүүд римчүүдээс хурдан суларч байсан учраас л римчүүд бүдүүлэгчүүдийг ялж байжээ гэж дүгнэж болно. 71. Гумилев Л. Н. Хунну. С. 171-173-ийг үз.
Гэхдээ II зуунаас бүх нийтийн сулралын үйл явц зөрчигдөв. Дорнод өргөргийн 20° болон 40°-ын хоорондын өргөн бүсэд тэр болтол зүгээр л байсан ард түмнүүдийн идэвхитэй үйл ажиллагаа эхлэв. Хамгийн эхлээд даки нар хөдлөв, гэхдээ азгүй явдал болж, тэд Траяны легиончуудад бут ниргүүлэв. Дараа нь хэт өндөр идэвхжлийг иллирийчүүд гаргав, тэд римийн армийг шаргуу нөхөн дүүргэж, цезарийн суудал дээр өөрийнхөө хойд зүгийн жолоологчийг тавив. Бараг бүр III зууны туршид энэ жижигхэн ард түмэн Римийн эзэнт гүрнийг ноёрхож байсан ч хэт хүчдэлээсээ болон задарч, тэдний удам дээрэмчин–арнаутууд болон хувирсан юм. Харин готууд илүү их азтай байж, Вислагийн эхээс Хар далайн эрэг хүртэлх асар том нутгийг хурдан эзэлж, Эгийн тэнгисийн эрэг хүртэл дайралт хийх болов. Тэдэнд хүнчүүд өгсөн цохилтын дараа ч гэсэн готууд Итали, Испани, богино хугацаанд Константинополийн Влахерн ордныг эзлэх хүчийг олж чадсан билээ. Цуст дайралтын хувь заяаг готуудтай вандалууд болон антууд хуваалцсан юм. II–III зуунуудад дорнод германы овгууд хэт хүчдэл гаргах чадвартай байсан нь хүнчүүдийн жижигхэн ордод эзлэгдсэн баруун герман, сармат-алануудын идэвхигүй байдлаас эрс ялгаран харагддаг юм.
Гэхдээ хамгийн чухал үйл явдал нь өөрсдийгөө “христиан” гэж нэрлэсэн шинэ угсаатан бий болсон явдал байлаа. Энэ угсаатан зарчмын хувьд гарал үүсэл, хэл, газар нутгийн талаар нэгдмэл байж чадахгүй байсан авч “Бүдүүлэгчүүд болон скифүүд, эллинчүүд болон иудей гэж бий” гэж хэлдэг байв. Шашинд нэн тэвчээртэй хандах явдлыг тогтоосон Римийн эзэнт гүрний системд христианууд ганцаараа тийм бус байсан. Мэдээжийн хэрэг үүний шалтгаан нь тэрхүү 325 онд тогтоогдоогүй байсан догматууд биш, эзэн хаад мөшгөлтөөс зайлсхийхийг эрмэлзэн тусгай зарлигаар христианчуудыг ховлосон бичгийг хүлээн авахыг хориглож байсан учраас засгийн газрын хядлага биш, ангийн ялгаа бас л биш юм. Учир нь бүх ангийн хүмүүс христианчууд болж, бусад бүх хүмүүст христианы “зан араншин харш” гэдгийг хурц мэдрэх болсон байна. I–III зуунд хүн бүр христиан болдоггүй, зөвхөн өөрийгөө “энэ ертөнцөд” харш гэж үздэг тэр хүмүүс, мөн нийтлэгтээ өөрийн гэгдэх хүмүүс л болдог байжээ. IY зуунд тэд зонхилж хараахан эхлээгүй байхад ийм хүмүүсийн тоо бүх үед ихэссээр байжээ. Тэгэхэд Рим нь Византи болон хувирсан юм.
Евангелийн номлолд юу ч хэлсэн байж болох боловч угсаатны нийлэгжилтэд эртний христианууд шинэ угсаатан бүрдэхэд зайлшгүй байх тэр бүх чанаруудыг үзүүлж байсан бөгөөд эдгээрийг зорилго чиглэлтэй, хэт хүчдэл гаргах чадвартай гэсэн хоёр шинжид нэгтгэж болно. I зууны түлхэлтийн инерцид бүтэн хагас мянган жил хангалттай байсан ба энэ хугацаанд Византи нь түүхэн үеийн бүх шат, мөн буурах шатыг дамжиж, үүний дараа фанариотууд үлдэц угсаатан болон хувирч, харин бусад византийчууд турк болон славянуудтай уусан нэгдсэн юм. 72. 1453 онд туркүүд Константинополь-ийг эзлэх үед Стамбулын Фанар дүүрэгт оршин сууж байгаад эсэн мэнд үлдсэн византийчүүдийн хойч үеийнхэн 1821 онд гарсан грекүүдийн бослогын үеэр Морее-д болсон мусульмануудыг яргалсны өшөөнд яргалагдсан юм.
Бидний зураглан буй зурвас бүсийн дорнод хязгаарт III зуун гэхэд перс хэмээх хуучин нэрээр нэрлэгдсэн шинэ ард түмэн өөрийгөө таниулж эхлэв. Эртний персүүдэд тэд италичууд римчүүдэд, эсвэл орчин үеийн грекүүд эллинчүүдэд ханддаг шиг хандаж байлаа. Ахеменидын хаант улс Ойрхи Дорнодын соёл, нийгэм, угсаатны сонгодог хөгжлийн урт удаан үеийн түүхэн төгсгөл байсан юм. Македоны довтолгоо нь энэхүү уламжлалын шулуун шугаман хөгжлийг тасалсан бөгөөд Ираныг Селевкидүүдээс чөлөөлсөн парфянууд орон нутгийн хүн амын хувьд бас л булаан эзлэгчид, “харийнхан” байлаа. 226 онд персүүд олон шашин, овгийн зарчмуудыг маш ухаантайгаар нэгтгэхэд үндэслэсэн өөрийн улс, өөрийн өвөрмөц угсаатан соёлын бүрдлийг бүтээж чадсан юм. Албан ёсны үзэл суртал болсон зороастризм христианаас ялгаатай нь прозелитизмд харш байв, гэхдээ энэ зайг ираны болон өөрчлөгдсөн манихейн болон гностик систем нөхөж байв. Византаас ялгаатай нь персийн өсөн үржихүйг гаднын довтолгооноос эхлээд арабууд, дараа нь сельжукудын довтолгоо зөрчиж байжээ. Сүүлчийн жинхэнэ перс улс–Саманидийн хаант улс 999 онд унаж, үүний дараа ираны соёлын уламжлал аажмаар алга болж, харин перс угсаатнууд мусульманы соёл гэж нэрлэгдэх системд орсон бөгөөд эртний зан үйлийн тогтсон үзлээсээ зөвхөн нэр болон ахуйн зарим шинжээ хадгалан дахин хэлбэршсэн юм.
Эцэст баруун хязгаар болох Ютландад мөн л дээр дурдсан сэргэлтэд автсан англо ард түмэн жаахан хожимдон байж өөрийгөө тодруулан Y зуунд Британид нэвтрэн оров. Генгист, Горз нарын цөөн хүнтэй бүлэглэл гэв гэнэт энэхүү шигүү хүн амтай баян орноос илүү хүчтэй болсныг ойлгоход бэрх юм. Эдийн засаг болон техникийн хувьд сакс болон англичууд нь романжсан бриттудээс сул байсан, гэхдээ тэд угсаатны хувьд залуу байсан ба насны эрчим нь кельтүүдтэй хийсэн тэгш бус тэмцэлд давуу тал олох боломж тэдэнд өгчээ. Британийн хоцрогдсон нутгууд л үүнээс өөр байсан бөгөөд тэнд буй кельтийн хүн ам хуучны дайчин чадвараа алдалгүй, түүнийгээ харийнхныг цохиход хэрэглэж чадсан байна. (Уэльс, Корнуэльс, Шотланд )
III–Y ЗУУНЫ ХҮНЧҮҮД
Европ дахь ард түмнүүдийн Агуу их нүүдэл Ижилийн чанадаас нүүдлийн хүнчүүдийн тэдэн рүү хийсэн довтолгооны улмаас болсон гэж үзэл бодол нэн тархмал. Гэхдээ үйл явдлын он цагтай танилцахад энэ үзэл бодол бүрэн няцаагддаг юм.
Хүнчүүд бол НТӨ IY зуунаас өмнө орчин үеийн монголын нутагт үүссэн нүүдлийн гүрэн юм. Ангит нийгмээс өмнөх үед байсан түрэг хэлт хүн нар “ард түмнийг ноёрхоход” үндэслэгдсэн гүрэн байгуулсан юм. НТӨ 209 оноос эхлэн НТӨ 97 он хүртэл Хүн гүрэн өсөж, хүчирхэг Хятадын шилдэг хүчийг бутниргэн, үүнийхээ дараа ялсан Хүн нар тасралтгүй суларч, харин бут цохигдсон Хятадууд тулалдалгүйгээр нөхцөлд байдлын эзэн болсон, өөрөөр хэлбэл ялалт нь хүн нарт ашиг болсонгүй.
НТ I зуунд хүнчүүд Хятадын эрх мэдлээс чөлөөлөгдсөн боловч дөрвөн мөчирт хуваагдаж, тэдний хамгийн зоригтой, эрх чөлөөнд дуртай хэсэг нь тал талаас нь отсон дайснуудыг няцааж, 155–158 онуудад Их Талын баруун хэсэгт нуугдан, Ижил– Урал хоёр мөрний угруудтай холилдон 200 жилийн дотор дорнод европын угсаатан болон хувирсан юм. Нэр томъёоны будлиан гарахаар зайлсхийж энэ хэсгийг “гунн” гэж нэрлэдэг болсон. 73. Иностранцев К. А, Хунну и гунны //Труды туркологического семинария. Т. I. Л., 1926.
III–IY зууны хооронд гунчууд “эцэс төгсгөлгүй дайнаар цуцааж” алануудыг ялсан, ингээд зөвхөн Y зуунд Карпатыг давж, Дунайн хөндийд ирсэн юм. Чингэхдээ тэдний хэсэг–акацирууд Дон болон Ижилийн төрөлх тал нутагтаа үлдсэн байдаг. 74. Иордан. Происхождение и деяния гетов//Пер. Б.Ч. Скржинской. М., 1961. С. 91. Под “гетами” подразумеваются готы.
Ийнхүү гунчуудын идэвхи бидний бичсэн идэвхийн тэсрэлтээс гэхээсээ илүү гурван зууны дараа ч байсаар л байжээ, Азиас бөөнөөрөө шилжсэн явдал бас л байсангүй, харин туршлагатай удирдагчид, дипломатын гайхалтай ухаалаг бодлого, стратеги л байлаа. Гунчуудтай харьцуулахад Готууд хөнгөн хийсвэр, хүүхэд шиг гэнэн хүмүүс байсан аж. Ийм учраас тэд дайнд ялагдаж, Хар тэнгисийн дэргэдэх сайхан улс орноо алджээ.
Хар тэнгисийн орчмын тал нутаг нь II–IY зууны үед Константинополийн хувьд талхны хоёрдахь (Египетийн дараа) эх булаг байлаа. Энэ нь аланы тал нутаг болон голын хөндийгөөр газар тариалан эрхэлж байсан хэрэг юм. Гунчууд Доныг гаталж, 371 онд алануудын ниргээд IY зууны төгсгөлд росомонуудын тусламжтайгаар готуудыг ялж, 420 оны орчим Паннонийг эзлэв. Эндээс гунчуудын орд бүх өмнөд тал нутгийг авахад хагас зуун жил ч болсонгүй. Энэ үедээ гунчууд тоогоор цөөн, харин дагаар орсон алан, ромосон, ант, остгот болон орон нутгийн бусад овгийн гараар тэд зэвсэглэдэг байлаа. 75. Гумилев Л. Н. Некоторые вопросы истории //Вестник древней истории. 1960. № 4.
Хэрэв Дорнод Европын бүх оршин суугчид бут цохигдсон байсан бол Римийн эзэнт гүрэн, Ирантай дайн хийх хүмүүсээ гунчууд хаанаас авах юм бэ ? Газар тариалангийн суурин аж ахуйг гунчууд сүйтгэсэн нь үнэн, гэхдээ эндээс Терек болон дунд Доны ойт хөндийн эсвэл Ижил мөрний бэлчрийн зэгслэг ургамалт нутгийн оршин суугчид нүүдэлчдийн богино хугацааны хөдөлгөөнийг өөрийнхөө нуувчид хүлээж суусан гэж үзэж болохгүй юм. Тэр тусмаа тэд газар тариалан эрхэлдэггүй, харин анчид, загасчид байсан юм. Тэр ч байтугай Хойд Кавказ, Доны талын аланууд X зуун хүртэл амьдарч байсан. Энэ нь яг тэр үед Төв Европт угсаатны идэвхитэй үйл явц явагдаж байхад Дорнод Европ тогтвортой байсныг илтгэж байна.
Гунчуудын амжилт тал нутгийн түр зуурын хуурайшлын оргил цэгтэй давхцаж аланы газар тариаланг сүйтгэж, ингэснээрээ аланы цэргийн хүчийг сулруулж байсныг тэмдэглэх нь чухал юм. Хуурай нөхцөлд дасаж сурсан гунчууд ганд бага нэрвэгдсэн нь тэдний 160–370 онуудад шийдвэрлэх амжилтгүй явуулж байсан дайнд ялахыг нь бүрдүүлж өгсөн юм. 76. Гумилев Л. Н. Истоки ритма кочевой культуры//Народы Азии и Африки. 1968. № 3.
Гэхдээ хуурайшлын хугацаа дуусмагц гунчуудын давуу тал ч бас дуусчээ. YI зуунд тал нутагт хүчний хуучин харьцаа эргэн тогтож, гэхдээ гунчуудын орыг болгарууд, алануудын орыг хазарууд эзлэв.
Эцэст нь хамгийн гол зүйлийг хэлье. Гунчууд, мөн азийн Хүнчүүд нь залуу ард түмэн биш байлаа. Тэдний түүх нь НТӨ 209 онд эцгээ алах замаар засгийн эрх авсан их жолоодогч Модунгийн их өөрчлөлтөөс дараалан үргэлжилсэн байдаг.
Одоо харьцуулсан аргад хандаж үзье. Хүн гүрэн НТӨ 209 онд үүссэн цагаасаа 48 онд самуурах хүртлээ нийт 257 жил оршин байжээ. Каролингийн эзэнт гүрний хэлтэрхий дээр үүссэн Франц 843 онд үүсчээ. 1100 жил (843 нэмэх нь 257)–энэ бол хамгийн харанхуй феодалын эрин үе байлаа. Энэ хугацаанд хүнчүүд соёлын хувьд францчуудаас илүү их зүйлийг хийж бүтээсэн юм. 77. Гумилев Л. Н. Хунну.
Овгийн гүрэн Хүн нь ангиас өмнөх нийгмийн хүчирхэг байгууллага бүтээсэн дэлхийн түүхэн дэх цор ганц тохиолдол биш юм. Өргөн далайцтай цэргийн үйлдвэрлэл нь хүчний зохицуулалтгүйгээр сэтгэшгүй юм. Бид НТӨ I мянганд кельтүүдийн хийсэн агуу их аян дайн, НТӨ II мянганд Энэтхэгийг арийчууд эзэлсэн, эцгийн эрхт боолчлолын институт үүсэхээс бүр өмнө XI зуунд Ана–уакад Нагуа улс үүссэн, XIX зуунд Өмнөд Африк дахь Амазулу гүрэн үүссэн, мөн түүнчлэн түүнтэй маш төстэй эртний түрэг угсаатан , цааш нь XII зууны Чингисийн өмнөх монголчуудын тухай мэднэ. 78. Кинжалов P., Белов А. Падение Теночтитлана. Л., 1956. С. 130-136. 79. Гумилев Л. Н. Древние тюрки. 80. Гумилев Л. Н. Поиски вымышленного царства.
VI ЗУУНЫ УГСААТНЫ НИЙЛЭГЖИЛТИЙН ТЭСРЭЛТ
Шинж чанар, үр дүнгээрээ дээрхтэй адил үйл явдал YII зуунд Төв Аравийд болжээ. Зөнч мэргэн Мухаммедийг тойрон хуучны овгийн харилцааг эвдэж, зан үйлийн шинэ тогтсон үзэл бүтээсэн дайчин залгамжлагчдын нийтлэг үүсчээ. Тархай бутархай бедуин болон йеменчүүд хүртэл араб гэсэн шинэ этноним (угсаатны оноосон нэр–Орч ) авч байжээ. Харин тэр үед тэдэнтэй хөрш байсан перс, сири, египет зэрэг ард түмнүүдэд идэвхийн ийм сэргэлт байсангүй.
Мөн тэр өргөрөгт мөн тэр үед Инд мөрний хөндийд орон нутгийн болон нүүн ирсэн угсаатны элементүүдийн хольцийн удам ражпут хэмээх шинэ ард түмэн бүрэлджээ. Ражпутууд Гуптагийн дарангуй залгамжлагчид (улс орон таслагдсаны дараа), буддын нийгэмлэг, хуучин дэг журмыг дэмжсэн бүхнийг устгасан байна. 81. Ражпут нар бол Индийн уугуулуудтай сак, кушан, эфталиф нар холилдсон гэр бүлийн үр удам юм. (см.: Синха Н. К., Бенерджи А. Ч. Указ. соч. С. 114). 82. Grousset R. Histoire de l’Extreme Orient. Paris, 1929. P. 125. 83. История Индии в Средние века /Под ред. Л. Б. Алаева. М., 1968. С. 76-83.
Балгасан дээр нь тэд индуист хааны хязгааргүй засаг, угсаатны нийлэгжилтийн инерцийн шулуун шугаман шинжийг зөрчигч, мусульманы довтолгоонд няцаалт өгч чадахгүй учраас л маш бага эрх мэдэлтэй жижиг ханлигуудын системийг бүтээжээ. Гэхдээ бидний хувьд судлан буй системийн улс төрийн амжилт нь чухал биш, харин ражпут нарт асар их хэмжээгээр тус болж байсан хэт хүчдэл гаргах чадвар зэрэг угсаатны нийлэгжилтийн шинжүүд байгаа нь чухал юм. Тодорхой утгаараа энэ нь ч гэсэн XYI зуунд тэдний ялагдлыг тодорхойлсон билээ. Учир нь хан толгойтой болгон ганц ганцаараа мусульманчуудтай тулалдаанд орж үхэцгээх болов, гэхдээ тэд хөршийнхөө тэргүүлэхийг хүлээн зөвшөөрөөгүй юм. Угсаатны идэвхитэй гадаад бодлогод хэт хүчдэл гаргах чадварын дээд биш, харин дунд эрэмбийн тархалт л хамгийн зохистой байдаг аж. Учир нь ийм байхад л хүчээ нэгтгэж, үйлдлээ зохицуулах боломжтой юм. Угсаатны хамт олны дотор цаашид хүчдэл сулрах үед удирдахад амархан болдог, харин гадаад үйлчлэлийг эсэргүүцэх хүч багасдаг байна. Иймээс дайчин арийчуудын удам–бенгалийн индусуудыг англичууд колонийнхоо цэрэгт элсүүлж байгаагүй юм. Учир нь тэд байлдах чадвартай цэргүүд болоход хэтэрхий их дуулгавартай байсан юм. Нийгмийн аль нэгэн бүлэг дэмжээгүй нь Ост–Энэтхэгийн компани Энэтхэгийг эзлэхэд хүргэжээ. Харин хамгийн олон тооны энэтхэг угсаатнуудын идэвхигүй байдал нь эх орноосоо дарлалыг зайлуулахыг хүсч байсан тэрхүү эрч хүчтэй ражууд болон султануудын гарыг хүлж байсан юм.
Цааш нь энэ үед дорно зүгт энэ болтол төвдийн уулаар тархай бутархай байсан Хойд Төвдийн овгуудыг шууд бөгөөд маш хурдан эзэлж нэгтгэсэн Төвдийн ард түмэн бүрэлджээ. Булаан эзлэгчдийн өвөг дээдэс нь өөрийнхөө орчинд Y зуунд Хэси-гээс хөөгдсөн зарим тооны сүмбэ нар, непалийн уулынхныг оруулж, YI зуунд угсаатны утгаараа холимог хүн ам бүрдүүлсэн, Цангло голын (Брахмапутра) дунд хавийн жижиг аймаг байсан юм. Эд нар Дорнод азид ноёрхох талаар Хятадтай маргалдаж байсан YII – IX зууны алдарт Төвдийн эзэнт гүрнийг бүтээсэн юм. 84. Гумилев Л. Н. Величие и падение древнего Тибета //Страны и народы Востока /Под ред. Д.А. Ольдерогге. Вып. 8. М., 1969. С. 156-157. 85. Гумилев Л. Н. 1) Хунны в Китае. С. 234-235; 2) Древние тюрки. С. 10
Эцэст нь баруун Хятадад мөн энэ үед бүдүүлэг Вэй эзэнт гүрнийг онхолдуулсан угсаатны хүчирхэг тэсрэлт болж байв. Үүний үр дүнд дундад зууны хятадын ард түмэн бүрэлдсэн бөгөөд энэхүү бие даасан эзэнт гүрний түүхэн уламжлалыг XYII зууны манжийн булаан эзлэлт тасалсан юм.
Угсаатны нийлэгжилтийн дурдсан үзэгдлүүд нь дуу дуугаа авалцсан төдийгүй, тэнхлэг нь Мекка болон Чанъань хоёрыг холбосон шугам болох нэг л зурвас бүсэд байрлаж байв. Энэ тэнхлэг цаашаа Дорно зүг мөн л угсаатны нягтрал болж байсан өмнөд Японыг дайран гарч, цаашаагаа Номхон далайд алдардаг юм. Үүнийг баруун зүгт үргэлжлүүлбэл, тэр нь огт хүнгүй Ливийн цөлийг дамжин, гэхдээ энэ үед угсаатны үйл явц тэмдэглэгдээгүй баруун Суданд хүрдэг. Үнэхээр хачирхалтай байгаа биз ?
XI ЗУУНЫ УГСААТНЫ НИЙЛЭГЖИЛТИЙН ТЭСРЭЛТ
Угсаатан үүсэх идэвхитэй үйл явц явагдан буй дэлхийн гадаргуугын шугаман хэсгүүд дэлхийн бөмбөрцгийг бүхэлд нь хамардаггүй, харин түүний мурийсалтаар хязгаарлагдаж байдаг ба энэ нь сургуулийн глобус дээр гэрлийн багц тусч, гэхдээ гэрлийн эх булаг руу харсан зөвхөн тэр талыг нь гэрэлтүүлдэг мэт сэтгэгдэл төрүүлж байна. Энэхүү адилтгал нь үзүүлэх л төдий юм. Түүний агуулгын тал нь юу юу болох тухайд өмнө ярьсан, харин одоо бусдыг нь (Европ болон Төв Америк) хожим гэж үлдээгээд бас нэг тохиолдлыг харгалзан үзье. Энэ нь дасаж дадсан, гэхдээ бүрэн тайлбар өгөхгүй өөр зүйлийг арилгахын тулд юм.
XII зуунд Дорнод азид нэгэн зэрэг хоёр хүчирхэг үндэстэн, мөн залуу насандаа мөхсөн нэг жижиг овог бүрэлдэв. Энэ удаад угсаатны нийлэгжилтийн талбар газар зүйн хувьд нарийн зурагдсан учраас ландшафт, нийгэм, соёлын зэрэг газар дэлхийн нөхцөл байдалд хамаатах юмгүй байв.
XII зууны эх хүртэл Амар мөрөн болон түүний цутгалангийн эргээр нийгмийн бүдүүлэг шинж, овгийн тархай бутархай байдал, өөрийн өмнөдийн догшин хөрш –киданы Ляо улсаас өөрийгөө хамгаалах чадваргүй зэргээрээ гомеостазийн буюу царцанги шатандаа яваа тунгус хэлт зүрчид нар оршин сууж байв. Зүрчдүүд киданы эзэн хаанд анд сургасан шонхороор өргөл барьж, түүнчлэн цэргийн алба хаах нөөц цэргүүд нийлүүлдэг байв.
Мөн ийм байдалд Байгал нуурын чанадын зүүнээ байх нийлүүлээд эсвэл “цзубу”, эсвэл “да–дань”–татаар гэж нэрлэдэг талын овгууд оршин байв. Соёлжсон киданууд тэднийг XIX зуунд хойт америкийн колонистууд хээр талын индианчуудтай харьцсан шиг яг тэгж яргалж байсан юм.
Гэхдээ 1115 он гэхэд бүх зүйл өөрчлөгдөв. Зүрчдүүд босч, 1126 онд Ляо гүрнийг сүйрүүлэв. Тэдний угсаатны нийлэгжилтийн чиглэл, өөрөөр хэлбэл угсаатны зонхилох хүч нь овог аймгийн нэгдэл болов. Энэ нь хуучин овгийн удирдагч Агуде–д “Алтан” нэртэй эзэнт гүрэн бүтээх боломж олгожээ.
Нүүдэлчдийн хувьд бол зонхилох хүчин зүйл нь өөр байдаг. Тэдний аймгуудаас “урт дурын хүмүүс” гэсэн баатар эрс ялгаран гарч, анхандаа маш ядуу зүдүү байдалд байсан ба XII зууны төгсгөлөөр Тэмүүжин нэрт удирдагчаа олж, түүнийгээ Чингис гэж өргөмжлөн хаанаараа сонгожээ. Харгис хатуу иргэний дайнд “урт дурын хүмүүс” овог аймгийн байгууллыг халж, ялагдсан болон ялагчид нэгдмэл угсаатан болж нийлэн нэгдсэн Монгол Улсыг байгуулав.
Эцэст нь Байгалийн өмнөд эргээр мэргэд хэмээх дайчин аймаг тодрон гарав. Мэргэд нь ямар гарал үүсэлтэйг тогтоогоогүй. Тэд бол монгол биш, түрэг биш, харин самодий нар байж магадгүй. Гэхдээ энэ нь бидэнд чухал биш, харин 1216 онд мэргэдүүд Төв азид ноёрхохын төлөө Монголчуудтай сөргөлдөж эхэлсэн нь чухал юм. Энд бидний дээр авч үзсэнтэй адилтгам угсаатны нийлэгжилтийн тэсрэлт бидний өмнө болж байна. Угсаатны нийлэгжилтийн энэхүү тэсрэлтийн талбар нь өмнөхтэйгээ адил маш нарийн зурагдсан байгаа нь онцгой юм. Энд хойд зүгийн эвенкүүд ч, якутууд ч, өмнө зүгийн солонгос, хятад, тангууд нар ч хөндөгдөөгүй билээ. Энэ талбар баруун зүгтээ Байгалийн өмнөд захтай хаяа нийлж орсон боловч нүүдлийн ахуйгаараа монголчуудтай адилхан амьдардаг ойрд болон куман (половчууд) нарыг шүргээгүй юм. Талбарын хязгаар зааг нь бидний өмнө, харин түүний үндсэн хэсэг нь Номхон далайн дэвсгэр дээр ногдоно гэж бодож болох юм. Хэрэв тийм ахул (үүнийг татгалзах үндэслэл байхгүй юм) угсаатны нийлэгжилтийн тэсрэлт буюу түлхэлт нь физик газар зүйгээс салшгүй үзэгдэл болно. Цаг уурч циклон болон муссоны шилжилтийг тэмдэглэдэг, түүхэн газар зүйч Евроазийн нүүдэлчдийн миграцийг тэмдэглэдэг шиг түүхчид ийм үзэгдлийг зөвхөн тэмдэглэдэг ажгуу. Одоо бол Евроазийн талын болон ойн угсаатнуудын идэвхи бий болох нь цаг агаарын хэлбэлзэлтэй дандаа холбогддоггүй, харин дундад зууны түүх бичигчдийн бодол санааг эргүүлсэн угсаатны нийлэгжилтийн тэсрэлтийн үр дүн болох нь илт байна. Энэ, зөвхөн энэ нь л монголчуудын нислэгийг ард түмнүүдийн Агуу их шилжилт, византийн үнэн алдартнууд эртний римийн буруу номтныг даран мандсан, исламыг номлох замаар ариун дайн хийсэн, ражпутуудын гавъяа, Тан гүрний эзэн хаадын ордны өнгө жавхаагаар Төвдийн хаант улс үүссэн зэрэгтэй амь нэгтэй болгодог юм. Тодорхой газар нутаг дээр салбарлан болдог иймэрхүү “түлхэлт” буюу “тэсрэлт” нь эхлэх үеүд нь Дундад зууных шиг тийм хэмжээгээр сурвалжуудад тэмдэглэгдээгүй эртний агуу их угсаатнуудын эх ундарга болдог байна.
Бид янз бүрийн үзэгдлүүдэд төстэй байдлын элемент олж авсны дараагаар бол тэдгээрийн ялгааг маш амархан тайлбарлаж болно. Гадаад орчны янз бүрийн нөхцөлд адилхан лугшилт янз бүрээр илрэх ёстой. Доошоо өнхрөх гэж буй чулууг хөлөөрөө түлхэн уулын орой руу явж хүнийг төсөөлөөд үзье. Заримдаа энэ чулуу хэд хэдэн сууринг бүрхэн унах их нуранги үүсгэж болно, заримдаа тэр ан цавд тээглэх буюу мөргөцөгт тулж тэндээ үлдэх болно. Чулууны түлхэлтийн хүч болон чиглэл, түүчлэн замд нь таарах бүхий л саад бэрхшээл зэргийн талаар бүх өгөгдөхүүн байвал энэ чулууны замыг тооцож, түүний хувь заяаг урьдчилан хэлж болох билээ. Гэвч практикийн хувьд ийм хэмжээний өгөгдөхүүн олж авах боломжгүй юм.
Угсаатны түлхэлтийн хувь заяаг хялбарчлан харуулахад үйл явцын хөгжлийг өөрчлөгч саад тотгорын үүргийг өнгөрсөн зуунуудад бүрэлдсэн нийгмийн нөхцөл, өвөг дээдсээс өвлөн авч хуримтлуулсан соёл, бүс нутгийн газар зүйн орчин, олон улсын харилцааг оролцуулсан угсаатны хүрээлэл, улс төрийн тооцоо, тухайн үеийнхний өдөөн хатгалга зэрэг олон үзэгдлүүд гүйцэтгэдэг байна. Гэхдээ энэ бүгд нь угсаатны системд түүнийг хувирган өөрчлөгч, агуу их үйлс бүтээхэд хүргэдэг эрчим хүч орж ирсэн тэр үед л хүч чадлаа хуримтлуулж чадна.
Энэ юун эрчим хүч вэ ? Түүний шинж чанарыг тодорхойлоод бид угсаатны нийлэгжилтийн асуудлаа шийдвэрлэх болно.
Угсаатны түүхийг судлахад бид зөвхөн нийгмийн харилцаа болон нийгмийн институтыг харж байдгийг тэмдэглэн хэлье. Гэхдээ энэ нь бидний ажиглалтаар сэдэв маань шавхагдаж байна гэсэн хэрэг биш. Цахилгаан юмуу дулааныг бид жишээлбэл, гэрлийн улайссан утас юмуу түүний халуун байгааг харж, энэхүү илрэлээр нь мэддэг. Гэхдээ энэ нь туршлагад нэгтгэн дүгнэлт хийх, оюун дүгнэлтийн ойлголтуудаар үйлдэл хийхэд бидэнд саад болдоггүй. Мөн “амьдрал”, “нийгмийн формаци” гэсэн ойлголтууд байдаг ч гэсэн эдгээр нь бас л олон ажиглалтын нэгтгэл юм. Угсаатан ч гэсэн мөн л ийм болох нь мэдээж.
Юмсын урт, жин, температурыг харьцуулан үздэг шиг угсаатныг ч гэсэн нийгмийн болон биологийн категориудтай харьцуулан жишиж болно. Эдгээрийн аль аль нь өөр хоорондоо үл авцах янз бүрийн мөн чанарын үйл явцын үзүүлэлтүүд (параметр) болдог. Бид нийгмийн үйл явц, угсаатны нийлэгжилт хоёр зэрэгцэн хөгждөг гэдгийг нэгэнт тогтоосон болохоор угсаатан нь бүл буюу бусад бүл – мөн тэр түвшний биологийн таксон буюу дараалсан эрэмбэтэй хэрхэн харилцан үйлчилдгийг шалгах хэрэгтэй болж байна. Биднийг багагүй сонирхолтой гэнэтийн зүйлс хүлээж байгаа.
