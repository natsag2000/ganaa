Гуравдугаар хэсэг

ТҮҮХЭН ДЭХ УГСААТНУУД
ЭНД УГСААТНЫ ҮЗЭГДЛИЙГ СУДЛАХ НИЙТЭЭР ЗӨВШӨӨРСӨН АРГУУДЫГ ҮЗЭХ БА ИНГЭЛЭЭ ГЭЭД ХАЙЖ БУЙ ҮР ДҮНГЭЭ ОЛЖ ЧАДАХГҮЙ ГЭДГИЙГ ҮЗҮҮЛНЭ.

XII. Дэлхий нийтийн түүхийн тухай эрэгцүүлэл
ДЭЛХИЙ НИЙТИЙН ТҮҮХИЙН ХОЁР АСУУДАЛ
Түүх бол холбоо харилцаа, дэс дараалалдаа байгаа үйл явдлын тухай шинжлэх ухаан юм. Гэтэл үйл явдал хэт олон, харин янз бүрийн ард түмнүүдийн харилцаа холбоо янз бүр. Ямар ч шинжлэх ухааны зорилт бол судлан буй зүйлийг бүхэлд нь харахад оршдог бөгөөд түүх ч үүнээс гадуур бус. Улмаар харах хамгийн тохиромжтой цэгийг олох хэрэгтэй болдог бөгөөд энд практикийн нягтлагч, өөрөөр хэлбэл аспектийг сонгодог онол зайлшгүй үүсдэг юм. Судалгааны аспект нь ямар нэгэн гүн ухааны хийцээс гардаггүй, зөвхөн практик сэтгэлгээгээр л нөхцөлддөг. Бид түүнийг шинжлэх ухааны онолын салбарт хамааруулдаг нь түүнийг сонголтыг материалын хуримтлалаар биш судалгааны эхэнд тавигдсан зорилгоор тодорхойлдогт л байгаа юм. Манай зорилго бол Дэлхийн бүрхэвчийн нэг угсаатны хүрээ тогтох байдлаар Дэлхийн түүхийг ойлгоход оршино.
Түүхэн сэтгэлгээний онолд эрт үеэс бүх нийтийн-түүхийн болон соёл-түүхийн гэсэн хоёр үзэл баримтлал бүрдсэн бөгөөд манай үед ч байсаар байна. Эхнийх нь ард түмнүүдийн түүхийг хүн ам суурьшсан бүх салбарыг их бага хэмжээгээр хамарсан, дэвшилтэт хөгжлийн нэгдмэл үйл явц хэмээн тайлбарладаг. Энэ нь анх удаа Дундад зууны үед ассир, перс, македон, рим гэсэн өнгөрсөн “дөрвөн эзэнт гүрний” болон тав дахь нь YII–IX зууны заагт (Chretiente) Баруун Европт үүссэн, папын сэнтийтэй хамт католик нэгдлийг толгойлсон “Герман Үндэстний Ариун Римийн Эзэнт Гүрэн” гэх концепци байдлаар томъёологдсон байна.
Үйл явдлыг тайлбарлах ийм системийн үед “дэвшил” гэдгийг эзэнт гүрний эрх мэдэлд захирагдах газар нутаг дэс дараалан өргөжих явдал гэж үздэг байв. XIY–XYI зуунд Реформаци Баруун Европын үзэл суртлын нэгдлийг эвдэж, Габсбургийн улсын эзэн хаадын улс төрийн ноёрхлыг задлах үед бүх нийтийн-түүхийн үзэл баримтлал тэсэн үлдэж, зүгээр л жаахан өөрөөр томъёологдсон юм. Одоо бол дэвшилтэт зүйлээр ахиад л Барууны, Европын роман–германы соёлыг, чингэхдээ хуучны “тэрс үзэлтэн”, “схизматик” буюу тусгаарлагч нарыг “зэрлэг”, “хоцрогдсон” ард түмэн гэж ойлгодог “соёл иргэншил” гэгчийг хүлээн зөвшөөрөх болжээ. Мөн “зэрлэг”, “хоцрогдсон” хоёрын аль алийг нь “түүхэн бус” гэж нэрлэхийг ч оролдож байна. “Евроцентризм” хэмээн зөв нэрлэсэн энэ системийг XIX зуунд голдуу ухамсаргүй, өөрөө аяндаа ойлгомжтой, нэмэлт нотолгоо шаардахгүй мэтээр хүлээн авч байлаа.
Соёл-түүхийн концепцийг анх удаа Европ, Ази хоёрыг сөргүүлэн тавьсан Геродот тунхагласан юм. Европ гэдэгт тэрээр эллиний хот–улсуудыг, харин ази гэдэгт персийн хаант улсыг ойлгож байлаа. Хожим нь энд Элладтай ч, Ирантай ч аль алинтай нь төсгүй байсан Скиф болон Эфиопийг нэмсэн бөгөөд цаашдаа соёлын бүс нутгууд бүх Эх газар буюу Ойкумен соёл–түүхийн мужуудад бичигдэх хүртэл өргөссөөр л байв. Ийм мужууд нь Хуучин Ертөнцөд анхдагч ойртолтоор Баруун Европоос гадна: Ойрхи Дорнод буюу Левант, Энэтхэг, Хятад, Номхон далайн арлын соёлт хэсэг, Евроазийн талын муж, Сахараас өмнүүрх Африк үлдэгдэл угсаатантай туйлыг тойрсон мужуудыг тооцдог байлаа.
Бүх нийтийн–түүхийн үзлээс соёл–түүхийн сургуулийн үндсэн ялгаатай нь соёлын муж бүр хөгжлийн өөрийн гэсэн замтай, иймээс европ бус ард түмнийг “хоцрогдсон” буюу “зогсонги” байдлын тухай ярих хэрэггүй, зөвхөн тэдний өвөрмөц байдлыг тэмдэглэж болно гэсэн үндэслэл юм. XIX зууны соёл–түүхийн сургуулийн томоохон төлөөлөгчид бол Ф. Ратцель, Н.Данилевский болон К.Н. Леоньев нар, харин ХХ зуунынх нь О.Шпенглер болон А.Тойнби нар болно.
БИ ЯАГААД А. ТОЙНБИЙГ ЗӨВШӨӨРӨӨГҮЙ ВЭ ?
Бид асуудлын түүх рүү гүнзгий орохгүй, энэ нь биднийг өөр зүг рүү хэтэрхий хол явуулна. Гэхдээ ямар ч гэсэн нэг зохиогчийг хараанаасаа гаргах ёсгүй юм. А.Тойнби газар зүйн сурвалжийг ашиглахад үндэслэсэн “соёл иргэншил” үүсэх үзэл баримтлалыг санал болгосон юм. Товчдоо энэ нь дараах зүйлст багтана.
Түүхийн нэгж нь “нийгэм” юм. “Нийгэм” нь “бүдүүлэг”, хөгждөггүй болон 16 бүсэд буй хорин нэгэн “соёл иргэншил” гэсэн хоёр зэрэглэлд хуваагдана. Улмаар, нэг газар нутаг дээр дэс дараалан 2–3 соёл иргэншил үүсдэг гэвэл ийм тохиолдолд тэднийг “охин” гэж нэрлэдэг. Ийм охин иргэншилд Месопотами дахь шумерийн болон вавилоны соёл иргэншил, Балканы хагас арал дээр минойн, эллиний, тууштай христианы соёл иргэншил, Индостанд энэтхэгийн (эртний), индусын (дундад зууны) соёл иргэншлүүд байсан. Түүнээс гадна “хаягдсан” соёл иргэншил–ирланд, скандинав, төв азийн несторианчууд, мөн “саатсан”–эскимос, осман, Евроазийн нүүдэлчид, спартчууд, полинезчүүд зэрэг байдаг аж.
А.Тойнбийн үзэж байгаагаар нийгмийн хөгжил нь “мимесис”–ээр, өөрөөр хэлбэл дуурайхаар дамжин хэрэгждэг. Бүдүүлэг нийгмүүдэд өвгөд, ахмадуудаа даган дуурайдаг, энэ нь эдгээр нийгмийг тогтонги болгодог. Харин “соёл иргэншилд” бүтээлч хувь хүнийг дуурайдаг, энэ нь хөгжлийн хөдөлгөөнийг бүтээдэг. Ийм учраас түүхийн гол асуудал нь хөдлөнги байдлын хүчин зүйлийг хайн олох явдал болно. Энд А.Тойнби арьсны үзлийг үгүйсгэсэн юм. Гэтэл газар зүйн орчны нөлөөлөл үлдэж байна, харин энд А.Тойнби нэн өвөрмөц шийдвэр санал болгож байна. “Хүн соёл иргэншилд биологийн өгөөж хишиг (удамшлаар) буюу газар зүйн хүрээллийн хялбар нөхцлийн үр дүнд биш, одоо болтол үзэгдэж сонсогдоогүй хүчдэлээр түүнийг урамшуулагч онцгой хүндрэлтэй нөхцөл байдал дахь дуудлагад өгөх хариултаар хүрдэг” 1. Toynbee A. J. Study of History /Abridienent by D. Somervell. London; New York, Toronto, 1946.P. 60 sqq.
Ийнхүү авъяаст чанар ба бүтээлч чадварыг гадаад өдөөгчид организмын хариу үзүүлэх төлөв байдал хэмээн авч үзээд үүнтэй холбогдуулан нэг бүлгээ (YI) “Аз жаргалгүйн давуу тал” гэж нэрлэжээ. “Дуудлагууд” нь гурван маягт хуваагдана :
1. Байгалийн тааламжгүй нөхцөл, жишээлбэл, Нил мөрний дээд хэсгийн намаг нь эртний египет хүний дуудлага, Юкатаны халуун бүсийн ой нь Майя нарын дуудлага, Эгийн тэнгисийн давалгаа нь эллинчүүдийн дуудлага, ой болон хүйтэн жавар нь Оросуудын хувьд дуудлага болно. Энэ үзэл баримтлалын дагуу бол английн соёл нь бороо, манангаас үүссэн байх ёстой. Гэхдээ үүнийг Тойнби батлаагүй юм.
2. Харийнхны довтолгоо, үүнийг бас газар зүйн (хэсэгчилсэн нүүдэл) агшин гэж үзэж болно. А.Тойнбийнхоор бол Бавари, Баден руу туркууд довтолсон учраас Австри хөгжлөөрөө тэднийг гүйцэж түрүүлсэн (119 дэх тал). Гэхдээ туркууд эхлээд Болгар, Серби, Унгарыг довтолсон бөгөөд тэд нь энэ дуудлагад бууж өгч хариулсан. Харин Австрийг Ян Собесскийн гусарууд хамгаалсан. Энэ жишээ нь үзэл баримтлалын концепцийн талд биш, харин эсрэгээр нотолж байна.
3. Өмнөх соёл иргэншлийн ялзрал, энэ дуудлагатай тэмцэх хэрэгтэй. Тухайлбал, эллин–римийн соёл иргэншлийн задрал нь эртний грекүүдийн замбараагүй байдалд византийн болон баруун европын соёл иргэншлүүд хариуд нь “дуудсан” мэт байна. Үүнийг бас цаг хугацааны координатыг (биоценезийн халаа) тооцож авсан газар зүйн нөхцөлд хамаатуулж болно. Гэхдээ харамсалтай нь Византи дахь зальхай явдал римийнхнээс дутаж байгаагүй, Баруун римийн эзэнт гүрэн унах болон амьдрах чадвартай феодалын хаант улс бий болох хүртэл 300 гаруй жил болсон юм. Хариу үйлдэл ихээхэн хоцорчээ.
Гэхдээ хамгийн чухал зүйл болох хүн ландшафтын харьцааг А.Тойнбийн үзэл баримтлал шийдэлгүй, будлиантуулсан. Нэг талаас хатуу ширүүн байгал нь хүний нэмэгдсэн идэвхийг урамшуулдаг гэх сэдэв, нөгөө талаас газар зүйн детерминизм буюу шалтгаант чанарын хувилбар нь үнэхээр буруу юм. Эртний оросын төр үүссэн Киев орчмын цаг агаар нь огтхон ч хүнд биш. “Тал нутагт ноёрхох нь нүүдэлчдээс асар их эрчим хүч шаарддаг бөгөөд үүнээс цааш юу ч үлддэггүй” гэсэн мэдэгдэл нь (167–169 дэх тал) зохиогчийн юу ч дуулаагүйг харуулж байна. Түрэг болон монголчуудын бүрэлдсэн Алтай, Ононгийн нарсан ой нь амралтын таатай газар мөн. Хэрэв Грек болон Скандинавыг угааж буй тэнгис нь “дуудлага” байсан юм бол яагаад грекүүд зөвхөн НТӨ YIII–YI зуунд, скандинавууд зөвхөн НТ IX–XII зуунд “түүнд нь хариулт өгсөн” вэ ? Бусад эрин үеүдэд бол ялан дийлэгч эллинүүд ч, зоримог махчин шиг финийкууд ч, догшин ширүүн викингүүд ч байгаагүй бөгөөд ердөө л далайн сархинцаг, май загас баригчид л байсан бус уу? Шумерууд Хоёр мөрнийг “усыг хуурай газраас салган” Эдем болгож, харин туркууд тэнд дахин намаг бий болгох гэж энэ бүгдийг хайхралгүй орхисон юм, хэрэв А.Тойнбийнхоор бол тэд нар Тигр, Евфрат мөрний “дуудлагад” хариу өгөх ёстой байлаа. Энд бүгд худал байна.
Соёл иргэншилд бүс нутгуудаар газар зүйн ангилал хийсэн нь бүр ч дур зоргийн харагдаж байна. А.Тойнби Византийн болон Туркийн эзэнт улсуудыг зөвхөн нэг газар нутаг дээр байрлаж байсан болохоор нь нэг соёл иргэншилд оруулжээ, чингэхдээ грек, албаничуудыг биш, харин османуудыг яагаад ч юм “саатсан”(?!) гэж зарлажээ. “Сирийн соёл иргэншилд” Иудейн хаант улс, Ахеменидийн эзэнт гүрэн, Арабын халифатуудыг оруулсан байна. Харин Шумер, Вавилон хоёрыг эх ба охин соёл иргэншил болгожээ. Энд ангиллын шалгуур нь зохиогчийн дур хүсэл байсан нь илт байна.
Шинжлэх ухааны үр ашигтай үзэл санааг үлбэгэр тайлбарлал, санаанаасаа олсон зарчмыг зохисгүй хэрэглэснээр амархан бузарладгийг үзүүлэх хэрэгтэй гэж үзсэн учраас би энэхүү хэсгийг тодорхой авч үзсэн билээ. Хэдийгээр үйл явдлын бодит явц болон цаг дарааллын талаар багагүй зөрчилтэй ч гэсэн А.Тойнбийн социологийн байгууламжийг санаатайгаар хөндөөгүй болно. Гэхдээ ихэнхи уншигчдад “дуудлага ба хариултын” газар зүйн үзэл баримтлалыг яаж ийгээд бас л олон хүн ноцтой хүлээн авдаг нь тодорхой. Хамгийн гачлантай нь ийм туршлагын дараа ерөөсөө газар зүйн өгөгдөхүүнийг тооцох, авч үзэхээс татгалзах, байгалийг түүхэн үйл явцад үл нөлөөлөх тогтвортой хэмжигдэхүүн мэтээр үг дуугүй үзэх хандлагууд ямагт бий болдог билээ. Хуурамч бүтэц хийх замаар дураараа сонгон авсан үндэслэлүүдийг хөгжүүлэх нь шинжлэх ухааныг мухардалд чихдэг юм.
Ийнхүү хоёр хандлага нь зарим давуу талтай, мөн томоохон дутагдалтай юм. Ялангуяа сүүлчийн хандлага нь бидний сэдвийг боловсруулах үед онцгой мэдрэгддэг билээ. Тухайлбал, бүх нийтийн–түүхийн сургуулийн үүднээс бол түрэг–монголын ард түмэн болон тэдний өвөрмөц нүүдлийн соёлыг дорно дахины соёл иргэншилд ч оруулж болохгүй, бас “зэрлэг” гэсэн ангилалд ч тооцож болохгүй байдаг. Улмаар эдгээр нь түүхч-онолчийн харааны гадна үлдэж байна. Ийм хэдий ч түргүүд болон монголчууд нь хүн төрөлхтний түүхэн дэх өөрийнхөө ач холбогдлыг нэн жинтэй мэдрүүлж байсан ба тэднийг Хятад, Иран, Византийн “бүдүүлэг зах хязгаар” мэтээр авч үзэх оролдлого хэд хэдэн удаа хийгдэж байсан нь асуудлыг өөрийг нь ингэж тавих үед ч гэсэн шинжлэх ухааны танин мэдэхүйд огтхон ч үл тохирох гажуудсан зураглалыг өгч байна. Ингээд л мухардал болж байна.
Түүний хамт хүн төрөлхтний түүхэнд түргүүдэд тодорхой байх суурь өгдөг соёл-түүхийн сургууль ч гэсэн тэдгээрийн түүхэн хөгжлийн дотоод зүй тогтолд тайлбарлал өгөх чадваргүй. Учир нь эдгээр зүй тогтол нь зөвхөн салбарын төдийгүй, нийт хүн төрөлхтний хувилбар байдаг юм. Ерөнхийг тооцохгүйгээр хэсэг нь ч гэсэн ойлгомжгүй болох ба учир нь ийм хандлагаар тэдгээрийг харьцуулах ч, хэмжих ч боломжгүй. Хүн төрөлхтний түүхийг ойлгоход цагаатгаж болшгүй тасралт үүссэн байна. Энэ бол бас л мухардал юм.
БИ ЯАГААД Н. КОНРАДЫГ ЗӨВШӨӨРӨӨГҮЙ ВЭ ?
Магадгүй, зорилгодоо дээд зэргээр ойртохын тулд үзэл баримтлал тус бүрээс ашигтай зүйлийг нь авч, тэдгээрийгээ нэгтгэн авах гэсэн гуравдахь зам ч зөв байж болох юм. Жишээлбэл, нэг формацаас гарч, нөгөөд орж буй: 1) эртний нийгмээс дундад зуун руу шилжсэн үе–эллинизм, 2) Дундад зуунаас Шинэ үед шилжсэн үе–Сэргэн мандалт, 3) Шинэ үеэс Шинэхэн үед шилжсэн үе–XIX зууны дунд үе гэсэн шилжилтийн эрин үеүүдийг ялгах санал тавьжээ. Энэ шалгууртаа утга зохиолын түүхийг оролцуулж, “тус бүр нь (гурван эрин үе) түүний давшилтыг өргөмжилсөн утга зохиолын суут бүтээлүүдээр эхэлж байна. Нэгдэхийг нь “Бурхны хотын тухай” (Августин), хоёрдахийг нь “Тэнгэрлэг туульс”, гуравдахийг нь “Коммунист намын тунхаг” магтан дуулж байна. 2. Конрад Н. И. Запад и Восток. С. 454.
Шинэ үзэл баримтлалыг зохиогч маш тууштай байна. Тэрээр Баруун Европын орнуудаас чанар муутай, соёлын талаар хамааралтай гэж үзэж огтхон ч болохгүй европын бус орнуудын соёлын хөгжил дэх адилтгах эрин үеүдийг хайж байна. Тэр: “Эртний үеэс Дундад зууны үе рүү шилжихэд Хятад, Иранчуудад бас л оюун ухааны хувьсгал гарсан бөгөөд …үүнийгээ Хятадад нийлбэр байдлаар даосизм гэж, Иранд бол манихеизм гэж нэрлэдэг. Энд гаднаас ирсэн үзэл суртлын систем–гадаад хүчин зүйл бас нэгдсэн. Энэ нь Хятадад буддизм, Иранд ислам байсан юм. Түүнчлэн “Дахин сэргэлтийн” буюу “Сэргэлт”-ийн үеүд ажиглагдаж байна. Энэ нь Хятадад YIII зуунд, Дундад Ази, Ази, Иран, баруун хойт Энэтхэгт IX зуунд, эцэст нь Италид XIII зуунд болжээ. Гурав дахь шилжилтийн эрин үеийг бичээгүй бөгөөд учир нь тэрээр дуусаагүй учраас ингэх нь ч зөв юм. 3. Там же С. 455. 4. Там же. С. 457.
Зохиогчийн үзэл санаа хамгийн нөлөөтэй, тод илэрхийлэгдсэн учраас л Н.И.Конрадын зузаан номноос энэ л хэсгийг сонгон авсан болно. Конрад бусад зохиолууддаа зөвхөн шилжилтийн үеийг биш, мөн нийгмийн ахуйчлалын тогтвортой хэлбэрийг хэвшиж тогтсон нэр томъёогоор: Эртний үе, Дундад зуун, Шинэ үе гэх мэтээр судлан үзсэн байна. Түүхэн үйл явцын үндсэн чиглэлийг тэрбээр ард түмэн бэхжих болон соёлын газар нутаг тэлэх явдал гэж үзэхийн хамт дэлхийн соёл иргэншлийн нийлэгжилт нь олон төвтэй, мөн ард түмнүүдийн хөгжилд тусгайлсан шинж байдаг гэдгийг хүлээн зөвшөөрсөн байна. Ингээд мухардлаас гарах гарц олдсон мэт боловч дээр дурдсан сэдвийн зарчмын талуудыг харж үзье.
Н.И.Конрадын үзэл баримтлалд шилжилтийн үеүдийн он дарааллын хэмжиж болшгүй шинж нь нүдэнд тусч байна. Эллинизмын эрин үе НТӨ IY зуунд эхэлсэн, тэгээд ч энэ нь эртний ертөнцийг үзэх үзлийн хямралтай давхцсан. Гэтэл Н.И.Конрад энэхүү “шилжилтийн үеийг” Августин хүртэл буюу НТӨ Y зуун хүртэл сунгажээ. Ялгаа нь 900 жил болж байна. 5. Конрад Н.И. 1) О рабовладельческой формации //Там же. С. 33- 53; 2) Средние века в исторической науке //Там же. С. 89-118. 6. Там же. С. 454.
Итали дахь Дахин Сэргэлтийн үе нь 150–200 жилд багтдаг, гуравдахь шилжилтийн үе нь хагас зуун жилд багтана. Зохиогч аберраци гэгч ойрхоны гажуудалд орж, өөрөөр хэлбэл ойрхон болсон үзэгдэл түүнд хол болсон үзэгдлээс илүү ач холбогдолтой болоо юу даа гэсэн бодол өөрийн эрхгүй төрж байна. Эдгээр үзэгдлүүдийг зэрэгцүүлэн хэмжиж болохгүйг үзүүлэхийн тулд эллинизмийг Дахин Сэргэлттэй харьцуулахад л хангалттай билээ. Хэрэв үүнээс их болох аваас энэ бол янз бүрийн эрэмбийн хэмжигдэхүүн болно. Шинэ мэдээлэл нэмэлгүйгээр (өөрийнхөө байран дээр юу ч хийсэн байж болох) энэ асуудлыг цоо шинээр дахин үзэхийг оролдъё. Чингэхдээ хуучны, ямар ч маргаангүй өгөгдөхүүнүүдийг шинэ, өвөрмөц үзэл санаатай харьцуулан жишихээр хязгаарлая. Хэрэв сүүлчийнх нь үнэн бол давхцал гарцаагүй юм.
ЭЛЛИНИЗМИЙН ТУХАЙ
НТӨ 336 он, өөрөөр хэлбэл Александр Македонский Фивийн ноёрхлыг сөнөөж, Афинд эрх чөлөө, Персэд агуу суу, Энэтхэг болон Египетийн эртний соёлд (Александрий хотыг бүтээсэн) тусгаар тогтнол олгосон тэр үеэс “Тэнгэрлэг туульсыг” зохиох хүртэл дараах зүйл болжээ.
Македоны богино хугацааны түрэмгийллийн дараа Иранд Арал орчмын тал газраас парфянууд ирсэн ба анхандаа тэд эллиний соёлын гэрэл гэгээг гайхан биширч байснаа дараа нь зороастризмын шашинд (НТӨ 250–НТ 224 он) гүн гүнзгий татагдах болжээ. Дараа нь өнөө үед бол үндэсний гэж нэрлэмээр Ираны харгислал эхэлжээ. Арташир Папаган язгууртны ноёрхлыг эвдэж, жижиг тайж (декхан), лам нарын (мобедууд) холбоонд тулгуурлан, үүндээ өршөөл үзүүлсэн парфяны язгууртнуудыг морин цэрэг болгон ашиглажээ.
YI зууны эхэнд ихэс Маздак засгийн эрхийг булаан авч, дээдэс болон лам нарыг хөнөөж эхэлсэн ба тэдгээрийн аль аль нь хүн амын хамгийн оюун ухаанлаг хэсгийг төлөөлж байсан юм. 530 онд болсон Хосров Ануширваны эргэлт нь шинэчлэл болон түүнтэй холбоотой аллагыг зогсоосон боловч үнэхээр үгийн бүрэн утгаар цэргэнцэрүүдийг засгийг эрхэнд хүргэжээ, учир нь мэргэжлийн дайчид өдрийн хөлс авдаг байсан аж. Цэргийн удирдагч Бахрам Чубин 590 онд хаан ширээг авсан боловч түүний эсрэг бүх нутгийнхан босч, бут цохижээ.
Сүүлчийн үе нь (591–651 он) төрийн систем болон соёлын зогсолтгүй задрал нь арабын булаан эзлэлтэд хүрч, энэ нь дүрвэлт болон бичиг үсэгтэй, боловсролтой бүх персүүдэд үхлийг дагалдуулсан юм. Үүний дараа шинэ соёл, тэр ч байтугай шинэ хэл бүхий шинэ ард түмэн бүрэлдэж эхлэв.
Дурдан буй үед соёлын хүрээнд таван өөрчлөлт гарсан бөгөөд эдгээр тус бүр нь тухайн системийн (манай тохиолдолд ираны соёл) доторхи ач холбогдлоороо италийн Дахин Сэргэлттэй адил юм, гэхдээ угсаатны нийлэгжилтээрээ ч, шинж чанар болон үр дагавраараа ч түүнтэй төсгүй юм. 1) Тал нутгийн парфянчуудыг эллинжүүлсэн, өөрөөр хэлбэл харийн соёл иргэншлийн хүлээж авсан, 2) Парфяны дээдэс иранжсан, өөрөөр хэлбэл өөрийн ард түмэнтэй ойртохыг оролдсон, 3) 224–226 онд Зороастрийн шашин парфяны язгууртнуудыг ялсан, өөрөөр хэлбэл хаан ширээ, гол шүтээн хоёр холбоо тогтоосон, 4) Маздакизм (Дундад зууны эртний үед Иран болон хөрш орнуудад дэлгэрч, үүсгэгч хүнийхээ нэрээр шашин–гүн ухааны урсгал -Орч), 5) Митраизмын (Грекийн өдрийн гэрэл, цэвэр байдал, үнэний бурхны нэрээр түгсэн шашны урсгал-Орч) харгислал, учир нь армянчууд Бахрам Чубиныг “Михр-т мөргөсөн босогч” гэж нэрлэдэг. Энэхүү суурь дээр бол тунгаасан, тогтворгүй сэхээтний өчүүхэн хэсгийг хөндсөн христианы болон эртний гностик үзэл санаа нь үл мэдэгдэм сэвшээ байлаа.
Үгүй ээ. Энэхүү хүнд хэцүү, бүтээлч амьдралын мянган жил нь ердөө л македоны болон арабын түрэмгийллийн хоорондын шилжилт байсан гэдэгт яагаад ч итгэж чадахгүй байна. Ираны хувьд бол энэ үе нь Сумбад Мага, Бабек нарын бослого, Афин ба арабын эсрэг тэмцлийн бусад илрэл болох хуйвалдаан мэтийн бүдэг бадаг дурсамж бүхий парфян–сасанидийн үе байсан бөгөөд италийн Дахин Сэргэлттэй биш, харин Каролингоос эхлээд Бонапарт хүртэл, Баруун Европын роман–германы бүх соёлтой төстэй мэт төсөөлөгдөж байжээ. Хэдийгээр харьцуулан буй соёлууд өөр хоорондоо ерөөсөө төсгүй ч мянган жил бол мянган жилтэй л тэнцдэг аж. Гэхдээ чухам энэхүү “төсгүй” байдал нь төстэй байдалтайгаа хамт Н.И.Конрадын үзэл баримтлалын үндэслэлүүдийн нэг болдог юм.
Римд дэх эллинизмийг хоёр тооцооллын аль нэгээс эхлэн тооцож болно. 1) Хөөгдөгсдийн бүлэг долоон толгодод суурьшиж, грекийн хот улсын жишгээр зохион байгуулагдаж байх үеийн Арван хоёр хүснэгтийн Эрин үеэс эхэлж тоолох. Хэрэв ингэх ахул энд Римийн бүгд найрамдах улсын бараг бүх үе багтана, гэтэл шилжилтийн үеийн эхлэлийн хувьд энэ хугацаа тохирохгүй, 2) Римийн соёлын эллинжүүлэлтийг голдуу НТӨ II зууны үеийн Сципионуудын дугуйлангийн үйл ажиллагаатай холбон бичдэг. Хэрэв ийм байх ахул Н.И.Конрад Римийн бүгд найрамдах улсыг шилжилтийн үед биш, харин боол эзэмшлийн формацийг батлахад хамаатуулах ёстой. Эндээс “шилжилтийн үе”-ийн хувьд Н.И.Конрадын “мандах үе ба түүний хамт задрах үе” хэмээн тодорхойлсон, зөвхөн эзэнт гүрнүүдийн эрин үе л үлдэж байна. Ингэж л байг. 7. Там же. С. 37.
Гэхдээ ийм байх аваас бид италийн Дахин Сэргэлттэй тус бүр нь тэнцэхүйц хэд хэдэн соёлын, түүний хамт нийгэм-улс төрийн үеүдийг ялгах ба чадах ёстой. Яг адил гэх зүйлийг огтхон ч үзэгдлийн шинж чанарын төстэй байдлаар нь биш, зөвхөн орчин үеийнхэнд зориулсан ач холбогдлоор нь баталдаг гэдгийг давтан хэлье.
Эртний Римчүүд өөрсдөө ч гэсэн НТӨ II–I зууны бүгд найрамдах улсаа бүрэлдэж буй улс төрийн хэлбэр гэж хэзээ ч үзэж байгаагүй. НТӨ 133 онд Тиберий Гракхыг алснаас эхлэн НТ 30 онд Антонийн үхэл хүртэл Рим амар заяагаа үзээгүй билээ. Иргэний дайнууд нь амьд үлдсэн хүмүүс нь ямар ч хатуу засагт баярлах хүртэл римийн сенат, ард түмнийг үгүйрүүлсэн билээ.
Октавиан Августын тунхагласан “Алтан дундаж” нь улс төрийг тогтворжуулах, цэргийн хүчийг нэмэгдүүлэх, сургамжтай жишээ бүхий өнгөрсөнд хандах лоозон болсон юм. Ертөнцийг үзэх үзлийн энэхүү систем нь Марк Аврелийн нас барах хүртэл, өөрөөр хэлбэл 200 орчим жил тогтсон юм. Хэрэв Хань, Юя болон бусад күнзийнхний үйл ажиллагааг Хятад дахь “Дахин Сэргэлт” гэж авч үзэх аваас Плиний, Тита Ливий, Састоний нарыг Римд өөрт нь “Эртний Дахин Сэргэлт” хийсэн гэж зөв, тууштайгаар тодорхойлж болно. Бид ийнхүү нэр томъёогоо тохиролцлоо.
Хоёрдахь үе нь азийн соёлууд Римийг эрчимтэй эзэлсэн явдал юм. III зуунаас эхлэн энд нөмрөгөөр халхалсан Изида, Гурван удаагийн агуу их Гермес, бурхдын Эх–Кибела, дур булаам Астарта, эцэст нь бүхнийг ялсан цэргийн бурхан “Митра–ялагдашгүй” нар зэрэг оюун ухаанууд захирч байсан юм. Аврелианаас Юлиан Апостат хүртэл митраизм нь төрийн шашин байж, Римийн эзэнт гүрний албан ёсны ертөнцийг үзэх үзэл нь болж байв. Соёл дахь энэ эргэлт нь гуманизм, тэр бүү хэл дахин сэргэлтээс ч илүү ихээхэн ач холбогдолтой байлаа. XYI зуун гэхэд италичууд болон немцүүд сайн христианаараа үлдсэн бөгөөд зөвхөн гоо зүйг болон улс төрийн төсөөллөө л өөрчилж, чингэхдээ эрс бишээр үүнийгээ хийсэн юм.
Бүр ч илүү агуу их гуравдагч өөрчлөлт нь I–IY зууны үед бүхий л Газрын дундад тэнгисийг хамарсан билээ. Түүнийг голдуу христийн ёс дэлгэрсэнтэй холбон үздэг боловч чингэхдээ христианы ёс нь Римийн эзэнт гүрнийг дагуулан байсан шинэ үзэл санааны урсгалын нэг л хэсэг байсан гэдгийг анхаарлаасаа орхичихдог юм. Хистианчуудтай нэгэн зэрэг египетийн гностик–Ариун Дар Эхийг хараасан Валентин Василид, сирийн Сайн, Муугийн гамшгийг тэнцүүлэгч Сатурн ба Мани, Могойн мэргэн цэцнийг хүргэгчийн үүрэгтэй офитууд–хорт демиургийн дайсан, Ариун гэрээний ариун байдлыг үгүйсгэсэн маркионитууд, түүний бэлэгдлийн тайлбарыг албаддаг ригенистууд, эцэст нь дээд монизмыг тунхаглагч-бүх оршин буйн бүрэн бүтэн–Тэнгэрлэг Плером зэрэг өрсөлдөж байлаа. Христианы теодицей буюу шударга ёсонд энэ бүхнээс илүү ойрхон нь Их Висилий болон Григорий Богослов нар байв, эртний платонизмаас хамгийн хол нь өөрийнхөө өвөрмөц сургаалийг нэрлэх гэж Платоны нэрийг яаж ийгээд зээлдсэн неоплатоникууд байсан юм. Н.И.Конрад үүнийг “Ухаантнуудын хувьсгал эхэлж, Рим дорно зүг рүү эргэсэн бөгөөд гэхдээ тэр нь ертөнцийг үзэх хуучин үзлийн өөрийн хямрал болж байсан “римийн эргэн тойрны нутгийн” грек-латины хэсгийг эзлэн авсан юм” хэмээн нарийн тэмдэглэсэн байдаг. 8.Там же. С. 455.
Энэ нь шударга явдал бөгөөд гэхдээ Европын соёлын түүхэн дэх энэ гамшигт үзэгдлийг шилжилтийн үе мэтээр авч үзэж болохгүй юм. Үнэн хэрэг дээрээ бол Сенекагийн бодит дүгнэлт, митреум дахь Аврелианы цуст тарнийн ёс буюу Гелиогабалын оргиастик зугаа зэрэг нь христианы болон манихейн ёстой ямар харьцаатай байх билээ? Ертөнцийг үзэх шинэ, бүтээлч урсгал нь аль алийг нь адилхан үгүйсгэж байлаа. Тэр нь өмхөрч муудсан эртний сэтгэлгээг халж, гэхдээ түүнийг үргэлжлүүлээгүй юм. Өөрөөр хэлбэл, энд “шилжилтийн үе” биш, харин хуучин болон шинэ уламжлалын тасралт болсон билээ.
Христианы болон манихейн сүм хийдүүд тухайн үеийнхнээ зожиг зангаараа гайхшруулж байв. Энэ нь эртний өнгөрснөөсөө бүрэн салсны мэдрэмжээс зүй ёсоор урган гарсан зүйл байсан юм. Эзэн хаан Константин тэрс үзэлтний бүх байр сууриа өгөх үед ч гэсэн христианы нийтлэгийн өмнө ертөнцийн эзнийг сүм хийдийн хэрэгт саналын эрхтэй байлгахын тулд түүнд диакон хэргэм олгох уу, аль эсвэл карфаген Донатын хэлснээр “Сүм хийдэд эзэн хааны хэрэг юу байна? “ гэж хэлсэн шаардлагыг тавьж түүнийг энгийн хүнээр нь үлдээх үү гэсэн зөвхөн ганцхан асуудал босч ирсэн юм. Энэ суурь дээр бүр Y зуунд эзэнт гүрнийг бүдүүлгүүд хэсэг хэсгээр нь тасалж байхад эхлээд манихей, дараа нь христиан болсон, авъяаслаг зохиолч, агуу их маргаанч Гэгээн Августин амьдарч, бүтээж, ажиллаж байлаа. Августины гол үзэл санаа нь католик биш, харин буруу номтны сэтгэлгээний дохио байсныг тэмдэглэх хэрэгтэй. Хүний эрх чөлөөт хүсэл зоригийн тухай католик догмыг бараг үгүйсгэж, урьдчилан тодорхойлох тухай сэдэв нь ертөнц дээр болж буй замбараагүй явдлын бүхий л хариуцлагыг Бүтээгчид ногдуулсан юм. Августины энэ сэдвийг мянган жилийн дараа Жан Кальвин ашиглаж, хөгжүүлсэн бөгөөд энэ нь Дундад зуунд тооцогдохгүй.
Өөрийнх нь цаг үед оршин байсан үзэл санаануудтай маргалдаж байгаагүй Данте-гаас ялгаатай нь өөрийнхөө үеийнхэнд сэтгэл нэн хангалуун бус байсан Августин авъяасынхаа бүх л хүчийг нэг үзэлтэн байсан манихейчууд болон Британийн лам Пелагийн үзэл бодолтой маргалдахад зарцуулсан билээ. Пелагий хүний нүгэл бол түүний тэнэг үйлдлийн үр дүн бөгөөд улмаар сайн тэрс үзэлтэн бол муу христианаас илүү гэж номлож байв. Харин Августин анхдагч нүглийн тухай сэдвийг дэвшүүлж, түүгээрээ бүх тэрс үзэлтнийг дорд гэж зарлан, онолын хувьд шашны үл тэвчих үзлийг үндэслэсэн юм. Ойрын таван зуун жилд энэ үзэл санаа тархаагүй байхад харин Дантегийн шүлгүүд найрагчийг бүр амьд ахуйд нь алдар суугаа зөвшөөрүүлж, түүнд гавъяа алдрыг авчирч байлаа. Үгүй, түүхэн үүргийн хувьд ч, нөлөө чансааны хувьд ч, хувийн шинж чанараараа ч Августин болон Данте Алигьери хоёр төстэй байгаагүй, тэдний амьдарч, бүтээж байсан үеүд нь бүр ч илүү төсгүй байлаа. Нэгэнтээ хэн Дантетэй төстэй гэх ахул энэ нь агуу их яруу найрагч, Ионнн Златоустын утгагүй байдлыг илчлэгч бөлгөө. Хэрэв бид энэ засварыг хүлээн авах болбоос цаашдын хэлэлцүүлэг өөр болно. Хэдийгээр дашрамд дурдахад нэлээд гэнэтийн боловч энэхүү шинэ зам нь илүү үр дүнтэй юм.
ВИЗАНТИЙН ТУХАЙ
Эртний христианы буюу нөхцөлт байдлаар византийн (огтхон ч улс төр биш, зөвхөн үгийн “соёлын” утгаар) гэж нэрлэж болох бидний авч үзсэн чиглэл нь иргэний түүхэнд ердөө л II зууны дунд үеэс, өөрөөр хэлбэл сүм хийдийн түүхээс 150 жилийн дараа тэмдэглэгдсэн билээ. Чухам тэр үед римийн гүн ухаантнууд болон христианы ёсыг магтагч Юстины хооронд алдарт ном хаялцах маргаан болсон бөгөөд Юстин маргаанд ялж, энэ ялалтаа зовлонт үхлээрээ төлсөн билээ. Хэрэв энэ цаг хугацааг эхлэл болгон авбал тохиромжтой, учир нь энд эргэлзээ, маргаан гарахгүй. Ингэвэл сэтгэлгээний шинэ чиглэл нь IY зууны төгсгөлд (Юлианыг тэрсэлсний дараа) Римийн эзэнт гүрний бүх нутаг дэвсгэрт төдийгүй, түүний чанадад ч түгэн дэлгэрсэн юм. Энэ нь Ирландад–баруун, Эфиопид-өмнө, Дундад азид–дорно, Оросод, нарийн яривал Днепр хавийн готуудад –хойт гэх мэтээр залуу мөчрүүдийг өгсөн юм.
Үгийн шууд утгаар соёлын гол баганатайгаа улс төрийн хувьд холбоогүй байсан Византийн эзэнт гүрний зах хязгаарын христианы соёлууд нь Евфратын чанадын азид несториан ёс, Сири, Армян, Африкт монофизитийн ёс ноёрхож байсан хэдий ч өөрсдийгөө дээр дурдсан Иран шиг, грек-римийн ертөнц шиг, хожим нь баруун европын Chretiente шигээр бүхэллэг хэмээн мэдэрч байлаа. Византийн соёл нь эллиний эрт үед өөрийн “Дахин Сэргэлттэй” байсан бөгөөд грек хэл төрийн удирдлагаас латин хэлийг шахсан (эзэн хаан Маврикийн үе), иконыг ялгасан өөрийн Шинэчлэлт, Македоны улсын үед өөрийн Соён гэгээрлийн эрин үетэй байлаа. Гэтэл эдгээр соёлын голомтуудад амь давчдалт бараг нэгэн зэрэг болов: XIII зуунд Ирланд унаж төв азийн несторианчуудад ниргүүлэв, Константинополь хэсэг хугацаанд догшин загалмайтнуудын олз болов, Абиссини нь исламд орсон галлас, сомаличуудад бүслэгдсэн уулын цайз болон хувирав. Байр сууриа хамгаалсан Никейн эзэнт гүрний амь тэмцсэн оролдлого энэхүү тарчлааныг зуун жилээр хойшлуулав. Гэхдээ XIY зууны дунд үе гэхэд л Палеологууд Баруунд бүрэн захирагдана гэдгээ харуулсан, өөрөөр хэлбэл Агуу Их Карлын байлдан дагуулалтын үндсэн дээр үүссэн, шинээр бүрэлдсэн соёлын тэрхүү бүхэллэгт орсноо харуулж Унийг хүлээн авчээ. (Уния гэдэг хоёр хаант улс нэг хааны захиргаанд орохыг хэлнэ–Орч). Чухам энэ бүхэллэгийг европын түүх судлалд эртний соёлын үргэлжлэл мэтээр авч үзэж заншсанаар энэ нь дунд сургуулийн сурах бичиг зохиоход ч тусах болсон байна. Гэхдээ “Эртний үе”-ийг “Дахин Сэргэлт”-ээс зааглаж буй мянган жилийг шилжилтийн үе гэхээсээ өмнө соёлын түүхийн бие даасан хэсэг гэж авч үзэх нь зөв юм. Тэр тусмаа католик рыцариуд болон прелат хэмээх дээд лам нар византийн соёлын ололтыг түүний грек болон ирланд хувилбараар нь ч өвлөн аваагүй бөгөөд зүгээр л үнсэн товрог болгосон юм.
Хэрэв ийм байх аваас Европийн Дахин Сэргэлтэд үйл явдлын дэс дараа, зүй тогтлын мөн л адил шугамыг хамааруулах ёстой бөгөөд үүнд нь түүнээс өмнө болсон загалмайтны аян дайнууд, мөн түүнээс хойш болсон колонийн булаан эзлэлтүүд орох ёстой. Үнэндээ ийм л байсан юм.
Баруун европын соёл нь үүссэн цагаасаа эхлэн өргөжин тэлэхийг эрмэлздэг байв. Барон агуу Их Карлын удмынхан баруун славян, англосакс, кельтүүдийг эзэлж, Пиренейн хагас арлаас арабуудыг шахан гаргаж, Энэтхэгийн далайн сав газрын мусульмануудын эсрэг дайн дэгдээсэн юм. Дундад зууны бюргерийн удмынхан Америк, Автрали, Өмнөд Африкийг булаан эзэлсэн. Тэдгээрийн аль аль нь Энэтхэг, халуун бүсийн Африк, өмнөд Америк, Полинез зэргийг дайлан авсан. Энэ бол орон зайн тэлэлт байсан юм. Харин гуманистууд юу хийж байсан бэ ? Тэд олз ашиг олох гэсэн мөн л тийм сэдлээр хөдөлж байлаа. Тэдэнд өнгөрснийг булаан эзлэх зорилго, тэгэхдээ өөрийн биш, хүнийхийг авах зорилго тавьсан юм. Тэдний хүчин чармайлтын үр дүн нь бусад соёлуудад адилтгал байхгүй үзэгдэл болох филологи буюу бичиг дурсгал судлахад суурилсан Дэлхий нийтийн түүх болсон бөгөөд ингээд түүх нь газар сайгүй, албатай мэт өөрийнхөө өвөг дээдсийг бичих, өөрөөр хэлбэл туйлшруулсан генеалоги буюу овог угсаагаа судлах явдал болсон юм. Хэрэв ийм байвал “Хятадын Дахин Сэргэлт” нь европынхоос зарчмын ялгаатай байх ёстой, харин төстэй шинжүүдийг нь тохиолдлын давхцал гэж үзэх ёстой. Гэтэл Н.И.Конрад үүний эсрэг үзэл барьж байна, иймээс энэхүү тулгуур асуудлыг шийдвэрлэхийн тулд Дорнод азийн түүхийг анхаарах болох нь.
ХЯТАДЫН ТУХАЙ
Эхлээд Дорнод ази бол газар тариалангийн Хятад, Төвдийн уулархаг нутгийг оруулсан Төв Азийн нүүдэлчид угсаатан ландшафтын хоёр мужид байрладаг гэдгийг тэмдэглэе. Хятад нь шигүү, түрэг болон монголын тал нутаг нь сийрэг хүн амтай ч гэсэн энэхүү соёлын бүс нутгууд нь түүхийн бүхий л үеийн туршид тэгш үндсэн дээр харилцаж байсан билээ. Энэхүү зогсдоггүй тэмцлийг харгалзахгүйгээр Азийн түүхийг ямагт буруу тайлбарлах болно.
Зураг № 1. Хятад ба Төв Ази
Хятад ба Төв Ази
Хятадын соёл тогтвортой бөгөөд зогсонги, харин унаж өгсөх хөгжил нь Баруун Европын ололт болох тухай хар аяндаа ойлгомжтой мэт үзэл бодол өнгөрсөн зуунд оршсоор байсан юм. Энэ үзэл баримтлал нь аберраци хэмээх алслалтын гажуудлын жишээ бөгөөд ийм үед жишээлбэл нар гэхэд тавтын мөнгөнөөс бага харагдаж болно. Хятадын түүхийг хангалттай нарийн судлаад үзэхэд энэхүү алслалтын гажуудал нь утаа мэт замхарч, Дорно болон Өрнө дэх бурангуй эрин үеүд болон уламжлалын тасралт адил байдлаар явагдсан нь тодорхой болдог. Түүхэн хөгжлийн тасралтат шинжийг эртний агуу их түүхчид Полибий, Сыма Цянь хоёр тэмдэглэсэн бөгөөд хоёулаа ажиглан буй үзэгдлийг тухайн үеийнхээ шинжлэх ухааны хөгжлийн түвшингөөс үүдэн тайлбарлахыг санал болгосон байдаг. 9. Там же. С. 54- 88.
Сыма Цянь өөрийнхөө “Түүхэн тэмдэглэл”–ийг НТӨ I зуунд бичихдээ түүний хувьд “эртний”, өөрөөр хэлбэл өнгөрсөн уламжлалаасаа тасарсан тийм үе байсныг нэгэнт тэмдэглэсэн байдаг. Сыма Цяны хувьд эртний үе нь “Ся, Инь, Чжоу анхны гурван гүрний эрин үе бөгөөд Чжоу унаснаар улс төр, соёлын задаргаа дагалдсан аж. “Гурван хаант улсын зам нь эргүүлэг адил болж: тэр нь дуусч, бас ахин эхэлж байлаа”. 10. Там же. С. 76.
Мэдээж энэ нь Хань гүрэн эртний үеэ дахин давтаад байсан гэсэн хэрэг биш билээ. Үгүй байх. Энэ нь өөрийн салбар шинжүүд бүхий бүрэн бие даасан үзэгдэл байсан байна. Адил байдал гэдэг нь Сыма Цяны бодлоор жинхэнэ бодит байдал байгаагүй, харин тэрээр түүхийн жам ёсны хууль гэж үзэж байсан дотоод зүй тогтол байжээ.
Түүхчдийн нээсэн зүй тогтол өнгөрснийг тайлбарлах төдийгүй, мөн таамаглал хийх боломж олгож байна. Хэрэв өнө эртний Хятад зайлшгүй дотоод хэмнэлээсээ болж задарсан ахул орчин үеийн Сыма Цяны хувьд ч, бидний хувьд ч Эртний Хятад, өөрөөр хэлбэл Хань гүрэн энэ хувь заяанаасаа зайлхийж чадахгүй байсан. Мэдээж хэрэг Сыма Цянь улс орныхоо мөхлийн нарийн ширийнийг урьдчилан хэлж чадахгүй юм, гэхдээ үр дүн нь нэг утгатай байх ёстой байсан. Тэгээд ч энэ нь болдгоороо болсон билээ. III зууны иргэний дайн Хятадыг үгүйрүүлж, харин 312 онд Тэнгэрийн дорх эзэнт гүрний нийслэлийг Хүнгийн цөөн тооны цэрэг гэнэдүүлэн эзэлж, улмаар Шар мөрний сав газрын уугуул хань газар нутгийг өөртөө захируулсан юм. Хятадын хамгийн тууштай эх орончид Хөх мөрний сав харь хязгаар руу зугтан одсон юм. Харин эртний хятадын соёлын амь тэмцэлт тэнд бүр 250 орчим жил буюу Римийн мөн адил амь тэмцэлтээс бараг хоёр дахин урт удаан үргэлжилсэн билээ. Харин Хятадын ард түмний эх оронд энэ бүхий л үед нүүдэлчид, уулынхан, хүнчүүд, табгачууд болон кянууд (төвдүүд) хэрцгийлж байлаа.
Хятадын шинэ сэргэлт YI зуунд болжээ. Хятадын хэт эх орончдын удирдагч, жанжин Ян Цзянь нүүдэлчин ноёдоос төрсөн удмынхантай тооцоо бодож, Суй гүрийг байгуулсан юм. Энэ бол дундад зууны Хятадын “үүрийн гэгээ” байсан бөгөөд харин “үдшийнх” нь манжууд Мин гүрний цэргүүд болон Ли Цзычэний боссон тариачин цэргийг ялсан XYII зуунд болсон юм. Тэгэхэд л уналтын үе эхэлсэн ба үүнийг нь европын юм анзаардаггүй эрдэмтэд Хятадын байнгын төлөв байдал хэмээн үзэж, “зогсонги” үе гэж зааж өгсөн билээ. Сыма Цяны үзэл баримтлал батлагдав.
Гэхдээ Баруунтай харьцуулахад Дорнодод соёлын харьцангуй ихээхэн уламжлалыг хангадаг иероглиф буюу дүрс үсгийн бичиг байсан юм. Тэр нь цагаан толгойн бичиг үсэгтэй харьцуулахад ихээхэн дутагдалтай байсан хэдий ч семантем буюу агуулгын тал нь ойлгомжтой байдлаа хадгалж, хөгжин буй хэлний авиан зүй солигдоход ч, үзэл суртлын төсөөлөл өөрчлөгдөхөд ч хэвээрээ байдаг аж. Бичиг үсгээ мэдэх цөөн тооны хятадууд Күнз болон Лао–Цзыгээ уншиж, Библийг үгчлэн судалсан дундад зууны лам нараас ч хавьгүй илүүтэйгээр тэднийхээ сэтгэлгээний яруу сайхныг мэдэрч байдаг юм. Учир нь үгнүүд нь: а) орчуулгаас, б) дуудлагаас, в) уншигчийн мэдрэмжээс, г) түүнийг холбох системээс хамааран утга санаагаа өөрчилж байдаг. Харин дүрс үсэг бол математикийн тэмдэг адил нэг утгатай. Ийм учраас Хятадын дотоод дахь соёл хоорондын тасралт нь эртний (грек–римийн), дундад зууны (роман-германы) соёл хоорондын буюу дундад персийн болон арабын, өөрөөр хэлбэл мусульманы гэх мэт соёлын ялгаануудаас хэд дахин бага байдаг юм.
Энэ нөхцөл байдал нь Хятадын түүхэнд улс төрийн хувьд ч, үзэл суртлын хувьд ч туссан байдаг. Чухамхүү төстэй байдлын энэхүү гадаад шинж нь Хятадын зогсонги байдлыг дүрс үсгийн бичгийн хөшүүн байдал гэж үзэн үндэслэл болгодог түүхчдийг төөрөгдөлд оруулсан нь онцгой чухал билээ. Үнэн хэрэг дээрээ Хятадын түүх нь Газрын дундад тэнгисийн сав газрын орнуудын түүхийг бодоход багагүй эрчимтэй хөгжиж байлаа. Гэхдээ Хятадын газар нутаг дээр үүсч, устаж байсан угсаатнуудын амьдралын гайхам хичээл зүтгэлийг харахын тулд урлагийн зүйлийг шохоорхох, хийсвэр сэтгэлгээний мушгиралтаас хөндийрөн салж, Цагаан хэрмийн зааг дээр Их Талын нүүдэлчидтэй хийсэн мянган жилийн дайны перитети буюу гэнэтийн явдлуудыг дэс дараатай мөшгөх ёстой. Бидний “Талын гурвал” (трилогия) хэмээх ном маань энэ сэдэвт бүхэлдээ зориулагдсан билээ.
XIII. Угсаатны түүхийн тухай бодлууд
ЭТНОЛОГИ ДАХЬ ТОДОРХОЙ БУСЫН ЗАРЧИМ
Дээр авсан бүх жишээнүүд нь соёл-түүхийн сургуулийн зөв болохыг баталж буй мэт байна, гэхдээ түүнд өөрт нь эсрэг үзлийг зөв хэмээн батлах нэг зүйл байдаг юм. Салбар онцлогтой хэдий ч бидний авч үзсэн бүх “соёлууд” адилхан байдлаар хөгжиж, мөхсөн байдаг. Энд ерөнхий диалектик үйл явцыг авч үзэх ч хэрэг байхгүй.
Хэрэв ийм байх ахул бид тавьсан зорилтоо шийдэх нь битгий хэл, түүнийг бүр хүндрүүлчихсэн юм биш үү ? Үгүй. Урагшлах зам бий, бид зөвхөн бусад шинжлэх ухаанд хандсан үед л тэр нээгдэх болно. XII зуунаас эхлэн гэрэл эгэл хэсгээс (корпускул) бүрдэг үү, эсвэл эфир дэх долгион уу ? гэсэн асуудлаар маргалдаж эхэлсэн билээ. Энэ хоёр үзэл баримтлал аль нь ч давамгайлж чадахааргүй тийм ноцтой дутагдлуудтай байсан юм. Энэ маргааныг XX зууны 20-иод оны дундуур квантын механик үүссэнээр шийдвэрлэсэн юм. Орчин үеийн физикчид гэрлийг долгион ч биш, эгэл хэсэг ч биш бөгөөд аль аль нь нэгэн зэрэг хоёр бүлгийн илэрхийлж чадна гэж үздэг. Энэ үндсэн дээр зэрэгцсэн хоёр физик хувьсагч (жишээлбэл, импульс ба координат буюу эрчим хүч ба цаг хугацаа) байгаа үед хоёулаа зэрэг биш, аль нэгнийх нь утгыг тогтоож болдог гэсэн ихээхэн нэрд гарсан тодорхой бусын зарчмыг томъёолжээ.
Угсаатны үзэгдэлд ч гэсэн нийгмийн болон биологийн гэсэн хөдөлгөөний хоёр хэлбэр бас л илт байдаг. Эндээс судалгааны аль нэг аспектаар, аль нэг аргаар нийлмэл үзэгдлийн эсвэл энэ, эсвэл нөгөө талыг нь тодорхойлж болох байна. Энэ үед тодорхойлолтын нарийн шинж, түүний олон талт байдал нь бие биенээ харилцан үгүйсгэдэг. Үүнийг тэмдэглээд тодорхой бусын зарчмыг манай материалд ашиглая.
Юуны өмнө аспектаа сольё. Хоёр сургуулийн аргуудыг нэгтгэхийн оронд тэдгээрийг хэрэглэх хүрээг хязгаарлая. Шууд ажиглагдан буй түүхэн үзэгдлийг соёл–түүхийн зарчмаар бүлэглэж, бүх нийтийн–түүхийн бүдүүвчийг баримтуудад албаар тааруулъя. Мөн түүнчлэн шууд ажиглалтад ордоггүй үзэгдлийн мөн чанар байдаг нь тодорхой, үүнийг салбар шинж нь шигшигдэж ялгагдсан үед бүх “нийтийн” түүхийн мэдэлд оруулна, харин соёл–түүхийн сургуулиар батлагдсан хөгжлийн дискрет чанарыг (тасралтат) нэгдмэл, гэхдээ маш нарийн үйл явцын чанаруудын зүгээр л нэг хэсэг гэж үзье.
Дараа нь хандлагаа сольё. Өнгөрсөн зуунд түүхийг гимназид газар зүйтэй хамт, дээд сургуулиудад хэл шинжлэлтэй хамт хоёрдмол байдлаар судалж байсан. Манай үед юмсыг ойлгох хоёрдахь арга нь давамгайлж, сурвалж судлалыг гол болгох болсон юм. Гэхдээ энд маш том аюул үүсдэг: түүхч хүн сурвалжийг зохиогчийн олзонд орох эрсдэл үүсч, сэдэвт аль болох ойр агуулгыг дамжуулахыг хичээхдээ уншсан зүйлээ зүгээр л дахин ярьж эхэлдэг. Эртний зохиогч бидний хувьд хүлээн авч болохооргүй үзэл санааг удирдлага болгосон, мөн түүнийг уншигчид нь биднээс өөр ойлголтын системтэй байсан, тэдний бичсэнийг манай үеийн уншигчид шиг хүлээн авдаггүй байсан нь магадтай юм. Хэрэв Геродот юмуу Рашид ад –Дин бидэнд зориулж бичсэн бол тэд мөнөөх үзэл бодлоо өөрөөр өгөх байхсан. Эцэст нь эртний сурвалжийг зохиогчид өөрийнх нь үед тодорхой, улиг болсон үнэнийг зүй ёсоор орхино. Харин бидэнд болохоор эдгээр нь тодорхойгүй, нэн сонирхолтой байх болно. Ийм учраас хойч үеийнхэнд зориулсан сурвалж–криптограмм (нэг аргаар буюу нэг гараар хийсэн баримт–Орч) бүрийн жинхэнэ утга санааг сэргээн босгоно гэдэг бол хүнд бөгөөд дандаа биелэгдэхгүй ажил мөн. Энд “Игорийн хорооны дуудал”- ыг тойрсон маргааныг санахад л хангалттай. Эдүгээ оршин буй таамаглалт тайлбарлалд түүн шиг үндэслэгдсэн бөгөөд итгэл татах өөр нэлээд баримт нэмэгдэнэ гэдэгт ямар ч баталгаа байхгүй юм. Товчоор хэлбэл, сурвалж судлалын асуудал бидний хувьд түүхэн үйл явцыг сэтгэн ойлгох гэж тавьсан зорилтоосоо хэзээ ч эргэж ирэхээргүйгээр холдон хөндийрөх шилдэг арга болно.
Харин гимназийн арга зүй бол өөр хэрэг. Сурвалжаас тэнд нь ямарч маргаангүй, нүцгэн баримтуудыг ялган авна, тэдгээрийгээ цаг хугацаа, орон зайн дэвсгэр дээр тавина. Байгалийн шууд ажиглалтаас материалаа олж байгаа бүх байгал судлалынхан яг ингэж ажилладаг. Чингэхэд өөрийн дотоод логик бүхий сэдвээс салсан баримтууд гарч ирнэ, эдгээр нь статистик зүй тогтолд захирагдана, адил болон ялгаатайн зэргээрээ бүлэглэгдэнэ, үүний ачаар л тэдгээрийг харьцуулсан аргын замаар судлах боломжтой болдог.
Энэ хандлага нь түүхэн ахуй–түүхэн бүхэллэгийн нэгэнт тэмтэрсэн эталоныг эрэгцүүлэх болгодгоороо ач тустай юм. Гэхдээ юуны ? Одоо бол салбар хоорондоо холбоо бүхий үйл явдал, үзэгдлийн хэлхээ нь шалтгаант чанараар дамжин хэрэгжинэ хэмээн хариулж болно. Энэхүү гинж нь эхлэл бөгөөд төгсгөлтэй байдаг, өөрөөр хэлбэл орчны эсэргүүцлээс унтарч байдаг инерцийн тэсрэлтийн шинжтэй байдаг гэдгийг шууд ажиглалт үзүүлж байна. Бүх маргаангүй ажиглалтыг тайлбарлагч, соёл–түүхийн сургуулийг нэгтгэн дүгнэсэн механизм нь энэ байна.
Гэхдээ тэсрэлт хаанаас гардаг, яагаад инерцийн үйл явцууд нь өөр хоорондоо ийм гайхалтай төстэй байдаг вэ ? Энэ асуултад бүх нийтийн – түүхийн үзэл баримтлал хариулах ёстой, гэхдээ харамсалтай нь түүхийн шинжлэх ухаанд байгаа тэрхүү хэрэгслүүд нь түүнийг зөвхөн дүрслэх боломжийг л олгодог юм. Хүмүүнлэгийн шинжлэх ухаанд дүрслэн бичих нь хязгаар юм, манай үед хуурамч гүн ухааны замаар тайлбар өгөх нь хэний ч сэтгэлд нийцэхгүй. Ингээд байгалийн шинжлэх ухааны бааз суурь дээр бүрэн шилжих болон “соёл” (аль нэгэн) гэсэн ухагдахууныг гүйцээх тухай, дүрсэлсэн өөрчлөлтөд өртсөн тэрхүү материаллаг субстанцийн тухай асуудлыг тавих явдал л үлдэж байна.
ТООЦООЛЛЫН ХОЁР СИСТЕМ
Ажиглан буй баримтыг хамгийн энгийн, хүрэлцээтэй тайлбарлах гэвэл түүнийгээ аль нэгэн үйлдвэрлэлийн аргад үндэслэсэн аль нэгэн формацитай харьцуулах оролдлого хийх явдал гэж хамгийн түрүүнд толгойд бууж байна. Энэ замаар Н.И.Конрад явсан бөгөөд тэрээр “Боол эзэмшлийн формаци бол яг боолчлолоороо биш, ард түмний түүхийн тухайн үе шатанд буй нийгмийн ахуйн эдийн засгийн үндсийг тодорхойлдог үйлдвэрлэлийн аргад боолын хөдөлмөр үүрэг гүйцэтгэж байгаа нийгмийн байгууллаар тодорхойлогдоно” гэсэн товч тодорхойлолт өгсөн байна. Энэ үе шатыг тэр “эрт үетэй” буюу бүх дэлхийн эртний түүхтэй шууд харьцуулж байна. 11. Там же. С. 33.
Мөн л ийм амархан “Дундад зуун” гэсэн ойлголтыг “феодализм бий болох, батжих, цэцэглэх үе” хэмээн тодорхойлж, ахиад л бүх Ойкуменд хамааруулж байна. Энд шинэ зүйл гэвэл зөвхөн нийгэм-эдийн засгийн категорийг үйл явдлын гинжин хэлхээний шалтгаан-үр дагаврын буюу зүй тогтлын хүрээнд дэлгэрүүлэх гэсэн оролдлого юм. Энэ нь дараах учраас буруу юм. Түүхэн материализмын онол нь спиралаар явагдах нийгмийн дэвшилтэт хөгжлийг тусгах гэж тусгайлан бүтээгдсэн болохоос биш, ерөөсөө улс гүрэн, цэргийн дэглэм солигдох, халдварт өвчин тархах буюу шашны үзэл баримтлалын өнгө аясыг тайлбарлах гэж хийгдээгүй юм.
Нийгмийн хөгжилд өөрийн гэсэн логик байдаг ба үйл явдлын дараалал нь ч бас өөрийн логиктой. Хоёр системийн хооронд харилцан холбоо ч бий, мөн урвуу холбоо ч бий. Чухам ийм зүйл байгаа нь энд тооцооллын нэг систем биш, дор хаяж хоёр систем байгааг харуулж байна. Ийм учраас “шилжилтийн үеүд” гэж нэрлэгдсэн тэр зүйлийг шинжлэх үед бидний дээр үзүүлсэнчлэн нэг “соёл” нь хоёр буюу гурван формацид, заримдаа нэг формацид байх нь байнга ажиглагддаг. Дараа нь “соёл” нь формациас хавьгүй том бөгөөд энэ нь ч гэсэн уг хоёр ойлголт давхцдаггүйг харуулж байна. Тэгээд хамгийн гол нь тооцооллын энэ хоёр систем өөр хоорондоо зөрчилддөггүй, харин ч нэг нь нөгөөгөө нөхөж байдаг юм.
Тайлбарлая. Египет, Вавилон, Эллада, Энэтхэг, Хятадад ажиглагдсан боол эзэмшлийн формацийн шинжүүд нь эдгээр нийгмийг таксоном дарааллын нэг эрэмбийн бүлэглэлд багтаах үндэслэл өгч байна, гэхдээ ямар ч тохиолдолд тэдгээрийг нийлэгжсэн уламжлалтай буюу бодитой ахуйжсан харилцаа холбоотой гэж баталж болохгүй юм. Харин ч дээр дурдсан улс тус бүр “соёл” болохынхоо хувьд нийгмийн хөгжлийн огт өөр түвшинд байгаа хөршүүдтэйгээ харилцаж байдаг. Жишээлбэл, Афин, Коринф зэрэг боол эзэмшлийн төвүүд нь газар тариалангийн Фиви, мал аж ахуйн Этолий болон Фессали, тэр ч байтугай овгийн байгуулал нь задран байсан Эпир, Македонитой нэгэн бүхэл бүрдүүлж байлаа. Гэхдээ энэ бүгд хамтдаа эртний грекүүд өөрсдөө бүхэллэг гэж үзэж байсан Эллада болдог юм. Тэгээд яагаав ?
Эртний Хятадад ч мөн л ийм байсан. Бүр “Байлдаант улсуудын” эрин үед Цинь болон Чу гэх хязгаарын улсууд Сычуань болон Зүүн өмнөд Хятад дахь ойн тархай бутархай овгууд, түүний дотор эртний малайн угсаатан Юе нарыг захирч байлаа. Тэдний болон хятадуудын нийгмийн байгууламж янз бүр байсан ч түүхэн хувь заяа нь нэгдмэл байлаа. Ийм учраас тэдгээрийг янз бүрийн формацид, гэхдээ хань гэсэн нэгэн бүхэллэгт оруулан тооцвол зохино. Харин газар нутгаа хятадын түрэмгийллээс хамгаалан тогтоож чадсан хүнчүүд бол аль ч талаараа (аспект) хятадуудаас ялгаатай ард түмэн юм аа. Тэднийг Евроазийн нүүдлийн бүхэллэгт хэдийгээр эдгээр ард түмэн өөр хоорондоо арьстны ч, хэлний ч талаараа ялгаатай байдаг ч усун, хагас нүүдлийн ди, хагас суурин сармат, газар тариалангийн динлин нартай хамт оруулан тооцох нь хамгийн зөв юм. Яагаад тэд нар бүхэллэг болох гэж ?
Судалгааныхаа төгсгөлд үүнд хариулахын тулд энэ асуултыг нээлттэй үлдээе. Ийм зүйлд л хариулах гэж энэ судалгааг хийж байгаа юм шүү дээ. Одоохондоо нийгмийн болон угсаатны гэсэн тооцооллын системд ялгаа байдгийг тэмдэглэснээр хязгаарлаж, материаллаг болон оюуны соёлын түүхэнд угсаатны нийлэгжилтийн асуудлуудын шийдвэрийг олох гэсэн хандлага нэг бус удаа үүссэн учраас соёлын үйл явцын асуудалд анхаарлаа хандуулъя.
СОЁЛЫН ТҮҮХ БА УГСААТНЫ НИЙЛЭГЖИЛТ
Угсаатны үйл ажиллагаа нь түүний гар болон ухааны бүтээлд, өөрөөр хэлбэл соёлд шингэсэн байдаг учраас бид салбар соёлын түүхийг судлаад л түүнийг бүтээгч угсаатнуудын түүх, тэр тусмаа угсаатны нийлэгжилтэд хамтад нь хүрнэ хэмээн бодож болох юм.
Хэрэв энэ үнэн бол судлаачийн зорилт маш их хөнгөрөхсөн билээ, гэвч харамсалтай нь угсаатны нийлэгжилт, угсаатны түүх болон соёлын түүх гурвын хооронд ийм холбоо байдаг ч гэсэн энэ нь дагалдах үзэгдлүүдээр халхлагдаж, бүх гурван тохиолдолд янз бүр болдог. Түүхэн синтезийн тусгай арга хэрэглэлгүйгээр харагдаж буй мэт соёлын түүхээс эхэлье.
Соёл, угсаатан, дээр нь хэт угсаатан гэсэн ойлголтууд давхцдаг уу ? Ийм тогтлыг баталсан тухайн тохиолдлыг эс тооцвол ёс мэт үл давхцана. Энгийн, бүхэнд тодорхой жишээ –Эллад дээр гэхэд л энэ бүхэн тодорхой харагдана.
Эллиний хот улсын соёл нь эх газрын Грект ч, түүний колонид ч бүр НТӨ YI –YII сонгодог үед эллиний биш газар нутаг, жишээлбэл, “эллиний хэрэг явдлыг” хамгаалагч, жолоодогчийн үүргийг Александр өөртөө авсан Македонд тархсан юм. Эллиний соёлын удаах тархалт нь Ойрхи Дорнод, Египет, Дундад ази, македончуудын эзэлсэн Энэтхэг, түүнчлэн эллиний соёлыг Афинаас зээлдэх замаар авсан Лациум зэрэг улс орон, ард түмнүүдийг хамаарна. Энэ бүхэн нь “эллинизм” гэж нэрлэгдэх зүйл, өөрөөр хэлбэл агуу их хэт угсаатны бүрэлдэлт билээ.
Гэхдээ эллиний соёлыг хүлээн авсан угсаатнууд бүгдээрээ энэ хэт угсаатанд орж чадаагүй юм. Парфянчууд грекээр ярьж суран, хаадынхаа ордны өмнө Еврипийн гунигт дүрийг тавьж, хотоо эллиний уран барилгачдаар бэхэлж, тэдгээрийгээ афин, милетийнхтэй адил уран баримлуудаар чимж байсан ч гэсэн Ираныг эзэмшигч, македоны дайсан–сирийн Селевкидүүдийн “туранчууд” хэвээрээ үлдсэн юм. 12. Иран болон Тураныг сөргүүлэх, өөрөөр хэлбэл зороастризмыг авсан суурин арийчууд болон дэв-үүдийн культаа хадгалан үлдсэн талын арийчуудыг сөргүүлэн тавих нь YII зууны арабын булаан эзлэлт хүртэл ач холбогдлоо алдаагүй байсан юм.
Карфагенчууд эллиний хот улсын хэв маягаар зохион байгуулагдсан ч оршин суугчид нь сири, бага азийнхнаас ялгаатай нь грекчүүдтэй төстэй болоогүй юм. Харин римчүүд Элладыг эзлэн аваад түүний соёлыг өвлөгч, хадгалагч нар болсон бөгөөд өөрийнхөө угсаатны шинжийг орон нутгийн онцлог байдлаар хадгалан үлдсэн байна. Тэгээд тэд эллиний соёлыг өөрийнхөө бүх хязгааруудад тарааж, Римийн улс төрийн хүчин чадал унасны дараа европын романууд болон герман угсаатанд хэсэгчлэн тараасан юм.
Ийм маягаар соёлын түүхийг судлахдаа бид угсаатны хил хязгаарыг байнга нааш цааш хөдөлгөсөн уламжлалын тасралтгүй шугамыг олж харж байна. Герман болон славяны удмынхан хэдийгээр эртний бурхдыг нь шашны газар биш, хошин дуурьт оруулсан боловч геометр, Платон болон Аристотелийн идеалист гүн ухааны систем, Гиппократын анагаах ухаан, сонгодог барилгын урлаг, театр, утга зохиолын төрлүүд, хууль зүйн хэм хэмжээ–римийн эрх, тэр ч бүү хэл домог зүйг эзэмшсэн юм. Агуу их соёл өөрийг нь бүтээсэн угсаатнаасаа илүү насалсан гэсэн хэрэг юм. Орон зайн хувьд ч, цаг хугацааны хувьд ч үл давхцал тодорхой байна.
Гэхдээ дасал болсон ч гэсэн соёлын “урт наслах” гэсэн нэр томъёог хэрэглэх нь зөв үү ? Соёл бол техникийн бүтээл, урлагийн сор, гүн ухааны систем, улс төрийн номлол, шинжлэх ухааны үзэл баримтлал, эсвэл өнгөрсөн зууны тухай энгийн домог байсан ч гэсэн хүмүүсийн бүтээл юм. Соёл оршин байдаг, амьдардаггүй. Учир нь түүнийг хүмүүсийн бүтээлч эрчим хүчээр хөтлөхгүй бол соёл нь эсвэл хадгалагдана, эсвэл эвдрэн сүйрнэ. Гэхдээ энэ “үл амьдрахуй” нь түүнийг бүтээгчдийн ухамсарт нөлөөлж, түүнээс гайхамшигтай хэлбэр гарган авч, дараа нь хойч үеийнхэн нь хүлээн авахаа болих хүртэл нэгэн хэвээр байдаг юм. Сүүлчийн энэ зүйлийг нь “зэрлэгшил” гэж нэрлэж заншсан бөгөөд энэ нь Римийн эзэнт гүрэнд байдаг олимпийн бурхад шиг нэрээ гутаасан, эртний ертөнцийг үзэх үзлийн хэм хэмжээний ач холбогдлоо алдсан, хуучирсан зүйлээс чөлөөлөгдөж буй хэрэг биш юм. Бүр НТӨ I зууны үед бурхдын баримал бүх замын уулзвар дээр шовойж байсан ч тэдэнд хэн ч итгэхгүй болсон байлаа. Эллинчүүд ба римчүүд янз бүрийн тэмдгийг сахиж, өөрийнхөө жанждыг зөвхөн долигнуур зангаараа бурхадтай адилтгаж байв, гэвч увайгүй болон их зантангууд хоосорсон бэлэгдлээ хадгалан үлдсэн ба учир нь соёлоо алдахын өмнөх айдас нь түүнийг сэдрэхээс илүү хүчтэй байлаа. Хүмүүс ямар нэг зургаа дахь мэдрэхүйгээрээ: соёл нь их хүнд, гэхдээ түүнгүйгээр амьдарч болохгүйг таамаглаж байсан юм. Ийм ч учраас хамгийн гүнзгий уналт ч соёлын түвшинг тэр болтол нь буулгаагүй билээ. Харин цаг хугацаа өнгөрөхөд шинэ өргөлт эхлэв…Эртний бус соёл байхгүй, харин газраас хуучин хэлтэрхий сонгон авч, тэдгээрийг өөрийнхөө хэрэгцээнд зохицуулан, түүгээр шинэ зэвсэг хийдэг угсаатан байдаг юм. Соёл өөрчлөгдөх бүдүүвч ийм л бөлгөө.
Харин угсаатны нийлэгжилт яах вэ ? Энэ бол түүнгүйгээр соёлыг бүтээх буюу дахин сэргээх боломжгүй тийм нөхцөл юм. Соёл нь хүмүүсийн гарын бүтээл бөгөөд харин манай үед угсаатангүй хүн гэж байхгүй. Угсаатан бүтээх, түүнийг хөгжүүлэх, өөрөөр хэлбэл угсаатны нийлэгжилт нь зогсох гэж буй моторт гүйдэл залгахтай адил эд бөгөөд үүний дараа тэр дахин ажиллаж эхэлдэг.
Угсаатны нийлэгжилт нь байгалийн үйл явц бөгөөд ингэхлээр соёл бий болсны үр дүнд бүрэлдсэн нөхцөл байдлаас хамаардаггүй. Энэ нь дурын үед эхэлж болох ба хэрэв түүний замд үйлчлэн буй соёлын бүхэллэгээс саад үзүүлбэл тэр нь түүний эвдэлнэ, эсвэл түүнийг хагална. Хэрэв тэр “дэлхий доргилж буй” үед эхэлбэл үүссэн угсаатан өөрийнхөө орших, хөгжихийн арга болсон соёлоо бүтээдэг. Энэ хоёр тохиолдолд үсрэлт бол хэний ч ухамсраар үл жолоодогдох байгалийн эрчим хүчний сохор хүчин болно. Асуудлыг ингэж шийдвэрлэх нь дээр дурдсан зарчмуудаас зөрчилгүйгээр урган гарч байна.
Гэвч өөр үзэл бодол бас байдаг. “Угсаатныг бүрдүүлэгч социал хүчин зүйл, түүний дотор угсаатны өөрийн ухамсар нь түүнтэй хавсарсан бүл бий болоход хүргэдэг, өөрөөр хэлбэл Л.Н.Гумилевын өгч байгаагаас шууд эсрэг дүр зураг бууж байна” Ийм маягаар ухамсрын үндэс нь ахуй юу, эсвэл эсрэгээр ахуйн үндэс нь ухамсар уу? гэсэн маргаан болж байна. 13. Бромлей Ю. В. Этнос и этнография. С. 122-123.
Үнэхээр асуудлыг ингэж тавих аваас маргааны зүйл байгаа юм. Ингээд учрыг нь тунгаая.
Эрдэмтэн бүр өөрийн гэсэн логик байгууламжаа сонгон авахдаа дурын үндэслэл, тэр ч бүү хэл угсаатны бодит ахуй нь зөвхөн түүгээр төдийгүй, өөрийнх нь ухамсраар тодорхойлогддог гэсэн үндэслэлийг ч сонгон авах эрхтэй. Ингэвэл түүний үзлийг сүсэгтэй христиан ч, материалист ч хүлээн авахгүй нь үнэн юм. Материаллаг бодит байдлыг бүтээх үйлийг хүний ухамсарт ногдуулж, дээр дурдсанаар Ертөнцийг бүтээгч буюу түүний байранд тавих аваас үүнийг христиан хүн зөвшөөрөхгүй. Мөн материалист гүн ухаантнууд ухамсар анхдагч болох тухай энэ сэдвийг хүлээж авахгүй.
Тэр байтугай эмпирик эрдэмтэд ч энэ нь энерги хадгалагдах хуулийг зөрчиж байгаа учраас дээр дурдсан сэдвийг зөвшөөрөх эрхгүй юм. Угсаатны нийлэгжилт нь ажлаар (физик утгаар) илэрч буй үйл явц юм. Энд аян дайнууд болж, сүм хийд, орд харш байгуулж, ландшафтыг дахин өөрчилж, системийн дотно болон гадна буй үл зөвшөөрөгсдийг даран доройтуулдаг. Харин ажил хийхийн тулд хамгийн энгийн килограмаметр буюу калориар хэмжигдэх эрчим хүч хэрэгтэй. Ухамсар, мөн түүнчлэн угсаатныг зүйлийг ч эрчим хүчний үүсгэвэр болж чадна гэж үзэх нь зөвхөн уран зөгнөлд л байх хийсвэр бодит байдлыг зөвшөөрнө гэсэн хэрэг болно.
Үүнийг тайлбарлая. Чулуун хийцүүдийг пирамидын орой руу угсаатны өөрийн ухамсар биш, харин египетийн ажилчид “нэг, хоёр, гурваа…” гэсэн зарчмаар булчингийхаа хүчээр гаргасан. Хэрэв татлагыг нь египетчүүдээс гадна ливи, иуби, хананеяне ….нар татсан байлаа ч үүнээс болж ажил өөрчлөгдөхгүй юм. Тухайн тохиолдолд ухамсрын үүргийг угсаатны биш, хувийн–инженер-барилгачид гүйцэтгэж, өөрийнхөө мэдэлд буй хүчийг зохицуулж байна. Удирдах үйл явц болон эрчим хүчний ялгааны ачаар энэ үйл явц явагдаж байгаа нь тодорхой байна.
Өнгөрсөн эрин үеүд болон хүмүүст бас л янз бүрээр хоол тэжээл өгдөг ландшафтаас өвлөн авсан янз бүрийн суурь дэвсгэр дээр янз бүрийн угсаатны нийлэгжилтийг нийгмийн үйл явцтай хослуулах нь угсаатны түүхийг өөр хооронд нь гайхамшигтайгаар сүлжилдүүлдэг юм. Угсаатны нийлэгжилтээс ялгаатай нь угсаатны түүх бол янз бүрийн хүчин зүйлд өртдөг, түүнд маш нарийн хариу үзүүлдэг олон хүчин зүйлт үйл явц болно. Түүний хамт угсаатны шат солигдохтой холбогдсон үйл явдлуудыг эх сурвалжууд тэмдэглээгүй байдаг учраас угсаатны түүх нь соёл, төр, нийгмийн институт, ангийн тэмцлийн түүхийн адил тийм ч тодорхой байдаггүй. Угсаатны түүх бол био хүрээний газар зүйд юунаас ч илүү ойрхон байдаг, эрээн алаг байдлыг нь бүр Р.Груссе хүртэл тэмдэглэж байсан түүхэн шинжлэх ухаан болно. Тэрээр ХХ зууны дунд үеийн түүхэн дэлгэцийг одод тэнгэртэй зүйрлэсэн бөгөөд энд бид аль эрт сөнөсөн авч гэрэл нь дөнгөж одоо л Дэлхийд ирж буй оддыг ажиглаж байдаг ба харин хэт шинэ туяа сансрын орон зайд цацарч л яваа, улмаар түүнийг газрын одон орны төвүүд хүлээн аваагүй байгаа оддыг хараагүй л байна. Р.Груссе адилтгалаа үргэлжлүүлж, XIY зууны европтой адилтгам насандаа байгаа исламын орнуудыг “треченто” буюу эртний италийн соёл гэж нэрлэсэн. 1940 онд немцүүд Францад довтолсныг тэр Y зууны үеийн Аларих болон Гензерихийн аян дайнтай харьцуулсан, Японы цэргийг орчин үеийн дүрэмт хувцас өмссөн самурай гэж нэрлэсэн. Түүний бодлоор Скандинав нь үүний эсрэгээр ирээдүйд, XXI зууны босгон дээр байгаа аж. Эндээс Р.Груссе нийгмийн ч, угсаатны ч үйл явцыг авч үзээгүй бөгөөд зөвхөн түүний гоёж чимсэн тал, өөрөөр хэлбэл соёлын ялгаврыг л үзсэн нь тодорхой байна.
Гэхдээ хэрэв тэнцвэржсэн, хотожсон соёл иргэншлийн нөхцөлд, ХХ зуунд ч гэсэн францын дорно дахиныг судлаач ийм ихээхэн үл зохицол илрүүлсэн гэвэл үүнийг нь ерөнхий техно хүрээгээр бага шиг арилгасан бусад эрин үед эдгээрийн ач холбогдол бүр ч их байхсан билээ. Р.Груссе: “Бидний зовлонгийн ихэнхи хэсэг нь ард түмнүүд нэг эрин үед амьдрахдаа ерөнхий логикт ч, нэгдмэл ёс суртахуунд ч захирагддаггүйгээс болж байна” гэж үзэж байна. Угсаатны хөгжлийн жигд бус байдлыг Р.Груссе немцийн бөөнөөр хорих лагерь шиг муу ёрын олон дайнуудын шалтгаан гэж үзэж байна. Үнэхээр ч төрийн бодит шаардлагаар цагаатгагдаагүй, сэтгэлийн зовлонтой гэмшилгүй байхад ийм аймшигтай үйлдэл хийхийн тулд зөвхөн ухаан солиотноор л төсөөлж болох сэтгэл зүйн тийм бүтэцтэй байх хэрэгтэй. Гэхдээ энэ бүхэн нь тохиолдлын ганцаарчилсан гажилт биш, олон түмний тогтвортой сэтгэл хөдлөлд хамаарах угсаатны үзэгдэл юм. Ингэхлээр угсаатны нийлэгжилтийн энэхүү үе нь бидний хэм хэмжээ гэж хүлээн авсан, тооцооллын анхлагч цэг эхэлдэг тэр зүйлтэй давхцдаггүй гэсэн хэрэг юм. Гэтэл бид тооцооллыг нөгөө талаас нь эхлэх гэхээр бидний хэвийн гэж үзэж буй зүйл ухаан гажуу мэт харагдах болоод байна.
Ийм болох ахул нийгмийн түүхэнд нийгэм-эдийн засгийн формаци ямар байдаг шиг угсаатны түүхийг хэмжих ямар нэг жишиг хэмжээ (эталон) олох ёстой болж байна. Гэхдээ үүнийг шийдвэрлэх замд бас л маш түргэн, заримдаа угсаатнаасаа ч хурдан өөрчлөгдөж байдаг угсаатныг багтаагч газар зүйн орчинтой угсаатны харьцах харьцаа гэсэн нэмэлт бэрхшээл үүсч бидний зорилтыг хүндрүүлж байна. Энд Каллиопа (туульсын бурхан) хүч хүрэхгүй учраас өөрийн эгч Уран (Тэнгэрийн бурхан )–аас тусламж хүсэх ёстой.
УРАН БА КЛИО ( ТҮҮХИЙГ ТЭТГЭГЧ БУРХАН )
Тодорхой асуудлыг шийдвэрлэхэд газар зүйг ашиглах явдал нь эсвэл бүрэн нэг санал, эсвэл хорлонтой зэмлэлтэй тулгардаг, одоо ч тулгарсаар л байна. Нэг талаас хуурай тал газар нь аж ахуй, соёлоо бүтээхэд нь халуун бүсийн ширэнгэ ой шиг тийм боломж олгоогүй нь тодорхой, нөгөө ийм хандлагыг “газар зүйн детерминизм” (шалтгаацал-орч) гэж нэрлэдэг.
Үүнийг эхэнд нь тодорхой болгоё. XYII–XYIII зууны нэрт сэтгэгчид Боден, Монтескье, Гердер нар тухайн эрин үеийнхээ шинжлэх ухааны түвшинд нийцүүлэн хүний үйл ажиллагааны бүхий л илрэл, түүний дотор соёл, сэтгэл зүйн янз маяг, засаглалын хэлбэр гэх мэт нь янз бүрийн ард түмнүүдийн амьдран суугаа улс орнуудын байгалаар тодорхойлогдоно хэмээн үзэж байлаа. Манай үед энэхүү үзэл бодлыг хэн ч хуваалцдаггүй болсон бөгөөд харин ч угсаатны түүхэнд газар зүйн орчны ач холбогдлыг огт үгүйсгэхээс илүү гарахгүй “газар зүйн нигилизм” гэх урвуу үзэл баримтлал бий болгосон байна. 15. Калесник С. В. Общие географические закономерности Земли. М., 1970.
Гэхдээ асуудлыг өөрөөр тавиад үзье. Газар зүйн орчин нь нийгэм –эдийн засгийн формаци солигдоход нөлөөлдөггүй гэдэг нь маргаангүй, гэхдээ зуун дамжсан ган буюу дотоод тэнгисийн (Каспий) хэлбэлзэл нь эдгээрийн байгаа бүс нутгийн аж ахуйд нөлөө үзүүлэхгүй гэж үү? 16. Гумилев Л. Н. Место исторической географии в востоковедных исследованиях //Народы Азии и Африки. 1970, № 1. С. 85-94.
Жишээлбэл, YI–XIY зуунд Каспийн тэнгисийн түвшин 18 метр өргөгдсөн нь өмнөд, уулархаг эрэгт их биш нөлөөлж, харин хойд зүгт Хазаруудын оршин амьдарч байсан асар их талбайг живүүлсэн билээ. Энэ гай нь Хазарийн аж ахуйг сүйдэлсэн бөгөөд нэг талаас хазарууд эх орноо орхиж, Доны нутаг, дунд Волгоор нутаглахад хүргэсэн, нөгөө талаас 965 онд Хазарын хаант улсыг Оросууд бутниргэхэд хүргэсэн. Түүхэнд ийм адилтгам тохиолдол маш олон бий. 17. Гумилев Л. Н. Открытие Хазарии. М.. 1966.
Угсаатны түүхэнд физик газар зүй оролцох хэмжээг зүгээр л тодорхойлох хэрэгтэй мэт санагдаж байв ч үүний оронд газар зүйн зүгээр л сайн мэдлэг мэтээр ойлгож эхэлж байгаа “газар зүйн детерминизм”–ийг ямар ч үр дүнгүй зэмлэх болж байна. Энэхүү гунигт үзлийн шалтгааныг газар зүйн түүхч В.К. Яцунский: “Түүхчид газар зүйг тун сул мэддэг, эргээд газар зүйчид нь түүхийг бараг мэддэггүй” гэж заасан байдаг. 18. Яцунский В. К. Предмет и задачи исторической географии //Историк-марксист. 1941. № 5 (93). С. 21.
Энэ ч бас гайгүй. Харин “газар зүйч нь газар зүйн судалгааны салбараа орхимогцоо түүхээр оролдож эхэлдэг, тэр байгал судлаач байхаа болимогцоо өөрөө түүхч болдог” нь бүр ч долоон дор юм. Ийм маягаар азгүйдлийн үндэс “цэцэглэдэг”: энд асуудал дэвшүүлэх, судалгааны арга зүйг боловсруулаагүй байна. Ингээд эдгээрээр оролдож эхэлье. 19. Там же. С. 27.
Түүхчийн хувьд ажил нь дуусч буй тэр зүйл этнологи болон газар зүйчийн хувьд эхлэх цэг нь болдог. Дараа нь шалтгаан нь тодорхой байгаа тэр үзэгдлүүдийг ялгаж, эсвэл нийгмийн гэнэтийн хөгжлийн хүрээнд хамааруулах, эсвэл үйл явдлын өөрийнх нь логикт (улс төрийн зүтгэлтнүүдийн хувь үйлдэл) хамааруулах ёстой. Эдгээр үзэгдлийг газар зүйтэй холбох нь үр дүнгүй юм. Энд угсаатны нийлэгжилт болон миграцийн (их нүүдэл –Орч) хүрээ л үлдэж байна. Энд хүний нийгэм байгальтай харьцах харьцаа гарч ирж байна. Натурал болон энгийн таваарын аж ахуй гол үүрэг гүйцэтгэж буй үед энэ нь нэн тод харагддаг. Үйлдвэрлэлийн арга нь овгийн бүлэг буюу аймгийг тэжээж буй газар нутгийн байгалийн нөхцөлд байгаа эдийн засгийн тэрхүү боломжуудаар тодорхойлогддог.
Ажил үйлийн төрлийг нь ландшафт зааж өгөх ба энэ нь аажмаар үүссэн угсаатны бүхэллэгийн соёлыг тодорхойлдог. Тухайн угсаатан өөрчлөлт, нүүдэл, хөршүүд нь ниргэсний улмаас алга болох аваас эртний ард түмний зан араншинг болон улмаар угсаатны оршин байсан эрин үеийн байгалийн нөхцлийн тухай гэрчилдэг археологийн соёл–тухайн эриний хөшөө дурсгалууд үлддэг. Ийм учраас бид түүхийг улс төрийн шинж чанартай үйл явдал, физик газар зүйн өөрчлөлтөөр давуутай нөхцөлдсөн үйл явдал гэж ангилах боломжтой болж байна.
Дэлхийн бүх ард түмнүүд байгалийг ашиглах ландшафтад амьдардаг, гэхдээ ландшафт нь нэгэнт олон янз болохоор ард түмэн ч олон янз болдог. Тэд хүний оршин үржихүйн тогтцыг бүтээх замаар юмуу эсвэл ургамал, амьтны аймгийг өөрчлөх замаар гэхчлэн ландшафтыг хэчнээн ч хүчээр өөрчиллөө гэсэн угсаатан эсвэл амьдран буй, эсвэл хянаж буй тэрхүү газар нутгийнхаа байгалийн өгч чадах тэрхүү зүйлээр л тэжээгдэх ёстой болдог.
Гэвч дэлхий дээр өөрчлөгддөггүй юм гэж үгүй, ландшафт ч гэсэн үүнд хамаарна. Ландшафтууд нь угсаатны адил хөгжлийн өөрийн гэсэн хөдлөнги шинжтэй, өөрөөр хэлбэл өөрийн түүхтэй байдаг. Хэрэв ландшафт танигдахгүй болтлоо өөрчлөгдлөө гэхэд энэ нь хүний үйлдэл, цаг агаарын өөрчлөлт, газар хөдлөлтийн үйл явц, эсвэл тахал тээгч хөнөөлт микроб бий болсоны алинаас үүдсэн эсэх нь огт ялгаагүй бөгөөд хүн эсвэл шинэ нөхцөлд дасан зохицох хэрэгтэй, эсвэл үхэх ёстой, эсвэл өөр газар руу дүрвэх ёстой болно. Энд бид миграцийн асуудалд шууд тулж ирлээ.
Ландшафтын янз бүрийн өөрчлөлт нь миграцийн цорын ганц шалтгаан биш. Эдгээр нь мөн хүн ам зүйн тэсрэлтийн үед буюу маш ховор нийгмийн түлхэлтийн үед үүсдэг. Гэхдээ энэ үед эдгээр нь анхныхаасаа шинж чанараараа ихээхэн ялгаатай байдаг учраас тэдгээрийг эндүүрэх бараг боломжгүй. Гэхдээ ямар ч тохиолдолд шилжин ирэгсэд эх орондоо дасаж дадсантайгаа төстэй нөхцлийг хайдаг юм. Англичууд зөөлөн уур амьсгалтай улс орнуудад, ялангуяа хонь үржүүлж болох Хойт Америк, Өмнөд Африк, Австралид дуртайяа тархан суурьшсан юм. Халуун бүсийн нутгууд тэдний дурыг татсангүй, тэнд англичууд голдуу колонийн түшмэд, худалдаачдын үүргийг гүйцэтгэж, өөрөөр хэлбэл байгаль ашиглаж биш, орон нутгийн хүн амыг ашиглагч хүмүүс болсон юм. Энэ нь мөн л миграци боловч огт өөр шинж чанартай болно. Испаничууд халуун бүсийн ойг анхааралгүй үлдээж, харин хуурай, халуун уур амьсгалтай нутгуудыг колоничлосон юм. Тэд ацтекуудын хүчин чадлыг нугалсан мексикийн намхан уулсад сайн идээшсэн юм. Харин Юкатаны майя нар Мексикийн засгийн газрын эсрэг “арьстны дайнд” тусгаар тогтнолоо хамгаалж, халуун бүсийн жунглидээ үлдсэн билээ. XI зуунд якутууд Лена мөрний хөндийд нэвтэрч, Байгал нуурын эрэг дэх өмнөх амьдралаа дуурайж тэнд адуу үржүүлэх болжээ, гэхдээ тэд тайгийн уулсын хагалбар хэсгүүдийг хөндөлгүйгээр эвенкүүдэд үлдээсэн байна. XYII зууны Оросын газар нээгчид бүх Сибирийг нэвт туулахдаа зөвхөн голын эрэг, тайгын ойн зах зэрэг газрууд, өөрөөр хэлбэл тэдний угсаатны өвөг дээдсийн амьдарч байсантай төстэй ландшафтуудад оршин суух болжээ. Яг ийм маягаар XYIII–XIX зуунд хуучин “Зэрлэг тал” байсан уудам орон зайг украинчууд эзэмшсэн байна. Манай үед ч гэсэн эх орноо орхисон түвдүүд цэцэглэн буй Бенгалиас Норвегийг илүүд үзэж, Осло хотноо өөрийн колонийг үндэслэжээ.
Томоохон миграци нь зөвхөн түүхийн үзэгдэл төдийгүй, мөн газар зүйн үзэгдэл болно. Иймээс тэр нь ландшафтын антропогенийн зарим өөрчлөлттэй ямагт холбогддог юм. Ийнхүү бид тавигдсан асуудлын шийдлийг хайхад бидэнд туслах ёстой түүхэн газар зүйн хилд явж ирлээ. Энэ нь юу өгөхийг харцгаая.
