\chapter{Хоёрдугаар хэсэг}

\section{УГСААТНЫ ЧАНАР}
ЭНД ЮУНД УГСААТАН ҮҮСЧ, БАС АЛГА БОЛДОГ ҮЙЛ ЯВЦ, УГСААТНЫ НИЙЛЭГЖИЛТИЙН ЕРӨНХИЙ ТАЙЛБАРЛАЛ ӨГӨХ БОЛОМЖТОЙ БАЙХЫН ҮҮДНЭЭС БАЙГААГААРАА БУЙ УГСААТНЫ ҮЗЭГДЛИЙН ОНЦЛОГИЙН ЖАГСААЛТЫГ АГУУЛНА
IY. Угсаатан ба этноним (угсаатны оноосон нэр)
НЭРҮҮД ХУУРАМЧ БАЙДАГ
Этнологийн ерөнхий зүй тогтлыг судлах үед нэг талаас бодит угсаатан, нөгөө талаас түүний гишүүдийн авсан нэр (этноним) нь өөр хоорондоо таардаггүй.
1. Энэхүү сэдвийн учрыг олох явдал нь угсаатны оршихуйг төдийгүй, мөн түүний үүсэн бий болохыг тодорхойлж байдаг нийгмийн хүчин зүйлийн нэг нь угсаатны өөрийн ухамсар болдог мэт үзэл өргөнөөр оршин буйгаар нөхцөлдөж байгаа юм. (Бромлей.Ю.В. Этнос и этнография. М. 1973. С. 121 – 123). Өөрийн ухамсар нь өөрийн нэрэнд илэрдэг. Эндээс хэрэв энэ хоёрын аль аль нь давхцахгүй бол тэдгээрийн функционал холбооны тухай асуудал байхгүй болох юм.
Бид нэг л адилхан нэр бүхий янз бүрийн угсаатан, эсвэл урвуугаар янз бүрийн нэртэй нэг угсаатантай ямагт тааралддаг. Тухайлбал, “римчүүд” (romani ) хэмээх үг нь анхандаа Римийн хот улсын иргэдийг хэлж байсан ба энд тэдний хөршүүд болох италикуудыг төдийгүй, Лациумын бусад хотуудад амьдарч агсан латинуудыг ч хүртэл огт оруулдаггүй байв. Римийн эзэнт гүрний үе I–II зуунд римчуудын тоо этруск, самнит, лигур, цизальпины галл зэргийн итали гаралтнууд болон огтоос латин гаралгүй мужийн олон оршин суугчдаар ихээхэн өсчээ. 212 онд Каракаллын зарлигаар Римийн эзэнт гүрний нутаг дэвсгэр дээрх муниципи хэмээх жижиг хотуудын бүх чөлөөт иргэд, түүний дотор грек, каппадок, еврей, бербер, галл, иллириц болон германчуудыг “римчүүд” гэж нэрлэх болсон байна. “Римчүүд” гэсэн ойлголт нь угсаатны утга холбогдлоо алдчихмаар байсан боловч хэрэг дээрээ ийм зүйл болсонгүй: энэ нь зүгээр л түүнийг өөрчилсөн ажээ. Гарал үүслийн болон хэлний нэгдлийн оронд соёлын нэгдэл ч биш, харин түүхэн хувь заяа нийтлэг агшин болжээ. Угсаатан энэ хэлбэрээрээ ихээхэн урт хугацаа болох гурван зууныг туулсан ч задраагүй билээ. Харин ч эсрэгээр тэр нь IY–Y зууны үед христианы ёсыг төрийн шашин болгон хүлээж авсны улмаас дотроо өөрчлөгдөж, энэ нь эхний гурван Их Чуулганы дараагаас тодорхойлох шинж тэмдэг болсон байна. Төрийн эрх мэдлээс зөвшөөрсөн энэхүү Чуулганыг хүлээн зөвшөөрсөн хүмүүсийг римчүүд гэж, зөвшөөрөөгүйг нь дайсан гэж үзэх болов. Энэхүү зарчим дээр бидний нөхцөлт байдлаар “византийн” гэж нэрлэж буй шинэ угсаатан бүрэлдсэн билээ. Гэхдээ бидний византчууд хэмээн нэрлэж буй тэр хүмүүс хэдийгээр грекээр ярьж байсан ч өөрсдийгөө “ромей” буюу “римчууд” гэж нэрлэж байв. Аажмаар ромейчуудын тоонд славян, армян, сирчүүд олон тоогоороо нийлж байсан ч “римчүүд” гэсэн нэр 1453 оныг хүртэл буюу Константинополийг унах хүртэл байсаар л байсан юм. Ромейчууд зөвхөн өөрсдийгөө л “римчүүд” гэж нэрлэж байснаас бус, лангобардууд нь феодал болсон Италийн хүн ам, үгүйрч хоосорсон Италид I–III зууны үед нүүн ирсэн сирийн семитуудын хотынхон, Эзэнт гүрний римчүүдийн хэзээ нэгэн цагт ялж байсан бүхий л ард түмний цэргийн олзлогдогсод болох тариачдыг римчүүд гэж үздэггүй байв. Тэгсэн атлаа флоренц, генуя, венеци хотынхон болон Италийн бусад оршин суугчид өөрсдийгөө “римчүүд” гэж тооцон, эртний хотуудаас нь зөвхөн балгас л үлдсэн грекчүүдийг энэ үндсэн дээр Римийн гол нэр биш гэж үздэг байлаа.
“Рим” хэмээх этнонимын гурав дахь салаа нь Дунайд үүсчээ. Римийн булаан эзлэлтийн дараа энэхүү Дакии нь (Дунай) цөллөгийн газар байжээ. Энд Римийн ноёрхлын эсрэг бослого гарган шийтгүүлсэн: каппадоки, фрак, галат, сири, грек, иллиричүүд, товчоор хэлбэл Римийн эзэнт гүрний дорно зүгийн бүх албатууд байлаа. Бие биенээ ойлгохын тулд тэд нийтэд тодорхой латин хэлээр ярилцдаг байв. Римийн легионууд явсны дараа цөлөгдөн суурьшигчдын үр удмынхан тэндээ үлдэж, XIX зуунд “румын”, өөрөөр хэлбэл “римчүүд” хэмээх нэр авсан угсаатныг бүрдүүлсэн билээ.
Хэрэв сайн ажиглавал Бүгд Найрамдах улсын эрин үеийн болон хожмын Эзэнт гүрний эрин үеийн “римчүүдийн” хооронд ядахдаа соёлын тархалттай функционал холбоо бүхий, ойлголтын аажим өргөсөлт мэтийн залгамж чанар олж харж бас болно. Харин визант болон римчүүдийн хооронд ийм холбоо ч байхгүй. Эндээс үг нь утгыг өөрчилж, агуулга нь ч гэсэн угсаатныг таних шинж тэмдэг болж чадахгүй нь харагдаж байна. Иймээс энэ үгийн утгыг ачаалж буй өөр сэдвийг, тэр тусмаа эрин үеийг нь харгалзах хэрэгтэй нь ойлгомжтой байна. Учир нь цагийн явцад үгийн утга өөрчлөгддөг аж. Энэ нь хажуугаар нь өнгөрч болохооргүй жишээ болох “түрэг”, “татар”, “монгол” гэсэн этнонимын учрыг олоход бүр ч илүү үзүүлэн болно.
\section{ӨНГӨЛӨН ДАЛДЛАЛТЫН ЖИШЭЭНҮҮД}
YI зуунд Алтай, Хангайн дорнод бэлд нутаглан байсан жижиг ард түмнийг түрэг гэж нэрлэж байлаа. Хэд хэдэн амжилттай дайны замаар түрэгүүд Хянганаас эхлээд Азовын тэнгис хүртэлх бүх тал нутгийг өөртөө захируулжээ. Их хаант улсын албатууд өөрийн этнонимыг дотоод хэрэглээндээ хадгалан түрэгийн хаанд захирагдаж байсан учраас өөрсдийгөө мөн л түрэг гэж нэрлэжээ. Арабууд Согдыг дагуулж, нүүдэлчидтэй мөргөлдөх болсон үед тэд бүх хүнийг, түүний дотор угр–мадъярыг ч түрэг хэмээн нэрлэх болов. XYIII зууны Европын эрдэмтэд бүх нүүдэлчдийг “Ies Tartars” гэж нэрлэсэн ба харин хэл зүйн ангилал моодонд орсон XIX зуунд хэлний тодорхой бүлгээр нь “түрэг” гэсэн нэрийг сонгосон юм. Ийнхүү “түргүүд”-ийн тоонд эрт дээр үеэс тэдний бүрэлдэхүүнд орж байгаагүй олон ард түмэн, жишээлбэл, якут, чуваш, турк–османууд багтах болсон юм.
Турк–османууд түүхчдийн нүдэн дээр Эргогрулаас Бага азид ирсэн туркмений орд болон исламын төлөө сайн дураараа тэмцэгч гази нар холилдох замаар бий болсон юм. Энэхүү гази нарын тоонд курд, сельжук, татар, черкес, янычар болгон авсан славян залуус, грек, итали, араб, кипрчүүд, мөн алдар нэр, аз жаргал хайж усан флотод элсэн орсон урвагч франц, италичууд, боолын зах дээр татааруудын зарсан олон тооны гүрж, украин, польш бүсгүйчүүд орж байв. Туркийн гэх юм зөвхөн хэл байсан бөгөөд учир нь энэ хэлээр армид ярьдаг байжээ. XY–XYI зууны үед энэ холион бантангаас 1000 жилийн өмнө Төв азийн тэгш газар алдраа дуурсгаж, удам хойчоо үлдээлгүй мөхсөн тэрхүү талын баатруудын дурсгал болгож, өөртөө “турк” нэрийг авсан нэгдмэл ард түмэн бий болжээ. 2. Гумилев Л. Н. Древние тюрки.
Ахиад л угсаатны нэр буюу этноним нь хэрэг явдлын үнэн байдлыг тусгасангүй, харин уламжлал, санаархлыг тусгаж байна.
“Татаар” гэсэн этнонимын хувирал нь шууд өнгөлөн далдлалтын жишээ болно. XII зуун хүртэл энэ нь Хэрлэн мөрний эргээр амьдарч байсан гучин томоохон овгийн бүлгийн угсаатны нэр байжээ. XII зуун гэхэд энэ аймгууд хүчирхэгжсэн бөгөөд хятадын газар зүйчид энэ нэрийг Төв азийн бүх нүүдэлчдэд, тухайлбал, түрэг хэлт, монгол хэлт нүүдэлчид, түүний дотор монголчуудад хамаатуулан хэрэглэж байв. 1206 онд Чингис хаан “монгол” нэрийг өөрийн албатуудын хувьд албан ёсоор авахад хөршүүд нь сурсан зангаараа нэг хэсэг хугацаанд монголчуудыг татаар гэж үргэлжлүүлэн нэрлэж байв. “Монгол”–той ижил утгатай “татаар” гэсэн үг нь энэ байдлаараа Дорнод Европод орж, орон нутгийн хүн ам Алтан Ордны хаанд төвч байгаагийн тэмдэг болгож өөрсдийгөө татаар гэж нэрлэх болсон Повольж хавиар тархсан юм. Гэтэл энэхүү нэрийг анхлан тээгч хэрээд, найман, ойрд, татаарууд өөрсдийгөө монгол гэж нэрлэх болов. 3.Грумм-Гржимайло Г. Е. Когда произошло и чем было вызвано распадение монголов на восточных и западных //ИРГО. 1933. Т. XVI. Вып. 2.
Ийнхүү нэрүүд байраа сольсон байна. Татаарын антропологийн хэв маягийг “монголжуу”, поволжийн түрэг–кипчакуудын хэлийг татаар хэл гэж нэрлэх үед шинжлэх ухааны нэр томъёо судлал цагаа олон үүсчээ. Өөрөөр хэлбэл бид шинжлэх ухаанд хүртэл цаанаасаа санаатай өнгөлөн далдалсан нэр томъёо хэрэглэж байна. 4.Никонов В. А. Этнонимия //Этнонимы /Отв. ред. В. А. Никонов. М., 1970. С. 10-11.
Үүнээс цаашаа бол жирийн будлиан биш, харин нэр томъёоны зөгнөлт хийсвэрлэл явагдана. Алтан Ордны нүүдэлчин албатуудын зарим нь түүний засгийн газартай харьцахдаа төвч байгаагүй юм. Уралаас баруун тийш талд амьдарч байсан босогчид ногай гэж нэрлэгдсэн. Харин Зүчийн улсын дорнод хязгаар, Тарвагатайн нуруу, Эрчис мөрний эргээр амьдрагсад нийслэлээс алс хол байсныхаа ачаар бараг тусгаар байсан бөгөөд казахуудын өвөг дээдэс болсон юм. Энэ угсаатнууд гурвуулаа угсаатны янз бүрийн бүрдлүүд эрчимтэй холилдсоны улмаас XIY–XY зууны үед үүссэн юм. Ногайчуудын өвөг дээдэс нь Батын цохилтоос амьд үлдээд монголын армийн бүрэлдэхүүнтэй хамт орж ирсэн хивчааг, талын алан, дундад азийн түргүүд, мөн тэр үед угсаатны эв нэгдлийн бэлэгдэл болж байсан исламын шашинд орсон Оросын өмнөд хязгаарын оршин суугчид байв. 5.Шенников А. А. Жилые дома ногайцев Северного Причерноморья //Славяно-русская этнография /Отв. ред. И. Н. Уханова. Л., 1973. С. 47-52.
Татаарын бүрэлдэхүүнд камын булгар, хазар, буртас, мөн половчуудын хагас, угр–мишарууд орж байв. XY зуунд казахын гурван жүзээс бүрдсэн Цагаан Ордны хүн ам ийм л холимог байлаа. Гэхдээ энэ нь бүх юм нь биш.
XY зууны сүүлээр оросын отрядууд Волгийн эхээс Волгийн дунд биеийн татаарын хотууд руу дайрч, хүн амынх нь зарим хэсгийг нутгаа орхин, Шейбан–ханы (1500–1510) удирдлагаар Дундад Азид очиход хүргэжээ. Тэнд тэднийг өстөн дайсан шиг угтжээ. Учир нь тэр үед “Цагаадай” нэртэй байсан (Дундад азийн улсын эзэн, Чингис хааны хоёрдугаар хөвгүүний нэртэй) нутгийн түрэгүүд 1395–1396 онуудад Повольжийг үгүйрүүлсэн талын болон повольжийн татааруудын дайсан Төмөрийн удмынхныг удирдаж байжээ.
Эх орноо орхисон Ордынхон өөртөө шинэ “узбек” гэсэн нэрийг Алтан Ордод исламыг төрийн шашин болгон тогтоосон Узбек (1312- 1341) хааны дурсгал болгон авчээ. XYI зууны үед “узбекүүд” Төмөрийн удмын сүүлчийн хаан Бабурыг бут ниргэж, тэр нь өөрийн талын үлдэгдлийг Энэтхэг рүү дагуулан явж, тэндээ шинэ хаант улс байлдан дагуулсан байна. Харин Самарканд болон Ферганд үлдсэн түргүүд өөрсдийг нь байлдан эзэлсэн узбекүүдийн нэрийг авсан аж. Энэтхэг рүү яваагүй үлдсэн мөн тэр л түргүүд гурван зуун жилийн өмнө монголын хаан хүүд захирагдаж байсныхаа дурсгалд “монгол” гэж нэрлэгдэх болжээ. Харин XIII зуунд Дорнод Иранд суурьшсан жинхэнэ монголчууд өөрийнхөө хэлээ хүртэл хадгалж, “хазарчууд” болсон байна. Перс хэлээр “хэзар” гэдэг нь мянга гэсэн үг аж. (байлдааны нэгж буюу дивиз гэж ойлгогдоно). Оросыг 240 жил дарлаж, “талхи” хэмээн нэрлэгдсэн монголчууд хаана байна? Угсаатны хувьд тэд байгаагүй бөгөөд учир нь Зүчийн гурван ордны бүх хүүхдүүдэд Чингис хааны гэрээслэлээр 4 мянган морин цэрэг л ногдсон юм. Эдний зөвхөн хэсэг нь алс Дорнодоос ирсэн байдаг. Сүүлийн энэ хэсгийг “татаар” гэдэггүй, зүрчидийн Кин (орчин үеийн Чин) улсын хятад нэрээр “хин” гэж нэрлэдэг. 6. Гумилев Л. Н. Поиски вымышленного царства. М., 1970. С. 311-313. Энэ ховор нэрийг сүүлчийн удаа “Задонщин” зохиолд Мамайг “хин” хүн хэмээн нэрлэсэн байдаг. 7.Рыбаков Б. А. 1) О преодолении самообмана //Вопросы истории. 1970. № 3, 2) “Слово о полку Игореве” и его современники. М., 1971. С. 28; Гумилев Л. Н. Может ли произведение изящной словесности быть историческим источником? // Русская литература. 1972. № 1.
Эндээс “талхидал”–ыг монголчууд огт биш, харин нүүдэлчин узбекуудын өвөг дээдэс хэрэгжүүлсэн бөгөөд энэхүү узбекуудыг хэдийгээр XIX зуунд холилдсон хэдий ч суурин узбекуудтай эндүүрч болохгүй юм. Эдүгээ тэд XYI зуунд өстөн дайсан байсан Төмөрийн болон Шейбанийн удмынхныг адилхан уншдаг нэгдмэл угсаатныг бүрдүүлж байна, учир нь энэхүү дайсагнал нь XYII зуун гэхэд утга, ач холбогдлоо алдсан юм.
\section{ТҮҮХЭН ДЭХ ХЭЛ БИЧГИЙН ХҮЧГҮЙДЭЛ}
Дээр дурдсан жишээнүүд нь угсаатны нэр буюу тэр ч байтугай өөрийгөө нэрлэсэн нэр, мөн Homo sapiens хэмээх зүйлийн амьтны тогтвортой хамтлаг болох угсаатны үзэгдэл нь бие биенээ огтхон ч хаацайлдаггүй гэдгийг нотлоход хангалттай юм. Ийм учраас үг судладаг хэл бичгийн арга зүйг этнологид хэрэглэж болдоггүй, бид энэхүү шинжлэх ухаан нь манай асуудлыг дэвшүүлэхэд хэр зэрэг тус болохыг нь шалгахын тулд түүхэнд хандах ёстой болж байна. Энд ч гэсэн бид гэнэтийн бэрхшээлтэй тулгарах болно. Түүхийн шинжлэх ухааны ашигладаг судалгааны нэгж нь нийгмийн институт байдаг. Энд нь төр, овгийн байгуулал, шашны сект, худалдааны компани (жишээлбэл Ост–Индийн), улс төрийн нам гэх мэт зүйлс, нэг үгээр хэлбэл ямар ч ард түмний ямар ч эрин зууны дурын байгууллага байж болно. Заримдаа төрийн институт, угсаатан хоёр давхцдаг, энэ үед бид хэд хэдэн тохиолдолд орчин үеийн хэв маягийн үндэстнийг ажигладаг. Гэхдээ энэ тохиолдол нь XIX –XX зуунд л хэвшмэл байдаг бөгөөд харин дээр үед бол ийм давхцал ховор байсан юм. Жишээлбэл, Энэтхэгт сикхи нар угсаатан болон нэгдсэн шиг шашны сект нэгэн санаатнуудаа нэгтгэх явдал тохиолддог, энэ үед ингэж нэгдсэн нийтлэгийн хүмүүсийн гарал үүслийг анхаардаггүй юм. Гэхдээ голдуу ийм нийтлэг нь YII зуунд Мухамедийн үндэслэсэн мусульманы нийтлэгт болсны адилаар тогтворгүй бөгөөд угсаатан болон задардаг. Хэрэв исламын орнуудад анхны дөрвөн халифын үед арабын овгууд, сири, персүүдийн зарим нь нэгдмэл угсаатан болж нэгдэх үйл явц явагдсан бол Омейядийн үе (661–750) гээд энэ үйл явц зогсож, харин Аббасидын үед булаан эзлэгчид болон эзлэгдэгсдийн хойч үеийнхэн угсаатан дундын нийтлэг соёлтой, нөхцөлт байдлаар “мусульман” хэмээгдэх, араб хэлтэй, христчүүд болон харь хэлтэнтэй харьцуулах үед өөрийн нэгдлээ ухамсарладаг, гэхдээ янз бүрийн сект, үзэл суртлын үзэл баримтлал бүтээхэд илэрдэг янз бүрийн түүхэн хувь заяатай, зан үйлийн янз бүрийн тогтсон үзэлтэй шинэ угсаатан болон бүрэлдсэн юм.
Угсаатан салан тусгаарласны улмаас үүссэн эмират, султанат зэрэг жижиг улсууд нь угсаатны хилд харгалзан бий болмоор мэт байсан ч ийм зүйл болоогүй юм. Азтай жанжингууд хэсэг хугацаанд олон хэлт хүн амыг газар нутагтаа захируулсан ч дараа нь хөршийнхөө золиос болцгоосон юм. Өөрөөр хэлбэл улс төрийн бүрдэл нь угсаатны бүхэллэгийг бодвол өөр хувь заяатай байдаг аж. Мэдээжийн хэрэг түүхэн хувь заяаны нийтлэг байдал нь угсаатан бүрдэх, хадгалагдахыг нөхцөлдүүлдэг, харин түүхэн хувь заяа нь хоёр–гурван ард түмэнд нэг, эсвэл нэг ард түмний хүрээнд янз бүр байж болдог.
8. Түүхэн хувь заяа гэдэгт бид өөрийнхөө дотоод логикоор казуаль байдлаар (нэгтгэн дүгнэлтэд орохгүй тохиолдол гэсэн утгатай латин үг -Орч) холбогдсон үйл явдлын гинжин хэлхээг ойлгож байна.
Жишээлбэл, англосаксууд болон уэльсийн кельтүүд төрийн хувьд XII зуунд нэгдсэн боловч тэд нэгдмэл угсаатан болон нэгдээгүй, гэхдээ энэ нь тэднийг энх тайвнаар амьдрахад саад болоогүй. Бүр III зуунд Иранд захирагдсан дорнодын армян, мөн энэ үед Византитай холбогдсон баруун армяны хувь заяа ялгаатай, гэхдээ угсаатны нэгдэл нь зөрчигдөөгүй. XYI –XYII зууны францын гугенотууд болон католикууд Нантын сургаалийг хэвлэх хүртэлх соёлын шинж чанараараа ч, түүнийг халсны дараа ч түүхэн хувь заяагаараа ихээхэн ялгагддаг байв. Гэсэн хэдий ч цус урсгасан дайныг үл харгалзан Францын угсаатны бүхэллэг байдал өөрчлөгдөөгүй үлдсэн билээ. Иймээс угсаатан бий болох, угсаатны нийлэгжилт нь түүхэн сурвалжид тэмдэглэдэг нүдэнд харагдах түүхэн үйл явцаас илүү гүнзгий юм байна. Түүх нь этнологид тус болж болно, гэхдээ түүнийг орлохгүй.
Y. Зүймэл байдал бол угсаатны чанар мөн
ОВГИЙН БАЙГУУЛАЛГҮЙГЭЭР ЯВЖ БОЛНО
Олонхи угсаатныг овог аймгуудад хуваадаг. Энэхүү хуваалтыг угсаатны хамааралд заавал байх зүйл буюу ядахдаа түүний бүрдлийн эхний шатанд, эцэст нь угсаатан өөрөө бий болохоос өмнө бий болсон хамтлагийн хэлбэр гэж үзэж болох уу? Бидний мэдэлд буй баттай материалууд үүнд үгүй гэж хариулах боломж олгож байна. 9. Итс. Р. Ф. Введение в этнографию. Л., 1974. С. 50.
Юуны өмнө орчин үеийн ард түмнүүд өөрийн оршин байх аль нэгэн хугацаанд заавал овгийн буюу аймгийн нэртэй байдаг буюу байгаагүй юм. Ийм зүйл испани, франц, итали, румын, англи, турк–осман, их орос, украин, сикх, грек (эллин бус) болон бусад олон угсаатанд байгаагүй буюу байхгүй. Гэвч кланы буюу овгийн систем кельт, казах, монгол, тунгус, араб, курд болон бусад олон ард түмэнд байдаг.
Кланы системийг илүү эртний үе шат гэж үзэх нь бэрхшээлтэй, учир нь монголчуудаас 1000 жилийн өмнө, казахуудаас 1200 жилийн өмнө үүссэн византийн болон сасанидийн ард түмэн овог болон фратри хэмээх овгийн хуваагдалгүйгээр сайхан л амьдарч байсан. Мэдээж эрт үед овгийн систем газар сайгүй байсан гэж таамаглаж болно, гэхдээ ард түмэн (угсаатан) түүхчдийн нүдэн дээр үүссэн түүхэнд үеүдэд ийм гаргалгааг хэрэглэж болохгүй. “Овог-аймаг-ард түмэн-үндэстэн” зэргийг өөр хавтгай дээр байгаа нийгмийн хөгжилд хамаарах бүдүүвч гэж үзэх нь зохилтой. Homo sapiens–ын оршин буй хугацаанд хүн төрөлхтний хамтын амьдралын бүхий л хэлбэрт энэ асуудалд шууд харьцаагүй гэр бүл зонхилогч нь байлаа. Учир нь угсаатны бүхэллэг нь гэр бүлийн үүртэй ч, үйлдвэрлэлийн болон соёлын түвшинтэй ч давхцдаггүй юм. Ийм учраас бид судалгаандаа өөр шалгуур, өөр танил шинж тэмдэг хайх ёстой болж байна.
Түүний хамт овог–аймгийн байгуулалтай ард түмэнд клан (кельтууд), фратри, яс (алтайчуудад “сеок”), овгийн нэгдэл ( казахуудад “жүз”) гэх мэтээр ангилах нь үр дүнтэй байсан гэдгийг тэмдэглэх ёстой. Энэхүү угсаатны доторхи нэгж нь угсаатны өөрийнх нь нэгдлийг тэтгэхэд зайлшгүй байжээ. Бүлэгт хуваах замаар нийт угсаатанд тухайлсан хүн, мөн овгийн болон гэр бүлийн хамт олны хоорондын харьцааг зохицуулж байжээ. Зөвхөн ийм хуваагдлын ачаар цусан төрлийн гэр бүлээс сэргийлсэн экзогами ёсыг хадгалж байв. Овгийг төлөөлөгчид ардын хурал дээр нэг овгийнхыхоо хүсэл зоригийг илэрхийлж, хамгаалах болон давших гадаад дайн явуулах тогтвортой холбоо байгуулж байв. Жишээлбэл, Шотландад кланы систем нь X зуунд викингуудын дайралт, XII–XY зуунд феодалуудын довтолгоо, XYII–XYIII зуунд английн хөрөнгөтнүүдтэй хийсэн дайныг даван гарсан бөгөөд хожим нь зөвхөн капиталист харилцаа л түүнийг эвдэж чадсан билээ. Харин кланы систем сул хөгжсөн, жишээлбэл полабийн славянууд буй газарт немцийн болон данийн рыцариуд Руги арлын оршин суугчид, лютичууд болон бодричуудын маргашгүй дайчин чанар, гайхалтай эр зоригийг үл харгалзан хоёр зуун жил ( XI–XII зуун) тэднийг дарласан юм. Угсаатныг овгуудад хуваах нь булчин шөрмөсийг хүчирхэгжүүлэх араг ясны үүргийг гүйцэтгэдэг бөгөөд ингэснээрээ хүрээлэн буй орчинтой тэмцэх хүчийг хуримтлуулдаг байна.
Ажиглан буй үзэгдлийн хэсэгт нь биш, харин бүхий л нийлбэрт тохирох тооцооллын өөр системийг санал болгоё.
ОВГИЙН БАЙГУУЛЛЫГ ЮУГААР СОЛИХ ВЭ
Ангит нийгмийн шатандаа буй бүрэн хөгжилтэй ард түмэнд овог төрлийн бүлэг байхгүй байгааг юугаар нөхдөг вэ ? Боол, феодал, капиталист нийгмүүдэд ангийн бүтэц, ангийн тэмцэл байдаг нь нэгэнт тогтоогдсон, дахин хянахааргүй баримт мөн. Эндээс үзэхэд ангийн хуваагдал нь функционал шинжийн хувьд овгийн хуваагдалтай адилтгал шинжтэй байж болохгүй. Бодит байдал дээр ч гэсэн ангийнхтай огтхон үл давхцах угсаатны бүлгүүдийн хуваагдлыг олж хардаг. Үүнийг нөхцөлт байдлаар “холбоо” (корпораци) гэж нэрлэж болно, гэхдээ энэ үг зөвхөн эхний ойртолтын үед л утгандаа тохирох ба хожим нь бид солих болно.
Жишээлбэл, феодалын Европт нэг угсаатны дотор, тодруулбал, францын ноёрхогч анги янз бүрийн холбоонуудаас: 1) шууд утгаараа феодал, өөрөөр хэлбэл хаантай вассаль тангаргаараа холбогдсон газар эзэмшигчид, 2) одонгуудад (орден) нэгдсэн рыцариуд, 3) хааны засгийн аппаратыг бүрдүүлэгч нотаблей хэмээх дээд хэргэмтнүүд (Nobless des robes), 4) дээд лам нар, 5) эрдэмтэд, жишээлбэл Сорбоннын профессорууд, 6) өөрөө газар нутгийн шинжээр хуваагддаг хотын патрициат хэмээх овгийн аристократууд гэх мэтээс бүрдэж байв. Энд авсан ойртолтын зэргээс хамааруулан их бага бүлгүүдийг ялгаж болно, гэхдээ ингэхдээ жишээлбэл, XY зууны эхэнд арманьякын болон бургундын намуудад хамаарагдаж байсан бүлгүүдийг ч заавал нэмэн тооцох хэрэгтэй.
Орден гэдэг нь дундад зууны шашин, бүлгийн, эсвэл аль нэг феодалд хамаарагдах цэргийн бүлгийг нэрлэж байжээ. Тэд хэдэн арав заримдаа хэдэн мянган хүнтэй байдаг байжээ. Тэд өөр хоорондоо ялгагдахын тулд малгай, цээжин дээрээ зүрх, сэлэм, загалмай зэрэг тусгай тусгай тэмдэгтэй байжээ. Ийм бүлгийг орден буюу одон гэдэг байжээ. Хаант засаг бэхжиж ирмэгц тэр нь хаанд гавъяа байгуулсан хүний энгэрт зүүх тусгай тууз, тэмдэг болжээ. Ингэж бидний сайн мэддэг одон үүссэн байна. (Цаашид бид одон гэдгээр нь явна –Орч)
Заримдаа уншигчдад энд заасан бүлгүүдийг язгуур угсаатай адилтгах хорхой төрдөг бөгөөд үүнийгээ тэд социал бүлгүүд гэж ойлгодог. Гэхдээ бид яв цав байх ёстой: социал хуваагдал бол анги, харин язгуур угсаа бол захиргааны хуваагдал, дундад зууны үед энэ нь “улс төрийн хувьд онцгой ач холбогдол …олж байгаагүй ба харин өөрсдийгөө л тэмдэглэж байсан юм.” 10. Маркс К.. Энгельс ф. Соч. 2-е изд. Т. I. С. 313.
Энд дурдсан эдгээр бүлгүүд нь үгийн бүрэн утгаараа ч язгуур угсаа биш бөгөөд харин “үйлдвэрлэлийн урьдач нөхцөл” болох нийтлэг юм. Энэ нь мэргэжлийн нийтлэгийн хувилбар болохын хувьд гэр бүл болон өргөжсөн овгийн нийтлэгт байлцаж болно. Ийм учраас К.Маркс ийм маягийн “холбоод” нь хуулийн ямар ч эрх байхгүй, гэхдээ гэр бүлийн холбоондоо эрчим хүч олж байдаг бастард хэмээх хэрмэл хүмүүсээр бүрддэгийг тэмдэглээд Дундад зууны түүхийг зоологийн буюу амьтан судлалын түүх гэж нэрлэсэн байдаг. Бастардууд Зуун жилийн дайны дараа маш томоохон үүрэг гүйцэтгэсэн байдаг. Тухайлбал, бастард Дюнуа гэгч Францын анхны рыцарь болж, граф хэмээх гүн болж байжээ. 11. Там же. Т. 13. С. 20, 12 Там же. С. 37, 13. Архив К. Маркс и Ф. Энгельса. Т. VI. М., 1939. С. 356 и сл.
Өргөн ард түмний хувьд энэ хуваагдлыг бүр ч их хэмжээгээр хэрэглэж, тэр үед феодалын муж тус бүр тод илэрсэн хувь шинжтэй байсан юм. Жишээлбэл, өөрсдийг нь англиас чөлөөлсөн Филипп II Августад дайсагнан хандаж, IX Людовик Египетэд олзлогдсон тухай мэдсэн Марсельчууд “сиричүүдээс” зайлахад найдан “Te Deum” дууг дуулж байв. 14. Тьерри О. Иэбр. сот. М., 1937. С. 214.
Хөрөнгөтний нийгэмд бид ийм холбоодыг олж харахгүй, гэхдээ зарчим нь өөрчлөгдөөгүй юм. Угсаатны дотоодод ангийн зэрэгцээгээр хүн бүрт “өөрийн” болон “өөрийн биш” хүрээллийн хүмүүс байдаг. Гэхдээ харийн түрэмгийллийн нөхцөлд энэ бүх бүлгүүд нь франц хэмээх нэгдмэл бүхэл болдог юм.
Бидний нөхцөлт байдлаар “холбоо” хэмээн нэрлэж буй тэр зүйл нь овог аймгийн бүлэглэлүүдийг бодвол харьцуулшгүй бага тогтвор, үргэлжлэлтэй. Гэхдээ овог аймгийн бүлгүүд нь гэсэн мөнхийн зүйл биш юм. Ингэхлээр аль алиных нь хоорондын ялгаа зарчмын шинжтэй биш бөлгөө. Адил тал нь гэвэл эдгээр нь функцээ дотооддоо хуваарилах замаар угсаатны нэгдлийг тэтгэж байдаг адилхан ачаалал болно.
Хамгийн чухал бөгөөд сонирхолтой зүйл бол “холбоод” яг үүсэх үедээ өөр хоорондоо зөвхөн сэтгэл зүйн өчүүхэн ялгаагаар л ялгагддаг, цагийн эрхээр ялгаа нь гүнзгийрэн талстжиж, зан заншил, ёс үйл болон шилжих ба энэ үзэгдлийг угсаатны зүй судалдаг юм. Жишээлбэл, эртний славяны үнсэх ёслол нь Орос болон Польшид нөхөртэй хатагтайн гарыг үнсэх болон өөрчлөгдсөн бөгөөд орон нутгийн тайж нарын дунд хадгалагдан үлдэж, чингэхдээ өөр язгуур угсааны ахуйгаас алга болжээ.
Повольж хавийн томоохон хотуудад янз бүрийн зиндааны сэтээтэн, хотын борчуулыг ажиглаж байсан А.М.Горький ихээхэн гүнзгий ялгааг тэмдэглэж, саяхан бүрдсэн хүн амын энэ бүлгүүдийг “янз бүрийн овог” гэж үзэж байсан юм. 15. Горький А. М. Сторож //Собр. соч.: В 30 т. Т. 15. М.. 1951. С. 81.
Энэ ойлголтыг хэрэглэсэн тэр утгаараа (ахуй, ёс авир, төсөөлөл зэргийг ойлгосон) Горький зөв бөгөөд түүний ажиглалт үр дүнтэй юм. Эдгээр нь 80 орчим жилийн богинохон үед хэвшмэл байсан, гэхдээ бид үзэгдлийн үргэлжлэх хугацаа нь ажил явдлын зарчмын талуудад нөлөө үзүүлдэггүйг бид нэгэнт ярьж байсан билээ.
УГСААТНЫ ДЭД БҮЛГҮҮД ҮҮСЭХ НЬ
Энд авч үзэж буй ойлголтоор “холбоо” гэсэн ойлголт нь илт тодорхой байна. Гэвч бидний шинжилгээнд тухайн нэгж нь угсаатны онцлогоос тогтох төдийгүй, мөн бусад “холбоодын” социал саадаас ялгараагүй байгаа учраас хангалтгүй байдаг. Энэ нь иш татсан жишээг хайж буй ерөнхий дүрмийн нэг л тохиолдол болохыг харуулж байна.
Францын угсаатны нийлэгжилтээс авсан жишээгээ үргэлжлүүлье. XYI зууны Шинэчлэл нь энэ ард түмнийг хөндөхдөө тэнд байсан бүх “холбоодыг” нь танихын аргагүй болтол нааш цааш болгожээ. Феодалын аристократууд, жижиг тайж, хөрөнгөтөн болон тариачид “папынхан” болон “гугенотууд” болон задарсан аж. Энэ хоёр бүлгийн социал үндэс нь ялгаагүй, харин угсаатны газар нутгийн хуваагдал тод харагдаж байв. Кальвинизм Луарын доод хэсгийн кельтуудын дунд амжилт олж, харин худалдааны Ла–Рошель хот шинэчлэгчдийн тулгуур болов. Гасконы сеньорууд, Наваррын вангууд кальвинизмыг хүлээн авав. Бургундын удмынхан, Севенны тариачид, альбигойчуудын удам болох Лангедокийн хөрөнгөтнүүд хөдөлгөөнд нэгдэв. Гэвч Париж, Лотарингия болон төв Франц Римийн сүм хийдэд үнэнч хэвээрээ үлджээ. Өмнө нь байсан “холбоод” устан алга болж, “нийтлэг” буюу “сүм” -д хамаарах явдал нь хоёр зууны турш аль нэгэн дэд угсаатны бүхэллэгт хамаарах шалгуур болсон байна.
Энд шашны сургааль шийдвэрлэх үүрэг гүйцэтгэсэн гэж хэлж болохгүй. Францчуудын ихэнх нь “улс төрийн бодлоготон” байж, өөрөөр хэлбэл Сорбонн болон Женевийн маргааныг сонирхохоос татгалздаг байжээ. Бичиг үсэггүй гасконы баронууд, хагас зэрлэг севенны уулынхан, Ла- Рошелийн эрэмгий корсарууд, Париж, Анжер хавийн гар урчууд нь Урьдчилан тодорхойлох, Урьдчилан оршихуйн номлолуудын нарийн ширийнийг огт ойлгодоггүй байлаа. Хэрэв тэд Библийн юмуу месса хэмээх католик мөргөлийн төлөө амиа өгсөн бол аль аль нь тэдний өөрийгөө бататгасан, мөн бие биендээ сөргөлцсөний бэлэгдэл болох ба ингэснээрээ гүн гүнзгий зөрчлийн шалгуур үзүүлэлт болох юм. Энэ зөрчил нь ангийн биш, учир нь аль ч талд нь тайж, тариачин, хөрөнгөтнүүд тулалдаж байсан билээ. Гэхдээ католик болон гугенотууд үнэхээр зан үйлийн тогтсон үзлээрээ ялгагдаж байсан бөгөөд энэ нь бидний урьдчилан тохирсон угсаатан задрах үндсэн зарчмын хувьд хангалттай үндэслэл болно.
Тэгвэл хэрэв гугенотууд өөртөө хэсэг газар булаан авч, тэндээ швейцар юмуу хойд америк шиг бие даасан улс байгуулсан бол яах байсан бол ? Тэднийг түүхэн хувь заяаны савалгааны улмаас үүссэн онцгой угсаатан гэж үзэх ёстой байсан байх. Учир нь тэд онцгой ахуй, соёл, сэтгэл зүйн хэв маяг, магадгүй хэлтэй байх байсан. Яагаад гэвэл тэд париж аялгаар ярих нь юу л бол, харин нутгийн аль нэг аялгыг сонгох байсан буй за. Энэ нь англичуудаас америкчууд салсантай адилтгам үйл явц байхсан.
Шотландууд гарцаа байхгүй угсаатан мөн, гэхдээ тэр нь гайлендер (уулынхан), кельт болон лоулендерүүдээс (Твид голын хөндийн оршин суугчид) бүрдэнэ. Тэдгээрийн гарал үүсэл янз бүр. Эртний хүн ам болох шивээсээр (пикты) чимэглэсэн каледонянчууд I –II зуунд римчүүдийн түрэлтийг няцааж байсан ба тэдэн дээр Ирландаас нүүн ирсэн скоттууд нэгдсэн байна. Энэ хоёр овог романжсан Британи руу хөнөөлт довтолгоо хийж, дараа нь Английн хойд мужуудад дайтаж, арлын зүүн эрэгт бэхэжсэн норвегийн викингуудтэй тулалдаж байв. 954 онд Шотландуудын аз гийж, сакс болон норвегийн викингуудын удмынхан суурьшсан Твид голын эрэг дэх тэр газар–Лотианыг эзлэн авчээ. Баян, эрч хүчтэй Лотианыхан өөрийн кельт эздийг Шотланд гэсэн жижигхэн вант улс болгон хувиргахад хүргэжээ. Учир нь Английн хилийг хамгаалах болсон байна. 16. Toynbee A. J. Study of History /Abr. by D. Somervell, London; New York; Toronto, 1946. P. 120-121.
XIY зуунд Англитай дайтах гэсэн Жан Балиол болон Робер Бюрсын талын, францын түрэмгийлэгчид Шотланд руу цутгасан юм. Францчууд хилийнхээ феодалыг тоо нэмэгдүүлсэн юм. Шинэчлэл (Реформаци) кельтүүдийг улам ихээр хамарч, энэ хөндийд кальвинистуудын зэрэгцээгээр католикууд ч байх болжээ. Товчоор хэлбэл, энэ ард түмэн нийлэгжих үед арьс угсаа, соёл, овгийн байгуулал болон феодализм хүртэл холилдсон юм. Гэхдээ энэ бүрэлдэхүүний нарийн нийлмэл байдал нь эхлээд англичуудтай, дараа нь ирландчуудтай тулгарахад илэрсэн угсаатны цул байдлыг арилгаж чадаагүй билээ.
Бүр өөр маягийн старообрядчууд буюу хуучин ёслолтнуудын жишээ ч бас байдаг. Энэ бол XYII зууны үеийн сүм хийдийн зарим шинэчлэлийг хүлээж аваагүй их оросуудын багавтар хэсэг болох нь тодорхой байдаг. Тэр үед сүмийн алба нь зөвхөн шашны төдийгүй, бас нийлмэл урлагийн үүрийг гүйцэтгэж, өөрөөр хэлбэл гоо зүйн хоосон зайг бөглөж байсан юм. Ийм учраас, ёс заншил гүйцэтгэхэд тавих шаардлага нэн өндөр байжээ. Манай үед ч гэсэн хэн боловч муу шүлэг унших юмуу утгагүй зургуудыг бясалгахаас сэтгэлийн таашаал авч чадахгүйн адилаар XYII зуунд үлэмж аллилуй–трегубой юмуу харласан бурхны дүрийг шинэ шинэ ягаан-хөх иконуудаар солих нь мөргөлчдийн тодорхой хэсгийг дуу алдуулам байжээ. Мөргөлчид уурыг нь хүргэж байсан нөхцөл байдалд төвлөрч чадахгүй байв.
Энэ нь үнэн чанартаа Реформацийн үед Баруун Европт болсон шиг угсаатны задрал жижиг цар хэмжээгээр болсонтой адил зүйл байв. Энэ үед бүх үнэн алдартан христианчууд хуучин ёслолтныг дагаагүй юм. Харин ингэхээр шийдсэн тэр хүмүүс цааз, тамлалтаас халшралгүй, хатуу чанд байхаар шийдсэн билээ. Тохиромжтой тохиолдолд тэд никонианчуудтай тэд өөрсдөө яадгийн адил хатуухан тооцоо боддог байв. Энэ нь хатан хаан Софийн урвасан стрелецкийн бослогын үед бий болсон. Аль аль нь тэсч ядан байжээ. XYII зуунд маргаан нь зөвхөн сүмийн ёслолын талаар болдог байсан ба ахуй боловсролын систем, дадал зуршил зэрэг бусад зүйлээрээ хуучин ёслолтнууд оросын нийт хүн амаас ялгардаггүй байсан. Петр I–ийн хоёрдугаар үед тэд хүн амын тодорхой тусгаарласан бүлэг болсон юм. XYIII зууны сүүл гэхэд зан заншил, ёслол, хувцас мэт нь зарим талаар хадгалагдсан боловч нийтээр хүлээн зөвшөөрснөөс эрс ялгагдах зүйлс бий болов. II Екатерина хуучин ёслолтнуудыг хөөн цээрлүүлэхийг зогсоосон боловч энэ нь тэднийг угсаатныхаа үндсэн олонлогтой нийлэхэд хүргээгүй билээ. Шинээр бүрэлдсэн угсаатны дотоодын энэ бүхэллэгт саятан–худалдаачид, казакууд, Волгийн чанадын хагас гуйранч тариачид орцгоосон байна. Анхандаа хувь заяаны нийтлэгээр, өөрөөр хэлбэл баримтлах зарчмаараа нэгдсэн энэ нэгж нь тэдний хувь маш үнэ цэнэтэй байсан агаад эдгээр зарчмынхаа төлөө тэд үхэл рүү явж, янз бүрийн чиглэл, аяс бүхий оюун санааны удирдагчдын (номлогчид) толгойлсон ахуйн нийтлэгээрээ нэгдсэн бүлгүүд болсон байна. XX зуун гэхэд үүсэн гарах шалтгаан нь аль эрт оршин байхаа больсны улмаас энэ нь аажмаар сарниж, одоо зөвхөн инерци нь л үлджээ.
Бидний авсан жишээнүүд хурц тод авч ховор юм. Угсаатны доторхи бүлэглэлүүдийн үүргийг жам ёсоор үүссэн газар нутгийн нэгдэл болох–нутгархах үзэл голчлон өөртөө авдаг юм. Ийм хуваагдал байгаа нь овгийн байгууллын үеийн фратри хэмээх угсаатны хуваагдал нь угсаатны нэгдлийг эвдээгүйтэй адил болно.
Одоо бид угсаатны дотоод дахь бүхэллэг байдал явагддаг социал хэлбэрүүд нь гайхамшигтай бөгөөд угсаатны хуваагдалтай тэр бүр давхцдаггүй, угсаатан дотоодын жижигрэл нь түүнд тогтвортой байдал олгож, угсаатны бүхэллэг байдлыг тэтгэж байдаг, энэ нь нийгмийн хөгжлийн бүх үед, бүх эрин үед хэвшмэл байдаг гэсэн дүгнэлтийг хийж болж байна.
УГСААТНЫ ХАРИЛЦАХ ХУВИЛБАРУУД
Одоо болтол бид том угсаатны дотоод дахь бутархай бүлгүүдийг авч үзлээ, ингэлээ гээд асуудал огтхон ч шавхагдахгүй юм. Түүхийн бодит үйл явцад хатуу тусгаарлагдсан угсаатан ажиглагддаггүй бөгөөд холимог угсаатны төр болон улс төрийн хувьд нэгдсэн, янз бүрийн угсаатан оршин суугаа газар нутаг дээр үүссэн угсааатны харилцааны олон янзын хувилбар л байдаг. Эдгээрийн харилцааг судлах үед дөрвөн хувилбарыг ялган үзэж болно. Энэ нь: а) зэрэгцэн орших. Энэ үед угсааатнууд холилдохгүй, бие биенээ дуурайхгүй, зөвхөн техникийн шинэ зүйлийг хуулан авна, b) ассимиляци буюу уусгах. Энэ үед нэг угсаатан нь нөгөөгийнхөө гарал үүсэл, хуучин уламжлалыг нь бүрэн мартагнуулж шингээн авдаг, c) эрлийзжих. Энэ үед угсаатны өмнөх уламжлал, өвөг дээдсийн талаарх дурсамж зэргийг хадгалж, хослуулж байдаг. Гэхдээ энэ хувилбар нь голчлон тогтворгүй байдаг бөгөөд шинэ эрлийзүүдийг нэмэгдүүлэн оршдог, d) уусан нэгдэх. Энэ үед анхдагч хоёр бүрдлийн аль алиний уламжлал мартагдаж, өмнө нь байсан хоёртой зэрэгцсэн (эсвэл оронд нь ) гуравдагч шинэ угсаатан үүсдэг. Энэ нь мөн чанартаа угсаатны нийлэгжилтийн гол хувилбар болно. Яагаад ч юм энэ нь бусдаасаа цөөн ажиглагддаг.
Энэхүү дөрвөн гишүүнт системийг илт жишээгээр тайлбарлая. А хувилбар хамгийн их тархсан. Трамвайд бүгдээрээ европжуу арьстан хамаарах, адилхан хувцастай, нэг ресторанд хооллосон, гартаа ижил сонин барьсан орос, немц, татаар, грузин хүмүүс орлоо гэж төсөөлөөд үзье, хувийн онцлогийн ялгааг нь арилгасан ч гэсэн тэднийг адилтгашгүй гэдэг нь хэн бүхэнд тодорхой. 17. Гумилев Л. Н. Этногенез и этносфера //Природа. 1970. № 1. С. 46-47.
“Тэгээд яагаав? Хэрэв энэ трамвайд үндэстний ноцтой будлиан гараагүй бол энэ дөрвөн хүн чинь угсаатнаасаа салангид буй хүмүүсийн жишээ болоод цаашаа тайван явна шүү дээ” гэж намайг шүүмжлэгч хүн нэг удаа татгалзан өгүүлж билээ. Бидний бодлоор бол ийм биш бөгөөд нөхцөл байдлын дурын өөрчлөлт нь энэ хүмүүст янз бүрийн, тэр ч бүү хэл тэр хүмүүс хамтдаа үйлчилж байсан үед ч хариу үйлдэл үзүүлнэ. Энэ трамвайд нэг залуу орж ирээд тэнд буй эмэгтэйтэй бүдүүлэг харилцая гэж бодъё. Манай нөхдүүд ямар үйлдэл хийх вэ ? Ямар ч гэсэн грузин хүн тэр залууг заамдан авч, трамвайнаас буулгаж хаяхыг оролдоно. Немц хүн ярвайн анивчиж, цагдаа дуудаж эхэлнэ. Орос хүн хэд хэдэн ариун нандин хараал хэлнэ. Татаар хүн хэрүүлд оролцохоос зайлсхийхийг оролдоно. Зан үйлийн өөрчлөлтийг шаардсан нөхцөл байдлын өөрчлөлт нь янз бүрийн угсаатны (супер угсаатны) төлөөлөгчдөд зан үйлийн тогтсон бодлын ялгааг ялгааг онцгой мэдэгдэм болгож өгдөг байна.
Энэ нь бүрэн ойлгомжтой юм. Бүхий л юмс үзэгдэл хослуулах үед танигддаг. Зэрэгцүүлэн асгасан сод, нимбэгийн хүчил хоёр хэрэв усаар норгосон нөхцөлд л хүчтэй шуугисан нийлэх урвал үзүүлнэ. Түүхэнд ч гэсэн усан уусмалын адил бүх үед урвал явагдаж, энэ нь дуусна гэсэн найдвар байхгүй. Тэр ч байтугай зүгээр л зэрэгцэн оршиж буй янз бүрийн угсаатнууд ойртох үедээ төвийг сахисан байдаггүй. Заримдаа энэ нь зүгээр л хэрэгтэй байдаг. Тухайлбал, Конгийн дээд биеийн банту нар пигмей хэмээх одойчуудтай симбиоз буюу хамтын амьдралд байдаг. Пигмейчүүдийн тусламжгүйгээр негрүүд ойгоор, болон цэвэрлэсний дараа буцаад л ургадаг жимээр явж чадахгүй. Банту негр европ хүний адилаар ойд төөрч, өөрийнхөө гэрээс хориод метрийн зайтай үхэж болно. Харин пигмей нарт хутга, сав суулга болон бусад ойр зуурын зүйлс хэрэгтэй байдаг. Энэ хоёр угсаатны хувьд төсгүй байдал нь найрамдал нь тогтож байдаг сайн сайхны үндэс болдог.
Байнгын дайсагналд удаан хугацаагаар оршин байх хувилбарыг гребений казак, чеченүүдийн мөргөлдөөнийг ажиглаж байсан Л.Н.Толстой маш сайхан дүрсэлсэн билээ. Тэрээр хөрш хоёр угсаатны харилцан хүндэтгэл, их орос казакуудыг уусгахын пионер болж байсан Терек дэх цэргүүдээс казахуудын болгоомжлол зэргийг тун зөв тэмдэглэсэн байдаг. Энэ уусах үзэгдэл ХХ зууны эхээр дууссан юм.
Уусах буюу b хувилбар голдуу цус урсгах гэхээсээ хэтэрхий доромж аргаар хэрэгждэг. Уусах объектод эсвэл сэтгэл зүрхээ алдах, эсвэл амьдралаа алдах ацан шийдэл тулгадаг. Ялагчдын дунд хоёрдугаар зэргийн хүн болон хувирахын тулд бүхий л үнэ цэнэтэй, дадсан зүйлсээсээ татгалзах замаар үхлээс мултарч болно. Ялагч нь ч гэсэн хожих зүйлээр маруухан, их зантай, голдуу яс муутай хамтрагчдыг олж авдаг ба эзлэгдсэн угсаатны зөвхөн гадаад зан үйлийг хянахаас биш, түүний сэтгэл санааг хянаж чаддагүй. XIX зуунд Англичуудад үүнийг Ирландууд харуулсан, испаничуудад Симон Боливарын партизанууд, хятадуудад дунган нар үзүүлсэн юм. Ийм жишээ хэтэрхий олон, байдал нь тодорхой.
Эрлийзжих хувилбар нь маш их тохиолддог боловч холимог гэр бүлээс гарсан удам нь гурав–дөрөвдүгээр үедээ эсвэл мөхөж, эсвэл эхийн юмуу эцгийн шугамаараа задран унадаг. Жишээлбэл, XYI зууны туркууд жинхэнэ турк байхын тулд исламын сургаалийн томъёоллыг мэдэх, султанд захирагддаг байхад л хангалттай гэж үзэж байв. Өөрөөр хэлбэл тэд угсаатны хамаарлыг дураараа өөрчилж болох “төлөв байдал” лугаа ойлгож байв. Ийм учраас туркууд хэрэв ямар нэг урлал юмуу цэргийн урлагийн мэргэжилтэн л байвал дурын түрэмгийчүүдийг албандаа авдаг байлаа. Үүний хор уршиг нь зуун жилийн дараа гарсан юм.
XYII зуунд Өндөр Порт унасан нь орчин үеийн туркийн зохиолч нарын анхаарлыг их татдаг юм. Тэдний бодлоор уналтын шалтгаан нь “ажемоглон”-ууд, өөрөөр хэлбэл урвагчдын хүүхдүүд байжээ. Тэгсэн мөртлөө неофитуудын үнэнч байсанд эргэлздэггүй аж.
18. “Урвагч” гэдэг нэр томъёо тэр үед муу утгатай байгаагүй бөгөөд дайсны талд алба хаах нь жирийн үзэгдэл байсан юм. Гэхдээ хэрэв нэг хэт угсаатны хүрээнд шилжихийг хүртэл урвалт гэж үзэж байгаагүй, хэрэв мусульман нар руу явсан бол урвагчийн угсаатны хамаарлыг үгүй болгодог байжээ. “Дунайн Запорожец” хэмээх хошин дуурийн гол баатар туркууд руу оргоод “Одоо бол би казак биш, турк мөн” гэж дуулдаг.
Жишээлбэл, франц Кеприлю, грек Хайраддин нарын эрч хүчтэй, ашигтай хүмүүс байжээ, харин тэдний ихэнхи нь тохитой байр хайж, польш, хорват, итали, грек зэрэг хүүхнүүдээр дүүрсэн визир хэмээх түшмэдүүдийн эхнэрүүдийн байраар дамжуулан амар хялбар ажил олж байжээ. Ni foi ni loi байхгүй эдгээр явуулын хүмүүс осман угсаатныг эвдэлсэн юм. Жинхэнэ османууд бүр XYIII зуун гэхэд өөрийнхөө эх оронд дарлагдаж буй угсаатны байдалд орчихсон байлаа. Харь гаралтны цатгалан нь зан үйлийн тогтсон бодлыг өөрчилж, энэ нь визирүүд биеэ худалдах, шүүгч нарын булхай хийх, цэргийн байлдааны чадвар унах болон эдийн засгийн задралд нөлөөлжээ. XIX зууны эх гэхэд Турк нь “өвчтэй хүн” болсон юм. Хүчирхэг ард түмэн ийм хачирхалтай байдлаар суларсны шалтгаан, урвагч нарын ролийн тухай шинжилсэн оросын дорно дахины судлаач В.Д.Смирнов өөрийн диссертацидаа: “Хэн нэг нь тоглоомоор ч гэсэн Чайковский, Лангевич нарыг славян, грек, мадъяр, италуудаас бүрдсэн хувь хүн, исламыг итгэл үнэмшлээрээ олж авсан гэж хэлж чадна гэж үү ? Хэн ч ингэхгүй нь эргэлзээгүй юм. Үүний хамт иймэрхүү маягийн эргэлтийн хувьд османы овгийг ялгуулсан гавъяаны үр шимийг ашиглах сугалаа ногддог юм. Ямар ч шашингүй тэд ёс суртахууны аливаа итгэл үнэмшлээс ангид байж, өөрөө засаглан буй ард түмэндээ ямар ч таашаал үзүүлдэггүй, ингэж тэд зөвхөн амьтны амьдралаар амьдарч байв. Лалынхны гарем хэмээх эхнэрүүдийн байрны хов жив нь жинхэнэ, аливаа үнэнч иргэний сонирхох улс төрийг тэдэнд орлох болжээ. Гэр бүлийн холбоо нь тэдний организмын өрөөл татуу байдлыг дуудах буюу жигшүүрт гажгийн дутууг нөхөж байв. Тэдний баялгийн тухай ойлголт нь өөрийнхөө халаасыг дүүргэхээс цааш явдаггүй байлаа. Үүргийн мэдрэмж нь өөрийнхөө хууль бус үйлдлийг хаацайлж болох хуулийн үндэслэл хайх, өөртөй нь төстэй нийгмийн өөр зүтгэлтэн далд хатгалын золиос болчихгүй байхаар л хязгаарлагдаж байв. Нэг үгээр хэлбэл, тэд нэр нь осман боловч бодит хэрэг дээрээ тэд османууд биш байв.” гэж бичжээ. 19. Смирнов В. Д. Кучибей Гомюрцжинский и другие османские писатели XVII века о причинах упадка Турции. СПб., 1873. С. 266-267. Тэгвэл шийдвэрлэх хүчин зүйл байгальд байна уу, иргэний төлөв байдалд байна уу ?
ЭКЗОГАМИЙН ҮҮРЭГ
Ингэж Туркэд харь овгийнхон нэвтэрснээр тэртэй тэргүй өсөн нэмэгдэж байсан ангийн хямралыг хурцатгаж, энэ нь угсаатны бүхэллэгийг хоосон бодол болгон хувиргахад дамжуулагчийн үүргийг гүйцэтгэсэн юм. Учир нь жинхэнэ төвч түшмэл их зантай, зарчимгүй түшмэлээс илүү үнэ цэнэтэй гэдэг нь хүн бүхэнд ойлгомжтой байсан билээ. Гэтэл урвуугаар ангийн зөрчил нь османы угсаатны нийлэгжилтийн хувьд задлагч векторын үүрэг гүйцэтгэсэн байна. Нэг бүс нутагт угсаатны болон нийгмийн үйл явц хосолсноор дээр үед “Ариун баялаг Хагас сарны” орон хэмээн нэрлэгдэж байсан дэлхийд хамгийн баян орны ландшафтыг антропогенийн эвдрэлд оруулсан хүчин зүйл болжээ. XYI зууны I Селим хааны булаан эзлэлт бүр III мянган жилээс эхлээд идэвхитэй газар тариалан эрхэлж байгаад Нийтийн Тооллын үед анхдагч ландшафт болгон өөрчилсөн Сири, Палестин, Месопотами зэрэг газруудыг османы султануудын гарт нь өгсөн билээ.
Тигр, Евпратын бэлийн шумерууд “усыг хуурай газраас салган”, ингэж бүтээсэн орныг нь орчин үеийнхэн “Эдем” гэж нэрлэсэн. Аккадийчууд дэлхийн анхны сая хүнтэй хот Вавилоныг байгуулж, энд алс орноос зөөлгүйгээр хүнсээр бүрэн хангаж байлаа. Антиохия, дараа нь Дамаск орон нутгийн нөөцөөр цэцэглэн хөгжсөн том, хөгжилтэй, соёлтой хотууд болсон юм. Бага Ази нь аварга том Константинополийг тэжээж байлаа.
Гэхдээ соёлын ландшафт нь түүнийг байнга тэжээн тэтгэхийг шаардаж байв. Үүнийг Занзибараас боол худалдан авч, Месопотамийн услалтын системийг хадгалж байсан арабын халифууд, тусгай зарлиг гарган тэр үеийн байгалийн нөхцөлд хамгийн эрчимтэй байсан тариачдын жижиг аж ахуйг бэхжүүлж байсан византийн дарангуйлагчид, тэр байтугай Хойт Хоёр мөрний хуурайшсан хэсэгт суваг бариулах ажлыг зохион байгуулж байсан монголын ильхан Газан нар ойлгож байлаа. Өмнөд Азийн соёлын ландшафтын задрал бүр хожим болсон бөгөөд XYII–XIX зууны үед гүн бат энх тайвны үед Османы эзэнт гүрэн унахад хэт их татвараас зүрхшээсэн сири, ирак, киликийн тариачид газраа орхин, эсвэл хялбархан баяжиж болох, эсвэл толгойгоо тавих эрэг хавийн дээрмийн хотууд руу сайхан амьдрал хайн очих болжээ. Харин залхуугаасаа буюу аймхайгаасаа болж гэртээ үлдсэн хүмүүс нь услалтын системийг тавьж, өмнө нь баян, элбэг дэлбэг байсан орныг үгүйрсэн газар болгон хувиргав.
Энэхүү аймшигтай бөгөөд хөнөөлт үйл явцын эхлэлийг бүр манай үеийн Францын адал явдал хайгч, гвардийн эмч Ауренгзеба, “Агуу их Могол” улсын захиргаанд байсан Энэтхэгийн ийм дэг журмыг ажиглаж байсан Франсуа Бернье нар Кольберт бичсэн захидалдаа мусульманы гурван том хаант улс Энэтхэг, Турк, Перс улсууд гарцаагүй доройтно гэж, чингэхдээ персийн орон нутгийн язгууртны хувьд Персийн доройтол нь удаан байна гэдгийг тэмдэглэсэн байдаг. 20. Бернье Ф. История последних политических переворотов в государстве Великого Монгола. М., 1936.
Тэрээр Кучибей Гомюржинскийтэй тохиролцоогүй байдаг. Энэхүү давхцал нь нэг ижил үйл явцыг хоёр ухаантай хүн ажиглаж, дүгнэлт, таамаглал хийж чадсан аж. Бид нийгмийн тогтвортой байгууламжид, нэг ижил формацийн нөхцөлд, гэхдээ төр хэмээх улс төрийн системд угсаатны бүрдлүүдийн харьцаа өөрчлөгдсөн үед ландшафтын төлөв байдал нь угсаатан үүсэх буюу сэргэх, унахыг, түүнчлэн тогтвортой үед ч маш нарийн барометр лугаа харуулдаг гэдгийг бид зөвшөөрөх хэрэгтэй юм.
Хэрэв ийм ахул бүс нутгийн ландшафттай холбоогүй, экзогам буюу өөр овгоос гэрлэх гэр бүлийн хязгаарлалтаас чөлөөтэй, тэгээд энэхүү хязгаарлалт нь бүс нутгийн угсаатны олон янз байдлыг дэмждэг угсаатны шинэ бүлэг системд бий болох нь тухайн ландшафтад багтан амьдарч буй угсаатны жижиг бүлгүүд хадгалагдахад хүргэдэг гэсэн дээр нэр дурдсан зохиогчдын заасан шалтгааныг үгүйсгэх үндэслэл бидэнд байхгүй юм. Тэгвэл байгал болон соёлыг чөлөөт харилцаа, чөлөөт хайр дурлал хоёр л хөнөөдөг байна.
Энэ гаргалгаа гэнэтийн бөгөөд сүрдмээр боловч Ньютоны хоёрдугаар хуулиас дахин ишилсэн зүйл юм. Энэ нь нийгмийн эрх чөлөөний хувьд гэвэл байгальтай, нарийн яривал газар зүйн орчинтой харилцах үед өөрийн физиологийг алддаг, учир нь байгал бидний бие дотор байдаг юм.
Учир нь ийм адилтгам үзэгдэл Рим, Эртний Иран болон бусад олон орнуудад байсан юм. Энд эндогам буюу овог дотроо гэрлэх ёс угсаатны саад тотгор байдлаар байгаа үед үйл явцууд нь удаан бөгөөд бага зовлонтой явагддаг, харин угсаатан гурван зуун жил , эсвэл мянган жил амьдарсан эсэх нь угтсаатны хувьд адилгүй байдаг ерөнхий зүй тогтлыг ажиглахад хялбар юм. Ийм учраас Ю.В.Бромлейн задралын эсрэг саад болсон эндогам буюу овог дотроо гэрлэхийн тогтворжуулагч ролийн тухай өгүүлсэн нь маргаангүй болно. 21. Бромлей Ю.В. Этнос и эндогамия //Советская этнография. 1969. № 6. С. 84-91.
\section{ТАЙЛБАРЛАЛЫН ТУРШЛАГА}
Энэхүү үзэгдлийг тайлбарлах гээд үзье. Хэрэв угсаатан нь үйл явц бол ижил хоёр үйл явц тулгарахад тус бүрийнх нь анхдагч давтамжийг зөрчсөн тайлбарлал үүсэх юм. Бүрдэн бий болсон нэгдэл нь хий зэрэглээ шинжтэй, ингэхлээр гаднын үйлчлэлийн өмнө тогтворгүй, удаан оршдоггүй. Хиймэл системийн мөхөл нь түүний бүрдэл хэсгүүдийг аннигиляци буюу оргүй болгодог бөгөөд энэхүү системд татагдан орсон хүмүүсийг үхэхэд хүргэдэг. Өгөгдсөн зүй тогтлыг зөрчих ерөнхий механизм ийм бөгөөд гэхдээ үүнээс өөр ч байх нь буй. Чухамхүү анхдагч хэмнэлийн тогтворгүй байдал нь шинэ хэмнэл, өөрөөр хэлбэл шинэ угсаатны нийлэгжилтийн энерцийн үйл явцын нөхцөл нь болж өгдөг. Энэ нь юутай холбоотой болох талаар түүнийг явдал дундаа шийдэхээр тийм ч амархан асуудал биш болохоор бид одоохондоо ярихгүй. Гэхдээ угсаатны уламжлалыг хадгалахад эндогами буюу овог дотроо гэрлэх ёс зайлшгүй бөгөөд эндогами гэр бүл хүүхдэд боловсорсон зан үйлийн тогтсон үзлийг олгодог, харин экзогами буюу өөр овгийн гэр бүл нь хүүхдэд бие биенээ үгүйсгэгч хоёр тогтсон үзлийг олгодог. Ийнхүү “социал төлөв байдалд“ огт хавьтахгүй, өөр хавтгайд байх экзогами буюу өөр овгийн гэр бүл нь угсаатны нийлэгжилтийн хүчин зүйлст ордог байна. Өөрөөр хэлбэл хэт угсаатны түвшинд харилцах үед бодитой эвдлэгч хүчин зүйл болдог юм. Зөрчлийн бүсэд шинэ угсаатан бий болдог тийм ховор тохиолдолд ч гэсэн тэр уусгаж, өөрөөр хэлбэл өмнөх хоёрыгоо устгадаг юм. Сүүлд нь хэлэхэд эдгээр жишээнд ч, түүнчлэн дийлэнхи олонхи тохиолдолд ч гэсэн арьстны зарчим ямар ч үүрэг гүйцэтгэдэггүй юм. Энд соматик буюу бие организмын биш, харин зан үйлийн ялгааны тухайд ярьж байна. Тухайлбал, талын хүмүүс, төвдийн уулынхан, хятадууд нь I түвшний нэгдмэл монголжуу арьстанд хамаардаг ба харин II түвшин хүртэл нарийвчлахад хойт хятадууд нь арьстны шинжээрээ өмнөд хятад гэхээсээ сяньби, төвдүүдтэй ойрхон байдаг. Гэхдээ нүд болон үсний өнгө, эпикантус болон гавлын ясны үзүүлэлтүүдийн гадаад төстэй байдал зэрэг нь угсаатны генетик үйл явцад ач холбогдолгүй байлаа.
Энэ авсан жишээнээс хааяа эргэлзэхэд хүрдэг угсаатаны ландшафттай холбогдох холбоо тодорхой байгаа юм. Хүн нар Шар буюу Хуанхэ мөрний хөндийг эзлэн аваад тэнд малаа маллаж, хятадууд газар тариалж, суваг байгуулдаг байсан, харин тэдний хольц эрлийзүүд мал аж ахуйн ч, газар тариалангийн ч дадлага байхгүй учраас хөршүүд болон албатуудаа араатан шиг цөлмөснөөр энэ нь хэдийгээр ой мод хяргасан, хааны ан хийхэд малын туурайнд талхлагдаж ядуурсан ч жам ёсны биоценез эргэн тогтоход хүргэж байв. Бүх юм таарч байна.
Ийм маягаар зөвхөн онолын хувьд биш, биет өгөгдөхүүнийг тайлбарлахын зайлшгүй нь угсаатныг төлөв байдал гэж үзэх үзэл баримтлалыг няцаахад хүргэж байна. Гэхдээ хэрэв угсаатан нь удаан явагдах үйл явц бол тэр нь Дэлхийн био хүрээний нэг хэсэг болно. Учир нь техник ашиглах замаар ландшафтыг өөрчилсөн нь угсаатантай холбогдоно. Иймээс хэдийгээр этнолог нь үгийн явцуу утгаар анхдагч материалаа түүхээс авч, үйл явдлыг түүхэн холбоо, дэс дараалалд нь судалдаг ч гэсэн этнологийг газар зүйн шинжлэх ухаан гэж тооцвол зохино.
YI. Угсаатны зан үйлийн тогтсон үзэл
\section{ТӨСГҮЙ БАЙДАЛ ЗАРЧИМ БОЛОХ НЬ}
Угсаатан бүр өөрийн гэсэн хувийн дотоод бүтэц, өөрийн үл давтагдах зан үйлийн тогтсон үзэлтэй байдаг. Заримдаа угсаатны бүтэц, зан үйлийн тогтсон үзэл нь үеэс үе дамжин өөрчлөгддөг. Энэ нь угсаатан хөгжиж байгааг болон угсаатны нийлэгжилт унтраагүй байгааг харуулдаг. Заримдаа угсаатны бүтэц тогтвортой байдаг, ийм учраас шинэ үе нь өмнөх үеийнхнийхээ амьдралын орчлыг нөхөн сэргээж байдаг. Ийм угсаатныг персистент буюу өөрийгөө туулсан гэж нэрлэж болох бөгөөд асуудлын энэ талыг хожим ярина. Харин одоо бол “бүтэц” гэсэн ойлголтын утгыг тогтсон үзэлтэй нь холбон түүний тогтвортой байдлын хэмжээ, өөрчлөлтийн шинж чанарынх нь гадуур нарийвчлан үзье.
Угсаатны зан үйлийн тогтсон үзлийн бүтэц нь а) хамт олон болон хувь хүний хоорондын, b) хувь хүмүүсийн хоорондын, c) угсаатны дотоод бүлгүүдийн хоорондын, d) угсаатны болон угсаатны дотоод бүлгүүдийн зэрэг харилцааны хатуу тогтсон хэм хэмжээ байдаг. Эдгээр хэм хэмжээнүүд нь тохиолдол бүрт өвөрмөц, нэг бол хурдан, нэг бол удаан өөрчлөгддөг, амьдрал болон ахуйн бүх салбарт далд байдаг, угсаатан бүр, тодорхой эрин цаг бүрт хамтын амьдралын цорын ганц боломжит арга гэж үздэг бөгөөд угсаатны гишүүдийн хувьд эдгээр нь огтхон ч түвэгтэй байдаггүй юм. Өөр угсаатны өөр хэм хэмжээтэй тулгармагц тухайн угсаатны гишүүн бүр гайхан төөрөлдөж, нөгөө ард түмний хөгийн гаж зүйлийг өөрийнхөндөө дуулгах гэж оролддог. Энүүхэндээ ярихад ийм яриа хөөрөөнүүд нь угсаатан хоорондын холбоо лугаа мөн л эртний шинжлэх ухаан болох угсаатны зүйг бүрдүүлдэг. Түүний анхдагч төлөв байдал болон шинжлэх ухааны нэгтгэсэн дүгнэлт хоёрын ялгаа нь гэвэл мэдээллийн өргөн цар хүрээ, системчилсэн байдал л байдаг, мөн угсаатны зүйч өөр угсаатны ёс заншил, зан үйлийг гайхан шогширдоггүй.
Жишээгээр тайлбарлая. Эртний афин хүн Ольвийд байхдаа скифүүд орон гэргүй бөгөөд баярынхаа үеэр тасартлаа архиддаг гэдгийг дургүйцэн ярьсан байдаг. Скифүүд грекийн вакханаличуудыг хараад мөн л ийм жигшил мэдэрсэн нь буй. Тэд нэг удаа Ольвийд зочлон хөөрөн баярласан эллинчуудын дунд цэцгэн хэлхээ, усан үзмийн модон таяг барьсан өөрийнхөө хааныг хараад түүнийг алсан байна. Иудей буюу жүүдүүд римчүүдийг гахайн мах иддэгт нь үзэн яддаг, харин Римчүүд агтлах заншлыг жам ёсны гэж тооцдог байв. Палестиныг эзэлсэн рыцариуд арабын олон эхнэртэй байх ёсонд дургүйцэж, арабууд франц хатагтай нарын халхлаагүй царайг ичгүүргүй явдал гэж үздэг байв. Ийм жишээ тоо томшгүй.
Угсаатны зүйн шинжлэх ухаан нь иймэрхүү үл ойлголцлыг даван туулсан бөгөөд ажиглалтдаа хувь хүмүүсийн харилцааны үйлчлэн буй хэм хэмжээ гэсэн зарчмыг оруулсан юм. Энэ хэм хэмжээ нь хүмүүсийн өөр хоорондын ч, мөн тэдний бүхэл хамт олонтойгоо харьцах харьцааг ч тодорхойлдог. Жишээ болгож, гэр бүл-бэлгийн харилцааны энгийн тохиолдлыг авч үзье. Бүдүүлгээр хэлбэл, ийм харилцааны хэлбэрүүд моногами буюу нэг эхнэр нөхөртэй гэр бүлээс авахуулан бэлгийн бүрэн чөлөөт харилцаа хүртэл маш олон янз байдаг. Жишээлбэл, зарим ард түмэнд эмэгтэй гэр бүл болохын тулд онгон байх ёстой, заримд нь урьдчилан хайр дурлалын аргуудыг заасан байдаг. Заримдаа гэр бүл салах нь амархан, заримдаа маш хүнд байдаг. Зарим ард түмэнд эхнэр нь гаднын эр хүнтэй явалдахыг гэр бүлийн үнэнч байдлыг зөрчсөн гэж шийтгэдэг, заримд нь урамшуулдаг. Жишээлбэл, Хамийн баян бүрдийн уйгарууд явуулын худалдаачдад эхнэрээ тавьж өгч сурсан байсныг бид дээр дурдсан, хожим нь Чингисийн удмынхны ивээлээр баяжсан хэдий ч хөршүүдэд нь ичгүүртэй санагдах энэ заншлаасаа тэд татгалзахыг хүсдэггүй байсан.
Яг энэчлэн бид үүргийнхээ мэдрэмжийг хүлээн авах хувилбаруудыг шинжилж болно. Феодалын Англи буюу Францад вассал хэмээх албат нь сюзерен хэмээх эрхэт ноёндоо зөвхөн бенефиция хэмээх “цалин” авсан тохиолдолд л зөвхөн үйлчлэх ёстой. Хэрэв үүнийг нь өгөхгүй бол тэрээр өөр сюзерен (жишээлбэл, испанийн хаанд) очих эрхтэй. Зөвхөн өөр шашинтанд, жишээлбэл мусульмануудад очсон үед л урвасан гэж үзэх ба энэ нь олон удаа тохиолддог байсан болохоор ренегат буюу урвагч гэсэн тусгай нэр томъёо үүссэн байдаг. Үүний урвуугаар Рим болон Грекэд нийгмийн үүрэг хүлээх нь цалин дагалдуулдаггүй, харин хот улсын иргэний үүрэг байв. Дашрамд дурдахад энэ иргэдийг өөрийгөө хэмжээнээс нь хэтэртэл нөхөн шагнадаг нийгмийн ажлаар дардаг байсан юм.
Угсаатны зан үйлийн тогтсон үзлийн хүч асар их бөгөөд угсаатны гишүүн түүнийгээ цорын ганц нэр төртэй зөв зүйл мэтээр хүлээн авч, бусад зүйлийг “зэрлэг бүдүүлэг” мэтээр ойлгодог. Чухам ийм учраас европын колоничлогчид индиан, африк, монгол, тэр ч байтугай оросуудыг хүртэл зэрлэгүүд хэмээн нэрлэж байсан ба нөгөөдүүл нь ч гэсэн англичуудыг ингэж хэлэх эрхтэй байж болох байсан. Хятадын их зан бол бүр ч илүү ичгүүр сонжуургүй байсан. Жишээлбэл, Мин улсын үеийн газар зүйн лавлахад Францыг “Баруун өмнөд тэнгист байдаг. 1518 онд эзэн хаан нь нутгийн захиргаанд бичигтэй элчээ явуулж, өөрийг нь хаан хэмээн хүлээн зөвшөөрөхийг хүссэн” гэж тэмдэглэсэн байдаг. 22. Бичурин Н. Я. (Иакинф). Собрание сведений по исторической географии Восточной и Срединной Азии //Сост. Л. Н. Гумилев, М.Ф. Хван. Чебоксары, 1960. С. 638.
\section{ЗАН ҮЙЛИЙН ТОГТСОН ҮЗЛИЙН ӨӨРЧЛӨЛТ}
Угсаатны зан үйлийн тогтсон үзэл нь угсаатантайгаа адил өөрчлөмтгий юм. Зан заншил, зан үйл, харилцааны хэм хэмжээ нь нэг бол удаан бөгөөд аажим, нэг бол маш хурдан өөрчлөгдөг. Английг жишээ болгон үзье. Кельт хүүхдүүдийг алж байсан зэрлэг догшин саксын удам, хөгжилтэй хориотой анчин Робин Гуд, “Цагаан отрядын” харваач, мөн түүний шууд удам корсар усан цэрэг Френсис Дрейк, Кромвелийн төмөр нударгат цэрэг зэргийг ялгаж болно гэж үү дээ ? Мөн тэдгээрийг залгамжлагч викторианы эрин үеийн нямбай, ёсорхуу лондон Ситийн конторын албан хаагч, ХХ зууны урт үст гутранги үзэлтэн болон хар тамхичныг юу гэхэв ? Англи нь цаг ямагт консерватив орон байсан билээ. Дүр төрхөд нь дотоод хөгжил төдийгүй, соёлын хуулбар, зан заншлын албадан өөрчлөлтийг дагалдуулсан булаан эзлэлт зэрэг гаднын үйлчлэл, эцэст нь угсаатны хэрэгцээг хүчээр тохируулагч, ажил төрлийн өөрчлөлт, эдийн засгийн дарамт учирсан өөр угсаатны тухайд юу ярих вэ ? 23. Жишээлбэл, XIX зуунд Хятадад хар тамхи оруулж ирэхэд ийм байв. Чингэхдээ эхэндээ хар тамхийг хямд үнээр тараах замаар түүний эрэлтийг бий болгосон. Арьс авахын тулд Канадын индианчуудад архи зарах болсон нь үүнтэй адилхан юм.
Угсаатны зан үйлийн тогтсон үзлийн тухай ярихдаа бид ярьж буй эрин үеийг нь ямагт дурдаж байх ёстой. Түүнчлэн “зэрлэг” буюу “бүдүүлэг” гэж нэрлэгдсэн овгууд нь соёлжсон үндэстнээс илүү” консерватив” гэж бодож болохгүй юм. Ийм үзэл бодол индиан, африк, болон сибирийн ард түмнийг зөвхөн бага судалсны үр дүнд л үүсдэг. Канадад архины, Таитад консервны худалдаа зохион байгуулах нь хангалттай бөгөөд ингэхэд л дакот, полинезчуудын зан үйлийн тогтсон үзэл шууд л өөрчлөгдөнө, чингэхдээ сайн тал руугаа ховор өөрчлөгдөнө. Гэхдээ бүх тохиолдолд өөрчлөлт нь өөрийн замаар нэгэнт бүрэлдчихсэн дадал, төсөөллийн бааз суурь дээр явагддаг. Энд л угсаатны нийлэгжилтийн дурын үйл явцын дахин давтагдашгүй чанар, түүнчлэн угсаатны нийлэгжилтийн үйл явцын бие биенээ хуулдаггүйн шалтгаан байдаг юм. Үнэхээр зүй тогтол байгаа, зөвхөн түүнийг олох л хэрэгтэй юм.
Дурын тооны жишээ, түүний дотор хууль зүй, эдийн засаг, нийгэм, ахуй, шашин болон бусад харилцааг хөндсөн зан үйлийн стандартын харьцааны жишээг татаж болно. Эдгээр нь хэчнээн нарийн байсан угсаатны дотоод бүтцийг дэмжих үндсэн зарчим болж байдаг. Хүмүүнлэгийн шинжлэх ухаааны талаасаа бол энэхүү үзэгдлүүд нь уламжлал, нийгмийн харилцааны өөрчлөлт гэдгээрээ тодорхой, харин байгалийн шинжлэх ухааны үүднээс бол зүйлийн үржилд болон салаалсан бүсүүдэд ялгаран тогтсон зан үйлийн тогтсон үзэл бодол тайлбарлагдах тийм л зүй тогтолтой юм. Хэдийгээр хоёрдахь асуудал нь дасаагүй зүйл боловч үр нөлөөтэй гэдгийг нь бид хожим үзнэ.
Ингээд угсаатан бол бусад бүх хамт олноос өөрийгөө ялгадаг амьтдын хамт олон юм. Угсаатан нь түүхэн цаг хугацаанд үүсч, алга болж байдаг хэдий ч ямар нэг хэмжээгээр тогтвортой байдаг. Бидэнд илэрхий бүх тохиолдлуудад хэрэглэж болох угсаатныг тодорхойлох нэг ч бодит шинж тэмдэг байхгүй. Хэл, гарал үүсэл, зан заншил, материаллаг соёл, үзэл суртал нь заримдаа тодорхойлогч зүйл болдог, заримдаа үгүй байдаг. Бид хаалтан дотор зөвхөн ганцхан зүйлийг хийж чадах ба энэ нь хүн бүр “Бид ийм ийм улс, бусад нь бүгдээрээ ондоо” гэдгийг хүлээн зөвшөөрдөг. Энэ үзэгдэл нь түгээмэл болохоор энэ нь бидний хайж буй хэмжигдэхүүн болох ямар нэг биет буюу биологийн бодит байдал гэж таамаглаж болно. Энэ хэмжигдэхүүнийг угсаатан үүсэх болон алга болохыг шинжлэх, угсаатнуудын хоорондын зарчмын ялгааг тогтоох замаар л зөвхөн тайлбарлаж болно. Тэдгээрийн ялгааг гаргахын тулд аль нэгэн угсаатны зан үйлийн тогтсон үзэл бодлыг дэс дараатай дүрслэх хэрэгтэй юм. Гэхдээ угсаатны зан үйл нь түүний наснаас хамааран өөрчлөгддөг, иймээс угсаатан түүхийн тавцанд гарч ирсэн үеэс эхлэн тооцох нь тохиромжтой гэдгийг санах хэрэгтэй. Ийм учраас бид шинжилгээндээ “угсаатан” гэсэн ойлголтын тодорхойлолтод хоёрдагч ойртолт хийхийн тулд этнодинамик буюу угсаатны өөрчлөлтийг илрүүлэх аргыг оруулах хэрэгтэй юм. Энэ нь нэг талаас бүхий л хүмүүст хэвшмэл байдаг сэтгэл зүйн агшин, нөгөө талаас угсаатны динамикийн шалгуур болж чадах хангалттай олон янз байдалтай угсаатан бүхэллэг шинжийнхээ хувьд цаг хугацаанд хэрхэн хамаарах байдал болно.
\section{УГСААТАН БА ЦАГ ХУГАЦААНЫ ДӨРВӨН МЭДРЭМЖ}
Цаг хугацаа юу вэ гэдгийг хэн ч мэдэхгүй. Гэхдээ хүмүүс түүнийг хэмжиж суржээ. Цаг хугацааны шугаман тооллын хэрэгцээ байхгүй хамгийн бүдүүлэг ард түмэн ч ямар нэг нөхцөлт (“Рим байгуулагдсан”, “Дэлхий бүтсэн”, “Христ мэндэлсэн”, “Хижир”, “Мухаммед Миккигээс Медин руу оргосон” гэх мэт) хугацааг мэдэж, өдөр шөнө, жилийн дөрвөн улирал, өөрийн амьдралын “амьд” хуанлигаар ялгаж, эцэст нь долоо хоног, сар, тус бүр нь амьтны нэр бүхий арван хоёр жилийн (түрэг–монголын цаг тооны бичиг) хэмнэлийг мэддэг. Харьцуулсан угсаатны зүйн өгөгдөхүүнээр цаг хугацааны шугаман тоолол нь угсаатан өөрийн түүхийг онцгой үзэгдэл мэтээр биш, зэргэлдээ орнуудын түүхтэйгээр холбоотойгоор мэдэрч эхлэхэд бий болдог байна. Харин мэдлэг хуримтлахын хэрээр хүмүүсийн ухамсарт хугацааг кванталж, өөрөөр хэлбэл түүнийг үргэлжлэл нь маш адилгүй, гэхдээ үйл явдлын дүүргэлт нь ижил эрин үеүдэд хуваадаг байна. Энд “цаг хугацаа” гэсэн категори нь хурдатгал, тухайн тохиолдолд түүхэн үйл явцыг хурдасгагч шалтгаан “хүч” гэсэн категоритой шүргэлцэж эхэлдэг байна. 24. Гумилев Л. Н. Этнос и категория времени //Доклады Географического общества СССР. Вып. 5. Л., 1970. С. 143-157.
Тооллын системийн ийм олон янз байдал нь угсаатны сэтгэл зүйг ноцтой өөрчлөлтөд хариулж, энэ нь өөрийн ээлжинд угсаатны нас халагдахыг тодорхойлдог байна. Манай зорилгын хувьд бол тооллын аль нэг систем нь чухал биш, харин өнгөрсөн, одоо, ирээдүйн ойлголт дахь ялгаа нь чухал юм.
Угсаатны нийтлэг өөрийн бүрдлийн эхний бүтээлч үед ороход угсаатны хөгжлийн замаар бүхий л системийг түлхэгч хүн амынх нь тэргүүлэх хэсэг материаллаг болон үзэл санааны үнэт зүйлсийг хуримтлуулдаг. Ёс зүйн салбар дахь энэ хуримтлал нь “империатив” буюу бүх нийтийн ёс суртахууны хууль болж, цаг хугацааны хувьд “пассеизм” хэмээх өнгөрсөн үед тэмүүлэх гэж нэрлэж болох мэдрэмж болон хуримтлагдана. Үүний утга нь угсаатны бүхэллэгийг идэвхитэй байгуулагч бүр өөрийгөө өвөг дээдсийн шугамыг үргэлжлүүлэгч хэмээн мэдэрч, үүн дээр нь тэр бас нэг ялалт, бас нэг барилга, бас нэг гар бичмэл, бас нэг цувисан сэлэм гэх мэтээр ямар нэг юм нэмж байдаг. Энэхүү “бас нэг“ гэдэг нь өнгөрсөн нь явчихаагүй, тэр нь хүн дотор байж байгаа учраас түүн дээр ямар нэг шинэ юм нэмэх ёстой ба ингэснээр өнгөрсөн хуримтлагдаж, урагшаа тэмүүлж байдаг. Энд амьдарсан агшин бүрийг оршин буй өнгөрсөн рүү урган орох мэтээр хүлээн авдаг (Passe existente).
Цаг хугацааг ингэж хүлээж авсны үр дүнд эх орныхоо төлөө сайн дураараа амьдралаа өгсөн баатруудын гавъяа төрдөг юм. Жишээлбэл, Фермопильд спартын үлгэрийн баатар Леонид, Карфагенд консул Аттилич Регула, Ронсевалийн хавцалд Роланд нар төрж, энэ нь бретонийн түүхийн маркграф болон “Роландын тухай дуульс” зэрэг уран зохиолдын баатруудыг бас хамардаг юм. Мөн ийм баатрууд гэвэл санваартан Пересвет ба Ослябь, Куликовын тулалдаанд амь үрэгдсэн Сергей Радонежскийн шавь нар, “өөрийн жам ёсны хаанд“ нуугдах боломж олгож, Чингисийн цэргүүдийг өөртөө дуудсан хэрээдийн Хадаах баатар нар болно. 25. Козин С. А. Сокровенное сказание. М.; Л., 1941. С. 140.
Ийм маягийн европчууд өөрийнхөө нэрийг мөнхжүүлэлгүйгээр готын сүмүүдийг босгож, индусууд агуйн сүмэнд онгон баримал сийлж, египетчүүд бунхан байгуулж, полинезчүүд эх орон нэгтнүүддээ Америкийг нээж, арал руугаа кумар (амтат төмс) авчирч байжээ. Тэдний хувьд хувийн ашиг сонирхол байгаагүй нь хэвшмэл байжээ. Тэд ажил хэрэгтээ өөрөөсөө илүү хайртай мэт байсан юм. Гэхдээ энэ нь аялдах үзэгдэл биш байсан, тэдний хайрлах зүйл нь зөвхөн тэдэнд байгаагүй ч тэдний дотор нь байсан юм. Тэд өөрийгөө агуу их уламжлалыг залгамжлагч төдийгүй, түүнийхээ хэсэг мэтээр мэдэрч, энэ уламжлалынхаа төлөө ариун амьдралаа цэрэг шиг хурдан, уран барилгач шиг удаан өгч, үйл ажиллагааных нь чиглэл, шинж чанарыг тодорхойлогч өөрийнхөө мэдрэл–сэтгэл зүйн хэв маягийн дагуу, тэр ёсоор л гүйцэтгэж байв. Ийм хэв маягийн хүмүүс бүх эринд байдаг бөгөөд гэхдээ угсаатны нийлэгжилтийн эхний шатанд эдгээр нь хэд дахин их болдог юм. Тэдний хувь буураад эхэлмэгц бидний “цэцэглэлт” гэж нэрлэж дадсан үе болох ба үүнийг яг зөвөөр нь “үрэгдэх” гэж хэлж болно.
Энэхүү пассеизм хэмээх өнгөрснийг санах үеийн оронд актуализм буюу идэвхижих үе болдог. Энэ хэв шинжийн хүмүүс нь өнгөрснийг мартаж, ирээдүйг мэдэхийг хүсдэггүй. Тэд одоо бөгөөд өөрийнхөө тулд амьдрахыг хүсдэг. Тэд эрэлхэг зоригтой, эрч хүчтэй, авъяастай, гэхдээ хийж байгаа зүйлээ зөвхөн өөрийнхөө төлөө хийдэг. Мөн тэд гавъяа байгуулах боловч өөрийн шуналын төлөө, өөрийн эрх мэдлээр сэтгэл ханахын тулд өндөр байр суурь хайж үүнийг үйлддэг, Учир нь тэдний хувьд бодит зүйл нь зөвхөн одоо үе байх бөгөөд үүнийгээ өөрийн, хувийн юм гэж гарцаагүй ойлгодог. Ийм хүмүүс гэвэл Римд Гай Марий ба Люций Корнелий, Сулла нар, Афинд Алкивиад, Францад “Агуу Их Конде” хэмээх хан хүү, XIY Людовик, Наполеон, Орост Догшин Иван, Хятадад Суйн эзэн хаан Ян Ди (605–618) нар байлаа. Харин ямар нэг агуу зүйл хийж, нэрээ алдаршуулах гэсэн зохиолч, зураач, профессоруудыг тоолох ч боломжгүй юм. Ийм хүмүүс нь зугаа цэнгэл хөөцөлдөгч, алиа марзангууд, амьдралаа шатаагчид зэрэг бөгөөд тэд хэдийгээр өөрийнхөө бүхэл амьдралыг туулах боловч мөн л өнөөдрийн өдрөөр амьдардаг. Ийм хүмүүсийн хувь угсаатны бүрэлдэхүүнд ихсээд ирэхэд тэдний золиос болсон өвөг дээдсийн хуримтлуулсан өв хөрөнгө маш хурдан зарцуулагдах бөгөөд энэ нь элбэг баян байдлын хуурамч сэтгэгдэл төрүүлдэг, ийм болохоор нь ч үүнийг “цэцэглэлт” гэж нэрлэдэг.
Зохиогч эдгээр хүмүүсийг муулж байна гэсэн сэтгэгдэл уншигчдад төрж магадгүй. Үгүй ээ. Тэдний цаг хугацааг хүлээн авч байгаа нь дээр дурдсантай адил тэдний хүслээс үл хамаарах, харин дээд мэдрэлийн үйл ажиллагаанаас нь хамаарах яг тийм л үзэгдэл юм. Тэд хүссэн ч гэсэн үүнээс өөрөөр байж чадахгүй. “Нэг өдөр ч болов минийх” , Надаас хойш усан галав юүлсэн ч хамаагүй” гэсэн алдарт үгс нь увайгүй явдал биш, харин чин сэтгэлийн үг болно. Угсаатанд ийм хэв маягийн хүмүүс байх нь түүнийг устахад хүргэдэггүй, заримдаа ашигтай ч байж болох өсөлтийг л зөвхөн зогсоодог, энэ хүмүүс өөрийгөө золиослолгүйгээр хөршүүдээ золиослох зорилго тавьдаггүй, ингээд угсаатны талбарыг хязгааргүй өргөтгөх эрмэлзэл нь жам ёсны хил тогтоох явдлаар солигддог.
Цаг хугацаа ертөнцөд харьцах гурав дахь боломжит бөгөөд бодитой оршин байх хувилбар нь зөвхөн өнгөрснийг төдийгүй, ирээдүйн үүднээс өнөөг үгүйсгэх явдал юм. Энд өнгөрснийг алба болгон, өнөөдрийг хүлээн авч болохгүй, бодит зүйлийг зөвхөн мөрөөдөл гэж үздэг. Энэ үзлийн хамгийн тод жишээ нь Элладын Платоны идеализм, Римийн эзэнт гүрний иудейн хилиазм, манихейн (альбигойн үзэл) болон маркионитийн (бугимели ёс) ёсны сектанд хөдөлгөөнүүд болно. IX зуунаас эхлэн Бахрейны бедунуудын авсан карматын үзэл суртлын систем Сири, Египет, Иранд тархаж эхлэхэд Арабын халифат хүртэл футурист (ирээдүйг таах) үзлийн (үүнийг ингэж нэрлэх нь зөв байх) нөлөөллөөс зайлж чадаагүй юм. Египетэд карматууд өөрийн Фатимидуудын эзэнт улсын тогтоож, Иранд Аламут, Гирдекух болон Люмбасарын уулын цайзуудыг эзэлж, тэндээсээ мусульманы султан, эмирүүдэд хүсэл зоригоо тулгадаг байжээ. Персүүд тэднийг измалийтүүд гэж, харин загалмайтнууд асасинууд хэмээн нэрлэж байв.
Карматын үзэл суртал илт идеалист боловч шашных биш юм. Тэдний сургаалиар ертөнц нь толин байдлаар тусч байдаг хоёр хагас хэсгээс бүрдэнэ. Өнөөгийн ертөнцөд карматуудад маш муу бөгөөд тэднийг дарлаж, гомдоож, дээрэмдэж байдаг. Харин эсрэг ертөнцөд бол бүх юм урвуугаар, мусульман, христианчуудыг карматууд дарлаж, гомдоож, дээрэмдэж болно. Эсрэг ертөнцөд зөвхөн тэдний ахмад багшийн томилсон “амьд бурхны” тусламжтайгаар очиж болох бөгөөд энэ хүнд үг дуугүй захирагдаж, мөнгө төлөх ёстой. Энэхүү системд шашинлаг юм юу ч байхгүй юм. Феодалуудад дарлагдсан тэмцэл мэт карматуудын үйл ажиллагааны тухай төсөөлөл нь ажил хэргийн хамгийн чухал бус, зөвхөн нэг талыг тусгадаг юм. Каир дахь Фатимадууд, Аламут дахь Хассан Саббах нар хэдийгээр заримдаа нийгмийн зөрчлийг улс төрийн бодлогынхоо ашиг сонирхолд ашигладаг байсан хэдий ч өөрсдийнхөө дайснуудын адилаар бас л тариачдыг дарладаг байсан юм. Дээрэмчид буюу сектууд өргөн олон түмний ашиг сонирхлыг илэрхийлж чадах уу даа ?
Гэхдээ эртний Хятадад III зууны үед үүссэн цаг хугацааг ирээдүйн үүднээс хүлээн авах үзэл нь ард түмнийг “шар алчууртны” хэмээх тариачны бослогод хүргэсэн юм. Ангийн зөрчлийн бодит байдлын зэрэгцээгээр Бага Хань (25-220) эзэнт гүрний засаглалын үед даосын эрдэмтдийг күнзийнхэн төрийн албаны бүх албан тушаалаас нь шахан гаргаж, тэднийг өвчин эмчлэх, цаг агаар таах зэргээр хоолоо олж идэхэд хүргэсэн юм. Энэхүү гуйранч оршихуй нь тэднийг сэтгэл санаагаар унагасангүй, харин тэдний орчинд “хүчирхийллийн хөх тэнгэр”-ийг “шударга ёсны шар тэнгэр”-ээр солих ёстой онол бий болжээ. Хэрэг дээрээ урсгасан цуснаас болж тэнгэр нь бараан болсон бөгөөд бослогыг удаалсан энэ самууны үеэр Хятадын хүн ам 50 саяас 7,5 сая болтлоо цөөрсөн байна. Энэ бүх гай зовлонд зөвхөн даосын суртал нэвтрүүлгийг буруутгах нь хөнгөмсөг явдал болох ба учир нь үйл явдалд оролцогчдын дийлэнхи олонхид нь гүн ухааны ямар ч үзэл харш байсан юм. Бидний судлан буй асуудлаар бол ирээдүйг таах үзэл бий болсон, түүний идэвхжилт нь нэгэн зэрэг пассеист үзлийг ард түмний амьдралаас шахан гаргасныг л тэмдэглэх нь чухал болой. III зууныг эртний болон дундад зууны хятадыг зааглагч эрин үе гэж үздэг нь тохиолдлын хэрэг биш юм. Шинэ үнэт зүйлсийн хуримтлал үзэл суртлын ч, материалын ч талаасаа YI зуунд Суй улсын үед эхэлж, пассеист урсгал YII зуунд Тан улсын үед хэлбэржсэн байна. Энэ үзэгдлийг Н.К.Конрад Хятадын Дахин Сэргэлт гэж нэрлэсэн ба энэ үед “хуучин руу буцах” лоозонгийн дор “Таван бүдүүлэг” гэж нэрлэсэн цэрэг болон нүүдэлчдийн ёс суртахууны задрал, бүдүүлэг ёсыг сөрөн зогссон шинэ жинхэнэ соёл бүтээгджээ. 26. Конрад Н. И. Запад и Восток. М.. 1966. С. 119-149, 152-281.
Цаг хугацааг ирээдүйчлэн хүлээн авах нь маш ховор тохиолдох боловч энэ нь гаж үзэгдэл болдог гэсэн дүгнэлт хийж болохоор ч юм шиг. Гэвч энэ нь буруу, энэ нь бусад хоёртойгоо адил зүй тогтолтой, гэхдээ энэ нь угсаатны нийтлэгт ямар ч угсаатныг бүхлээр нь мөхөөдөг тийм хөнөөлт үр дагавартай байдаг. Энд эсвэл “мөрөөдөгчид” мөхдөг, эсвэл “мөрөөдөгчид” мөрөөдлөө хэрэгжсэн гэж үзээд актуалистууд болдог, өөрөөр хэлбэл бусдын адил амьдарч эхэлдэг.
Ертөнцийг ирээдүйчлэн хүлээж авах нь эргэн тойрны хүмүүст зөвхөн цэвэр хэлбэрээрээ буюу өндөр “найрлагатай” үедээ л аюултай байдаг юм. Энэ ертөнцийн өөр үзэлтэй холилдох үедээ бол таашаал төрүүлэх ч чадвартай болдог. Жишээлбэл, Ионн Лейденский Мюнстерт зайлшгүй цус урсгахтай холбоотой сэтгэлийн ихээхэн хөөрлийг бий болгож чадсан, гэвч орчин үеийн баптист–бэртэгчингүүд бидний хийж буй ангиллаар бол өөрсдийнхөө үзэл санаа, оюун санааны өвөг дээдэстэй гэхээсээ католик, протестант, атеист зэрэг хүмүүстэй илүү ойрхон юм. Өөрөөр хэлбэл үзэл санааг номлох нь цаг хугацааны харьцааг тусгадаггүй, түүнтэй холбоогүй байдаг. Цаг хугацааг ирээдүйчлэн хүлээн авахуйн олон хувилбар нь түүний мандалт нь угсаатны задралын үйл явцыг өдөөдөгт оршино. Ийм үйл явц нь бидний судлан буй бүх үеүдэд ажиглагдаж байгаа учраас угсаатан мөхөх, түүчлэн шинэ угсаатан бий болох зэрэг нь тохиолдол биш гэдэг нь тодорхой байна. Үүний аль аль нь угсаатны нийлэгжилт хэмээх диалектик нэг л үйл явцын бүрдлүүд болно. Хэрэв бид хүн болохынхоо хувьд дурын оюун сэтгэгдэл буюу хэв шинжид талархалтай байж болно, гэхдээ эрдэмтний хувьд судлан буй үзэгдлийнхээ ерөнхий чиглэл дэх бүрдүүлэгч хэмжигдэхүүний харьцаа, болон векторуудыг зүгээр л тодорхойлох ёстой.
Пассеизм, актуализм, футуризм гурав нь угсаатны хөдлөнги гурван үе шатыг тусгадаг, гэхдээ түүнээс гадна угсаатны тогтонги төлөв байдалд харгалзах цаг хугацааны категорийг үнэлдэг өөр систем үнэхээр оршин байх ёстой юм. Энэ нь цаг хугацааг байгаа байдлаар нь үгүйсгэхэд оршдог. Энэ хэв шинжийн хүмүүс цаг хугацааг сонирхдоггүй, учир нь тэд цаг хугацааны тооллоос тэднийг өөрсдийг нь тэжээж буй үйл ажиллагаандаа ямар ч ач тустай зүйл гаргаж авдаггүй. Энэ хүмүүс (бид тэднийг дээр бэртэгчин гэж нэрлэсэн) бүхий л үе шатанд амьдардаг бөгөөд гэхдээ бусад категорийнхан байгаа үед тэд бараг ажиглагдаггүй. “Футуризм” мандаад ирмэгц л тэдний бүх өрсөлдөгчид алга болж, цоорхойнуудаас үхэж өгдөггүй тааруухан амьтад мөлхөн гардаг ба ингээд түүхэн цаг хугацаа зогсож, газар дэлхий ууран дор орно.
Ингээд бид өөрсдийнхөө шинжилгээний бүх шугамыг хааж, угсаатны бүрэлдлийн дөрвөн гишүүнт бүтцийн тухай таамаглалынхаа батлагааг оллоо. Эн бол тохиолдлын давхцал биш, дур зоргоороор хийсэн зохиомж биш, харин угсаатны задралын үйл явцын мөн чанарыг тусгасан хэрэг юм. Хэрэв бидний шинжилгээ энэ сэдвийг шавхчихсансан бол зөвхөн этнологи төдийгүй, угсаатан ч аль хэдийнээ байхгүй байх байсан. Учир нь тэд бүгдээрээ өнгөрсөн түүхэн үеүдэд задарчихсан байх билээ. Угсаатны дотоод хувьслын эвдлэгч үйл явцын зэрэгцээгээр угсаатны шинэ нийтлэг үүсэхэд тус болдог бүтээн байгуулагч зүйл оршин байгаа нь тодорхой байна. Ийм учраас хүн төрөлхтний угсаатны түүх дуусахгүй, Дэлхий дээр хүн байсан цагт энэ нь зогсохгүй юм. Учир нь угсаатан бол хүн хэмээх нэгжийн арифметик нийлбэр биш, харин нарийн нээвэл зохих “систем” хэмээх ойлголт болно.
YII. Угсаатан систем болох нь
ДЭЛГЭРСЭН ТАЙЛБАРЛАЛ ДАХЬ “СИСТЕМ”
Социал систем бүхний тодорхой жишээ нь нэг гэрт амьдардаг гэр бүл юм. Системийн элементүүд нь гэр бүлийн гишүүд, амь зуулгын зүйлс, түүний дотор нөхөр, эхнэр, хадам эх, хүү, охин, байшин, худаг, муур зэрэг болно. Эдгээр нь гэр бүлийг эхнэр нөхрүүд нь салаагүй, хүүхдүүд нь өөрсдөө ажил хийж салан яваагүй, хадам эх нь хүргэнтэйгээ хэрэлдээгүй, худаг нь ногоороогүй, муур зулзгуудаа дээвэрт орогнуулаагүй байх хүртэл гэр бүлийг бүрдүүлнэ. Үүний дараа ч гэсэн тэд гэртээ үлдлээ гэхэд тийш нь ядахдаа усан хоолой татахад л энэ нь гэр бүл биш, хүмүүс амьдардаг газар, өөрөөр хэлбэл амьд болон байгалийн амьгүй бүх элементүүд байрандаа үлдэж, гэхдээ гэр бүлийн систем устсан газар болох юм. Эсвэл урвуугаар хадам эх нь нас барж, шинэ байшин барьж, муур нь алга болж, хайртай хүү нь алс явсан ч гэсэн элементүүдийн тооны өөрчлөлтийг үл харгалзан гэр бүл хэвээр үлдэнэ. Энэ нь системийн бодитой оршин буй болон үйлчилж буй хүчин зүйл нь юмс биш, харин жин, цэнэг, халуун байхгүй ч гэсэн холбоо болж байна гэсэн хэрэг юм. Энэхүү харилцан төсөөгүй үед тухайлсан хүмүүсийн хоорондын дотоод холбоо нь системийн холбооны бодит илрэл болддог ба үүнийг өөр ямар ч үзүүлэлтээр тодорхойлж болдоггүй.
Систем дэх холбоонууд нь эерэг ч, мөн сөрөг ч байж болно, учир нь дэд системийн зарим холбоо хүний амьдралын туршид тэмдгээ өөрчилж болно. Энэ жишээ үргэлжлүүлье. Ахмад хүнтэй шинэ төрсөн хүний холбоо нь тодорхой чиглэмж, “жинтэй” болдог. 27. Холбогдох баримтын үеийн (кибернетик утгаар) коэффициент нь жишээлбэл, эцэг нь хүүдээ халамж тавих явдал болно.
Түүнд халамж тавьж, түүний хүмүүжүүлж, сургадаг. Тэрээр насанд хүрч гэрийн эзэн болмогц холбооны тэмдэг нь эсрэгээр өөрчлөгдөж, тэр настнууддаа санаа тавьж, хүүхдүүдэээ хүмүүжүүлдэг. Эцэст нь тэр хөгшин болоод ахиад л халамж, анхаарал шаарддаг. Энэ зүй тогтол нь аливаа систем тогтонги биш, харин эсвэл өөрчлөгдөх тэнцвэрт байдалд (гомеостаз) байна, эсвэл тухайн системээс гадна орших ямар нэг түлхэлт, лугшилтаас болж хөдөлгөөнд ордгийг харуулж байна. Мэдээжийн хэрэг энэ лугшилт нь дээд түвшний системийн хувьд хязгаарлагдмал байхыг үгүйсгэхгүй, гэхдээ л үйлчлэлийн механизм нь үүнээс болж өөрчлөгдөхгүй.
Гэр бүл нь системийн илт жишээ болно. Гэхдээ илүү нарийн систем, жишээлбэл, угсаатан, социал организм, биогеоценез зэрэг дэд систем нь системийн бүхэллэгийг бүрдүүлж, тэр нь хэт системийг, хэт систем нь гипер системийг бүрдүүлэх гэх зэргээр шаталсан зарчмаар байгуулагддагийг нь харгалзсан ч гэсэн мөн ийм зүй тогтолд захирагддаг.
Ийм маягаар хөдлөнги тогтсон үзлийг бий болгогч түгээмэл холбоо нь их бага хэмжээгээр тогтвортой авч хэзээ ч мөнх байдаггүй юм.
Ингээд системийн хувьд угсаатны тогтвортой байдлын хэмжээ нь түүний массаар, өөрөөр хэлбэл хүн амын тоо, өвөг дээдсээ яг хуулсан байдал, мөн статистик дундаж холбооны нийлбэрээр тодорхойлогдохгүй. Тодорхой хязгаараас эрс гарах нь эсвэл үхэл, эсвэл эрчимтэй хөгжил болно. Чухам үүгээр л угсаатны уян хатан байдал бүтээгддэг бөгөөд энэ нь угсаатанд гадаад үйлчлэлийг зөөлрүүлэх, тэр байтугай заримдаа нөхөн сэргэхэд нь тус болдог. Учир нь “олон холбоост” систем нь холбооны дахин өөрчлөлтийн алдагдлыг нөхөж байдаг.
Энэхүү хялбарчилсан тайлбарлалын дараа шинжлэх ухааны тодорхойлолт руу шилжиж, өөрөөр хэлбэл бидэнд хэрэгтэй буй кибернетик болон системчлэлийн ухаанд оръё.
ЭТНОЛОГИ ДАХЬ “СИСТЕМ”
Н.Винер кибернетикийг амьтан ба машин дахь холбоо болон удирдлагын тухай шинжлэх ухаан гэж тодорхойлсон. “Кибернетикийн давуу тал нь нарийн системийг судлах аргадаа байдаг бөгөөд энгийн системийг судлах үед кибернетикт давуу тал байдаггүй” Кибернетикийн судлах зүйл нь объектын зан үйлийн аргууд байх бөгөөд кибернетик нь “энэ юу вэ ? “гэж асуудаггүй, харин “энэ юу хийж байна ?“ гэж асуудаг. “Ийм учраас объектын шинж чанар нь түүний зан үйлийн нэр байдаг”. “Кибернетик нь зан үйлийн бүх хэлбэрийг судалдаг, учир нь эдгээр нь тогтмол, буюу шалтгаацсан, буюу нөхөн сэргээгдсэн байдаг. Түүний хувьд материаллаг байдал нь физикийн ердийн хуулийг мөрдөх, эс мөрдөх нь адилхан байдаг шиг ач холбогдолгүй байдаг”. 28. Росс Эшби У. Введение в кибернетику. М., 1959. С. 13, 29. Там же. С. 18, 30. Там же. С. 13, 31. Там же. С. 21, 32. Там же. С. 14.
Энд үзсэн сэдвүүд нь угсаатны үзэгдлийн мөн чанарыг сонирхож буй этнологи хүн бүхэнд тодорхой байгалийн хуулийг ажигласнаас гарган авсан өөрийн жишээг Винерийн кибернетикийн аргуудад бүрэн итгэж албаар зохицуулах нь буруу гэдгийг харуулж байна. Кибернетик аргуудыг хэрэглэхдээ эмпирик дүгнэлтийн экстраполяцид л засвар хийхэд ашиглаж болохоос илүү зүйлд хэрэглэж болохгүй юм. Ийм учраас угсаатныг системийн хувьд судлах арга зүйн үндэс болгон Н.Винерийн санааг биш, харин кибернетикийг физик, хими, термодинамиктай хослуулсан Л.Берталанфийн үзэл санааг авах нь зүйтэй.
Л.Берталанфийн системийн хандлагын дагуу “систем бол харилцан үйлчлэлд оршиж буй элементүүд юм” , өөрөөр хэлбэл мэдээллийн хэвийн элемент нь тусгай тусгай баримтууд биш, харин баримтуудын хоорондын холбоо юм” М.М.Малиновскийн үзэж байгаагаар “систем нь нэгжүүдээс бүрдэх бөгөөд тэдгээрийг бүлэглэх нь салбар, дэд систем гээд бие даасан ач холбогдолтой болдог, эдгээр тус бүр нь доод зэргийн нэгж болох ба энэ нь өгөгдсөн түвшинд судалгаа явуулахад хүргэдэг шаталсан зарчмуудыг хангаж байдаг” аж.
33. Берталанфи Л. Общая теория систем – критический обзор //Исследования по общей теории систем? Под ред. В. Н. Садовского, Э. Г. Юдина. М., 1969. С. 28., 34. Садовский В. Н., Юдин Э. Г. Задачи, методы и приложения общей теории систем //Там же. С. 12 , 35. Малиновский А. А. Общие вопросы строения системы и их значение для биологии //Проблемы методологии системного исследования /Под редакцией И. В. Блауберга и др. М., 1970. С. 145-150.
Энэ зарчмыг үндэслэн бид угсаатныг өөртөө элемент бүхий нийгмийн болон байгалийн нэгжтэй систем гэж үзэх эрхтэй болж байна. Угсаатан бол өөр хоорондоо аль нэг шинжээрээ төстэй хүмүүсийн жирийн бөөгнөрөл биш, хүсэл сонирхол, чадвараараа янз бүрийн хувь хүмүүс, тэдний үйл ажиллагааны бүтээгдэхүүн, уламжлал, газар зүйн орчин, хөршүүдийн хүрээлэлд багтан амьдардаг, түүнчлэн системийн хөгжилд нь ноёрхогч тодорхой хандлага бүхий систем болно. Хөгжлийн чиглэл болох хандлага нь “тогтонги болон хөдлөнги бүтэц бүрэлдэхэд хүргэдэг бүхий л зүйлийн идэвхийг эзэмшигч элементийн шинж чанар болж, бүх олонхи тохиолдлын хувьд нийтлэг байдгаараа” маш чухал юм. Энэ хандлагыг угсаатны нийлэгжилтийн үйл явцад хэрэглэх нь түүхчлэлийн асуудлыг шийдвэрлэхтэй холбоотой юм, яагаад гэвэл бүхий л ажиглан буй баримтууд нь түүхэн хөгжлийн хөдлөнги системд багтаж байдаг. Иймээс бидэнд манай сэдэвт шууд холбогдсон Дэлхий дахины түүхийн тэрхүү хэсгийг л шинжлэх л үлдэж байна. 36. Рашевский Н. Организмические множества. Очерк общей теории биологических и социальных организмов //Исследования по общей теории систем. М., 1969. С. 445.
Ийнхүү угсаатны бодит бүхэллэг байдлыг өөртөө зөвхөн хүмүүсийг төдийгүй, мөн ландшафтын элементүүд, соёлын уламжлал, хөршүүдтэйгээ харилцах зэргийг багтаасан хөдлөнги систем гэж тодорхойлж болно. 37. Энэ бол биологийн болон социал түвшний адилтгал нь үндэслэгдээгүй (Машновский А. А. Общие вопросы. С. 182) социал систем биш төдийгүй, биологийн бус систем юм.
Ийм системд эрчим хүчний анхдагч цэнэг нь байнга зарцуулагдаж, энтропи буюу замбараагүйдэл байнга ихэсч байдаг юм. Иймээс систем нь хүрээлэн буй орчинтой эрчим хүч, энтропи байнга солилцож хуримтлагдан буй энтропийг байнга зайлуулах ёстой. Энэ солилцоог удам дамжин үлддэг, мэдээллийн нөөцийг ашигладаг удирдах систем зохицуулдаг юм. 38. Свиридов М. Н. На переднем крае космической науки //Природа, 1966. № 8. С. 112.
Манай тохиолдолд удирдах системийн үүргийг материйн хөдөлгөөний нийгмийн болон байгалийн хэлбэрүүдтэй жигд харилцан үйлчилж байдаг уламжлал гүйцэтгэдэг. Туршлагаа үр удамдаа дамжуулах нь халуун цуст ихэнхи амьтдад ажиглагддаг. Гэхдээ зэвсэг, хэл болон бичиг байгаа нь хүнийг бусад сүүн тэжээлтний тооноос гаргадаг бөгөөд угсаатан бол зөвхөн хүнд л байх хамтын ахуйн хэлбэр болно.
УГСААТНЫ СИСТЕМИЙН ТҮВШИН БА ХЭВ МАЯГУУД
Бидний авсан хандлага угсаатны ангиллыг угсаатны систем зүйгээр солих болгож байна. Ангиллыг хэл, арьстан, шашин, ажил төрөл, аль нэг улсад хамаарах зэрэг дураар авсан ямар ч шинж тэмдгээр хийж болдог билээ. Энэ нь ямар ч тохиолдолд маш нөхцөлд хуваалт болно. Систем зүй нь чухамхүү байгальд байгаа юмыг тусгаж, хүнийг техник болон доместикаттай (гаршуулсан мал, таримал ургамал) нь судлах боломж олгодог юм. Хүн төрөлхтний дараах томоохон бүхэл (Дэлхийн бүрхэвчийн нэг болох–аморф буюу хувирдаггүй антропо хүрээ) нэгж бол хэт угсаатан бөгөөд өөрөөр хэлбэл нэгэн зэрэг тусгай бүс нутагт үүссэн, угсаатнуудаас бүрэлдсэн зүймэл маягийн бүхэллэг байдлаар түүхэнд илэрдэг угсаатнуудын бүлэг юм. Чухамхүү эдгээр нь шууд ажиглагдах угсаатны таксономи буюу эрэмбэт байрлал болно. Угсаатнуудыг өөрийг нь угсаатны бүрэлдэхүүнд орж байдгийнхаа ачаар л оршин байдаг салбар буюу дэд угсаатан гэж хуваадаг. Эдгээр нь угсаатангүйгээр сарниж, мөхдөг.
Таксономи буюу эрэмбэт байрлалын аль нэгэн хэсэгт хамаарагдах нь байгальд хэзээ ч байдаггүй хүний туйлын адил байдлаар биш, харин өгөгдсөн түвшинд тодорхой аспектаар төстэйнхээ хэмжээгээр тодорхойлогдоно. Мусульманчууд ч гэсэн хэт угсаатны түвшинд (Дундад зууныг жишээ болгон авъя) араб, перс, туркмен, берберүүд нь бүх католикуудыг нэрлэж байсан “франк” хэмээх баруун христианы угсаатны гишүүдийг бодоход өөр хоорондоо хавьгүй ойрхон байсан. Харин ерөнхий хэт үндэстэнд орсон франц, кастил, шотландчууд нь мусульман, үнэн алдарт гэх мэтийн бусад хэт угсаатнаас өөр хоорондоо ойрхон байсан. Франц угсаатны түвшинд тэд өөр хоорондоо англичуудыг бодоход бас л ойрхон байлаа. Энэ нь хэдийгээр тэд өөрсдийнхнийхээ эсрэг явж байгаагаа ойлгож байсан ч Бургундчууд IY Генрихийг дэмжиж, Жанна д’Аркийг олзлон авахад нь саад болоогүй юм. Гэхдээ ямар ч тохиолдолд харагдах түүхийн бүхий л олон янз байдлыг зөвхөн заримдаа л хүний зан үйлийг тодорхойлогч хүчин зүйл болдог угсаатны нийтлэгийг ухамсарлахад оруулж үзэх ёсгүй юм. Гэхдээ угсаатны ойр дотнын мэдрэмж ямагт байдаг ба магадгүй хүний мөн чанарын инвариант буюу өөрчлөгдөөгүй үлдсэн зүйл байж болох юм. Өөр үгээр бол угсаатан хэчнээн зүймэл байсан ч, түүний бүтэц нь хэчнээн олон янз байсан ч өгөгдсөн түвшиндөө тэр нь бүхэллэг зүйл л байдаг.
Энд хамгийн сонирхолтой нь энэхүү хандлагын боломжийг түүхчид нэгэнт практикаар мэдэрч эхэллээ. Тэд санамсаргүйгээр угсаатнуудыг эсвэл “соёл”, эсвэл “соёл иргэншил”, эсвэл” ертөнц” хэмээн нэрлэж, бүтэц болгон бүлэглэж байна. Жишээлбэл, XII–XIII зууны хувьд тэр үедээ бодитой оршин байсан бүхэллэгийг тэмдэглэж байсан тийм ойлголтын утгыг бид олж байна. Тухайлбал, Римийн папын үзэл суртлын ноёрхол дор оршин, түүнийгээ хэрэг дээрээ албан ёсоор хэзээ ч хэрэгжүүлж байгаагүй, германы эзэн хааны хараат байсан Баруун Европ өөрсдийгөө “Христианы ертөнц” гэж нэрлэж байв. Чингэх мөртлөө Баруун европчууд Испани, Палестинд өөрсдөө байлдаж байсан мусульмануудын эсрэг төдийгүй, бас үнэн алдарт грекууд, оросуудын эсрэг, бүр гайхалтай нь ирландын болон уэльскийн кельтүүдийн эсрэг өөрсдийгөө тавьдаг байв. Тэд шашны нийтлэгт биш, дураараа сонгосон шалгуур нэрнээс гарган авсан системийн бүхэллэгийг төлөөлж байсан нь нэн ойлгомжтой байна.
Үүнтэй адилаар “Исламын ертөнц” гэдэг нь өөрсдийгөө грек, франц, шашны үзлээрээр нэгдмэл байгаагүй хэлний туркуудын эсрэг тавьдаг байв. Шийт (теист), кармат (атеист), суфийн (пантеист) сургаалиуд нь өөр хоорондоо болон исламын ортодоксиаль сургааль–суннизмтай ойртох юмаар бага байлаа. Дэргэд нь христиан–европууд ч гэсэн өөр хоорондоо огтхон ч найзлаад байгаагүй юм. Гэхдээ мусульман буюу хэлтнүүдтэй тулгармагцаа тэд шууд л хэл ам, эвслийн замаа олдог байжээ. Энэ нь жишээлбэл, венец болон генуя хоёр хүн тэдний дээрээс араб буюу бербер–мусульман хүн иртэл л зодолдоно, харин нэмж ирсэн нийтийн өрсөлдөгч рүү саяхны хоёр дайсан хамтран дайрна гэсэн үг юм.
Хатуу ширүүн дайнууд цаг ямагт ойрын төрлүүдийн хооронд болдог гэдэг нь түүхнээс тодорхой. Түүний хамт эдгээр дайн нь том системийн түвшинд явагдах дайнуудаас язгуурын ялгаатай байдаг. Том дайнд эсрэг талынхнаа ямар нэг өөр төрлийн, саад бологч, устгавал зохилтой гэж үздэг. Гэхдээ уур, үзэн ядалт, атаархал гэх мэтийн хувийн сэтгэл хөдлөл энд илрэх харгис байдлын шалтаг болдоггүй юм. Систем нь бие биенээсээ хэр зэрэг хол байх тусам тэр чинээгээр хайр найргүй, харилцан устгаж, энэ нь аюултай агнуур мэт болдог. Бар юмуу матарт уурлаж болно гэж үү ? Үүний урвуугаар системийн дотоод дахь дайн нь өрсөлдөгчөө устгах биш, харин ялах зорилготой байдаг. Учир нь дайсан нь мөн л системийн хэсэг болдог учраас түүнгүйгээр систем оршиж чадахгүй юм. Жишээлбэл, флорентийн гибеллинуудын удирдагч Фарината Дельи Уберти эх орныхоо дайснуудад ялалт байгуулахад нь туслаад Флоренц хотыг устгуулаагүй байна. Тэгээд тэр “Би энд амьдрахын тулд энэ хоттой тулалдсан юм” гэж хэлж байжээ. Ингээд тэр Арби гол түүний дайсан флорентийн гвельфуудын цусаар хүрэнтсэний дараа тэндээ нас эцэслэтлээ амьдарсан юм.
Гэхдээ энэ бол юу ч биш. Алдарт гибеллин Эццелино да Романы ах Альберригог венецчүүд бүр ч аймаар шийтгэжээ. 1260 онд тэрээр Тревизогийн орчим байх цайзаа тэдэнд тавьж өгөхөд түүний зургаан хүүг нь өөрийнх нь нүдэн дээр алж, дараа нь өөрийнх нь толгойг авч, улмаар түүний эхнэр болон хоёр охиныг нь Тревизогийн талбай дээр амьдаар нь шатаасан байна. Яалаа гэж ийм утгагүй харгислал хийдэг байна аа ?
Энэ нөхцлийг ойлгохын тулд ард нь ямар ч утга санаа нуугдан байж болох алгебрын тэмдэг “гвельфы болон гибеллин” гэдгийн учрыг ойлгох ёстой. 39. История Италии: В 3 т. Т. 1. /Под ред. В. Д. Сказкина. М., 1970. С. 233. Гибеллин нь феодалууд, гвельф нь бюргерүүд гэж тооцогддог байж: гэвч хэд хэдэн хот хуваагдаж гибеллины талд орох, зарим гвельф нь гибеллин болох, үүний эсрэгээр ч болж байжээ. Энэ хоёр нам араб, грекүүдийн эсрэг хамтран тэмцэж ч явж, Ингээд Генуя болон Венец шиг том хотын бүгд найрамдах улс нэг биш удаа нэг лагераас нөгөөд шилжсэнээр энд зөвхөн улс төрийн тооцоо удирдлага болох болжээ. Юунаас болж цус урссан юм бэ ? 40. Соколов Н. П. Венеция между гвельфами и гибеллинами //Вопросы истории. 1975. № 9. С. 142-153.
Системийн бүхэллэг чанарыг тогтоох арга нь эрин үеэс, нарийн яривал угсаатны нийлэгжилтийн шатнаас хамаардаг. Залуу системд элементүүд маш хүчтэй, тэсгэлгүй гэж хэлж болохоор харилцаж мөргөлдөөнд хүрдэг. Цуст алаан нь ямагт үзэл суртлын ч, ангийн утгыг ч авчирдаггүй ч жишээлбэл, Англи дахь Улаан болон Цагаан сарнайн дайн, Франц дахь арманьяк болон бургундын дайн лугаа нийгмийн нэг давхаргын хүрээнд л явагддаг. Гэхдээ энэхүү дотоодын дайн нь угсаатны системийн бүхэллэг байдал ба төрийг сайнаар тэтгэдэг. Хэдийгээр хүн ам гөлгөр байдлаар тэнд амьдрахад амархан ч угсаатнууд бүхэллэг байдлын хувьд задарч, алга болж байдаг.
Бидний нэгэнт дээр дурдсанаар угсаатны системүүд нь голдуу төрийн бүрдлүүдтэй адилтгашгүй байдаг: зарим угсаатан янз бүрийн улсад амьдрах буюу хэд хэдэн угсаатан нэг улсад амьдарч болдог. Тэгвэл бид ямар утгаар нь тэдгээрийг систем гэж тайлбарлах вэ ?
Системийг идеал хэв маягийн хоёр хэв маягт хувааж заншсан байдаг. Энэ нь хатуу болон корпускуляр буюу тасралттай систем болно. Хатуу системд бүх хэсэг нь (элементүүд) нэгэн зэрэг оршин байхад зайлшгүй хэрэгтэй хэвийн үйл ажиллагааны тулд бие биендээ таарч байдаг. Корпускуляр системд элементүүд нь чөлөөтэй харилцан үйлчилж, адилтгах элементээр амархан солигдож, ингэсэн ч систем ажиллахаа болихгүй, тэр ч байтугай элементүүдийнхээ зарим нь алдагдсан ч дахин сэргэдэг. Хэрэв ийм зүйл дахин явагдахгүй бол түүнийг устгах хязгаар бүхий системийн хялбаршил явагдана.
Системийн өөр ангилал ч бий. Энд эрчим хүч байнга хүлээн авч, орчинтойгоо эерэг болон сөрөг энтропигоо сольж байдаг нээлттэй, орчны потенциалтай өөрийнхөө потенциалыг тэнцтэл нь зөвхөн анхдагч цэнэгээ зарцуулдаг хаалттай гэж хоёр хувааж болно. Энэ хоёр шинж тэмдгийг харьцуулахын тулд системийн дөрвөн боломжит хувилбарыг авч үзье : 1 ) хатуу нээлттэй, 2) хатуу хаалттай, 3) корпускуляр нээлттэй, 4) корпускуляр хаалттай. Энэ хуваалт нь нөхцөлт шинжтэй бөгөөд үйлчлэн буй аль ч систем нь хоёр хэв маягийн шинжүүдийг багтаадаг, гэхдээ аль нэгэнд нь ойрхон байдаг, Иймээс энэ хуваалт нь практикийн хувьд зүйтэй бөгөөд элементүүдийн хавсран захирагдсан хэмжээгээр системийг ангилж болдог.
Төрийн ч , угсаатны ч түүхийг судлах үед захынхаас бусад, өөрөөр хэлбэл зөвхөн хатуу буюу зөвхөн тасралттай, эдний аль аль нь амьдрах чадваргүй байдгаас бусад дээр дурдсан хэв маягийн системийн дурын бүлэгтэй бид тааралддаг. Хатуу систем нь эвдрэлийн үед дахин сэргэдэггүй, тасралттай систем нь гаднын цохилтыг эсэргүүцэх чадваргүй байдаг. Ийм учраас практик дээр хатуулгын янз бүрийн зэрэгтэй системтэй тааралдана. Хатуулаг нь их байх тутам түүнд хүний хөдөлмөрөөр оруулсан зүйл их байна, хатуулаг нь бага байх тусмаа системийг бүтээхэд түүнийг бүрдүүлэгч элементүүдийг байнга өөрчилж байдаг байгалийн үйл явцын оролцоо ихэснэ. Энэ хязгаар дотор техно болон био хүрээг сөргүүлэн тавьж болно.
Хэрэв хүний организм өөрөө байгалийн хэсэг юм бол био хүрээ болон техно хүрээний зааг ялгаа хаана байдаг вэ ? Социо (техно) хүрээ болон био хүрээний зааг нь зөвхөн хүний биеийн хязгаарт биш, харин түүний дотор байдаг нь тодорхой байна. Гэхдээ үүнээс болоод ялгаа нь алга болчихгүй. Харин ч урвуугаар бид энд биологийн зүйлтэй социал зүйлийн харилцан үйлчлэх бодит агшинг мэдэрч байна. Энэхүү байгалийн бие даасан үзэгдэл бол бүхэнд маш тодорхой буй угсаатан юм.
Угсаатан нь төгс байдлаараа корпускуляр систем бөгөөд хөршүүд, хүмүүсээр устгуулахгүй байхын тулд түүнийг бүрдүүлэгчид угсаатны хувьд туслах чанартай хатуу систем болох бий болгосон буюу зээлж авсан институтыг тогтоож өгдөг. Энэ зүйл нь жишээлбэл, овог доторхи ахмадын эрх мэдэл, ан буюу дайнд удирдан жолоодох, гэр бүлийн хувьд хүлээх үүрэг, эцэст нь боловсрол болон төр болно. Ийм маягийн хатуу систем гэвэл нийгэм-улс төрийн бүрдэл: төр, овгийн холбоо, клан, нөхөрлөл зэрэг болно. Ингээд хоёр хэв шинжийн систем, өөрөөр хэлбэл угсаатан төр, овгийн холбоо зэрэг давхцах нь хэдийгээр жам ёсны мэт боловч заавал биш юм. Янз бүрийн угсаатныг нэгтгэсэн эртний агуу их эзэнт гүрнүүд, угсаатнуудын дундад зууны феодалын бутралыг санацгаая. Үзтэл хослолын гайхамшиг нь давхцалтайгаа яг адилхан жам ёсны ажээ. Хоёр хэв маягийн систем нь өөрчлөгддөг, өөрөөр хэлбэл түүхэн хугацаанд үүсч, алга болж байдаг. Өөрчлөлт нь зөвхөн гадаад нөлөөтэй холбоотой байдаг гомеостатик систем л энэ зүй тогтлоос өөр байдаг аж. Гомеостаз нь хүчилсэн хөгжлийн дараа, системийг бүтээж, хөдөлгөсөн хүч дуусахад л бий болдог аж. Иймээс статистикийн бараг хүрч болшгүй тэг хязгаарт байгаа удаасгасан инерцит хөдөлгөөн гэж үзэх хэрэгтэй юм.
YIII. Дэд угсаатнууд
УГСААТНЫ БҮТЭЦ
Угсаатны бүтэц нь их бага хэмжээгээр нарийн, гэхдээ чухамхүү энэ нарийн байдал нь энх тайвнаар орших, үймээн, самууны зуунуудыг даван туулах боломж олгодгоороо угсаатны тогтвортой байдлыг хангадаг юм. Угсаатны бүтцийн зарчмийг дэд угсаатны бүлгүүдийн шаталсан хамтран захирагдсан байдал хэмээн нэрлэж болно. Угсаатны дэд бүлэг гэдэгт угсаатны дотор илтэд бүхэллэг байдлаар оршиж, түүний нэгдлийг зөрчдөггүй таксоном буюу эрэмбэт нэгж гэж ойлгож болно. Өнгөц харахад ингэж томъёолсон сэдэв нь энгийн бүхэллэг байдлаар угсаатан оршдог тухай бидний үндэслэлтэй зөрчилдөж байна, гэхдээ гэхдээ бодисын молекул ч гэсэн атомаас, атом нь эгэл хэсгүүдээс, бүрддэг ба энэ нь молекул буюу атомын аль нэг түвшин дэх бүхэллэг шинжийн тухай нотолгоог үгүйсгэдэггүй. Бүх хэрэг бүтцийн холбооны шинж чанарт байдаг. Үүнийг жишээгээр тайлбарлая.
Тверь губерний карел хүн тосгондоо бол өөрийгөө карел гэнэ, Москвад суралцахаар ирэхдээ орос гэнэ, учир тосгонд бол карел орос хоёрыг сөргүүлэн тавих нь ач холбогдолтой байсан, харин хотод бол байхгүй, энд ахуй, соёлын ялгаа нь харагдахааргүй тун өчүүхэн. Хэрэв энэ хүн карел биш, татаар хүн байсан бол тэрээр Москвад өөрийгөө татаар гэж үргэлжлүүлэн нэрлэнэ. Учир нь энд шашны ач холбогдол нь оростой угсаатны зүйн төсгүй байдлыг нь гүнзгийрүүлж байна. Иймээс өөрийгөө орос гэж зарлахад багагүй зүйл хэрэгтэй. Харин энэ татаар Баруун европ юмуу Хятадад очвол тэнд өөрийгөө орос гэхээс гадна өөрөө ч үүнийг зөвшөөрнө. Харин Шинэ Гвинейд бол мань хүн европ болох бөгөөд гэхдээ зөвхөн англи, голланд “овгоос” л биш. Энэ жишээ нь угсаатны диагностик, тэр тусмаа хүн ам зүйн газар зүй, угсаатны зүйн зурагт маш чухал. Угсаатны зургийг хийх үед ойртолтын журам, зэрэг мэтийн тухай нөхцлийг тохирох зайлшгүй хэрэгтэй байдаг. Ингэхгүй бол угсаатны бүтцийн элемент мэтээр оршин байгаа дэд угсаатныг жинхэнэ угсаатнаас ялгах боломжгүй болдог.
Одоо угсаатны захирагдах чанарыг авч үзье. Жишээлбэл, цул угсаатны тод жишээ болох францууд нь өөртөө дээр ярьсанчлан, бретоны кельт, баск гаралтай гаскон, алеманнуудын удам дотаринг, романы бүлгийн бие даасан ард түмэн провансальчуудыг багтаадаг. IX зууны дунд үед “франц” хэмээх угсаатны нэрийг анх удаа баримтаар тэмдэглэхэд дээр дурдсан бүх ард түмэн, түүнчлэн бургунд, норманн, аквитанц, савояри нар хараахан нэгдмэл үндэстэн болоогүй байсан билээ. Зөвхөн угсаатны нийлэгжилтийн мянган жилийн үйл явцын дараагаас бидний франц гэж нэрлэж буй угсаатан бүрэлдсэн юм. Ойртон нийлэх үйл явц нь харин ч салбар ёс заншил, зан үйл зэргийг арилгаагүй билээ. Эдгээр нь францын угсаатны бүхэллэг чанарыг үл зөрчин орон нутгийн зах хязгаарын онцлог байдлаар хадгалагдан үлдсэн байна.
Францад бид угсаатны интеграцийн үр дүнг нэн тодорхой хардаг юм. Дахин Сэргэлтийн үеийн үйл явдлын явц нь франц–гугенотуудыг XYII зуунд эх орноо орхиход хүргэсэн билээ. Тэд амиа аврах гэж өмнөх угсаатны хамаарлаа алдаж, немцийн тайж, голландын бюргер, олонхи нь колоничилсон өмнөд Африкийн бурчууд болцгоосон юм. Тэртэй тэргүй олон янз байсан Франц угсаатан тэднээс бүтцийнхээ илүүдэл элемент мэт салсан юм. Гэхдээ Франц нь нийгэм–улс төрийн бүхэллэг шинжээрээ сулраагүй, харин ч хүчирхэгжсэн юм. Хичээл зүтгэлтэй гугенотуудын орхисон талбай, цэцэрлэг дотоод дайнд нэрвэгдээгүй, XYIII зуунд аж ахуйг сэргээн босгосон ямар ч ялгаагүй хүмүүс нүүн ирсэн билээ. Ингэж үүссэн угсаатны цул байдал нь Наполеонд хүн амын дайчилгаа явуулах, хамгийн олон тоотой бөгөөд дуулгавартай арми бий болгох боломж олгосон юм. Ялагдлын дараа ч орон нутгийн салан тусгаарлах үзлийн үлдэгдлийг үл харгалзан Франц улс задраагүй үлдэж чадсан юм.
УГСААТНЫ ӨӨРИЙН ЗОХИЦУУЛАЛТ
Угсаатанд өөрийгөө зохицуулах чадвар байгаа гэж бид хэлбэл чамирхалтай байж болох юм. Гэхдээ угсаатан нь түүхэн хөгжилдөө динамик буюу өөрчлөмтгий шинжтэй, улмаар аливаа удаан явагддаг үйл явц шиг өөрийн оршихуйг тэтгэхийн тулд эрчим хүчний хамгийн бага зардал гаргадаг. Бусад нь шалгарлаар тасарч, мөхдөг. Бүх амьд системүүд устгах явдлыг эсэргүүцдэг ба, өөрөөр хэлбэл антиэнтропи шинжтэй байж гадаад нөхцөлд боломжийнхоо хэрээр дасан зохицож байдаг. Бүтцийн зарим нарийн нийлмэл байдал нь угсаатны гадаад цохилтыг даах эсэргүүцлийг нэмэгдүүлдэг, угсаатан төрөхдөө зүймэл шинж хангалтгүй байсан тэр газар, жишээлбэл XIY – XY зууны Их Орос язгуур угсаатны байдлаар заримдаа төлөвшдөг дэд угсаатны бүрдлүүдийг өөрөө гаргаж байсан нь гайхалтай зүйл биш юм.
41. Бид байгалийн иймэрхүү үйл процессийг ярихдаа антропоморфизм буюу амьгүй зүйлсийг амьтай мэт үздэггүй, харин “горхи гольдрилоо тавьж, булан тохой үүсгэсэн” гэх мэтийн хэвшмэл үгсийг хэрэглэдэг.
Өмнөд хязгаараар казакууд, хойд хэсгээр поморууд ялгаран гарчээ. Хожим тэдэн дээр газар нээгчид нэмэгдсэн бөгөөд тэд анх харахад тодорхой ажил төрлийн жирийн л төлөөлөгчид байж, тэднийг дагасан тариачид Сибирийн унаган иргэдтэй холилдон сибирчүүд гэсэн дэд угсаатан буюу “челдончуудыг” бүрдүүлжээ. Сүм хийдийн хагарал нь хуучин ёслолтнууд хэмээх дэд угсаатны бас нэгэн бүлгийг бий болгосон бөгөөд тэд нь оросын үндсэн олон түмнээс угсаатны зүйн хувьд ялгардаггүй байв. Түүхийн явцад эдгээр угсаатны дэд бүлгүүд угсаатны үндсэн хэсэгтээ уусаж, гэхдээ энэ үед шинэ дэд угсаатан төрөн гарсаар байлаа.
Жишээлбэл XYIII зууны хоёрдугаар хагаст баян тайж, язгууртнууд хүүхдүүддээ франц–губернер хэмээх гэрийн багш авчрах болов. 1789 оноос хойш Орост ирэх францчуудын цуваа ихсэж, үүний хамт өөрийн хэл, зан аяг, сонирхол бүхий францын үзэл үйлдэл дэлгэрч, энэ нь дэд угсаатны түвшинд зан үйлийн шинэ тогтсон үзлийг бүтээжээ. Энэ дүрвэгсэд Наполеонтой хийсэн дайны үед оросуудыг дэмжиж байв. Цаашдаа бол европын соёлд сурах уламжлал инерци байдлаар бүрдэж, амьдралын үндсэн урсгал, өөрөөр хэлбэл угсаатны нийлэгжилт өмнөх гольдролдоо оржээ. Европжсон Онегины удмынхан өдөр хоногоо Чеховын “Интоорын цэцэрлэгт” өнгөрөөж, амьдралынхаа байр суурийг бусад дэд угсаатанд тавьж өгсөн юм.
Дэд угсаатныг ялгах маш амархан, XIX зууны төгсгөлд угсаатны зүйчид чухамхүү энэ түвшинд ажиллаж байлаа. Угсаатны зүйчид ахуйн ёс заншлыг, өөрөөр хэлбэл нийслэлийнхнээс эрс ялгагдах хүн амын тэр бүлгийн зан үйлийн тогтсон үзэл, жишээлбэл, олонецкийн тариачдын ахуйг тэмдэглэн судалж, гэхдээ Петербургийн профессорын амьдралыг үгүйсгэж байв. Энэ нь дэмий байгаагүй, учир нь манай үед ч гэсэн ийм бичиглэл нь маш ашигтай, сонирхолтой байхсан. Харин одоо бол А.П. Чеховыг уншиж, чингэхдээ түүнийг субъективизмийн засвартай унших ёстой болж байна.
Товчоор хэлэхэд дэд угсаатнууд шууд ажиглагддаг, учир нь нэг талаас тэд угсаатны дотор байдаг, нөгөө талаас зан авир, харьцаа, мэдрэхүйгээ илэрхийлэх арга гэх мэтээрээ бусдаас ялгагдах дэд угсаатны зан үйлийн тогтсон үзлийг тээгчид байдаг. Дэд угсаатнууд нь түүхийн янз бүрийн нөхцөл байдлын улмаас үүсдэг, заримдаа язгуур угсаатай давхцдаг, гэхдээ хэзээ ч ангитай давхцдаггүй, ингээд зовлон багатайгаар хөхөгдөн шингэж, гаднаа адилгүй өөр зүйлээр солигдон, гэхдээ хувь заяаныхаа тэр л функцтэйгээ үлддэг. Дэд угсаатны энэ бүрдлүүдийн зориулалтууд нь антагонист биш дотоод өрсөлдөөний замаар угсаатны нэгдлийг хангахад оршино. Угсаатны системийн механизмын биет деталь болох энэхүү нарийн нийлмэл байдал нь ийм л байдлаараа угсаатны нийлэгжилтийн үйл явцад өөрт нь үүссэн байна. Угсаатны системийг хялбарчлах үед дэд угсаатны тоо нэг хүртэл багасдаг ба энэ нь угсаатны персистент (үлдэгдэл) төлөв байдлыг харуулдаг юм. Гэхдээ л дэд угсаатны үүсэх механизм нь юу вэ ? Үүнд хариулахын тулд консорции ба конвиксии гэх хоёр зэрэглэлд хуваагдсан таксономи буюу эрэмбэт нэгжээр нэг зэргээр доош уруудах хэрэгтэй болно. Энэ зэрэглэлүүдэд жижиг овог, клан, дээр нэгэнт дурдсан холбоод, салаа бүлгүүд болон бүх эрин үеийн хүмүүсийн бусад нэгдлүүд багтана.
КОНСОРЦИИ БА КОНВИКСИИ
Нэр томъёогоо тохиръё. Консорци гэдэгт бид түүхэн нэг хувь заяагаар нэгдсэн хүмүүсийн бүлгийг ойлгож байна. Энэ ангилалд “дугуйлан”, артель, сект, дээрэмчид болон түүнтэй адилтгах тогтворгүй нэгдлийг оруулдаг. Эдгээр нь голдуу задардаг, харин заримдаа хэд хэдэн үеийн хугацаанд хагалагдан үлддэг. Энэ үедээ тэд конвикси болдог, өөрөөр хэлбэл нэг адил ахуй, гэр бүлийн холбоо бүхий хүмүүсийн бүлэг болдог. Конвекси нь бага чадвартай. Тэднийг экзогами буюу өөр овгоос гэрлэх ёс идэж, сукцесси буюу дотроо гэрлэх явдал хольж, өөрөөр хэлбэл түүхэн хүрээллийн эрс өөрчлөлт нэрвэдэг. Эндээс бүтэн үлдсэн конвикси дэд угсаатан болон өсдөг. Дээр дурдсан газар нээгчид нь арга ядсан аялагчдын консорци байсан агаад тэсгэл тэвчээртэй сибирчүүд болон хуучин ёслолтны үеийнхнийг бий болгожээ. Америк дахь анхны колонийг конвикси болон хувирсан англичуудын консорци бий болгосон юм. Шинэ Английг пуританууд, Массачусетсийг баптистууд, Пенсильваныг Квакерүүд, Мериландыг католикууд, Виржинийг роялистууд, Жоржийг Ганноверийн гэрийн талынхан үндэслэсэн билээ. Англиас эсвэл Кромвельтэй, эсвэл Стюарттай эвлэрээгүй консорци гаран явсан бөгөөд харин хуучны маргаан элэгдсэн шинэ хөрсөн дээр тэд индиан болон францчууд хэмээх шинэ хөршүүддээ сөрөн зогссон конвикси болж чадсан юм.
Газар нээгчид болон хуучин ёслолтнууд өөрийнхөө угсаатны бүрэлдэхүүнд үлдсэн, харин испанийн конкистадорууд болон английн пуританууд Америкт онцгой угсаатныг бүрдүүлсэн юм. Чухам энэ түвшинг угсаатны дивергенци буюу салалт гэж үзэж болно. Хэзээ нэгэн цагт байсан хамгийн эртний овгууд мөн л ийм замаар бүрдэж байсан нь илэрхий байсан гэж тэмдэглэвэл зохино. Тусгаарлагдмал нөхцөлд эрчимт хүмүүсийн анхны консорци нь эртний эрин үеүдэд нь бидний “овог” гэж нэрлэдэг угсаатан болон хувирсан байна.
Консорцийн дээд түвшинд этнологи дуусах бөгөөд гэхдээ хэрэгцээтэй үед шаталсан захирах ёсны зарчим цаашдаа ч үйлчилж болох юм. Үүнээс нэг зэргээр доошилбол бид хүрээлэлтэйгээ холбогдсон нэг хүнийг олж үзнэ. Энэ нь агуу их хүмүүсийн намтарт маш ашигтай байж болох юм. Ахиад нэг зэргээр доошилбол бид хүний бүрэн намтартай биш, харин жишээлбэл илрүүлбэл зохих гэм хэрэг үйлдсэн, түүний амьдралын нэг үзэгдэлтэй тулгарна. Ахиад нэг доошилбол томоохон үр дагавар дагуулдаггүй тохиолдлын сэтгэл хөдлөлтэй л таарах болно. Юмсын мөн чанарт буй энэхүү хязгааргүй хуваагдал нь тавигдсан зорилтыг шийдвэрлэхэд амин чухал, өгөгдсөн түвшний бүхэллэг чанарыг олохын зайлшгүйг арилгахгүй гэдгийг бид санаж явах ёстой.
IX. Хэт угсаатнууд
\section{ХЭТ УГСААТНЫ БОДИТ БАЙДАЛ – “ФРАНКУУД”}
Хэт угсаатан гэж бид тодорхой бүс нутагт нэгэн зэрэг үүссэн, эдийн засаг, үзэл суртал, улс төрийн нийтлэгээрээ холбогдсон, хоорондын мөргөлдөөнийг огтхон ч үгүйсгэдэггүй угсаатнуудын бүлгийг хэлж байна.
Дайныг устгах буюу дарангуйлалд хүргэдэг (жишээлбэл, XYI–XIX зуунд Европчууд Америкийн угуулуудтай тулгарсан) хэт угсаатны түвшний мөргөлдөөнөөс ялгаатай нь дэд угсаатны дотоод дахь дайн нь зөвшилцөлд тэмүүлсэн үед түр хугацаагаар давуу байдал олж авахад л (жишээлбэл дундад зууны европын гвельфүүд болон гибеллинууд буюу эртний оросын хаадын өөр хоорондын дайн) хүргэдэг. Хэт угсаатан нь угсаатны адилаар өөрийн төлөөлөгчдийг бусад бүх хэт угсаатанд сөргүүлэн тавьдаг, гэхдээ угсаатнаас ялгаатай нь хэт угсаатан дивергенци буюу задралд хүрэх чадваргүй. Хэт угсаатан нь хэмжээ, хүчин чадлаараа биш, харин зөвхөн угсаатан хоорондын ойртолтын зэргээрээ тодорхойлогдоно. Би энэ сэдвийг нотолгоогүйгээр түр хүлээн авахыг хүсч байна, номынхоо төгсгөлд үүнийг хийх болно.
Өнгөц харахад энэ нь хачирхалтай юм. Учир нь хэт угсаатан хаанаас бий болдог нь ойлгомжгүй гэж үү ? Тэдний үүсэх шинж байдал нь угсаатны, тэр тусмаа дэд угсаатны бүхэллэгийг бодоход хавьгүй өөр бололтой. Хэрэв ийм ахул угсаатны үүслийн оньсого ингээд л шийдэгдээгүй байгаа, түүнийг шийдэхийн тулд нэг зэргээр дээшилж, үүнийгээ дүрст багана нь ордны бүхэллэгт орж, хэдийгээр кариатид хэмээх эмэгтэйн дүрст баганыг зэрэгцэн зогсоод харж болдог, харин бүтэн ордныг зөвхөн холын зайнаас тоймолж болдогтой адилтган үзэж, улмаар бидэнд үзэгдэж мэдрэгдэж буй угсаатны үзэгдэл энэ тэр нь системийн бүхэллэгт элементүүд зүйж ордог шиг хэт угсаатны ердөө нэг л хувилбар гэж таамаглах хэрэгтэй. Гэхдээ нэг ч баганагүй ордон байж л байна, харин салсан баримал нь их л сайндаа музейн үзмэр, муу тохиолдолд барилгын хог болдог юм даа. Үүнийг түүхийн жишээгээр тайлбарлая.
Хэт угсаатны нэгдэл нь дэд угсаатныхаас дутахааргүй бодитой байдаг. Франц угсаатан бүр Дундад зууны эхээр Европын католик орнуудыг багтаан, хүн амынх хэсэг ариан (бургунд) болон хэлтнүүд (фризүүд) болж байсан Chretiente хэмээн нэрлэгдсэн бүхэллэгт багтаж байлаа. Гэхдээ тэр үед ийм нарийн зүйл хэний ч санааг зовоохгүй байв. Каролингуудын нэгтгэсэн газар нутаг дээр герман хэлт тевтонууд (teutskes) болон латин хэлт волохууд ( welskes ) гэсэн угсаатны хоёр том бүлэг нутаглаж байжээ. Агуу Их Карлын ач нарын үед эдгээр угсаатнууд өөрийнхөө эздийг эзэнт гүрний төмөр хүзүүвчийг таслахад хүргэн 841 онд Фонтаны дэргэдэх тулалдаанд зорилгодоо хүрчээ. Ингээд Халзан Карл, Немцийн Людвиг хоёр Страсбургт эзэнт гүрнээ үндэстнээр хувааж чадсан юм.
Гэхдээ энэ бол анхдагч ойртолтын үеийн хуваалт юм. Баруун франкын хаант улсаас Бретань, Аквитани болон Прованс зэрэг тусгаарлаж, харин өчүүхэн Франц нь Маас, Лаура хоёрын улангасал дунд байрлаж байлаа. Энэхүү “газар нутгийн хувьсгал” нь Каролингуудын хууль ёсны тевтоны улс Париждаа нуран унаж, 895 онд Анжуйн Робертын хүү, гүн Эд ноёрхох болов. Каролингууд зуун жилийн турш улс орныхоо задралын эсрэг тэмцсэн хэдий ч ихээхэн олон янзийн хольцийн суурь дээр үүссэн угсаатнууд тэдэнд захирагдахаас эрс татгалзжээ. 42. Тьерри О. Избр. соч. С. 244- 247.
X зуунд дууссан “феодалын хувьсгалын” үр дүнд Баруун Европ улс төрийн хувьд задарч, гэхдээ мусульман–араб, үнэн алдарт–грек, ирланд, түүнчлэн харь хэлтэн славян, нормануудын эсрэг сөрөн зогссон хэт угсаатны бүхэллэг байдлаар оршсоор байжээ. Хожим нь тэрээр өргөсөн тэлж, англосаксийн католикуудад хандах замаар, дараа нь баруун славян, скандинав, унгаруудыг залгижээ. Угсаатны зүймэл байдал хэт угсаатны хөгжилд саад болсонгүй.
\section{ХЭТ УГСААТАН ТӨРӨХ НЬ – ВИЗАНТИ}
Хоёрдахь жишээ. Эрт дээр үеэс Газрын дундад тэнгист эллины нэгдмэл соёл оршин байсан бөгөөд хөгжлийнхөө явцад өөртөө Лациум болон финикийн хотуудыг нийлүүлжээ. Угсаатны агуулгаараа бол тэр нь баруун европыг санагдуулах бөгөөд учир нь эллиний үндсэн цөм нь олон талт эллиний соёлын бүх хувилбаруудыг шавхдаггүй байв. Мэдээжийн хэрэг Рим, Карфаген, Пепла зэрэг нь өөрийн салбар онцлогтой, бие даасан угсаатан байсан боловч хэт угсаатны утгаараа эллиний соёлын өргөн хүрээнд багтаж байлаа. Дашрамд дурдахад энэ нь шинэ зүйл биш боловч, бидний хувьд эхлэх цэг гэдэг утгаараа чухал юм. Римийн ноёрхол нь угсаатнуудыг тэгшитгэн арилахад хүргэсэн бөгөөд грек болон латин хэлний эрхийг тэгшитгэсэн нь Газрын дундад тэнгисийн бараг бүх хүн ам нэг угсаатанд уусан нэгдэхэд хүргэжээ.
Гэхдээ НТ–ын I зуунд Римийн эзэнт гүрэнд шинэ, хөршүүдтэйгээ хэнтэй нь ч төсгүй, сүүлийн хоёр зуун жилийн туршид бүрэлдсэн бүхэллэг хүмүүс бий болжээ. Би болсон үеэсээ л тэд өөрсдийгөө “харь хэлтний”, өөрөөр хэлбэл үлдэх бүх хүмүүсийн эсрэг сөргүүлэн тавьсан байна. Тэд анатоми, физиологийн шинжээрээ биш, харин зан үйлийн шинж чанараараа үнэхээр бусдаас ялгаран гарсан байна. Тэд бие биедээ өөрөөр хандаж, өөрөөр сэтгэж, тэдний үеийнхэд утгагүй мэт санагдах амьдралын зорилго тавьж, үхлийн дараах сайн сайхан байдалд тэмүүлдэг байв. Эллиний ертөнцөд даяанч байдал харш байсан, шинэ хүмүүс Фивиадыг бүтээж: грек, сиричүүд орой үдшийг театрт өнгөрөөж, “зөгийн бүжиг” (эртний стриптиз) сонирхож байхад энэ хүмүүс ярилцлагад цуглаад гэр гэртээ чимээгүйхэн тардаг байв. Эллин болон Римчүүд өөрийнхөө бурхныг утга зохиолын дүр мэт үзэж, тэдгээрийг шүтэхээ төрийн уламжлал мэт хадгалж, амьдралд ахуйдаа олон тооны эд зүйлсийг хэрэглэдэг байв. Шинэ номлогчид болон неофитууд бүрэн итгэлтэйгээр өөр ахуйг бодит зүйл хэмээн үзэж, энэ ертөнцийн бус амьдралд бэлтгэдэг байв. Мөн тэд Римийн засгийн газарт төвч хандаж, хэдийгээр энэ нь амь насанд халгаатай ч гэсэн түүнийг тэнгэрлэг мөн чанарыг хүлээн зөвшөөрөхөөс татгалзан, эзэн хаадын барималд мөргөдөггүй байв. Тэдний зан үйлийн энэ гоц байдал нь нийгмийн бүтцийг эвдээгүй, гэхдээ угсаатны бүхэллэгийн энэ шинэ хүмүүс хотын доодсыг үнэхээр үзэн ядаж, төсгүй байдлын эрхийг үгүйсгэх зарчмаас үүдэн тэдгийг устгахыг шаарддаг байв.
Үүсэн бий болсон үзэн ядах байдлын шалтгаан нь итгэл үнэмшлийн ялгаа байсан гэж үзэх нь буруу юм. Учир нь бүрэлдээгүй байсан харь хэлтнүүдэд энэ үед тогтсон, нарийн итгэл үнэмшил байгаагүй юм, харин шинэ хэв шинжийн хүмүүст энэ нь олон янз байлаа. Гэхдээ яагаад ч юм эллинчүүд болон римчүүд Митра, Исида, Кибела нартай маргалгүй, ганцхан Христостой л маргалдсан юм. Хашилтын гадна үзэл суртлын буюу улс төрийн шинж тэмдгийг биш, эллиний соёлын хувьд үнэхээр шинэ, ер бусын байсан этнологийг, өөрөөр хэлбэл зан үйлийн шинж тэмдгийг гаргасан нь ойлгомжтой.
Асар их хохирол хүлээсэн ч шинэ бүхэллэг ялсан нь тодорхой билээ. Гностикууд буюу эртний христиан грекийн холимог шашны номлогчид алга болж, манихейн шашин дэлхийгээр нэг сарниж, маркионитууд (хожим нь павликанууд) явцуу нийтлэгтээ хаагдсан юм. Зөвхөн христианы сүм хийд л амьдрах чадвартай үлдэж, өөрийн гэсэн нэр байхгүй бүхэллэгийг төрүүлж байв. Үүнийг бид византийн буюу тууштай христианы гэж нөхцөлт байдлаар нэрлэх болно. Y зууны үед Римийн эзэнт гүрэн болон хөрш хэд хэдэн орнуудын хязгаарт хүртэл өсөн тэлсэн эртний христианы нийтлэгүүдийн бааз суурь дээр өөрийгөө “ромеи” хэмээх хуучин нэрээр нэрлэсэн угсаатан үүссэн юм. Y–X зуунуудад үнэн алдарт шашинд болгар, серб, унгар, чех, орос, аланууд орж, тэр үед үнэн алдартний ертөнцийн хэт угсаатны соёлын бүхэллэг бий болсон бөгөөд үүнийг нь XIII зуунд “франкууд”, “турк”, монголчууд эвдсэн байна. XIY зуунд үнэн алдартны уламжлал их оросын ард түмэн үүссэнтэй холбогдон дахин амьдарсан юм. 43. XIII зууны Ойрхи Дорнодод бүх баруун европчуудыг “франк” гэж нэрлэж байв.
Гэхдээ Москвагийн Оросыг Византийн соёлын зах хүрээ гэж үзэж болохгүй, орон нутгийн уламжлал нь Оросыг бие даасан бүхэллэг болгосон юм. Y зууны үед Вселенскийн сүм хийдээс салан гарсан несториан болон монофизитууд нь Веселенскийн сүмийн хараал зүхлийг үл харгалзан, өөрсдийгөө үнэн алдартнуудтай нэгтгэн мэдэрсээр байсан билээ. Маргалдагч талууд нөгөөгөө буруу номтон гэж зарласан 1054 оны сүмийн жирийн хагарал нь нэгэнт болж байсан нэгдмэл хэт угсаатны бүхэллэгийн задралыг хэлбэржүүлсэн юм. Ингээд католик ёс нь “Христианы ертөнц” системийн шинэ бүтэц болж, “Католик” Европын талбар амьдран буй хүмүүсийн зан үйлийн шинж чанараараа “византийнхаас” ялгагдах болсон юм. Баруун Европт дундад зуунд Nationes гэгч үүсч, эндээс орчин үеийн үндэстэн, рыцарийн ёс, хотын коммун зэргээр европыг дэлхийн бусад хэт угсаатнуудаас ялгах бүх зүйл үүссэн юм.
Гэхдээ 1054 оны задралын дараа христианы догм ёс урьдынхаа адилаар үлдсэн нь хэрэг явдал үүнд биш, харин шашны түүх нь нийгмийн үйл явцыг ч, угсаатны түүхийг ч нарийн шалгуур үзүүлэлт байдлаар тусгаж байдаг гэсэн хэрэг юм.
\section{ХЭТ УГСААТНЫ ЭВДРЭЛ – YII – X ЗУУНЫ АРАБУУД}
Арабууд бол манай эриний эхэн гэхэд л угсаатны нэгдлээ алдчихсан байсан тийм эртний ард түмэн юм. Хамгийн боловсронгуй арабууд нь эсвэл византийн Сири, эсвэл ираны Иракт амьдарч, эдгээр эзэнт гүрний улс төр, соёлын амьдралд оролцож байсан юм.
Эртний арабуудын гарал үүслийн тухай зөвхөн Ахуйн номон дахь домгууд л гэрчилдэг. Харин түүхэнд бараг мянган жилийн турш Аравийд замдаа худалдаа наймаа хийдэг, ерөнхий нэр байхгүй бедуин, цэцэрлэгчид зэрэг хэсэг бусаг овгууд амьдарч байсан нь тэмдэглэгдсэн байдаг. Тэдний ахуй, овог төрлийн байгуулал нь ихэнхдээ натурал аж ахуйгаар болон улмаар тэдний амьдран буй улс орны ландшафтаар тодорхойлогдож байв. Нэгдэн нийлэх ямар ч хандлага үүсээгүй ба арабуудын байлдааны чадвар хамгийн доод түвшинд байв. Ийм учраас ч YII зуун хүртэл Аравий нь Римийн эзэнт гүрэн, сасанидын Иран, Аббисини (Аскум) гэх хөрш гурван орны өрсөлдөөний талбар болж байв. Аравийд өөрт нь хамгийн идэвхитэй нь Хижас болон Йемений иудейн нийтлэгүүд байлаа.
(Энд Аравий гэж тэр чигээр нь хэрэглэж буй нэрээр эдүгээ арабын олон улсуудыг бүхэлд нь багтаасан газар нутгийг ерөнхийд нь хэлж байна – Орч.)
Н.Т- ын YI зуунд бүх Аравии даяар идэвхжлийн хөдөлгүүр гэж үзмээр яруу найргийн гэнэтийн сэргэлт болсон юм. Сэтгэлийн их тэмүүлэлгүйгээр сайн шүлэг туурьвих боломжгүйг нотлох хэрэг юун билээ. YII зуунд хатуу нэг бурхныг шүтсэн, фанатик, хүсэл эрмэлзэлтэй, хязгааргүй эр зоригтой дагалдагч багавтар бүлгээр өөрийгөө хүрээлүүлсэн Мухаммед сургааль айлдан гарч ирээд хамгийн эхлээд яруу найрагч нарыг өөрийн өрсөлдөгч мэтээр устгажээ. Мусульманы энэ нийтлэгийн гишүүд хуучны овгийн холбоогоо тасалж, византийнх лугаа адил шинэ, онцгой хамт олныг бүрдүүлжээ. Энэ нь олон шашны бүрдэлтэй, мөн угсаатны генетикийн шинжтэй байлаа. Мухаммед мусульман хүн боол байж болохгүй гэж зарлан өөрийн нийтлэгт исламын томъёоллыг авсан боолчуудыг оруулах болсон байна. Энэхүү шинэ итгэл үнэмшлийн суртал нэвтрүүлэгт угсаатны эрчим хүч хуримтлах далд үе бас нуугдан байжээ.
Үүсэн бий болсон энэ консорци хэмээх нөхөрлөл бүр Мухаммед болон Абу–Бекрагийн амьд ахуйд хэт угсаатан болон хувирсан билээ. Хэдхэн арван хүнээс хэд хэдэн арван мянган хүн болтлоо өсч тэлсэн мусульманы хэт угсаатан бүх Аравийг эзэлж, нэг бурхны үзлээ арабуудад тулган хүлээлгэжээ. Цөлийн юунд ч анзааргагүй мекканы худалдаачид болон бедуинууд исламд дээрэнгүй хандсанаас үхэл буюу боолчлолыг илүүд үзэх болов. Ингэж зан үйлийн тогтсон үзэл нь өөрчлөгдсөн “арабууд” хэмээх шинэ угсаатан үүсчээ.
Исламд захирагдсан болон гаднаас хандсан хүчийг ашиглан хоёрдугаар халиф Омар Сири, Египет, Персийг эзэлсэн байна. Харин гуравдугаар халиф Османы үе гэхэд хуурамчаар элсэгчид шинэ төрийн дээд албан тушаалуудад нэвтрэн орж, анхны хамт олны шашны тэмүүллийг хувийн баяжих зорилгодоо ашиглаж эхэлжээ. Шашны итгэлээ хамгаалагсад Османыг алсан нь фанатик биш хүмүүсийн дунд дургүйцлийн тэсрэлт үүсгэн, номлогчийн найз болон түүний дайсны хүү Моавийн хооронд овог хоорондын тэмцэл үүсч, “хуурамч мусульман” Моави ялалт байгуулжээ. Гэхдээ тэд улс төрийн бодлого, албан ёсны үзэл суртлыг өөрчлөөгүй бөгөөд исламын уриагаар булаан эзлэлтийг үргэлжлүүлжээ. Моави болон Омейядийн удмынхны улс зөвхөн араб төдийгүй, Атлантын далайгаас Энэтхэг хүртэл тархан суурьшсан сири, иран, согд, испани, африкийн болон кавказын зэрэг бусад олон элементүүдийг шингээн авчээ.
Арабууд Халифатын улсынхаа олон угсаатны хүн амд өөрийн хэл, өөрийн оюуны соёлоо тулган хүлээлгэж, эзлэгдсэн ард түмнүүдийн ихэнхи нь араб хэлтэй болж, өөрийнхөө хэлийг авч үлдсэн Персэд гэхэд л утга зохиолын хэлний талаас илүү нь араб болжээ.
Гэхдээ X зуун гэхэд Халифат овгуудын газар нутаг нь давхцаж байсан тусгай мужуудыг задласан юм. Идрисүүд (789-926), рустамидууд (777-909), Зиридүүдийн (972-1152) улс берберүүдийг ялав. Бундууд (932-1062) гиляны болон дейлемийн уулынхныг, Саманидууд (819-999) тажикуудыг гэх мэтээр ялжээ. Арабчууд өөрсдөө ч гэсэн задран хуваагдав. Испаний арабууд Абас-Сидуудын ногоон тугийг өргөж, египетийн арабууд Фатимидуудын цагаан тугийг мандуулж, харин бахрейн хавийн бедуны овгууд эхлээд нийтлэгээ байгуулж, дараа нь карматын улсыг байгуулжээ. Эдгээр нь үнэн хэрэг дээрээ бүгдээрээ бие биенээ дайсагнагч тусгай тусгай угсаатнууд болон салцгаасан билээ.
Товчоор хэлбэл IX– зууны Халифатын улсуудад Агуу Их Карлын эзэнт гүрэнтэй тохиолдсон шиг тийм л явдал болсон юм. Тэхэд өвс ногоо цардмал замыг цоолон гардаг шиг угсаатны амьд хүч нь христианы ч, мусульманы ч эзэнт гүрний төмөр хүзүүвчийг тас татсан билээ. Гэхдээ улс төрийн хуваагдал нь тэнд ч, энд ч араб болон латины соёлын болон утга зохиолын хэлний зарим элементүүдийн тодорхой төстэй байдлыг тусгаж байсан хэт угсаатны нэгдлийг эвдэж чадаагүй юм. Мусульманы хэт угсаатан өөрийг нь төрүүлсэн араб угсаатныг бодоход хавьгүй илүү амьдрах чадвартай байсан юм. XI–XII зуун гэхэд л Халифатын улсын үзэл санааг туркмен-сельжукууд, авч, харин XIII зуунд боолын зах дээр худалдагдаж, армид элссэн куман (половчууд) болон суданы негрүүд авсан билээ. Мухаммедийн хамтрагчдын бүтээсэн системийн инерци нь агуу их байлаа.
Одоо дээр дурдсан үйл явцын гол хөдөлгөгч нь шашны үзэл баримтлал байсан гэж тооцож болох уу ? гэсэн асуулт тавья. Гадаад илрэлээрээ бол энэ нь эргэлзээгүй юм. Гэхдээ дотооддоо, агуулгаараа энэ хэрэг явдал их нарийн юм. Гүн ухааны үзэл баримтлалаараа карматын шашин нь христиан, тэр ч байтугай иудаизмыг бодоход исламаас хавьгүй их ялгагддаг хэдий ч өөрөө болохоор мусульманы соёлын–хэт угсаатны бүтцэд төдийгүй, тэр ч бүү хэл араб угсаатныхаа дотор байдаг юм. 44. Бертемс А.Е. Насир-и Хосров и исмаилизм. М., 1959. С. 202-247.
Туркийн хөлснийхөн, мароккогийн алуурчид шашны талаар хамгаас бага боддог байсан, гэхдээ зөвхөн тэд нар л XI зуунд өөрийнхөө сэлмээр суннитийн үнэн итгэлийг дэмжиж байсан билээ. Мухаммедийн өмнө харь хэлтэн, христиан, иудей зэрэг арабын яруу найрагчдын одод байсныг санацгаая, учир нь яруу найргийн цэцэглэлт нь дурдан буй үйл явцын эхний алхам байсан бөгөөд энэ нь зуучлалын худалдаа, негрүүдийг боол болгон худалдах гэж ангуучлах, овгийн ахлагчид кондотьер хэмээх хөлсний хүн болж хөгжсөнтэй адил юм.
Гэхдээ араб угсаатан бүрдэх бүхий л үйл явцын гол хөдөлгөгч нь (хэт угсаатны утгаар бол бүхий л мусульманы соёл) ямар гэсэн Мухаммедийн эхлүүлсэн ислам байсан агаад энд түүнээс өмнө байсан арабын яруу найргийн цэцэглэлийн эрин үе тохирох хөрс нь болж өгчээ. Ислам нь өөрийгөө бататгах фанатик объектын бэлэгдэл, нэгдмэл байдалд орох арга болж өгсөн байна. Шашны шинэ систем эрчлэн гарч ирэх үед голчлон янз бүрийн буруу номтон, шашин-үзэл суртлын агуулгатай хувилбарууд үүсэх нь (өвөрмөц маягийн гарцаагүй үүсэх антитез) үндсэн үйл явц өрнөх эрчмийг л урамшуулдаг аж. Цаашдаа бол араб угсаатны болон хэт угсаатны соёлын хүрээнд шинжлэх ухаан, урлаг, ахуйн өвөрмөц хэлбэрийн цэцэглэлд хүргэдэг олон талт оюуны амьдрал хөгжсөн байна. Энэхүү үйл явц нь гадаад талаасаа шашин-үзэл суртлын хөдөлгөгчтэй мэт харагдах хэт угсаатны бүрэлдэхүйн жишээ болно. Ийм бүхэллэг нь шинжлэх ухаанд эртнээс тодорхой байсан ба заримдаа тэдгээрийг “соёлын хэв маяг”, заримдаа “соёл иргэншил“ гэж нэрлэдэг.
Ингээд X зуун гэхэд араб–мусульманы угсаатны эдийн засаг цэцэглэж, нийгмийн харилцаа нь тогтворжиж, чухам энэ үед гүн ухаан, утга зохиол, газар зүй, анагаах ухаан зэрэг нь хамгийн олон тооны нандин бүтээлийг өгч байсан хэдий ч тэдний эрчим хүч дуусав. Арабууд цэрэг эрээс яруу найрагч, эрдэмтэн, дипломатууд болон хувирчээ. Тэд уран барилгын гялалзсан арга барил бий болгож, зах, сургууль бүхий хотууд барьж, өсч буй хүн амаа хүнсээр хангах усжуулалтын систем, үзэсгэлэнт цэцэрлэгүүдийг барьж байгуулжээ. Гэхдээ дайснаас өөрийгөө хамгаалах аргаа арабчууд мартсан байна. Байлдан эзлэлтийн эрин үеийн оронд алдагдлын үе эхэллээ.
Францын нормандууд мусульманчуудаас Сицилийг булаан авч, Астурийн уулынхан төв Испанийг булаан эзэлж Кастили хэмээх “цайзуудын орон” болгон хувиргав. Византичууд Сирийг (Дамаскаас бусад) эргүүлэн авч, грузинууд арабын гарнизоноос Тифлисийг чөлөөлөв. Эд нар тус хүргэжээ. XI зуунд Аль–морравидууд испаничуудыг хойт зүгт хөөж, сельжукууд Армени болон Бага азийг захирах болов. Гэхдээ энэхүү ирэгсэд нэг ч зоосны хэрэггүй араб угсаатныг хамгаалсангүй, харин “Исламын ертөнц” хэмээх хэт угсаатныг хамгаалав. Учир нь энэ нь тэдний хувьд угсаатан соёлын хөдөлгөгч нь болой.
45. Өнөөдөр арабаар ярьдаг Ойрхи Дорнодын хүн амыг араб гэж нэрлэж байна. Энэ бол буруу юм. Сири, Иран, Умард африкийн хүн амын олонхи нь харилцааны бүс дэх эртний угсаатнуудын хольц юм. Жинхэнэ арабуудын удам бол Саудын Арабын бедуин нар болно.
Дундад азийн түргүүд, суданы негрүүд, зэрлэг курдууд задран буй Халифатын улсын бүтцэд орж, тэнд зуршсан зан үйл, ёс заншил, үзэл бодлыг эзэмшин, өөрөөр хэлбэл Мухаммедийн бүтээсэн нийтлэгийн ажил хэргийг үргэлжлүүлэгчид болжээ. Чухамхүү эдгээр ард түмэн л загалмайтны түрэлтийг зогсоосон билээ. Энэ бүх үед өөрийн хөгжил байхгүй, эвдрэн сүйрэх тавилантай хүний гараар хийсэн бүтээл болох соёл л үлдсэн юм. Энэ эвдрэл удаан явагдаж байна, харин энэ соёлыг гайхан алмайрах явдал Африк, Энэтхэг, Малайн ольтриг, Хятадад улам бүр шинэ салбаруудыг хамарсаар байна. Тэнд угсаатныг төрүүлж, түүний мандал бадралыг туулан гарсан энэ соёл одоо ч оршин байна.
Энэ соёл нь X–XII зуунд ирсэн, нэгдсэн угсаатнуудын өөртөө харш үй түмэн элементүүдийг хүлээн авснаар өөрийн дүр төрхөө өөрчилж, мангас болтол нь өөрчилсөн шинэ хэлбэрүүдийг төрүүлсээр байна. Угсаатны хувьд арабуудад харш мусульманчууд шиит, исмаилит, суфия болон гаднаа мөн юм шиг мөртлөө уг чанараараа Мухаммедийн хамтрагчид болон анхны халифуудын анхдагч ертөнцийг үзэх үзлээс хаа хол сургаалийг номлогчид болцгоосон байна. Гэхдээ л энэ эрин үед угсаатны санал зөрөлдөөнийг конфессиал буюу шашин хоорондын хэлбэрээр бүрхэж, соёлоосоо угсаатны нийлэгжилт рүү буюу буцах замаар явсныг нь Дундын азиас Алс Дорнодын жишээн дээр аль ч хэт угсаатны угсаатны харилцаанд дээр илрүүлж, мөн тодорхойлж болох юм. Зохиогч уншигчтайгаа хамтран этнологийн арга зүйн хэд хэдэн аргыг эзэмшин энэхүү нарийн асуудлаар онцгой аялал хийх болно.
X. Угсаатны нийлэгжилтийн алгоритм
\section{УГСААТНЫ ҮЛДЭЦ}
Угсаатны түүх нь түүхэн үед алга болсон болон одоо байгаагаар солигдсон хорь гаруй хэт угсаатныг тоолж чаддаг. Зорилт маань хэдийгээр хэт угсаатан устан алга болдог механизмыг дүрслэхэд оршиж буй боловч бид тэдгээрийн үүсэл болон тархалтын талаар онцгой ярих болно. Түүхэн үйл явцаар угаагдсан агуу том хэт угсаатны суурин дээр цэцэглэлт, уналтынхаа эрин үеийг даван туулж чадсан бяцхан арлууд үлддэг гэсэн чухал зүйлийг тэмдэглэн хэлье. Иймэрхүү жижиг угсаатны жишээ нь баскууд, албаничууд, кавказын хэд хэдэн угсаатан, хойт Америк дахь сонирхолтой, маш тогтвортой ирокез угсаатнууд болно. Хойт болон төв америкийн мөхсөн болон уусан нэгдсэн ихэнхи овгоос ялгаатай нь ирокезууд өөрийнхөө тоо (20 мянган хүн), өөрийн хэлээ хадгалан үлдэж, мөн өөрийгөө ирокез бус бүхэнд сөргүүлэн тавьдаг. Тэдний ахуйн хэвшил өөрчлөгдөж, цэргүүд нь “музейн үзмэр” болсон нь үнэн бөлгөө.
Үлдэц угсаатнууд хангалттай олон, чингэхдээ тэдний зарим нь мөхөж, зарим нь бусад угсаатанд уусан нийлж, харин зарим нь ирокезуудын адил өөрийн ухамсар, их бага хэмжээний тогтвортой хүн ам, газар нутгаа хадгалан үлдсэн байдаг. Эдгээр угсаатныг бид персистент шинжтэй, өөрөөр хэлбэл өөрийн юмаа туулсан, гомеостаз буюу зогсонги шатандаа байгаа гэж үзэж байна. Угсаатны зүй газар зүйн байрлалынхаа ачаар бусад угсаатантай харилцаанд татагдан ороогүй буюу зөвхөн сүүлийн зуун жилд татагдан орсон маш олон зожиг угсаатныг мэддэг юм. Арьс үсний компаниуд тэнд байгуулагдахаас өмнөх Канадын олон овог, каучукийн чичрэгэ эхлэхээс өмнөх дотоод Бразилийн индианчууд, европчууд бий болох хүртэл тэнд байсан автраличууд, Кавказын зарим уулынхан (Оросын цэргүүд Гунибыг эзэлсний дараа ч) ийм хүмүүс юм. Энэтхэг, Африк, тэр ч байтугай Европт их бага хэмжээгээр тусгаарлагдсан бусад олон ард түмэн, овгууд бий. Гэхдээ хамгийн чухал зүйл нь эдгээр тусгаарлагчид түүхчдийн нүдэн дээр үүсч байна. Тухайлбал, IX зууны үед арал дээр суурьшиж, гурван зуун жилд өвөг дээдсийнхээ дайчин үзэл санааг алдсан викингуудын удам исландчууд болно. Норвеги, дани, шведийн викинкууд болон Ирландаас булаан авсан боол бүсгүйчүүдийн удмынхан IX зуун гэхэд хуучны зарим уламжлалаа хадгалсан, өөрийн арлын хязгаарт гэрлэцгээсэн, том биш боловч бие даасан угсаатныг бүрдүүлжээ. 46. Стеблин-Каменский М. И. Культура Исландии. Л., 1967.
Өөр овгийнхонтой байнгын харилцаа байхгүй байх нь угсаатны дотоод харилцаа гарцаагүй тогтворжиход хүргэдэг. Бидний “зогсонги” гэж нэрлэдэг бүтэц үүсч, угсаатны дотор “системийн хялбаршил” болдог. Үүнийг тодорхой жишээнд дээр авч үзье. Эртний Египетэд нэгдсэн хамит овог хүчирхэг угсаатан болон бүрэлдэж, салбарласан нийгмийн систем бүтээжээ. Тэнд фараон, зөвлөгч, номын буюу эртний грекийн дууны ноёд, цэрэг, төлөгч, бичээч, худалдаачин, газар тариалагчид болон гуйлгачин зарц бүгд байжээ. Систем харийнхантай тулгарахын хэрээр нарийссаар байв. Нуби, Сири дэх байлдааныг мэргэжлийн дайчид, Вавилонтой хийх гэрээг туршлагатай дипломатууд, суваг, ордон зэргийг хар багаасаа суралцсан мэргэжлийн инженерүүд хийдэг байв. Салбарласан систем нь гискосуудын довтолгоог давж, шинэ хүчин чадал цутгасан мэт сэргэсэн юм. Гэвч НТӨ XI зуунд хялбарших үйл явц эхэлж, системийн эсэргүүцэх чадвар унасан билээ. НТӨ 950 онд Египетийн эрх мэдэл ливичүүдийн гарт оров. НТӨ 715 онд ноёрхол эфиопчуудад шилжиж, тэд нар нь Ассируудтай хийсэн дайнд ялагджээ. Дараа нь азийнхан өөрөө нэгэнт хамгаалахаа байсан Египетийг эзлэн авав. Саиссийн улс энэ орныг чөлөөлж, гэхдээ ливи, эллинчүүдийн жадан дээр тогтоон барьж байлаа. НТӨ 550 онд энэ нь бас унаж, үүний дараа Египетийг перс, македон, рим, араб, бербер, мамлюк, турк нар дараа дараалан ноёрхож байлаа. Нийгмийн бүх бүлгүүдээс НТ I зуунд зөвхөн газар тариалан эрхлэгч-феллахууд болон эллинжсэн хотын иргэдийн жижиг тасархай–коптууд л үлдсэн юм. Феллахууд зожигрон үлдэв. Хэдийгээр түүний эргэн тойронд түүхийн идэвхитэй амьдрал буцалж байсан ч эдэнд тэр нь ямар ч хамаагүй, угсаатны хувьд харь болсон нийгэмдээ л амьдарсаар мянган жилээр өөрийнхөөрөө үлдсэн билээ. Үүнийг бид угсаатны тогтмол байдал буюу тайван байдал гэж нэрлэж болно. Энэ нь хөгжил нь түүнийг дүрслэхэд анхааралд орохооргүйгээр саарсан гэсэн хэрэг юм.
\section{ТОГТОНГИ БА ХӨДЛӨНГИ (СТАТИКА И ДИНАМИКА) ШИНЖ}
Нэр томъёогоо тохирч авъя. “Тогтонги” буюу “персистент” ( “зожиг” – Орч ) гэдгийг бид амьдралын хэмнэл нь үе бүртээ өөрчлөлтгүй давтагдаж байдаг ард түмнийг нэрлэж байна. Энэ нь мэдээж ийм ард түмнүүд гаднын үйлчлэлд өртдөггүй гэсэн хэрэг биш. Тэд голдуу хүрээлэн буй орчны өөрчлөлтийн үед ч мөхдөг, жишээлбэл, тасманийчууд устгагдсан, Патагони дахь арауканууд бут цохигдсон. Заримдаа тогтвортой угсаатны бүлгүүд, овог буюу ард түмэн соёлжсон хөршүүдээсээ юм авах зээлдэхээс зайлхийдэг, гэхдээ амьдралынхаа дадсан хэмнэлийг өөрчлөлгүйгээр голдуу өөртөө тохирсон юмыг амархан хүлээн авдаг. Жишээлбэл, алконкикс овог бүр XYII зууны үед мушкет буугаар зэвсэглэж, франц болон английн колоничлогчдоос муугүй буудаж сурсан, патагончууд ганцхан үеийн дотор XIX зуунд явган анчдаас морьтнууд болон хувирсан, тунгусууд шүдэнз, урцанд нэн тохиромжтой төмөр зуух зэргийг эзэмшсэн. Гэхдээ эдгээр ард түмнүүдийн угсаатны дүр төрх ХХ зуун хүртэл хуучнаараа үлдсэн юм. Алконкинууд ч, арауканууд ч франц, испани болоогүй юм.
“Хөдлөнги” ард түмнүүдэд ямагт “эцэг”, “хүүхдүүд”-ийн асуудал босч байдаг. Залуу үе нь өмнөхтэйгөө төсгүй. Төгс эрхэмлэл, таашаал, зан заншил нь өөрчлөгдөж, “моод” гэсэн категори гарч ирдэг. Шинэ юм бий болохын зэрэгцээгээр хуучин нь мартагдаж, ийм өөрчлөлтийг соёлын хөгжил гэж нэрлэдэг.
Хөдлөнги ард түмнүүд ч гэсэн бас л мөнх биш юм. Тэд эсвэл ул мөргүй алга болдог, эсвэл хөгжлийн тодорхой орчил болсноор тогтонги болон хувирдаг, эргээд янз бүрийн өөрчлөлтийн дараагаас хөдлөнги болдог, гэхдээ огт өөр болно. Устаж алга болох нь заримдаа угсаатныг бүрдүүлэгч хүмүүсийн үхэлтэй холбогддог, харин голдуу угсаатны нийтлэг задрахад бүтэн үлдсэн нь хөрш угсаатны нийтлэгүүдтэй уусан нийлж хүмүүс нь үлддэг, гэхдээ угсаатан нь системийн бүхэллэгийн хувьд устан алга болдог. Хэрэв угсаатны хэсэг үлдэц байдлаар хадгалагдан үлдэх аваас тэр нь зожиг хэсэг болно.
Эдгээр жишээ нь хурц тод боловч уламжлалт шинжийн хэд л бол хэдэн шатлал байдаг болохоор хэрэв бидэнд мэдэгдэж буй бүх угсаатныг буурах хуучинсаг шинжээр нь байрлуулах аваас “тэг түвшин” буюу өөрөөр хэлбэл уламжлал байхгүй зэрэгт нэг ч угсаатан хүрээгүй, хэрэв ийм байсан бол тэр нь зүгээр л оршин байхаа больж, хөршүүдийнхээ дунд уусах байсан. Хэдийгээр үеэс үед ажиглагддаг хэдий ч энэхүү уусах явдал нь угсаатны хамт олны өөрийнх нь зорилго чиглэлтэй хүчдэлийн үр дүн хэзээ ч байдаггүй. Гэсэн хэдий боловч угсаатнууд мөхдөг, энэ нь мөхлийг бий болгогч эвдлэгч хүчин зүйлүүд байна гэсэн үг юм. Гадаад үйлчлэлээс бүрэн тусгаарлагдсан угсаатан гэж байдаггүй учраас бүх угсаатан мөхдөг гэж үзвэл зохино. Хамгийн сонирхолтой нь угсаатнууд заримдаа тэдний оршихуйд хүлээн авч болохооргүй мөхлийг илүүд үздэг. Яагаад ?
Ажаад байхад чухам энэ үхэх эрхээрээ амьтны аль нэг зүйлийн бүлээс орчинтойгоо зогсонги тэнцвэрийн үедээ байгаа угсаатан ялгагддаг байна. Угсаатны үхэл бол системийн бүхэллэг шинжийн задрал болохоос түүнд орж буй бүх хүмүүсийг толгой дараалан хүйс тэмтэрнэ гэсэн үг биш. Хэдийгээр түүх америкчууд тусгай тусгай индиан овгийг, хятадууд хукн хэмээх багавтар угсаатныг хүйс тэмтэрсэн шившигт жишээ байдаг боловч мөхсөн угсаатны хавьгүй олон гишүүн шинэ, хөрш угсаатны бүрэлдэхүүнд ордог юм. Ийм учраас угсаатныг бүрэн зайлуулах буюу экстеминаци гэдэг нь биологийн гэхээсээ нийгмийн үзэгдэл юм.
Диалектик материализмын ёсоор үхэл бол организмын амьдрах үйл ажиллагааны зайлшгүй агшин, зүй ёсны үр дүн мөн. Ф.Энгельс: ”амьдралын үгүйсгэл мөн чанартаа амьдралд өөрт нь байдаг ба амьдрал нь төрөхөд нь түүнд дотор оршиж байдаг өөрийнх нь зүй ёсны үр дүн үхэлд харьцах харьцаагаар ямагт сэтгэгдэж байдаг” Диалектикийн энэхүү түгээмэл хууль угсаатны нийлэгжилтийн үйл явцад ч үйлчилж байдаг. 47. Маркс К., Энгельс Ф. Соч. 2-е изд. Т. 20. С. 610.
Хүн ямар ч насандаа үхэж болдог шиг угсаатны нийлэгжилтийн үйл явц аль ч үедээ тасарч болно. Гэхдээ угсаатан хүч чадлаа авч чадаагүй байгаа, эхэлж буй, эсвэл хүч чадлаа нэгэнт барж, төгсч буй угсаатны нийлэгжилтийг таслахад амархан юм. Энэ үед текник, соёлын түвшин нь хүн амын тоотой адил их ач холбогдол байхгүй. XY зуунд ирокезууд таван овгийн холбооны хамтран амьдрах өвөрмөц, хөгжиж буй хэлбэр, нэг ёсны бүгд найрамдах улсыг байгуулсан. Нахау ацтекуудад эхлэл тавьж өгсөн, мөн Монтесума улсыг XIY–XYI зуунуудад хөгжөөгүй байсан гэж тооцож болно гэж үү ? (Нарийн яривал 1325 онд Теночтитлан үндэслэгдэх үед, 1521 онд тэр Кортест эзлэгдэх үед). Энд авсан жишээ нь хөгжлийнхөө эхний шатанд гаднын цохилтын улмаас тасарсан угсаатны нийлэгжилтийн үйл явцын жишээ болно.
Эртний еврей нарын жишээ бүр ч тодорхой. XY зууны үед хабиру хэмээх хэрмэл овог Палестинд нэвтрэн орж, Иорданы газар нутгийн хэсгээс булаан авчээ. Техникийн түвшин, аж ахуй эрхлэх арга, цэргийн заль мэхээр тэд Сири, Аравийн бусад семит овгуудаас ялгаагүй, Египет болон Вавилоны ард түмнээс дутмаг байлаа. Гэхдээ энэ бол угсаатны утгаараа эрчимтэй хөгжиж буй ард түмэн байсан бөгөөд хөршүүдийгээ римийн явган цэргийн богино сэлмийн дор угсаатны нийтлэгийн хувьд мөхөх хүртлээ тэсэн гарч чадсан юм. 48. Римийн эзэнт гүрний Рейн дэх хил болон Парфи руу зугтаж гарсан урвагчид амьд үлдсэн юм.
Гэхдээ энэ мөхөл нь мэдээж тохиолдлын бус байдлаар еврейн ард түмний өөрийх нь дотор болсон угсаатны задралтай давхцсан билээ. Еврейн фарисей, саддукеи, ессеи нар өөрсдийнхээ нийтлэгийг мэдрэхээ байж, эсвэл бие биенээ няцаагч, урвагч (фарисей, ессеи нар саддукеичуудыг), эсвэл зэрлэгүүд (садукеичууд ессеи болон ард түмнийг), эсвэл ард түмнээс тасарсан тайлгачдын бүлэг (саддукеи, ессеи нар фарисеичуудыг) гэж үзэх болсон байна. I зууны үед еврей нар соёлын түвшингийн хувьд римээс ч, грекээс ч дутаж байгаагүй юм.
Энд авсан жишээнүүдийн үндсэн дээр чухамхүү бүдүүлэг байдал л (варварство–Орч) соёлыг хөгжлийн үед нь цуцаах хүчийг өөртөө агуулж байдаг хэмээн бодож болохоор юм. Гэхдээ энэ үзэл түүхэнд баталгаагаа олоогүй байна. Европын ард түмнүүд XIX зуунд Африк, Зүүн Өмнөд азийг булаан эзэлж, ХХ зууны эх гэхэд дэлхийн бараг бүх хуурай газрыг хамарсан колони улсуудын систем байгуулсан билээ. Үүнийг зарим тохиолдолд цэргийн техникийн давуу талаар тайлбарлаж болох авч дандаа тийм биш. Жишээлбэл, Энэтхэгт сипайчууд английн зэвсгээр зэвсэглэсэн хэдий ч тоогоор цөөн англичуудад ялагдсан. XYII–XYIII зуунд Туркийн арми Орос, Австрийн зэвсгээс чанарын хувьд дор байгаагүй, гэхдээ Евгений Савойский, Суворов нар армийн хүний тоо цөөн, хангамжийн баазаасаа хол байсныг эс харгалзан ялан дийлэгчид болсон билээ. Францчууд Алжир болон Аннамыг шилдэг их бууныхаа хүчээр биш, бага (партизаны) дайнд хэрэглэсэн, эрэлхэг зоригоороо алдаршсан зуавчуудын хүчээр дагуулсан билээ. Үүний эсрэг жишээ ч бий. Хамгийн боловсронгуй зэвсэг бүхий италичууд 1896 онд цэргүүд нь жад, чулуун зэвсгээр зэвсэглэсэн ч эрт үеэс соёлоороо унаган италичуудаас дутаж байгаагүй Менеликийн негусуудтай хийсэн дайнд ялагдсан юм. Ийм л байдаг байна даа.
Дээр дурдсан бүхий л байлдан дагуулалт нь Баруун Европын угсаатны нийлэгжилтийн үйл явцаас салшгүй бөгөөд үүний үр дүнд бүр феодализмын үеэс эхэлсэн үндэстэн болон колонийн эзэнт гүрнүүд бүрдэх боломж өгсөн юм. Гэхдээ Европ угсаатнуудын талбарыг өргөтгөх явдал ХХ зуунд дууссан, энэ нь дэлхийн түүхэнд төдийгүй, мөн Европын өөрийнх нь түүхэнд ч чухал, цус урсгасан, баатарлаг, зөрчилт үзэгдэл байсан, гэхдээ зөвхөн үзэгдэл болохоос биш хувьслын оргил биш байсан нь тодорхой болов. Бидний гэрч нь болсон колонийн эзэнт гүрнүүдийн сүйрэл нь угсаатны нийлэгжилтийн үйл явц цэцэглэх үеэ өнгөрч, түүх өмнөх чиглэлээ авч-Европ өөрийнхөө газар зүйн хилд ахин орсныг харуулж байна. Иймээс ажил хэргийг техникийн буюу соёлын түвшингөөр биш, харин угсаатны хөгжлийн загварын огт өөр зарчим дээр байгуулах ёстой. “Нэг ч ард түмэн, нэг ч арьстан өөрчлөгдөхгүй үлддэггүй, Тэд бусад ард түмэн, арьстнуудтай байнга холилдож, байнга өөрчлөгдөж байдаг. Тэд бараг үхсэн мэт болж болно, харин дараа нь шинэ ард түмэн байдлаар буюу хуучныхаа өөр нэг төрөл байдлаар дахин сэргэдэг”. 49. Неру Д. Открытие Индии. М., 1955. С. 53.
Гэхдээ энд яагаад зожиг–угсаатан дайсагнагч хүрээллийг эсэргүүцэх чадвараа алддаг нь тодорхойгүй үлдэж байна. А. Тойнбийн “дуудлага”, “хариулт”-ийн концепци ёсоор тэд дайсны дуудлагад хүчтэй хариу өгөх ёстой, харин тэд эсвэл бууж өгдөг, эсвэл таран зугтдаг байна. Угсаатныг зожиг оршиход хүргэн буй гомеостаз буюу зогсонги шилжилт нь эртний үед уг угсаатны резистент буюу өсгөх чанарыг тэтгэж байсан ямар нэг шинж тэмдгээ алдсантай холбоотой мэт. Тэд зөвхөн нэг юман дээр хатуу байдаг бөгөөд энэ нь харь юмыг өөрийнхөө орчинд оруулдаггүй явдал юм.
НЭГДЭХ БУЮУ ИНКОРПОРАЦИЯ
Угсаатны үзэгдлийн энд тэмдэглэж дурдсан онцлог нь өөр өөр овгийнхон нэгдэх үед байнга үүсдэг бэрхшээлийг тайлбарлана. Өөр угсаатанд орохын тулд хувийн хүсэл, тэр ч байтугай хүлээн авч буй хамт олны жирийн зөвшөөрөл ч хангалтгүй. Харь орчинд сайхан гэгч нь тохижиж болно, гэхдээ л өөрийн хүн болж чадахгүй. Хэрэв Шотланд хүн Перугийн иргэн боллоо гэж бодоход хэдийгээр Перу нь харьцангуй саяхан XIX зуунд бий болсон нарийн бүрэлдэхүүнт угсаатан ч гэсэн тэрээр перу хүн болж чадахгүй. Харин маори угсаатны оюутан английн их сургуулиудад амархан дасна. Үүний тайлбарлал нь бидний дурдсан зан үйлийн тогтсон үзэл болно.
Шотланд эр инкийн удмууд шиг анхлах ёслолд ороогүй, конкистадоруудын удмууд шиг долоо хоног бүрийн үдийн мөргөлд яваагүй, мөн түүний сайхан муухай, эрхэм, доод, тансаг, утгагүйн тухай ойлголт нь перу хүнийх шиг биш, хэрэв ирсэн хүн сайн санаагаараа өөр орны иргэн болох нь хангалттай гэх ахул энэхүү сайн санаа нь угсаатнаа солиход багадна. Харин маоричууд аль хэдийнээ 100 жил англичуудтай зэрэгцэн амьдарч, бага байхаасаа англи маягаар юу хийхийг мэднэ. Сургууль, хамт тоглох, ажил хэргийн харилцаа зэрэг нь холимог гэр бүлээс ч илүүтэй нөлөөлж байдаг. Шинэ Зеландад бол маориг англичуудтай эсрэгцүүлэн тавих учир утга байдаг бол Кембрижид Шинэ Зеландын оршин суугчийг “хуучны, сайн английн” оршин суугчтай сөргөлдүүлэн тавих нь чухал ач холбогдолгүй болдог.
Хэдийгээр зарим нэг тохиолдол байдаг авч жижиг, натурал аж ахуйгаар амьдрагч угсаатанд орох нь маш их түвэгтэй байдаг. Жишээлбэл угсаатны зүйч Морганыг ирокез гэж, францын аялагч, арьс үсний наймаачин Этьен Брюлег гурон гэж хүлээн зөвшөөрсөн байсан. Ийм жишээ олон бий. Гэхдээ Морган америкийн эрдэмтэн хэвээрээ л үлдсэн, харин 1609–1633 онуудад үйл ажиллагаагаа өрнүүлж байсан Брюле хуучны ёслолын эсрэг залуучуудыг турхирсныхаа төлөө овгийн ахлагчид алуулсан гэдгийг тэмдэглэх нь шударга хэрэг болно. Түүнээс гадна В.Г.Богораз чукчи нарын хүмүүжүүлсэн, орос хэлээ мэдэхгүй өнчин хүү-“орос-чукчи”-гийн тухай өгүүлсэн байдаг. Чукчи нар түүнийг орос гэж хатуу тооцож, хүү ч мөн адил ингэж үзэж байжээ.
Ийнхүү инкорпораци буюу нэгдлийг практик зорилгоор эрт дивангалавийн үеэс хэрэглэгдэж ирсэн, байгалийн үзэгдэл болох нь тодорхой байдаг мэдрэхүйн салбар дахь ухамсар, өөрийн ухамсрын чанадад байх хүчин зүйлсийн эсэргүүцэлтэй ямагт тулгардаг, ухамсрын аппаратаар дандаа зөв тайлбарлаад байдаггүй юм. Асуудал хэчнээн нарийн түвэгтэй байлаа ч одоо бол угсаатны үзэгдэл материаллаг бөгөөд тэрээр хэдийгээр бидний бие махбодь болон дээд мэдрэлийн үйл ажиллагаагаар хязгаарлагдаж байдаг ч бидний ухамсрын гадна буюу хажууд оршин байдаг. Тэр нь хүмүүсийн үйл ажиллагаа, зан араншингийн онцлогоор илэрдэг ба угсаатны сэтгэл зүйд хамаарна. Энэхүү этнопсихологи буюу угсаатны сэтгэл зүйг ”хүмүүсийн юу хийж байгааг үйлдэл, ач холбогдол, хувь хүний роль, бүлэг гэсэн таван функциональ нэгжийг….судлах замаар тайлбарлах”-ыг эрмэлздэг нийгмийн психологитой хутгах ёсгүй билээ. 50. Шибутани Т. Социальная психология. М., 1969. С. 28.
Социал бүлэг бол “нэгдмэл бүхэл байдлаар хамтран ажиллаж буй хүмүүсээс бүрдэх” бүлэг юм. 51. Там же. С. 32. Ийм зүйлийн жишээ нь хөл бөмбөгийн баг, “Линчийн шүүх” зэрэг болно. Ийм байлаа ч гэсэн эдгээр нь угсаатан биш юм. Мөн тэнд “Америкт шилжин ирсэн европын сэхээтнүүд…түүний түүх, хууль , ёс заншлыг америкчуудаас илүү сайн мэддэг байсан. Гэхдээ эдгээр хүмүүс америкийн амьдралын “мэдлэг”-тэй байсан болохоос биш, түүнтэй ”танилцаагүй” байлаа. Тэд АНУ –д өссөн дурын хүүхэд зөнгөөрөө мэдэрдэг маш олон зүйлийг ойлгох чадваргүй байлаа” гэж тэмдэглэсэн байдаг. 52. Там же. С. 42.
Түүний хамт (энэ нь хэвшмэл юм) зарим хүмүүс Америкт хурдан идээшиж байхад зарим нь сайн хөлс мөнгө олж байсан ч эргэн зүтгэж байсан юм. Үүнийг В.Г.Короленко “Хэлгүйгээр” туужиндаа гайхамшигтай өгүүлсэн байдаг.
Угсаатан нийцэх янз бүрийн зэрэг хэмжээ байдаг нь илт байна. Зарим үед нэгдэх нь амархан, зарим үед хүнд, гурав дахь тохиолдолд боломжгүй байдаг аж. Ийм хачин үзэгдлийн шалтгаан нь юу юм бол ?
Дэлхий дээр неоантропууд бий болсны дараагаас угсаатнууд ямагт оршсоор ирсэн. Хүн төрөлхтний түүхийн харуулж байгаагаар тэдний оршихуйн арга нь нэг л адил: үүсч төрөх, өргөсөн тэлэх, идэвхийн зэрэг нь буурах ба эсвэл задран мөхөх, эсвэл орчинтойгоо тэнцвэрт байдалд шилжих зэрэг болно. Орчинтойгоо мэдээлэл, энтропийг онцгой, дахин давтагдахгүй байдлаар солилцож байдаг системийн энэхүү нэг хэвийн, инерцийн үйл явцыг бодит хэмнэлдээ буй хэмээж болно. Чухамхүү энэ нөхцөл байдал нь нэгдлийг хязгаарладаг. Жинхэнэ “өөрийн хүн” болохын тулд үйл явцад оролцох, өөрөөр хэлбэл элссэн хүн өөрийн жинхэнэ эцэг эхээр хүмүүжүүлээгүй нөхцөлд уламжлалыг залган авах явдал юм. Үүнээс өөр бусад тохиолдолд нэгдэл буюу инкорпораци нь угсаатны харилцаа болон хувирна.
ТЭНЦВЭР БОЛОН ХӨГЖЛИЙН ХООРОНДЫН ЯЛГАА
Одоо зожиг–угсаатан болон эрчимтэй хөгжиж буй угсаатны хоорондын ялгаа нь юу вэ гэсэн асуулт тавъя. Үлдэц угсаатны системд угсаатны гишүүдийн тэмцэл байдаггүй, хэрэв өрсөлдөөн тохиолдсон ч ялагдсан хүн нь үхдэггүй. Зөвхөн хэн ч үл хүсэх шинэ зүйлийг мөрдөн мөшгөдөг. Гэхдээ хэрэв ийм ахул хувьслын нэг хүчин зүйл болсон байгалийн шалгарал унтрана. Энд зөвхөн нийгмийн дэвшил буюу бууралт л боломжтой байх суурь үндсэн дээр угсаатан–ландшафтын тэнцвэр л үлддэг. Гэхдээ нарийн бөгөөд түвэгтэй реадаптаци буюу дахин дасах, мөн зан үйлийн тогтсон үзлийг солиход жам ёсны зүйл ахин үүсч, ингэж бүрэлдсэн бүл эсвэл мөхнө, эсвэл шинэ угсаатан болно.
Ийм маягаар угтсаатны анхдагч ангилал нь (тэднийг бий болох гэсэн утгаар) бие биенээсээ хэд хэдэн шинж тэмдгээрээ эрс ялгагдах хоёр зэрэглэлд хуваах явдал болдог. (Хүснэгт 1 –ийг үз )
Хүснэгт 1
Энд санал болгож буй хуваалт нь одоо болтол хэрэглэгдэж ирсэн антропологи, хэл шинжлэл, нийгмийн болон түүх–соёлынхоос ялгаатай зарчимд үндэслэгдсэн болно. Хүснэгтэд өгсөн 12 ялгааны шинж тэмдэг нь бүх эрин үе, бүр газар нутаг дээр өөрчлөгдөшгүй юм. 53. Эдгээрээс хамгийн нарийн, чухал асуудлуудыг тусгайлан авч үзсэн. ( Гумилев Л. Н. Этнос и категория времени //Доклады отделений и комиссии ВГО. Вып. 15. 1970.).
Ангит нийгэмд тогтонги угсаатнууд оршин байж болдог, мөн овгийн байгууллын үед хэсэг хүмүүс дахин бүлэглэгдсэний ачаар овгийн шинэ холбоо буюу цэргийн–ардчилсан нэгдэл үүсдэг. Үүний эхний хувилбарын жишээ гэвэл Аравийн бедуин овгийн дунд, Баруун африкт: Бенин, Дагомед зэрэг болон Баруун хойт америкад тлинкуудад байдаг хуучирсан боол эзэмшлийн харилцаа, XIX зуун хүртэл Кавказын уулынханд грузины боол эмэгтэй болон боолчуудын харилцаа зэрэг болно. Феодалын царцанги харилцаа XIX зуунд Төвдийн баруун болон зүүн–хойно, уулын Дагестан, якут, малайз нарт ажиглагдаж байсан. Үүний эсрэгээр XY зуунд үүссэн ирокезийн холбоо бол ангит нийгмээс өмнөх нөхцөлд шинэ угсаатан бүтээсэн тод жишээ болох юм. Мөн ийм үйл явц овог төрлийн гүрэн Хүнд НТӨ III зуунд, YI–YIII зуунд цэргийн ардчилал бүхий түрэгийн “Мөнхийн Иль” улсад, кельтүүдэд НТӨ 1 мянган жилийн үед байсан ба эдгээр нь нийгмийн харилцааны кланы системтэй, ямар ч маргаангүй угсаатны бүхэллэгийг бүрдүүлж байсан юм.
Ийм жишээг нэмж болох авч эдгээр нь хангалттай юм. Ангилах үед аливаа материалыг хуваах нь нөхцөлт шинжтэй байдаг, гэхдээ чухамхүү ийм учраас л эдгээрийг судлаачдын тавьсан зорилгоор тодорхойлж болдог юм. Бидний зорилт бол ажиглагдан буй олон янзийн үзэгдлүүд дотроос угсаатны үүсэн бүрэлдэх байр суурийг тодорхойлох явдал болно. Тэгээд юу болох гэж ? Угсаатан үүсэх нь аль нэг ард түмний дутуу дулимагаас болчихсон мэт “хоцрогдол” буюу “зогсонги” гэж үзэж болохооргүй угсаатны ландшафтын ерөнхий тэнцвэрийн суурин дээр болдог ховор тохиолдол мэт санагдана. Орчин үеийн бүх “зогсонги” угсаатнууд хэзээ нэг цагт хөгжиж байсан, харин одоо хөгжиж буй тэр угсаатнууд хэрэв устдаггүй юмаа гэхэд хожим хэзээ нэг цагт “тогтвортой” болох юм.
УГСААТНЫ НИЙЛЭГЖИЛТ БА БАЙГАЛИЙН ШАЛГАРАЛ
Угсаатны үзэгдлийг дээр авч үзсэн бүхнээс дүгнэхэд нийгмийн үйл явцууд нь өөрийн мөн чанараараа янз бүр байдаг гэсэн гаргалгаа бий болж байна. Нийгмийн болон угсаатны хэмнэл (ритм) давхцах нь тохиолдлын явдал байдаг авч чухамхүү энэ нь өнгөц ажиглахад шатууд нь давхцах үед нийлмэлжиж, үр нөлөө нь хүчтэй болдог мэт харагддаг байна. Энд үүссэн асуудлыг: угсаатан бүтээгдэх хүч хаанаас гардаг вэ гэж томъёолох нь зүйтэй юм. Ийм хүч байх ёстой, хэрэв байхгүй байсан бол байгалийн шалгарлаар тодорхойлогдох энтропи аль эрт, бүр палеолитийн эрин үед угсаатны бүх ялгааг арилгачихаад хүн төрөлхтний олон янз байдлыг нэгдмэл өвөрмөц шинжгүй антро хүрээ болгох байсан билээ.
Байгалийн шалгарал нь оршихуйн төлөө тэмцэлд хамгийн их дасан зохицсон амьтан нь тэсэж үлдэхэд ямагт хүргэх ёстой гэж нийтээр үздэг. Гэхдээ Ж.Б.С.Холден энэ дүгнэлт нь амьд бус байгаль, амьтны бусад зүйлүүдээс өөрийгөө арга буюу хамгаалдаг, ховор, хэсэг бусаг тарсан зүйлийн хувь зөв гэж тэмдэглэсэн байдаг. 54. Холден Дж. Б. С. Факторы эволюции. М., Л., 1935. С. 71 и сл.
Гэхдээ тоо толгой нягт болмогц л зүйлийн тодорхой төлөөлөгчид өөр хоорондоо өрсөлдөөнд ордог байна. Хэрэв тодорхой нэг амьтан ялагч боллоо ч гэсэн зүйлийн тэмцэл нь өөрөө биологийн хувьд хортой болж эхэлдэг байна. Жишээлбэл, эр амьтдад аварга том эвэр, хатгуур хөгжих нь тэдэнд хувийн ялалт байгуулахад нь туслах боловч голдуу тухайн зүйл устахын эхлэл болдог аж.
Энэхүү эрэгцүүлэл нь ноёрхогч зүйл, биоценезийн дээд, төгс салаа болох хүнд ч гэсэн хамаарна. Ж.Холдены тэмдэглэснээр зүйлийн дотоод дахь амьтны тэмцэл нь хоол хүнсний төлөөх зүйлийн дотоод тэмцэлтэй нийтлэг зүйл юу ч үгүй байдаг бөгөөд түүний зүй тогтлыг хүний нийгэмд шилжүүлэн авчрах нь буруу аж. Энд огт өөр асуудал гарч ирдэг: бүл буюу сүрэгт давамгайлахын төлөө тэмцэл хурцдах нь гэнэтийн биш ч гэсэн чухамхүү ялагч нар үр удмаа үлдээдэггүй аж. Улмаар бид хамгийн чадвартай нь тэсэн үлддэг дарвиний хуультай биш, хамт олны нийт хувьсалд тусдаггүй өвөрмөц маягийн экцесс буюу гажуудалтай тулгарч байна. Залуу эрэгчингүүд мөргөлдөх буюу зулзага нь өсөхөд сүргээс хөөн гаргах үед болдог шалгарал нь шинэ бүл үүсэхэд хүргэдэггүй. Харин эсрэгээр энэ нь зан үйлийн тогтсон авирыг оролцуулаад ихэнхи амьтны шинж тэмдгүүдийг хадгалагч хүчирхэг хүчин зүйл болдог байна. Энэ нь ч бүрэн ойлгомжтой юм: тодорхой нутагт амьдран буй зүйл тус бүр нь түүний биоценозод орж, түүндээ хамгийн дээд зэргээр дасан зохицдог. Энэ байдал нь жишээлбэл, удаан хугацааны ган болох буюу хөрсний давхаргыг шороогоор булдаг үер болох зэрэг физик газар зүйн өөрчлөлт, эсвэл биоценезийн тэнцвэрийг өөрчилдөг нутаг газарт нь өөр амьтан бөөнөөр нүүн ирэх зэрэгт л өөрчлөгддөг.
Гэхдээ ямар ч байгалийн хүчин зүйл, яагаад ч гамшиг сүйрэл болоогүй байхад зарим угсаатан нөгөөгөөр солигдож, хойч үедээ зөвхөн уран барилгын туурь, уран баримлын хэлтэрхий, ном зохиолын тасархай, хагархай сав суулга, өвөг дээдсийн алдрын тухай будилсан дурдатгал үлдээдэг гэдгийг тайлбарладаггүй. Хүний хувьд шалгарал нь өөр утга агуулгатай байдаг нь илт байгаа бөгөөд Ж.Холден үүнийг онцгой анхаарсан юм. “Биологийн шалгарал нь угсаатны гишүүн бүр бусдаасаа олон хүүхэдтэй байхын тулд хамаардаг тийм л шинж тэмдгүүдэд чиглэгдсэн байдаг. Ийм зүйлүүд гэвэл өвчин эсэргүүцэх чадвар, биеийн хүч зэрэг болно, харин угсаатныг (буюу зүйлийг) бүхлийнх нь хувьд үржүүлэхэд үйлчилдэг, иймээс ч хүмүүст онцгой үнэлэгддэг тэр чанарууд биш юм.” Холдены үзэж байгаагаар үзэл санаа, шинжлэх ухааныхан, эрэлхэг дайчид, яруу найрагчид, жүжигчдийн генүүд дараа дараагийн үедээ улам улам цөөрөх болж байна. Бидний шинжилгээний хувьд бол өөр зүйл, угсаатны цаашдын хувь заяанд энэ үйл явцын үр дүн нь юу болох вэ ? Энэ асуудлыг бид мэдээж хэрэг, нийгэм судлалын талаасаа биш, яг одоо бидний сонирхлыг татаж буй үржил–генетикийн тал нь илүү татаж байна. Ж.Холден энэ байдлыг “Байгалийн шалгарал нь дасан зохицох шинж чанар бүхий өөрчлөлтөд үйлчилнэ, гэхдээ энэ өөрчлөлт нь дурын чиглэлд явдаггүй. Тэдгээрийн ихэнхи хэсэг нь бүтцийн нарийн байдлыг алдахад буюу эрхтний редукци буюу буцаж жижигрэх, өөрөөр хэлбэл дегенераци буюу дахин бий болоход хүргэдэг” гэсэн. 55. Там же. С. 82. Яг ийм “нэг төрлийн хувиргалтад өртөж буй олонлог дахь олон янз байдал ихсэхгүй, харин ч голдуу буурдаг” гэсэн дүгнэлтэд кибернетикийн арга зүйг ашиглан хүрч болно. 56.(Росс Эшби У. Введение в кибернетику. С. 193).
Холдений баталсан энэ сэдэв нь хувьслыг дэвшилтэт хөгжил мэтээр үздэг сургуулийн төсөөлөлтэй анх харахад зөрчилдөж байна. Гэхдээ бид диалектик аргыг хэрэглээд эхэлмэгц энэ зөрчил утаа мэт замхрана. Зүйлүүд эсвэл дахин төрнө, эсвэл тогтворжих буюу персистент (өөрийн хөгжлөө барсан үлдэц) болон хувирна, гэхдээ өмнөхөө бодвол илүү төгс шинэ зүйлүүд үүсч байдаг. Гэлээ ч гэсэн тэд жам ёсны зүйл боловсорч, залгамж нь гарч ирэх үед нарт хорвоо дахь зайгаа тавьж өгдөг. Мөлхөгчид аварга том хуягтнуудыг сольсон, сүүн тэжээлтэн динозавруудыг сольсон, харин орчин үеийн хүн неандерталь хүнийг халсан. Өөдлөлт бүрийн өмнө гүнзгий уналт түрүүлж байлаа.
Үүнийг этнологийн хэл рүү буулгаж, манай материалд энгийн жишээ маягаар хэрэглэвэл салбаржсан (газар нутгийн), хаалттай (генетикийн), өөрөө бүрэлдсэн (нийгмийн хувьд ) угсаатны хамт олон юм.
АЯЛДАХ ҮЗЭЛ. НАРИЙН ХЭЛБЭЛ АНТИ АМИНЧ ҮЗЭЛ
Шинэ төрсөн угсаатан өөрийнхөө оршихуйг мэдэгдмэгцээ шууд л дэлхий түүхийн үйл явцад оролцож эхэлдэг. Энэ нь тэрээр түүнд ямагт дайсагнаж байдаг хөршүүдтэйгээ харилцаж эхлэж буй хэрэг юм. Үүнээс өөрөөр ч байх аргагүй бөгөөд шинэ, идэвхитэй, дасаж дадаагүй зүйл бий болсноор амьдралын нэгэнт тогтсон бөгөөд дуртай болсон хэвшлийг эвдэлж эхэлдэг. Угсаатан төрөн гарсан нутаг орны баялаг ямагт хязгаарлагдмал байна. Энэ нь юуны өмнө хоол хүнсний нөөцөд хамаарна. Хуучин дэг журамд айван тайван оршин байсан хэн бүхэн өөр, харь, үл ойлгогдох, тааламжгүй хүмүүст шахагдаж буюу байраа тавьж өгөхийг огтхон ч үл хүсэх нь бүрнээ ойлгомжтой юм. Шинийг эсэргүүцэх нь өөрийгөө хамгаалах жам ёсны хариу үйлдэл маягаар үүсч, ямагт хурц хэлбэртэй, голдуу хөнөөлт дайн болдог. Ялах буюу наад зах нь өөрийгөө хамгаалахын тулд хамтын ашиг сонирхлыг хувийнхаас дээгүүр тавьсан аялдах ёс зүй угсаатны дотор үүсгэх хэрэгтэй болдог.
Аялдах үзэл болон аминч үзэл нэр томъёонд бид чанарын үнэлгээ огтхон ч өгөхгүй. “Сайн” болон “муу” гэсэн ойлголт эдгээрт огт хамаагүй бөгөөд цаашид энэ утгаараа явна. Энгийн үгийг шинжлэх ухааны нэр томъёо болгох нь уншигчдад үзэл баримтлалаа яг байгаагаар нь ойлгуулах нь зайлшгүй үед л хэрэглэнэ. “Аялдах үзэл”-ийг “анти аминч үзэл” гэх нь илүү дээр байх.
Ийм ёс зүй нь сүргийн амьтдын дунд ажиглагддаг, гэхдээ зөвхөн хүнд л зүйлээ хамгаалах цорын ганц хүчин зүйлийн ач холбогдолтой болдог. Гэхдээ энэ хувийн эерэг зүйл нь гэр бүлийн зүйл гэсэн аминч ёс зүйтэй ямагт хөршлөн оршиж, гэхдээ хувийн болон хамт олны ашиг сонирхол ямагт давхцаж байдгаас хурц зөрчил маргаан цөөн гардаг. Зүйлүүдийн нэг мөчир хүний адилтгал буюу угсаатныг хамгаалах үүднээс бол ёс зүйн энэхүү хоёр үзэл баримтлалыг хослуулах нь зохистой нөхцөл байдлыг бүрдүүлж байдаг. Энд функцүүд нь хуваагдмал болдог. “Аялдагчид” нь угсаатныг бүхлийх нь хувьд хамгаалдаг бол “аминчид “ нь түүнийг үр удамдаа нөхөн сэргээдэг. Гэхдээ байгалийн шалгарлаар “аялдагсдын” тоо цөөрч, угсаатны хамт олныг хамгаалалтгүй болгон, цаг хугацаа өнгөрөх тутам өөрийн хамгаалагчгүй болсон угсаатан хөршүүддээ идэгддэг. Харин “аминчид” –ын үр удам амьдран үлдэж, гэхдээ нэгэнт өөр угсаатны дотор үлдэж, “аялдагсдыг” өөрсдийг нь хамгаалагч баатрууд биш, харин зөрүүд, зожиг, тэнэг араншинтай хэмээн дурсдаг. Энэ томъёоллыг түүхэн материал дээр дэлгэрэнгүй өгүүлбэл зохих ганцхан аргаар шалгаж болдог. Ёс зүй нь оршихуйн зүйлийг байх ёстой харьцаанд нь авч үздэг, харин байх ёстой зүйл нь эрин үе болгонд өөрчлөгддөг. Энэхүү өөрчлөлтийг түүхэн сурвалж зохиогчид маш нарийн тэмдэглэдэг ба өөр тохиолдолд бол баримтыг гуйвуулахаас ч санаа зовдоггүй юм. Энд тэд үнэнээсээ ханддаг бөгөөд учир нь тэд бодит байдлыг биш, өөрсдөд нь тухай бүр эргэлзээтэй санагдах идеал буюу төгс эрхэмлэлийг дүрслэн бичдэг байна. Ийм учраас зан үйлийн гол шинжийн өөрчлөлтийг тэмдэглэхдээ бид түүх судлал, тэр ч байтугай өнгөрсөн үеийн уран сайхны бүтээлийг хүртэл ашиглаж болох ба гэхдээ тэдгээрийг мэдээллийн эх сурвалж биш, шүүмжлэлт судалгаанд хамаарах баримт гэж үздэг. Ингээд эдгээрийн тусламжтайгаар үйл явц жинхэнэ байдал дээрээ хэрхэн болсныг тогтоодог юм. Жишээ болгон уншигчдад хангалттай тодорхой, ард түмний түүхийн ямар нэг дууссан хэрчмийг (төр биш, улс төрийн институт биш, нийгэм-эдийн засгийн харилцаа биш, чухамхүү угсаатан) түүний үе шатыг гүйлгэн авч үзье. Тохиромжтой жишээ нь Эртний Римийн хот–улс болно. Хэрэв түүний хаад ноёдын баатарлаг түүхийг орхин нийгмийн системийг тодорхойлсон анхны секцесси буюу тусгаарлалтаас (плебей нар Ариун уул руу гарч, үүнийхээ дараа нь тэд патрицуудтай зөвшилцсөн) эхлээд Каракаллын зарлиг хүртэл (хязгаар нутгуудыг Римийн албат–римчүүд гэж хүлээн зөвшөөрсөн), өөрөөр хэлбэл НТӨ 949 оноос НТ 212 онууд хүртэл авч үзвэл “аялдагчид” болон” аминчид”-ын харьцааны хувьслыг амархан ажиглаж болно. Дашрамд дурдахад энэ бүхнийг Римийн түүхчид аль хэдийн хийсэн бөгөөд энэ үйл явцыг “ёс суртахууны уналт” гэж нэрлэсэн байна.
Түүх бичигчдийн мэдээлж байгаагаар Пунийн дайны төгсгөл хүртэл эхний үеүдэд эх орныхоо төлөө үхэхийг хүсэгч баатар эрсүүд дутагдахгүй байв. Муций Сцевола, Аттилий Регул, Цинцинат Эмилий Павел болон тэдэнтэй адил олон олон баатар ихээхэн хэмжээгээр баатарлаг домгоор бүтээгдсэн нь магад, гэхдээ чухам ийм хүмүүс л зан үйлийн төгс эрхэмлэл болж байлаа. Иргэний дайны эрин үед байдал эрс өөрчлөгдөв. Баатрууд нь намын жолоодогчид Марий болон Сулла, Помпей, Крас буюу Цезарь, Серторий, Юний Брут буюу Октавиан нар болцгоов. Тэд нэгэнт эх орныхоо төлөө амиа өргөхгүй болж, харин намынхаа ашиг сонирхлын үүднээс эх орноороо эрсдэл хийж, өөртөө ээлжлэн ашиг тус гаргаж байлаа. Принципиатын буюу зарчимтны эрин үед бас л зоригтой, эрчимтэй зүтгэлтнүүд цөөнгүй байсан ч тэд бүгдээрээ хувийнхаа ашиг сонирхлыг нуулгүй ажиллаж байсан ба үүнийг нь олон нийт байх ёстой зүйл, тэр ч байтугай зан үйлийн цорын ганц боломж гэж үзэж байв. Одоо бол эзэн хаад болон жанждыг үүргээ чин сэтгэлээсээ биелүүлсэн, өөрөөр хэлбэл шударга бус, учир утгагүй харгислал хийгээгүй гэж магтах болов, гэвч энэ нь тэднийг “ухаалаг аминч үзэлтэн” гэж хүлээн авч буй хэрэг бөгөөд энэ нь ч тэдэнд өөрт нь ашигтай байжээ. Оптимат болон популяруудын намын үе өнгөрч, жишээлбэл, сирийн, галлийн, паннонны гэх мэт аль нэгэн легионуудын бүлэглэлүүд гарч ирэн зөвхөн эрх мэдэл, мөнгөний төлөө хоорондоо тулалдах болов. Умард улсын үед төгс эрхэмлэл, ашиг хонжоо цэцэглэн мандсан ба чухам энэ үед Populus, Romanus хэмээн нэрлэгдэж байсан римийн угсаатан өөрийнхөө эзлэн авч байсан ард түмний дунд ууссан нь тохиолдлын хэрэг биш билээ.
Иймэрхүү дүр зургийг Дундад зууны Баруун Европоос ч харж болно. Тэнд хамгийн гол ажил нь мусульмануудтай хийх дайн байсан юм. Баатарлаг туульсын анхны дүрүүд нь Роланд болон христианы хөлөг баатар Сид нар байв. Үнэн хэрэг дээрээ эхнийх нь Бретони нутгийн жижиг эзэн байсан ба мавруудад биш, харин баскуудад алагдсан юм, хоёр дахь нь зүгээр л ямар ч зарчимгүй түрэмгий нөхөр байв. Яриан байхгүй эрмэлзэл нь аялдах, баатарлал байв. Хожим нь хоёрдугаар үед баатрууд гарч ирсээр л байлаа. Тэдгээр нь Кортес ба Писарро, Васко да Гама ба Албукерки, Фрэнсис Дрейк, Лепантийн дэргэд ялагч Австрийн Хуан нар байсан юм. Тэднийг эр зоригтон болгосон зүйл нь ил тод шунал байсан ба хэн ч тэднийг буруутгадаггүй, харин ч бахархал, талархал төрүүлдэг байв. Цаг хугацаа явсаар байгаад зөвхөн өөрийнх нь арьс чухал болсон хөлсний цэрэг баатар болдог болсон ба түүний зөвхөн ухаан, тэвчээр, өөрийгөө эзэмдэх чадварыг нь л үгүйсгэж болохгүй юм.
Бидний харж байгаагаар тодорхой чиглэлд өөрчлөгдөх төгс эрхэмлэл нь хамт олны сэтгэл санааны шалгуур үзүүлэлт болж байна. Учир нь баатарт хандах зохиогчийн харьцаа сэтгэл хөдлөлийн шинжтэй болохоор энд санаатай худал зүйл үгүй болой. Энэхүү сэтгэл санаа нь илүү гүн мөн чанарыг тусгадаг ба хүний хамтын ахуйн угсаатны мөн чанарын бодит үндэс болсон зан үйлийн тогтсон үзлийн өөрчлөлтийг тусгадаг юм.
Энэ үед ухамсрын хүрээг тооцохоос татгалзах хэрэггүй бөгөөд учир нь зөвхөн тэр л хурц, байхгүй байж чадахгүй нөхцөлд зохистой шийдвэр олох боломж олгодог юм. Угсаатны шинэ систем бүрэлдэж, эрчээ авах болтол үйл явц гаднын оролцоотойгоор тасарч болох ба эндээс хатуу фатализмд (муу ёрд) байр байхгүй юм.
ҮЛДЭГДЭЛ УГСААТНЫГ НИРГЭХ НЬ
Асуудлыг ингэж тавихад яагаад угсаатнууд мөхдөг вэ ?, НТӨ III мянган жилд бичмэл түүхийн эхэн үед тэмдэглэгдсэн угсаатнуудаас нэг нь ч үлдээгүй, мөн НТ- ын эхэн үед амьдарч, байж байсан угсаатнуудаас тун цөөхөн нь үлдсэн вэ гэдэгт хариулж болно. Эртний рим, эллин, ассирийн шууд биш удмынхан танигдахгүй болтлоо зүсээ өөрчлөн одоо болтол амьдарч байгаа ба гэхдээ аль хэдийнээ рим ч биш, эллин ч биш, ассир ч биш болж, өвөг дээдсээсээ зөвхөн генийн санг нь өвлөн авчээ. Бусад асуудлын зэрэгцээгээр бүлийн мөхлийг судалдаг палеонтологид адилтгал хийж үзье. Энд судалж буй объектын хэмжээ ямар байх нь чухал биш байдаг ба эндээс мөхөх үйл явц нэг л зүй тогтолтой байх ёстой гэж үзэж болно.
Гайхалтай нь өнгөрсөн эрин үеүдийн байгалийн нөхцөл байдалд хамгийн бага дасан зохицсон, хамгийн бага хөгжилтэй зүйлүүд тэсэн үлдсэн байдаг юм. Амьдралын хуучны хаад байсан динозавр, мастодонт, махайрод, агуйн баавгай, агуйн арслан зэрэг тэнцэхээр өрсөлдөгч байхгүй хэдий ч цэврээрээ алга болжээ. Зүйлийн мөхөл нь газар нутгийн аажим багасалтыг дагуулж, заяа нь орхисон тэр зүйлийг био эгнээнээс шахан гаргах хөрш зүйлүүдийн өрсөлдөөнийг бий болгодог. Гэхдээ энэхүү “заяа нь орхих”-ын учир юу вэ гэдэг нь тодорхойгүй байгаа юм. Эртний биологийн зорилтыг шийдвэрлэх гэлгүйгээр энэ нь этнологид угсаатны бүтцэд нуугдан байдаг гэж бид хэлж болох байна. Бусад адил нөхцөлд (тоо хэмжээ, техник гэх мэт) бүтцийн нарийсал нь дайсагнагч хүрээллийн эсэргүүцэх чадварыг нэмэгдүүлж, харин хялбаршил нь түүнийг багасгадаг. Биет болон оюун санааны талаар бүрэн дүүрэн ард түмэн, жишээлбэл, индианчууд болон полинезчууд ард түмнийхээ огтхон ч шилдэг төлөөлөгч биш колоничлогчидтой харьцуулахад хүчгүй байсны учир нь энэ юм. Ийм маягаар угсаатны хувьд ч, байгалийн хувьд ч хамгийн их аюулыг хөгжлийн явцдаа дасан зохицох чадвараа алдаагүй, иймээс газар нутгаа өргөтгөж буй хөршүүд нь тарьдаг юм. Ийм дайсан бий болохгүй бол үлдэц угсаатан хязгааргүй удаан оршин байж чадна. 57. Рычков Ю. Г. Антропология и генетика изолированных популяций (Древние изоляты Памира). М., 1969.
Гэхдээ хэрэв угсаатан илүү махчин хөршийнхөө эргэлтгүй эсэргүүцэлтэй тулгарвал хөгжиж байсан угсаатан ч гэсэн толгой дараалан устгагдах хүртлээ мөхөх явдал ч байдаг. Ганц тодорхой жишээ авч YI–YIII зууны түргүүдээр хязгаарлая.
550–581 онуудад алтайн жижиг угсаатан түргүүд бүхий л Их Талыг Хятадаас авахуулаад Дон хүртэл, Сибирээс авахуулаад Иран хүртэл ноёрхлоо тогтоожээ. “Мөнхийн Иль” хэмээн нэрлэж байсан систем нь уян хатан бөгөөд салбарласан байв. Түүнд талын болон уулын овгууд, согдын баян бүрдийн оршин суугчид, тэр үед өргөн уудам Волгийн доод бэлчирийн худалдаачин, малчид, буддистууд, гал шүтэгчдийн зэрэгцээгээр “хөх тэнгэр–хар газар“ гэж тооцогдсон түргийн өөрийнх нь цэргүүд байр байраа олон оршин сууж байв. Гэхдээ Суй гүрэнд нэгдсэн (589-618) Хятад, ялгуулсан табгачийн эзэнт гүрэн Тан (619-907) хүчтэй бөгөөд харгис байлаа. Хятадууд цэргийн хүчээр түргүүдийн эсэргүүцлийг нугалж чадахгүй байсан, гэхдээ тэд дипломатын замаар нэгдмэл хаант улсыг Баруун, Зүүн гэж гэж хоёр хуваагдахад хүргэсэн ба дараа нь талынхныг Таримын бүсийн баян бүрдүүдээс тусгаарлан эзэлж авсан байна. Ингээд согдчуудыг арабуудын золиосонд гаргажээ. Үүнийхээ дараа хятадууд түрэгүүдийн эсрэг уйгар, карлук, басмалуудыг босгож, 747 он гэхэд түрэгийн ордыг бут ниргэсэн бөгөөд чингэхдээ ялагчид олзонд хүн аваагүй юм. Гэхдээ хятадууд өөрсдөө түргийн оргодлуудыг хүлээн авч, тэдний хилийн цэрэгтээ оруулжээ. Энэхүү “азтангууд” 756-763 онуудад хятадын түшмэдүүдийн үймээний эсрэг Ань–Лушаны бослогод оролцож үхсэн байна. Босогчдын эсрэг хятадуудаас гадна хаашаа ч гарах газаргүй байсан талын уйгарууд болон уулын төвдүүд тэмцсэн ажээ. Тусгаарлагдсан, тэр тусмаа хялбаршуулсан систем мөхсөн нь энэ ээ. Үүнтэй адил зөрчилдөөн ажиглагдсан газар бүрт үйл явцын механизм нь огт өөрчлөгддөггүй юм.
XI. Угсаатны харилцаа
УГСААТНЫ ЭРЭМБЛЭГДСЭН ШАТЛАЛ
Угсаатны систем зүй нь нийгмийн ангиллаас ялгагддаг. Зөвхөн ховор тохиолдолд л тэд давхцдаг. Эдгээрийн алийг нь хэрэглэх нь судалгааны аспектаас, өөрөөр хэлбэл түүхэн үйл явдлын хэлхээг харах өнцгөөс хамаарна. Түүхэн үйл явдлын хэлхээс нь судлаачийн зорилгод нийцэж буй ойртолтын зэргийг сонгон авч буй судлаачийн тавьсан зорилгоор тодорхойлогдоно. Энэ зорилтыг одоо болтол нэг бус удаа тавьж, хангалттай шийдвэр гаргаж чадаагүй байгаа нөхцөл байдал нь (Д.Вико, О.Шпенглер, А.Тойнби) судлаачдыг хэчнээн хүнд хэцүү байлаа ч гэсэн эмпирик нэгтгэн дүгнэх оролдлогоосоо дургүйцэхэд хүргэх ёсгүй юм. 58. Берталанфи Д. Общая теория систем. С. 65-68.
Үйл явц хэрхэн болдгийг тайлбарладаг нэлээд зохиогчдоос ялгаатай нь бид үйл явцын тодорхой аспектийг тодорхойлогч зарчмын өрөөсгөл загвар гарган авсан байсан хэдий ч чухам юу өөрчлөлтөд хүргэв гэдэг асуултад хариулах боломжтой байдаг. Гэхдээ үзэл баримтлал бүтээхийг түүхэн дурын тайлбарлалын үндсэн дээр хийдэг бөгөөд энэ нь түүхийг (“үнэний эрэл”) үйл явдлын дараалал буюу үйл явдлыг тоочих явдлаас ялгаж өгдөг. Бид түүхийн шинжлэх ухааны хуримтлуулсан янз бүрийн материал дээр үндэслэдэг, иймээс бидний судалгааны объект бол Шпенглерийн “соёлын санаа” биш, Арнольд Тойнбийн “Судалгааны ухаан хүрэх талбар” биш, аль нэгэн тодорхой эрин үед буй, аль нэгэн түвшиндөө буй угсаатны нийлэгжилтийн шатуудын (фаз) систем болдог. Түүхэн цаг хугацаанд урсаж буй дараагийн эрин үед бүрдэл хэсгүүдийн байршил нь нэгэнт өөр байх болно.
Одоо бид санаа нэгтэйгээр нэр томъёогоо тодотгож, ерөнхий байдлаар угсаатны шатлалыг байгуулж болно.
Хүснэгт № 2.
Угсаатны нийлэгжилт гэдгийг зөвхөн түүний бий болох агшин буюу түүхийн тавцан дээр угсаатан бий болох төдийгүй, мөн угсаатны хөгжлийн бүхий л үйл явц, угсаатан үлдэц болон хувирах буюу алга болох бүх үйл явц гэж ойлгохоор тохиролцоод дараах гаргалгааг хийж болно: Дурын, шууд ажиглагдаж буй угсаатан бол угсаатны нийлэгжилтийн аль нэг шат юм. Харин угсаатны нийлэгжилт бол материйн хөдөлгөөний нийгмийн хэлбэртэй харилцан үйлчилсэн үед л илэрч байдаг, био хүрээн дэх гүний үйл явц мөн. Ингэхлээр судалж болох угсаатны нийлэгжилтийн гадаад илрэл нь нийгмийн дүр төрх болдог гэсэн үг юм.
Энд угсаатны зүйчдийн судалдаг угсаатныг төрүүлж байдаг угсаатны нийлэгжилтийн үйл явц яагаад үүсдэг вэ ? гэсэн үндсэн асуулт босч ирнэ. Нийтэд тархсан үзлээр бол шинэ угсаатан нь нягт хамтран амьдрах үед, угсаатны анхдагч хольцууд харилцан уусан нийлсний үр дүн болж үүсдэг аж. 61. Итс Р. Ф. Введение в этнографию. С. 43-46.
Угсаатан бүрэлдэхэд харилцан уусан нийлэх нь чухал боловч хангалттай биш. Рейн мөрний эрэг дээр франц болон немцүүд мянга илүү жил хөрш зэргэлдээ амьдарч, нэг шашин шүтэж, ахуйн нэг зүйл хэрэглэж, бие биенийхээ хэлийг сурсаар буй боловч уусан нийлсэнгүй, үүнтэй адилаар австрийчууд унгар, чехүүдтай, испаничууд каталончууд болон баскуудтай, оросууд удмуртчууд, вепсүүд болон чувашуудтай уусан нэгдээгүй юм.
Заримдаа маш ховор тохиолдолд нэг бүс нутагт буй угсаатнуудын нөлөөлөл болдог ба тэгэхэд уусан нийлсэн угсаатнууд нь алга болж, оронд нь өмнөх хоёртойгоо төсгүй шинэ угсаатан бий болдог. Эхний үедээ шинэ угсаатны гишүүд өөрийнхөө ахуй заншилд дасаж өгөхгүй байснаа хоёр буюу гуравдугаар үеэсээ өвөг дээдсээсээ ялгаатайгаа мэдэж эхэлдэг. Энэ үзэгдлийг харилцан уусан нэгдсэний үр дүн гэж үзэж болохгүй, учир нь тэд дандаа үүсээд байдаггүй, мөн маш хурдан болдог. Энд тэсрэлт лугаа нэг зүйл байх ёстой. Иймээс эдгээрийн үүсэхэд шаардагдах ямар нэгэн хүчин зүйл байгаа ба үүнийг нээж олох хэрэгтэй.
Дээр дурдсанаас гадна эхний тохиолдолтой адилгүй угсаатан үүсэх өөр нэг арга байдаг. Түүхэн хаялгийн үр дүнд угсаатнаас ямагт хүмүүсийн бүлэг салан гарч, амьдрах газраа өөрчилдөг. Хугацаа өнгөрөх тутам энэ хүмүүс зан үйлийн тогтсон шинэ үзэлтэй болж, төв хэсэгтэйгээ холбоогоо алддаг. Заримдаа энэ бүлэг мөхдөг, гэхдээ нутгийн уугуулууд юмуу өөр шилжин ирэгчидтэй холилдон бие даасан угсаатан бүрдүүлэх нь цөөнгүй байдаг.
Хоёрдахь хувилбарын жишээ нь XYIII зууны төгсгөл үед англичуудаас холбоогоо тасалсан англосаксон гарлын америкчууд, испанийн конкистадоруудын удам креолууд, голланд, франц, хойт немцийн тариачдын ач гуч нар болох бурууд, XYI зуунд кастын системийг халж, бусад индусуудаас ялгаран гарсан сикхи нар, 1688 оны Хурилтай дээр манжаас орост захирагдахыг илүүд үзсэн буриадууд болон үндсэн угсаатнаасаа тасран гарч, түүхэн заяагаараа явж буй олон угсаатан бий.
Энэ хоёр хувилбарын нийлэгжилт ялгаатай, мөн хоёр хувилбар тус бүрийн өөрчлөлтийн шинж чанарт нийтлэг зүйл юу ч байхгүй болохыг тэмдэглэх нь амархан бөгөөд чухал юм. Хоёр дахь хувилбарт дахин бий болсон угсаатнууд өөрийнхөө соёлын хүрээнд үлдэж, зөвхөн салбар онцлогтой л болж байна. Нэгдэх тохиолдолд бол огт шинэ үзэгдэл болж, түүнийг төрүүлсэн ард түмнүүдийг хадгалсан институтууд зөвхөн өнгөрсөн байдлаар буюу зээлдмэл байдлаар оршиж байна. Нэгдэх хувилбар нь шинэ хэт угсаатан төрсөн жинхэнэ угсаатны нийлэгжилт болох нь илт байхад хоёр дахь хувилбар нь зөвхөн хэт угсаатны олон янз байдлыг нэмэгдүүлж байна. Тухайлбал, АНУ бол огтхон ч хэт угсаатан биш, зүгээр л Европын роман–германы далайн чанад дахь үргэлжлэл, мөн боолын худалдаагаар олсон Африкийн зарим хэсэг юм. Америкийн хэт угсаатны үлдэгдлүүд бол атабаски, сиу, алконкин болон бусад овгууд мөн. Ийм учраас бид цаашдаа зөвхөн нэг дэх хувилбарын тухай ярих бөгөөд учир нь түүх бол үйл явдлын тухай шинжлэх ухаан, харин үйл явдал бол харилцах явцад учрах үед болдог юм. Чухам иймээс энэхүү харилцаанд л давуутай анхаарвал зохино. Энэ сэдэв хөндөгдсөн, гэхдээ хангалттай биш.
\section{ЯНЗ БҮРИЙН ТҮВШНИЙ ХАРИЛЦАА}
Угсаатны харилцааны асуудалд эргэн орохдоо юуны өмнө харилцаа хэрэгжиж буй түвшингийн тухай асуудлыг тавих ёстой юм. (Хүснэгт 2–ыг үз). Хоёр буюу түүнээс илүү консорци болон конвиксийн хослол тогтворгүй байдаг. Энэ нь эсвэл задарна, эсвэл хэт угсаатны тогтвортой хэлбэрийг бүрдүүлнэ. Дэд угсаатны түвшинд уусан нийлэхийг “манай хүрээний биш” хүмүүстэй тогтоосон “тэгш бус гэр бүл” шигээр тайлбарладаг ба учир нь нийгмийн шатлалын түвшин нь голдуу ач холбогдолгүй байдаг. Тухайлбал, бүр XIX зуунд казакууд тариачинтай, тэр ч байтугай казакуудыг бодвол хавьгүй илүү баян, нэр төртэй байхад ч тайж нартай гэр бүл болохыг “тэнцүү бус” гэж үздэг байв. Миний бие ч гэсэн бодвол бурангуй цагийн хөтлөгч эхлэлтэй “Номонд жийдүүд самарчуудтай, казакууд тайж нартай бүү явалд” гэж заасан байдаг гэж сонсож байсан юм. Мэдээж хэрэг энэ нь ямар ч “номонд” байхгүй бөгөөд гэхдээ курдууд перс, армянуудад ханддагтай төстэй юм. Гуйлгачин–малчин курд хүн хэрэв перс эхнэрийнх нь гайхамшигт овог тодорхой биш бол эхнэрээ хамаатнууддаа танилцуулж зүрхэлдэггүй байна. Османы эзэнт гүрэн дэх албаничууд, Испани дахь баскууд, Англи дахь шотланд – гайлендерүүд, Гидукуш дахь патанууд мөн л ийм байдлаа хадгалан үлдсэн юм. Тэд эндогами буюу өөрийн овгоос гэрлэх ёсоор бэхжсэн хамтын амьдралын үндсэн дээр бусад дэд угсаатнуудтай тогтвортой угсаатны бүхэллэгийг бүрдүүлдэг билээ. Евроазийн эх газрын төв хэсэгт угсаатны хамтын амьдралын хэлбэрүүд бүр эрт дээр үеэс маш хурц илэрч байсан юм.
Угсаатнууд соёл–аж ахуйнхаа дадал заншилд нийцсэн ландшафтын янз бүрийн бүс нутгуудыг эзлэн сууж, бие биендээ саад болдоггүй, харин туслалцдаг байсан. Тухайлбал, якутууд Лена мөрний өргөн татамд, харин эвенкүүд тайгын салаануудад амьдарч байв. Их Оросууд голын хөндийгөөр суурьшиж, уудам тал газрыг нь казак, халимагуудад, ойн хэсгийг нь угорын ард түмэнд үлдээж байв. Угсаатны ийм бүхэллэг хэчнээн нарийн бөгөөд салбарласан байх тутмаа улам бат бэх, хүчтэй байдаг байна.
Хүснэгт 3
* Симбиоз – угсаатнууд харилцан ашигтай зэрэгцэн орших хэлбэр, симбионтууд нь өөрийн өвөрмөц байдлыг хадгална.
** Ксения (грекийн зочин, зочлох гэсэн үгнээс гаралтай) – угсаатнууд өөрийн өвөрмөц байдлаа хадгалан төвийг сахин зэрэгцэн амьдрах хэлбэр
*** Химера (барын толгой, ямаан бие, луун сүүлтэй домгийн амьтан); энд янз бүрийн хэт угсаатны системд угсаатнуудын өвөрмөц байдал устсан, угсаатны үл нийцэх харилцааны хэлбэр
**** Аннигиляция (юу ч үгүйд хувирах гэсэн физикийн гаралтай үг ) – гэрэл цацрах (фотонууд), масс алдагдахад янз бүрийн тэмдэг бүхий эгэл хэсгүүд харилцан устгалцах үзэгдэл.
Нэгдмэл нийгмийн организмд хоёр буюу түүнээс дээш угсаатан хослох нь өөр асуудал юм. Аль нэгэн нийгмийн организмын шинж чанар нь холимог угсаатны харилцан үйлдэлд ул мөрөө үлдээдэг бөгөөд нэлээд тохиолдолд тэд тухайн бүс нутагт, оршин буй баримттайгаа эвлэрэх ёстой болдог, гэхдээ бие биедээ татагдахгүй байж чадахгүй. Эдгээрийг “ксени” гэж нэрлэдэг. Тухайлбал, Бельгид валлончууд болон фламандчуудыг нийтийн орон сууцны оршин суугчид болгон “өөд нь татсан”. Тухайлбал, Канадад англи, франц, франц–энэтхэгийн эрлийзүүд, бас дээр нь славянууд зэрэгцэн амьдарч, чингэхдээ уусан нийлдэггүй, өөрийн симбиозод байх үүргээ хуваадаггүй байна.
Хоёр буюу түүнээс дээш хэт угсаатны харилцаа маш зовлонтой. Энд зөвхөн угсаатан юу ч биш болох төдийгүй, хүн ам зүйн уналт болж, зүгээр л хэлэхэд сул тал нь тэсвэрлэшгүй нөхцөл буюу шууд устгаснаас болж мөхдөг. Ийм нөхцөл байдал XYIII–XIX зууны үед АНУ – д болж, индианчуудыг хуйхны арьсны хөлсөнд буудан алдаг байсан, мөн Бразильд каучукийн чичрэгийн үед, Австралид түүнийг англичууд эзлэх үед, Шар мөрний хөндийд эртний Хятадын соёл иргэшил эртний жун овгийн соёлтой тулгарах үед тохиолдож байсан. Жун овгоос юу ч үлдээгүй юм.
Гэхдээ түүний хамт түүхэнд хэт угсаатнууд дандаа энх тайван байгаагүй ч, харилцан устгалцдаггүй байсан бүхэл бүтэн эрин үеүд байдаг юм. Харин заримдаа нэг бүхэллэгийн дэд угсаатнууд үзэн ядах шалтгаан заримдаа олж, заримдаа ололгүйгээр өөр хоорондоо хөнөөлт дайнууд явуулдаг байсан билээ. Зарим тод жишээг авч, энэ нь яагаад болдгийг авч үзье. Төр улсуудын түүх нь үйл явдлын явцыг хангалттай тайлбарлаж чадах болов уу ?
\section{ЯНЗ БҮРИЙН ЭРЭМБИЙН УГСААТНЫ БҮХЭЛЛЭГИЙН ХАРЬЦАА}
Санал болгож буй угсаатны энэ хуваалт нь зөвхөн орчин үеийн төдийгүй, түүхэн угсаатны зүйд нэн ашигтай юм. Үүнийг бид сайн судалсан, саяхан дууссан эрин үеийг жишээ болгон XII зууны Евроази тив, мөн нэн тархсан, утгагүй хуваагдлаар эсвэл “Баруун”, эсвэл “Дорно”–д ордог, өч төчнөөн маргааны талбар болсон Эртний Оросыг тодорхой жишээ болгон энэ бүхнийг харуулахыг хичээе. 62. Энд тусгайлан судлавал зохих Америк, Далайн орнууд, Сахараас өмнүүрх Африк зэргийг оруулаагүй болно.
Энэхүү огт утгагүй хуваагдал нь Римийн сүм хийдэд үзэл суртлын хувьд нэгдсэн, өөрийгөө бусад бүхний эсрэг тавьсан роман–германы хэт угсаатны бүхэллэг дотор бий болсон юм. Товчдоо энэ бол Дундад зуунд утга санаатай байсан бэртэгчин евроцентризм, өнөөдөр ч Баруун Европ, түүний далайн чанад дахь үргэлжлэл Америкт оршсоор байна.
Хэрэв барууны “Христианы ертөнцийг хэт угсаатны эталон (1) гэж авч үзвэл түүнтэй адил тэгш хэт угсаатнууд: 2) Левант, буюу “Исламын ертөнц”-Испаниас Кашкар хүртэл тархсан огтхон ч шашны биш, харин угсаатан соёлын бүхэллэг, 3) Энэтхэг–мусульманчууд ноёрхсон түүний зарим хэсгээс бусад, 4) Хятад-өөрийгөө “Дундад гүрэн” гэж нэрлэдэг, бүдүүлэг зах хязгааруудтайгаа, 5) Византи–улс төрийн хил нь эртнээс цаг ямагт хэт угсаатных байсан, дорнод–христианы бүхэллэг, 6) Кельтийн ертөнц–XIY зуун хүртэл английн феодалуудаас өөрийнхөө язгуур уламжлалаа хамгаалж чадсан, 7) Балтийн славян–литовын хэлний бүхэллэг–XII зуунд үлдэц болон хувирсан, 8 ) Дорнод Европын хэт угсаатны бүхэллэг–Оросын газар зэрэг болно. Үүн дээр бид анхаарлаа хандуулъя, гэхдээ түүний угсаатны хувь заяаг дээр дурдсан бусад бүх хэт угсаатны зөрчлийн огтлолцлын суурин дээр авч үзэх болно. Евроазийн эх газар дээр тусгаарлах нь Сибирийн туйлыг тойрсон ард түмнүүдийн арав дахь хэт угсаатны хувьд л боломжтой байх бөгөөд үүнийг нь эсвэл эвенкүүд, эсвэл якутууд ямагт зөрчиж байдаг.
63. Энэтхэгийн хагас арал бол нутаг дээр нь эрт дээр үеэс хэт угсаатны бүхэллэг үүсч байсан хагас тив юм. Одоогийн бүхэллэг бол YIII зууны үед ражпут нарын бий болгосон хэт угсаатны угсаатны нийлэгжилтийн дурсах шат болно. “Ражпутын хувьсгал” гэдэг нь YII зуунд араб-исламын болон табгачийн хувьсгал (хятадын Тан гүрэн), мөн эртний Түвдийн сэргэлттэй нэгэн зэрэг болсон юм. (см. Гумилев Л. Н. Величие и падение Древнего Тибета //Страны и народы Востока. М., 1968. С. 153-182). Мусульман болон английн булаан эзлэлт нь олон хүний амийг авч одсон гаднын нөлөөлөл байсан ба Энэтхэг дэх угсаатны нийлэгжилтийг эвдээгүй юм.
64. Нэгдэх мянган гэхэд кельтүүд баруун талдаа римчүүдэд, зүүн европт талдаа маркоманн нарт, тал нутагт сарматуудад эзлэгдсэн юм. Гэхдээ “Кельтийн ертөнцийн” эхлэл нь тэд Умард Кавказаас (киммер-чүүд) Исланд хүртэл (кельтибер-үүд) тархсан үеэс эхлэнэ. Хэт угсаатныг тодорхойлохын тулд орон зайн төдийгүй, мөн цаг хугацааны зааг хилийг харгалзаж, улмаар хэт угсаатныг бүрдүүлэгч угсаатнуудын насыг харгалзах нь зүйтэй. Хэт угсаатныг тодорхойлоход тэдний цэргийн хүчин чадал гэх мэт өргөсөлт, агшилт ач холбогдол байхгүй, харин угсаатан хоорондын ойртолтын зэрэг л хамгийн чухал байдаг.
Дорнод Европод славянууд бий болох үедээ XII зууны үе гэхэд л зөвхөн “Анхдагч он тоолол”-ын зохиогчдын ой ухаанд үлдсэн овог аймгуудад хуваагджээ. Энэ бол жам ёсны зүйл юм. Угсаатны ойртон нягтралт нь шинэ нөхцөлд өмнөх овгийн ялгаа ач холбогдлоо алдсан том хотуудын эргэн тойронд эрчимтэй явагдах болов. А.Н.Насонов XI–XII зууны Оросыг “хагас улсуудын” систем гэж тодорхойлсон байдаг. 65. Насонов А. Н. “Русская земля” и образование территории древнерусского государства. М., 1951.
Энэ нь “Оросын газар нутаг”–аас бүтэн эрэмбээр доогуур байсан:1) Новогородын бүгд найрамдах улс зах хязгаарын хотуудын хамт, 2) Полоцкийн ванлиг, 3) Смоленскийн ванлиг, 4) Ростов–Суздалийн газар, 5) Рязаний ванлиг, 6) Турово–Пинскийн газар, 7) Киев, Чернигор, Переяславскийн гурван ванлигийг багтаасан Оросын газар, 8 ) Волынь, 9) Червонная Русь буюу Галицийн ванлигаас тогтож байлаа. Энэ жагсаалт дээр Владимир Мономахийн эзлэн авсан Дон болон Карпатын хоорондын половийн тал нутгийг нэмэх хэрэгтэй. Харин энэхүү XII–XIII зууны үед Их Булгар, половчуудын Доны чанад дахь бэлчир газар, Хойт Кавказ дахь аланы газар, Саксин хот бүхий Волжийн хазарууд Оросын нутгийн гадна байсан юм.
Энэ үед булгар болон хазарууд нь левантийн буюу мусульманы хэт угсаатанд хамаарагдаж байв. Ландшафтад дасан зохицох аргаараа тэд хөршүүдээсээ ялгагддаггүй байв. Гэхдээ Ирантай тогтоосон Булгарын хотуудын худалдаа, соёлын системт холбоо нь газар зүйн орчноос хавьгүй илүү үр нөлөөтэй байсан билээ. Тэд Волжийн Болгарыг “мусульманы” хэт угсаатны өмнө тулгуур, владимирийн ванлигийн дайсан болгож чадаж байлаа.
Нийтээр зөвшөөрсөн зарчмаар бол аланууд болон крымийн готуудыг византийн хэт угсаатанд, литов, латыш, ятвягуудыг балтийн хэт угсаатанд оруулна. Польш, Унгарууд бүр X зуунд баруун европын хэт угсаатанд орчихсон байсан. Харин немцийн загалмайтнууд полабскийн славянуудыг ялснаар католик шашин Баруун Европыг хэдийгээр угсаатны зүймэл бүтэцтэй ч соёлын хувьд нэгтгэсэн ба XII зуунд аажмаар, дандаа азтай байгаагүй ч газар нутгаа өргөжүүлсэн юм. Харин XIII зуунд загалмайтны аян дайнууд сүйрлийн ялагдлууд хүлээсэн байдаг.
Ахиад нэг эрэмбээр доошилж, өөрөөр хэлбэл оросын дэд угсаатны нэг Киевийг аваад үзэх юм бол бид тэнд барууны (II Святополка вангийн талынхан, түүний дотор Киев–Печерийн лавр), грекфилийн (Владимир Мономахын талынхан, Ариун Софид байрлаж байсан митрополиуд), Киевээс хөөгдсөн Всеславт талтайн улмаас хүчтэй нэрвэгдсэн үндэсний гэсэн гурван идэвхитэй консорци байсныг харж болно. 66. Гумилев Л. Н. Сказание о хазарской дани //Русская литературы 1974. № 3.
Консорци нь ангийн, язгуур угсааны, шашны, овгийн хуваагдлуудтай давхцдаггүй, тооцооллын бие даасан системийн үзэгдэл болохыг хялбархан ажиглаж болно. Энэ системийг нэн ашигтай гэж үзэж болох ба учир нь чухамхүү түүний ачаар жишээлбэл, дээр дурдсан улс төрийн чиглэлийн талынхны сэдлийг олж мэдсэн билээ. Ангийн зөрчлийг шинжлэх үед үүнийг хийж болохгүй бөгөөд учир нь үйл явдлын бүх оролцогчид нэг ангид хамаарч, хүчээ ард түмний дундах адил сэтгэгчдээсээ олж авдаг юм. Гэхдээ тэмцэл нь ямар ч гэсэн идэвхитэй бөгөөд харгис байлаа. Юунаас болж ? Юуны төлөө ?
\section{“ТАВАН ОВГИЙНХОН” БА “ДУНДАД ТЭГШ ТАЛЫН” ОРШИН СУУГЧИД}
НТӨ III зуунаас НТ III зууны төгсгөл хүртэл газар тариаланч Хятад, малчин–хүнчүүдийн нутагладаг Их Тал хоёр зэрэгцэн оршиж байлаа. Угсаатан тус бүр өөрийнхөө ландшафтад амьдарч, гэхдээ энэ хөршүүд нийлбэрээрээ нүүдлийн соёлын хэт угсаатны бүтэц (хийц) болон алс дорнодын соёл иргэншилд орж байлаа. Аль аль нь холимог угсаатан байсан. Нүүдлийн ертөнцөд хүнчүүдээс гадна сүмбэ (эртний монголчууд), цянь (нүүдлийн төвдүүд), бага юечжи, усун, кыпчак болон бусад овгууд байжээ. Хятадад хятадуудаас гадна уугуул: жун, ди, мань болон хэлээрээ төвд–бирм, тай болон малайн бүлэгт багтдаг юе нар амьдарч байв.
Нийтлэг соёлоор холбогдсон энэхүү хоёр хэт угсаатны оршихуйн үргэлжлэх хугацаа нь орчин үеийн хүмүүст “төлөв байдал” мэт түүхэн бодит байдал гэж үзэх шалтгийг өгч байна. Үнэн хэрэг дээрээ энэ нь удаан урссан үйл явц байсан юм. Хятадын нэгдсэн төр бий болж, ангийн зөрчил хурцдаж, энэ нь НТӨ III зуунд антагонист болон хувирч оршин суугчдын 60 хувийн амь насыг аван оджээ. НТ III зуунд энэ гүрэн задрахад хүн амын 80 хувь нь үхсэн байна. III зууны төгсгөлд хүнгүй болж, үгүйрсэн энэ орныг Цинь гүрэн нэгтгэжээ. III зуун гэхэд өмнө нь 50 сая байсан хятадын хүн ам 7,5 сая болсон байв. Дараа нь IY зуунд энэ нь 16 сая болтлоо өссөн байна.
Ямар ч гэсэн хамгийн хатуу ширүүн нийгмийн донсолгооны үед ч хятад угсаатан мөхөх дотоодын хангалттай шалтгаан байгаагүй юм. Сул чөлөөтэй газар их байлаа, улмаар хүн ам өсөж болох байв. Төрийн систем ажиллаж, соёлын хувьд гадуурхах явдал зогсов. Нүүдэлчидтэй хэтэрхий нягт холбоо тогтоосонгүй, Цинь гүрний засгийн газарт ч гэнэтийн зүйл байсан эсэргүүцэх чадвараа гайхалтайгаар алдаагүй бол эртний Хятадын угсаатан үргэлжлүүлэн оршиж чадах байв.
Тал нутагт овгийн байгуулал ноёрхож, задрал нь нүүдэлчдэд томоохон алдагдал учруулахгүйгээр маш удаан явагдаж байв. Гэхдээ тэднийг НТ I зуунд эхэлж, III зууны үед дээд хэмжээндээ хүрсэн тал нутгийн хуурайшил зовоож байлаа. Бэлчээр нутаг багассан нь хүн болон сүмбэчүүдийг Шар болон Ляохэ мөрөн рүү шахаж, хятадуудтай харилцаанд оруулах болов. Газар нь хоосорсон байсан учраас Цины засгийн газар хил дээрээ 400 мянган нүүдэлчин, янз бүрийн овгийн 500 мянга орчим төвдүүдийг оруулав. III зууны хятадын улс төрийн бодлоготнууд угсаатны хамаарал бол нийгмийн төлөв байдал, мөн тооны хувьд өчүүхэн хэсгийг уусган нэгтгэхэд хялбархан, ноёдыг нь соёлд сургаж, сурвалжтнуудыг нь дуулгавартай язгуур угсаатнууд болгоно гэж үзэж байв. Энэ тооцоо шазруун боловч муу байсан юм. Сурвалжтанууд түшмэдүүдийн дур зоргийг авир, газрын эздийн мөлжлөгийг тэсч байсан ч хятад болон хувирсангүй, ноёдууд дүрс үсэг, сонгодог яруу найраг сурсан байна. Гэвч 304 онд болсон тохиромжтой тохиолдолд овог нэгтнүүд рүүгээ эргэн очиж, “алдагдсан эрхийг зэвсгээр буцааж авах” зорилго тавьсан бослогыг толгойлов. Бодит байдалд хэрэглэсэн хуурамч онол сүйрэл дагуулав.
316 онд 40 мянган Хүн цэрэг бүх хойт Хятад, түүний дотор хоёр нийслэл, хоёр эзэн хааны эзэлж, хуримтлуулсан бүх баялгийг нь булаан авчээ. Хятадууд тэр үед хятадын зах хязгаар байсан Хөх мөрний эрэг хүртэл хөөгдөж, халуун бүсийн жунглид мань овгийнхонтой арга буюу холилдон амьдрах болов. Харин мань-ы дүр төрх, сэтгэл зүйн хэв маягийг ихээхэн өөрчлөв. Үр дүнд нь өмнөд хятадын угсаатныг бүрдэхэд хүргэсэн угсаатны нийлэгжилтийн өвөрмөц үйл явц эхэлсэн юм. Харин эх орондоо үлдсэн хятадууд хүнчүүдтэй холилдож…ингэснээрээ тэднийг хөнөөсөн юм. Ялагч хүн, хятад эмэгтэйн хүүхдүүд талын нүүдэлчний ёс зангаа аль хэдийн мартав. Ордны харшуудад хүмүүжсэн тэд эрч хүч, эрэлхэг зоригоо хадгалсан ч “өөрийн” гэсэн мэдрэмж, нөхрийн тохойны мэдрэмж, үнэнч байх гол шинжээ гээцгээжээ. Маргаан тэмцэл тэдний хүчийг барав, харин энэ хүртэл тэдний эцгүүд зохицол дунд амьдарч чадаж байсан юм. Ач нар нь хүүхнүүдийн дунд эрхэлсэн, хүн идэж зугаагаагаа гаргадаг, ойр тойрныхноосоо урвадаг болон хувирчээ. Давших дайны тухай яриа ч байхгүй болж, тэр байтугай хамгаалах тулалдаанд хүнчүүд ялагдал хүлээх болов. Эцэст нь 350 онд эзэн хааны өргөмөл хятад хүү өөрийн ах дүү нараа алж, өвч залгамжлагч болон засгийн эрхийг авч, улс доторхи бүх хүнчүүдийг алахыг тушаажээ. Үүнийг нь маш шаргуу гүйцэтгэснээс хүнчүүдтэй төстэй сахалтай болон монхор хамартай олон хятад үрэгдсэн юм.
Энэ яргалал урвагчийг аварсангүй. Сүмбэ–муюнчууд түүний хятад цэргийг бутцохиж, түүнийг өөрийг нь цаазлав. Хятадуудад тооны олон нь тус болсонгүй, тэд мөн соёлтойгоо хамт алдар гавъяаны уламжлалаа алдсан юм. Муюныг хүнчүүдийн хувь заяа нэрвэв. Тэд хятаджиж, талын табгач нарт ялагдсан юм. Тэд эхлээд өөрийгөө тойруулан нүүдэлчдийг нэгтгэж (угсаатан эрлийзжих түвшний холилдолт), харин дараа нь гай нь дуудаж, хятадын бат цул хүн ам амьдран суудаг Хэнанийг байлдан эзлэв. Y зууны төгсгөл гэхэд тэд хятадуудтай холилдон тэдний хан нь эзэн хаан цолтой болж, төрөлх хэл, табгач хувцас, үс засалт, мөн нэрээ хүртэл хоригложээ. Өөрийнх нь хэвшмэл зан үйлийн тогтсон үзлээсээ хагацсан олон албатууд улс гүрнийг нурааж, азгүй улс орныг үгүйрүүлсэн түрэмгий хөлснийхний золиос болсон юм. Дээр нь өлсгөлөн болж, хүн амын 80 хувь нь үрэгдсэн байна. Хоёр хэт угсаатны холигдолт нь ард түмэнд ингэж нөлөөлсөн бөгөөд амьд үлдсэн хүмүүс нь YI зуун гэхэд тэр үед табгач (сүмбэ буюу эртний монгол-Орч) гэж нэрлэдэг, хятад хэл (эртний хятад хэлнээс ялгагдах) хэрэглэдэг, харийн үзэл суртал–буддизмыг хүлээн авсан шинэ угсаатан болж гэнэт нэгдсэн юм. Энэ бол дундад зууны хятад угсаатны эхлэлийг тавьсан Тан улсын агуу их эрин үе байлаа. Тэд Хятадыг манжууд эзэлсэн зөвхөн XYII зуунд л бие даасан байдлаа алдсан билээ. Энэ нь Эртний Хятадад хамаарах, Византи, Эртний Римийн адил угсаатны нийлэгжилтийн шинэ орчил байлаа. 67. Гумилев Л.Н. Хунны в Китае. М., 1974, 68. Wieger L Textes historiques. Hien-Hien, 1905-1907. P. 1428, 69. Grousset R. L. Empire des Steppes. Paris, 1960. P. 96.
Энд авсан жишээнээс угсаатан ландшафттайгаа холбоотой нь тодорхой байна. Хүнчүүд Шар мөрний хөндийг эзлэн малаа бэлчээж, хятадууд газар хагалан суваг байгуулж, харин хүн–хятадын эрлийзүүд мал аж ахуйн ч, газар тариалангийн ч дадлага байхгүйгээс хөршүүд болон албатуудаа хэрцгийгээр цөлмөсөн нь атаржсан газар бий болж, хэдийгээр ой хяргаснаас үгүйрч, хааны ан агнуурын үед малын туурайнд эвдэрсэн байгалийн биоценез дахин сэргэхэд хүргэжээ. Гэхдээ YII зуун гэхэд л хятадууд алдагдсан газар нутгаа эргүүлэн авч, идэвхитэй цөлмөгч газар тариалан эрхлэх замаар ландшафтаа дахиад л эвдэх болов. Энэ тухай хожим ярина.
\section{БҮДҮҮЛГҮҮД БОЛОН РИМЧҮҮДИЙН ХАРИЛЦАА}
Угсаатнууд эмх замбараагүй холилдсон иймэрхүү дүр зургийг бид тэр үед Римийн эзэнт гүрэн хөгжиж буй эх газрын баруун хэсэгт мөн олж харж болно. Энэ талаар маш олон дүгнэлт хэлэгдсэн бөгөөд гэхдээ бараг бүгдээрээ Э.Гиббоны хэлсэн: “ Антонины эрин үеийн буурсан байсан үед Римийн эзэнт гүрэнд шингэсэн Эллиний нийгэм Дунай, Рейний чанад дахь цөлөөс гарч ирсэн хойт европын бүдүүлэгчүүд, эзлэгдсэн хүмүүсээс үүссэн, гэхдээ дорнод мужуудад уусан нийлээгүй христианы сүм хийд гэсэн хоёр фронтоор түүн рүү довтолсон, хоёр дайсны нэгэн зэргийг дайралтаас болж нуран унасан юм” гэсэн алдарт сэдвээр дуусгавар болдог юм. Энэ нь зөв эсэхийг харцгаая. 70. Цит. по: Toynbee A. J. Study of History /Abridiement by D. Somervell. London; New York; Toronto, 1946. P. 260-261.
I –ээс IY зуун хүртэл үргэлжилсэн Римийн эзэнт гүрний германчуудтай хийсэн дөрвөн зуун жилийн дайн Римийн ялалтаар дуусчээ. Рейний ч, Дунайн ч хил хязгаарыг эвдсэнгүй. Римчүүдээс готуудад өгсөн хэд хэдэн цохилт, Эгийн тэнгисийн эрэг хавиас готын флотыг зайлуулсан зэрэг нь Феодосийн ялалтын цагаатгал байлаа. Рим зөвхөн Германы Рейны цаад мужийг алдаж, сайн дураараа Дакийг цэвэрлэв, гэхдээ эдгээр орнуудад итали гаралтнууд амьдран суухыг хүссэнгүй, харин зах хязгаарынхан сенат болон римийн ард түмний санааг огтхон ч үл зовоох тэр газар руу цөлөгдөн очсон юм. Харийн түрэмгийллийн эсэргүүцэл эрс багассан Y зуунд нөхцөл байдал эрс өөрчлөгджээ. Тэгэхэд Стилихон, Аэций нар өөрсдийг нь римчүүд нь буюу улаан титэмтэн өдөөн хатгагч, урвагч нар алах хүртэл герман, хүнчүүдийг ялан дийлж байсан билээ. Римийн гай тотгор түүнд өөрт нь байгаагүй, харин түүний гадна байлаа. 71. Үүнийг А. Тойнби зөвшөөрсөн юм. (см.: Ibid. Р. 262)
Бүдүүлгүүд өөрсдөө ч гэсэн Римийн соёлыг устгахыг эрмэлзээгүй юм. Визиготын хаан Атаульфын өөрийнх нь хүлээн зөвшөөрснөөр “амьдрал эзэнт гүрэн дэх римийн бүхий л эзэмшилд хандах хүслээр эхэлсэн хугацааны хувьд бэлэн болоод байтал гэнэт түүний туршлага нь төрийн амьдралаас хуулийн удирдлагыг хөөн гаргах нь гэмт хэрэг болно, ингээд хэрэв түүнд хууль шахахаа болих аваас төр улс өөрөөрөө байхаа болино гэдгийг итгүүлсэн байна. Атаульф энэхүү үнэнийг ухамсарлаад римийн нэрийг бүх юманд, магадгүй хуучин агуу ихээс нь ч илүү бүх юмыг сэргээн тогтоохын тулд амьдралынхаа хүчийг хэрэглэхэд бэлэн болж, …алдар хүндэд хүрэхээр шийдсэн” байна. Хэрэв ийм байсан юм бол яагаад римийн соёл алга болж, түүний хамт энэхүү соёлд хайртай эрэлхэг, хүчтэй готууд дэлхийн угсаатны зургаас арчигдсан юм бэ ? 72. Цит. по: Ibid. Р. 409-410.
Германчуудын ялалт цаг хугацааны хувьд Римийн эзэнт гүрний бүх нутаг дэвсгэр дээр христианы ёс мандаж байсантай давхцсан нь энэ номлолыг ард түмэн, улс гүрэнд хөнөөлтэй мэт батлахад хүргэжээ. Энэ үзэл баримтлалыг бүр 393 онд Капитолийн Ялалтын шүтээнийг дахин сэргээхийг оролдож байсан тэрс үзэлтнийг хамгаалагч Евгений ба Арбогаст нар дэвшүүлсэн байна. Гэхдээ христианы легионууд 312 оных шигээ илүү зоригтой байлаа. Тэрс үзэлтний армийн удирдагчид тулалдаанд баатарлагаар үрэгдэж, ингэснээрээ хөгжлийн хэтийн төлөв нь аль нэгэн сургааль номлолд бус, илүү их газар дэлхийн хэрэг байдгийг үзүүлсэн байна. Хамгийн сонирхолтой нь мөн энэ зүйлийг сүм хийдийн эцгүүд нотолдог юм.
Гэгээн Августин болон Y зууны христианы дурын сэтгэгчид “Бид эзэн хаан Авреалинд Дакийг орхиж, хил дээр цэргийн хүчтэй пост тавьж байх бодлогыг орхи гэж хэлээгүй. Бид Каракаллад римийн иргэн цолыг алив төрлийн хүмүүст олгох, мөн хүн амыг нэг газраас нөгөө газар руу цэрэг болон иргэний албан тушаалтнуудаар хүчлэн хөөхийг албадаагүй. Эх оронч мэдрэмжийн хувьд гэвэл түүнийг эзэн хаад та бүхэн өөрсдөө сүйтгэсэн бус уу ? Римийн иргэд галл болон египет, африк болон хүнчүүд, испаничууд болон сиричүүдэд хандахдаа ийм хэсэг бусаг овгийн бөөн хүмүүс өөрсдийг нь эзлэн авсан Римийн ашиг сонирхолд үнэнч байна гэж тэд итгэж байлаа гэж үү ? Эх оронч үзэл нь хичээл зүтгэлээс хамаардаг болохоос, задралыг тэвчдэггүй юм” гэж хэлж чадах байсан. Гэтэл гэнэтийн: яагаад ийм олон янзийн овог аймагтай христианы ертөнцөд золиос болох чадвар, нөхрийн тохойг мэдрэх мэдрэмж, барилдлага болон хөгжлийн боломж үүсэв ээ ? гэсэн гэнэтийн асуулт гарч ирээгүй бол Гэгээн Августины зөв байхсан.
Хариулт тун энгийн юм. Арьстны ялгаа шийдвэрлэх биш, ерөөсөө бол ёс мэт их ач холбогдол байдаггүй, харин угсаатны ялгаа нь зан үйлийн хүрээнд байдаг байна. Христианы нийтлэгийн зан үйлийн тогтсон үзэл хатуу жаягтай байдаг. Неофит буюу шинээр шашинд элссэн хүн түүнийг биелүүлэх үүрэгтэй, эсвэл нийтлэгийг орхин явах ёстой. Улмаар хоёрдахь үе болоход христианы консорцийн суурь дээр гетерозигот буюу нэг төрлийн бус дэд угсаатан бий болдог аж. Гэхдээ энэ нь бат цул, хэл нь явцуу, үнэхээр шашингүй, улс гүрэн нь албатуудынхаа сэтгэл зүйн өөрчлөлтийг хэрэгжүүлдэггүй. Янз бүрийн угсаатны гишүүд нэгдмэл нийгмийн хүрээний дотор зэрэгцэн оршоод өөрийнхөө хүндээс болж задардаг байна. Энд байгалийн хуулийн өмнө римийн эрх ч хүчгүйдэж байна.
Августины хэлснээр эзэнт гүрний дотоод дахь хүн амын шилжих хөдөлгөөн багагүй хөнөөлтэй байсан юм. Байгалийн хүрээлэл болох ландшафттай хүний харилцах нь тохиолдол бүрт дасан зохицохуйгаар тодорхойлогдож байдаг тогтмол хэмжигдэхүүн болно. Сири, Британи, Галли, Фракийн ландшафтууд нэн их ялгаатай. Улмаар шилжигчид хана нь харь, дасаж дадаагүй, дургүй хүрмээр байгалийг тэднээс салгаж байдаг хотуудын амьдралыг илүүд үздэг болжээ. Энэ нь тэдний байгальд харилцах харилцаа нь цэвэр хэрэглээний шинжтэй, шууд яривал цөлмөгч шинжтэй гэсэн хэрэг юм. Үр дүнд нь Галлийн ойн 2/3, Аппенины өтгөн шугуй алга болж, Атласын уулан дахь хөндийнүүдийг хагалж, шимгүйдүүлж, Эллади болон Фригийн толгодын ямаануудад золиос болон өгсөн байна. Хамгийн хөнөөлт хоосролыг цэргийн удирдагчид өөрсдөө биш, боолчлолоос зугтахыг нь хязгаарлах гэж эх орноос нь алсад суурьшуулсан тэдний колониуд буюу цэргийн олзлогдогсод үйлдсэн аж.
Өөр үгээр бол Римийн эзэнт гүрний улс төрийн системийн хатуу социал холбоо хөндөгдөөгүй байсан үед ч угсаатны холбоо нь бүр IY зуунд бүрэн эвдэрсэн байна. Харин дараа нь Y зуунд болсон германчуудын довтолгоо энэ үйл явцыг гүнзгийрүүлжээ. Учир нь идэвхитэй эрлийзжилт гот, бургунд, вандал зэрэг угсаатныг идэж, тэднийг задралын ерөнхий үйл явцад татан оруулжээ. Тэр үед мөн тэнд шинэ угсаатан үүссэн бөгөөд энэ нь эзэнт гүрний дорнод хязгаарт болсон байна. Энд ач холбогдлыг нь нээн олбол зохих нэмэлт “икс хүчин зүйл” үйлчилсэн нь тодорхой байна.
\section{УГСААТНУУД ЯМАГТ ХАРИЛЦААНААС ҮҮСДЭГ}
Хэт угсаатнууд юугаараа ялгагддаг, тэднийг өөр хоорондоо нийлэхэд нь юмуу өмнөх үеийнхээ баялгийг өвлөн авахад юу саад болдог вэ ? Хэт угсаатны дотоодод оршин байгаа угсаатнууд байнга, саадгүйгээр уусан нийлдэг. Хэт угсаатны энэхүү өндөр тогтвортой чанарыг угсаатны ноёрхогч үзэл, хэт угсаатан тус бүрт нэгдмэл ач холбогдол бүхий аль нэгэн төгс эрхэмлэлийн үгэн илэрхийлэл, тухайн системд орж буй бүх угсаатнуудын хувьд анхдагч утга санааны хөдлөнги шинж байгаа зэргээр тайлбарлаж болно. Төгс эрхэмлэлийг зөвхөн их зангаар л сольж болно, тэгвэл хэт угсаатны уусан нэгдэх нь хуурамч зүйл болно. Янз бүрийн хэт угсаатны төлөөлөгч тус бүр сэтгэлийнхээ гүнд жам ёсны бөгөөд цорын ганц зөв хэмээн төсөөлж буй тэр зүйлтэйгээ л үлдэнэ. Төгс эрхэмлэл нь түүнийг дагагчдад шалгуур гэхээсээ түүний амьдралын баталгааны бэлэгдэл мэт санагдана. Ингээд ноёрхогч үзэл санаа гэдгийг бид угсаатан соёлын олон янз байдлыг зорилгод чиглэсэн нэгдмэл байдал болгох угсаатны нийлэгжилтийн үйл явцын хувьд анхдагч шилжилтийг нь тодорхойлдог үзэгдэл буюу үзэгдлийн бүрдэл (шашин, үзэл суртал, цэрэг, ахуйн зэрэг) гэж үзэж байна.
Угсаатны үзэгдэл бол түүнийг бүрдүүлэгч хүмүүсийн зан үйл ч бас мөн гэдгийг сануулья. Өөрөөр хэлбэл тэр хүмүүсийн биед байдаггүй, харин тэдний үйлдэл, харилцаанд байдаг байна. Улмаар шинэ төрсөн хүүхдээс өөр угсаатны гадна орших хүмүүн гэж үгүй. Хүн болгон өөрийгөө ямар нэгэн байдлаар авч явах ёстой, чухам зан үйлийн ийм шинж чанар нь түүний угсаатны хамаарлыг тодорхойлдог. Ийм ахул шинэ угсаатан үүсэх нь өмнөхөөсөө ялгаатай зан үйлийн шинэ тогтсон үзлийг бий болгоно гэсэн үг юм. Шинэ тогтсон үзлийг хүмүүс бүтээх нь бүрэн ойлгомжтой боловч энд шууд л эргэлзээ бий болдог. Нэгдүгээрт, энэхүү шинийг санаачлагчид үүнийгээ ухамсартай үйлддэг үү, эсвэл ухамсаргүй үйлддэг үү ? Хоёрдугаарт, шинэ нь хуучнаасаа ямагт сайн байх уу ? Гуравдугаарт, тэд хэрхэн мах цусны тасархай нэг овгийнхоо байтугай, өөрийнхөө уламжлалын инерцийг нугалж чадаж байна вэ ? Онолын хувьд энэ эргэлзээ шийдвэрлэгдэшгүй юм. Гэхдээ эмпирик дүгнэлт томъёолж болохоор эртний угсаатны зүйн ажиглалтын материал тус болж байна. Энэ нь угсаатан бүр хоёр буюу түүнээс илүү угсаатны, өөрөөр хэлбэл тэднээс өмнө оршин байсан угсаатны хослолоос бий болдог ажээ.
Тухайлбал, орчин үеийн испаничууд харьцангуй хожим Дундад зууны үед эртний ибер, кельт, римийн колоничид, иберийн шууд удам-баскуудын нэгдэн орсон германы свев болон вестгот овог, сармат болон түүний ойрын төрөл осетинуудын удам–аланууд, араб–семитүүд, мавр болон туареги–хамитууд, угсаатныхаа өвөрмөц байдлыг хэсэгчлэн хадгалсан норманчууд болон каталончуудын хольц дундаас энэхүү нэрийг зүүсэн угсаатан болон бүрэлджээ.
Англичууд л гэхэд англо, сакс, нөхрүүд нь тулалдаанд үрэгдсэн кельтийн эмэгтэйчүүд, дани, норвеги болон Анжу ба Пуату–гийн баруун францчуудаас бүрдсэн нийлмэл угсаатан юм.
Их Оросууд өөрийнхөө бүрэлдэхүүнд: Киевийн оросын дорнод славян, баруун славяны-вятичей нар, финийн–меря, мурома, весь, заволоцкийн чудиуд, анхлан дээр дурдсан финн овгийнхонтой холилдож байсан угрууд, балтийн-голядь, туркийн–загалмайлсан половчууд болон татаарууд, мөн цөөн тооны монголчуудыг багтаадаг.
Эртний хятадууд бол монголжуу антропологийн янз бүрийн хэв маягт, тэр ч байтугай европейд маягийн ди нарт хамаардаг Шар мөрний хөндийн олон овгуудын хольц юм. Иймэрхүү зураглал Японд ч бий бөгөөд эрт дээр үеэс полинезчүүдтэй төстэй өндөр биет монголжуу хүмүүс, Солонгосоос гаралтай навтгар монголжуу хүмүүс, сахалтай айнууд болон Хятадаас гарагсад нь нэгдмэл угсаатан болж гагнагдсан байдаг.
Түүх эрин зууны харанхуйд живүүлсэн цөөн тооны, тусгаарлагдсан угсаатнууд хүртэл үлдэгдэл антропологийн болон хэлний шинжээрээ угсаатны хольцынхоо хуучны ялгааг хадгалсаар байдаг юм. Тухайлбал, эскимосчууд, Пасхи арлын оршин суугчид, мордва нар, марийчууд, эвенкүүд, Гидикушийн уулсын патанууд ийм байдаг. Энэ нь дээр үед эд нар нарийн бүрэлдэхүүнт угсаатны хамт олон байсан бөгөөд одоо ажиглагдаж буй нэг хэвийн байдал бол янз бүрийн уламжлалын өө сэвийг нь арилгасан угсаатны нийлэгжилтийн урт удаан үйл явцын үр дүн гэдгийг харуулж байна.
Гэхдээ энэ нь бие биенээсээ хол угсаатнууд холилдохын хөнөөлтэйг дөнгөж саяхан баталсан дүгнэлттэй зөрчилдөж байна. Нэгдэх ажиглалт ямар байна, хоёр дахь ажиглалт нь маргаангүй байдаг. Дотоод зөрчил агуулсан дүгнэлт үнэн байж болох болов уу ? Хэрэв бид нээж ололгүйгээр зорилтоо шийдвэрлэх боломжгүй, ямар нэг “икс хүчин зүйл”, ямар нэг маш чухал жижиг хэсгийг (деталийг) тооцоолоогүй байгаа зөвхөн нэг л тохиолдолд ийм байж болно. Ийм учраас бүхий л илэрхий баримтуудыг тайлбарлах зөрчилгүй хувилбар олохын тулд алдаа онооны аргаар цаашаа хөдөлье.
\section{“ИКС ХҮЧИН ЗҮЙЛ”}
Одоо өөр нэг саналыг шалгаад үзье. Магадгүй энэ нь урт удаан үйл явц биш, харин шинэ угсаатан бүрэлдэх шалтгаан болсон агшин зуурын үсрэлт байж болох бус уу ? Үүнийг зөвхөн үйл явдлыг нь хангалттай тодорхой дүрсэлсэн нэн шинэ түүхийн жишээн дээр шалгаж болох юм. Латин Америкийн түүхийг аваад үзье. Испаний конкистадорууд тулалдаанд догшин байсан бөгөөд харин 1537 онд III Павлын тушаасны дараагаас л индианчуудыг “доод арьстан” биш, харин нэр төртэй өрсөлдөгчид гэж үзэх болсон байна. Амьд үлдсэн индианы удирдагчдыг загалмайлж, өөрийнхөө орчинд авдаг болсон ба харин жирийн индианчуудыг тариан талбайд пион буюу тариачин болгодог байв. Ингэж хоёр зуун жилийн туршид Мексик, Перугийн хүн ам бүрэлдсэн бөгөөд харин уулс болон халуун бүсийн ойд цэвэр индиан овгууд хадгалагдан үлдсэн юм. Боолын худалдаа Америкт негрүүдийг бий болгосон ба мулат болон самбо (негр болон индианы удам) бий болсонд атгаг санаа байгаагүй юм. ХХ зуун эхэлж Францад эзлэгдсэн Испанийн эзэнт улсаас чөлөөлөгдөхийн төлөө тэмцэл эхлэхэд бослогын хөдөлгөөний удирдагчдын ихэнхи нь испаничууд биш, харин метис болон мулатууд байсан билээ. 1819 онд энэ талаар Боливар өөрөө: ”Манай ард түмэн европ ч биш, хойд америк ч биш, тэр нь европчуудын удам гэхээсээ африк болон америкчуудын хольц болох нь илүүтэй, учир нь Испани өөрөө ч гэсэн Европт өөрийнхөө африк цусаар, өөрийн байгууллагаар, өөрийн зан аашаар хандахаа больсон. Хүний ямар гэр бүлд хамаарагдаж байгааг нарийн хэлэх боломжгүй. Индиан хүн амын ихэнхи хэсэг устгагдсан, европчууд нь америкчуудтай холилдсон, харин америкчууд нь индиан болон европчуудтай холилдсон. Нэг эхийн хормойд төрсөн, гэхдээ цус, гарлаараа янз бүр, манай эцгүүд гадаадынхан, янз бүрийн өнгөтэй арьстай хүмүүс юм. Энэ ялгаа нь маш үр дагавар дагуулж болно” гэж хэлсэн байдаг. 73. Боливар С. Избр. произв. М., 1983. С. 83.
Түүхчдийн нүдэн дээр үүссэн энэ ард түмэн нэн тогтвортой бөгөөд бусад хөрш ард түмнүүдтэйгээ төсгүй ажээ. Хэл, соёл, шашин гэх мэтийн бүхий л гадаад шинж тэмдгүүдээрээ Венесуэл болон Колумбийн оршин суугчид испаничуудтай ойролцоо байсан. Эдийн засгийн талаар чөлөөлөх дайны үр дүнд английн болон хойд америкийн худалдааны кампаниудаас хамааралт байдлаа тэд испанийн протекционизм буюу гаалиар тэтгэх бодлогоор сольж, зөвхөн ялагдаж байлаа. Тусгаар тогтнолын төлөө дайн нь шигүү хүнтэй Европт наполеоны бүх дайны үед алдсан шиг сийрэг хүн амтай газар нутагт сая хүний амь үрсэн тийм харгис болсон билээ. Гэхдээ босогчдын нүдэн дээр энэ бүх золиос нь тэд испаничууд биш байсан, улмаар тусдаа амьдрах ёстой гэдгээрээ цагаатгагдаж байлаа. Сонирхолтой нь тэр үед индианчууд испанийн засгийн газрыг дэмжиж байв. Ийнхүү гарал үүслийн холимог байдал нь бүхэл угсаатан үүсэхэд саад биш болой.
Гэхдээ үнэхээр ийм гэж үү ? Амьтанд бол эрлийз хэлбэрүүд нь ямагт тогтворгүй бөгөөд хоёр эцэг эхийнхээ олдмол чадварыг голдуу гээж, үүнийгээ нэгдүгээр үедээ амьдралд хэт дурлахад зарцуулж, дараах үедээ энэ нь голдуу алга болдог нь тодорхой юм. Холимог гэр бүлийн удмынхан эсвэл анхдагч хэв маягийн (эцгийн буюу эхийн) аль нэг рүү буцаж, эсвэл аль нэг орчинд дасан зохицох явдал нь хэд хэдэн үе дамжиж байж боловсрогддог учраас мөхдөг. Энэ нь уламжлал бөгөөд хоёр уламжлалын холимог нь дасан зохицохуйн тогтворгүй шинжийг бий болгодог.
Ихэнхи тохиолдолд, магадгүй заримдаа хүнд ч ийм байж болно, гэхдээ энэ нь ямагт ийм байсан бол нэг ч шинэ угсаатан үүсэхгүй байх байсансан. Холимог гэр бүлд эртнээс явж ирсэн хүн төрөлхтөн бүр чулуун зэвсгийн (неолит) эртний үеэс үүлдэр нь доройтох байсан. Үнэн хэрэг дээрээ угсаатны зүйн зураг дээрээс маш цөөн угсаатны бүлгүүд төрж, алга болж байдаг. Харин хүн төрөлхтөн зүйлийнхээ хувьд нэн эрчимтэй хөгжиж, одоо хүн амын өсөлтийг хүн ам зүйн тэсрэлт гэж нэрлэж байна. Байгалийн шалгарлын эвдлэгч нөлөөллийг тэнцвэржүүлэгч, мөн уламжлал буюу удамшлын дохиоллын үүргийг тогтворжуулагч хүчин зүйл оршиж байгаа нь илт байна. Энэ нь “икс хүчин зүйл” бөгөөд тэр нь зан үйлийн өөрчлөлтөөр буюу хүмүүс өөрөө сэтгэл зүйн бүтцийн онцлог гэж хүлээн авах байдлаар илрэх ёстой. Улмаар энэ нь угсаатны нийлэгжилтийн үйл явцыг өдөөгч бөгөөд урамшуулагч шинж тэмдэг байх юм. “Икс хүчин зүйлийг” олж, хайж буй шинж тэмдгийнхээ агуулгыг нээх юм бол бид угсаатны нийлэгжилтийн тусгай тохиолдлын хувьд болон бүх нийлбэр цогцсынх нь хувьд үйл явцын хөгжлийн механизмыг тайлбарлаж чадах болно.
Тавьсан зорилгодоо хүрэхийн тулд бидэнд хүн төрөлхтний түгээмэл түүхийн арвин баялаг, шалгаж нягтласан, хатуу он цагт оруулсан материал хэрэгтэй. Хэрэв бид байгалийн шинжлэх ухаанд хэрэглэдэг аргуудын тусламжтайгаар түүнийг боловсруулж чадваас тавигдсан зорилгыг шийдвэрлэх өгөгдөхүүнийг гарган авч чадна. Харин энэ болтол дээр томъёолсон эргэлзээтэй хариултуудаар хязгаарлагдах болно. Энэ нь: 1) Зан үйлийн шинэ тогтсон үзлийг бодож олж болохгүй, учир нь хэрэв хэн нэгэн гайхал өмнөө ийм зорилго тавьсан ч гэсэн тэрээр өөрийгөө ямар ч гэсэн хуучнаараа, дадсан сурснаараа буюу угсаатны хамт олны ахуйчлалын оршин буй нөхцөлд хамгийн илүү дасан зохицсоноороо авч явах болно. Угсаатнаас гарна гэдэг бол зөвхөн барон Мянгуужин хийснээрээ тодорхой болсон шиг өөрийнхөө үснээс татан намгаас гарна гэсэн хэрэг юм. 2) Зан үйлийн шинэ тогтсон үзэл нь хүмүүсийн замбараагүй үйл ажиллагааны үр дүнд үүсдэг учраас түүнийг сайн уу, муу юу гэдэг асуулт тавих нь утгагүй юм. Энд харьцуулах хэмжээс байхгүй. 3) Гэхдээ хэрэв зан үйлийн тогтсон үзлээр илэрч буй ахуйжсан уламжлалыг эвдвэл энэ нь боломжгүй, утгагүй, хэн ч үүнийг ухамсартайгаар хүсэхгүй. Тэгэхлээр энэ нь нөхцөл байдлын онцгой урсгалын хүчээр болдог нь илт байна. Ямар хүчний нөлөө вэ ? Чухам энэ асуултад л хариулт олох хэрэгтэй байна.

