Оршил

ЮУНЫ ТУХАЙ ЯРЬЖ БУЙ БОЛОН ЯАГААД ЭНЭ НЬ ЧУХАЛ БОЛОХ ТУХАЙ.
ЭНД ЭТНОЛОГИЙН ЗАЙЛШГҮЙГ БОЛОН УГСААТНЫ НИЙЛЭГЖИЛТИЙН (ЭТНОГЕНЕЗ) ТАЛААРХ ЗОХИОГЧИЙН ҮЗЭЛ БОДЛЫГ ТАЙЛБАРЛАХ БӨГӨӨД УНШИГЧДЫГ ЗӨРЧЛИЙН ТӨӨРҮҮЛЭГТ ХӨТӨЛСӨН ЗОХИОГЧ ЭНЭХҮҮ БИЧВЭРИЙН ҮЛДЭХ ХЭСЭГТ ҮНДЭСЛЭЛИЙГ НЬ ГАРГАНА.

ИТГЭЛ АЛДРАХААС АЙХ НЬ
Манай үеийн уншигчид түүхийн болон угсаатны судлалын шинэ ном худалдан аваад сөхөж үзэхдээ түүнийг ядаж дунд нь хүртэл уншиж чадах болов уу гэж эргэлзэж эхлэдэг юм. Ном нь түүнд уйтгартай, утгагүй, эсвэл зүгээр л түүний сонирхолд үл нийцэх мэт санагддаг. Уншигчид ч яахав арай гайгүй, ганц нэг рубль л алдана биз. Харин зохиогч ямар байх бол? Мэдээ баримт цуглуулна. Зорилгоо тодорхойлно. Олон арван жил шийдэл хайна. Олон жилээр бичгийн ширээний ард сууна. Шүүмжлэгчидтэй учраа ололцоно. Редакторуудтай тэмцэлдэнэ. Ингээд эцэст нь гэв гэнэт энэ бүхэн нь хоосон зүйл болж, ном нь сонирхолгүй болдог. Тэр нь номын санд л хэвтэх бөгөөд хэн ч түүнийг нь авахгүй. Ингээд л амьдрал талаар өнгөрөх нь тэр.
Энэ нь аймшигтай зүйл бөгөөд ийм үр дүнгээс зайлсхийх бүх арга хэмжээг авах нь чухал. Тэгээд ямар арга хэмжээ авах гэж ? Их сургууль болон аспирантурт суралцаж байх үед нь ирээдүйн зохиогчид түүний зорилт бол эх сурвалжуудаас аль болох их ишлэл авч, тэдгээрийгээ ямар нэг дараалалд оруулж, гаргалгаа хийх, жишээлбэл, дээр үед боолын эзэн болон боолчууд байсан. Боолын эзэн нь муу, тэдэнд сайхан байсан, боолчууд сайн, тэдэнд муу байсан. Тариачид бол бүр ч дор амьдарч байсан гэсэн санааг албадах нь цөөнгүй.
Энэ бүхэн нь мэдээж зөв боловч гол гамшиг нь энэ талаар хэн ч, зохиогч нь ч уншихыг хүсдэггүй. Нэгдүгээрт, учир нь энэ бүхэн мэдээжийн зүйл,, хоёрдугаарт энэ нь жишээлбэл, яагаад нэг арми нь ялж, нөгөө нь ялагдсаныг, юунаас болж нэг улс хүчирхэгжиж байхад нөгөө нь мөхөж байдгийг тайлбарладаггүй. Эцэст нь яагаад хүчирхэг угсаатнууд үүсч, мөн гишүүд нь мэдээж бүрэн устаагүй атал мөнхүү угсаатнууд хаашаа ч юм алга болдгийг мөн тайлбарладаггүй.
Энд дурдсан асуудлууд аль нэг ард түмэн гэнэт хүчирхэгжиж, хожим нь алга болдог тухай бидний сонгосон сэдэвт бүхэлдээ хамаарах юм. Үүний тод жишээ нь XIII – XYII зууны монголчууд болно. Мөн бусад ард түмэн ч энэхүү зүй тогтолд захирагддаг. Талийгаач академич Б.Я. Вернадский “Энэ бүхэн хэрхэн, яагаад болж өнгөрснийг би ойлгохыг хүсч байна” гэж яг таг тодорхойлсон атлаа бусад судлаачдын адил хариулт өгөөгүй юм. Бид уншигч энэхүү номыг хоёр дахь хуудаснаас нь хаахгүй гэсэн хатуу итгэлтэйгээр энэ асуудлыг дахин дахин авч үзэх болно.
Бид энэ зорилтоо шийдвэрлэхийн тулд юуны өмнө судалгааны арга зүйг өөрийг нь судлах ёстой гэдэг нь нэн ойлгомжтой юм. Эсрэг тохиолдолд энэхүү зорилтыг аль хэдийнээ шийдчихсэн байгаа. Учир нь тоо томшгүй олон баримтууд байгаа, гол асуудал нь тэдгээрийг нэмэхэд биш, харин ажил хэрэгт хамаатайг нь сонгоход байгаа юм. Орчин үеийн баримт судлаачид мэдээллийн далайд живсэн ч асуудлыг ойлгоход ойртож ч чадсангүй. Сүүлийн зуун жилд археологууд нээж, баримт сэлтийг цуглуулж, янз бүрийн тайлбартайгаар хэвлүүлсэн асар их бүтээлүүд бий болов. Дорно дахины судлаачид хятад, перс, латин, грек, армян, араб зэрэг янз бүрийн сурвалжуудыг ангилж ялган, мэдлэгийн нөөцийг бүр ч их нэмжээ. Баримт мэдээллийн тоо хэмжээ өссөн ч шинэ чанар руу огтхон ч шилжсэнгүй. Яагаад жижиг овог заримдаа дэлхийн хагасыг эзэрхдэг, улмаар тооны хувьд өсч, дараа нь алга болдог гэдэг нь урьдын адил тодорхойгүй үлджээ.
Энэхүү номыг зохиогч бидний мэдлэгийн хүрсэн түвшин, тодруулбал, судалж буй зүйлээ мэдэхгүйн тухай асуудлыг дэвшүүлэн тавьжээ. Эхлээд харахад энгийн, амархан мэт зүйл уншигчдын сонирхлыг татсан хэсгийг нь эзэмдээд ирэхэд оньсого болон хувирдаг. Иймээс тулхтайхан ном бичих шаардлагатай юм. Харамсалтай нь бид яг таг тодорхойлолт (ер нь судалгааг нэн хөнгөвчлөх) гаргаж чадахгүй, гэхдээ бид анхдагч дүгнэлт хийх боломжтой. Эдгээр нь тулгамдсан асуудлын бүхий л нарийн ширийнийг гаргаж чадахгүй байлаа ч анх удаа дөхөж үзвэл хараахан бичээгүй байгаа угсаатны түүхийг тайлбарлахад нэн тохиромжтой үр дүнг өгч болно. Хэрэв зөрүүд шүүмжлэгч гарч ирээд номын эхэнд “угсаатан” гэдэг яг таг ойлголт өгөхийг шаардах аваас бид : Угсаатан бол гэдэг бол амьд биеийн геобиохимийн эрчим хүчээр ажилладаг, түүхэн үйл явдлын диахроник дэс дарааллаар батлагддаг, термодинамикийн хоёрдугаар эхлэлийн зарчимтай нийцэн явагддаг био хүрээний үзэгдэл буюу тасралтгүй хэв шинжийн бүхэллэг гэж хэлмээр байна. Хэрэв энэ нь ойлгоход хангалттай бол энэ номыг цааш нь уншихгүй байж ч болно.
УГСААТНУУД БОЛ HOMO SAPIENS ЗҮЙЛИЙН ОРШИХУЙН ХЭЛБЭР БОЛОХ НЬ
Homo sapiens хэмээх биологийн зүйл өөрчлөгдөж байгаа юу, эсвэл социал зүй тогтол зүйл бүрдүүлэгч үйлчлэлийн механизмыг бүрэн шахан гаргасан уу ? гэдгээр зуу гаруй жил маргалдаж байна. Хүн болон бусад амьтдад нийтлэг юм нь орчинтойгоо бодис болон эрчим хүчийг солилцох зайлшгүй чанар байдаг, харин хүн амьтдаас өөртөө хэрэгтэй оршихуйн бүхий л хэрэгслийг хөдөлмөрөөрөө олж авдаг ба чингэхдээ байгалтай зөвхөн биологийн төрөл биш, юуны өмнө нийгмийн амьтан байдлаар харилцан үйлчилдгээрээ ялгаатай. Нөхцөл ба хэрэгсэл, үйлдвэрлэх хүчин, эдгээрт харгалзсан үйлдвэрлэлийн харилцаа тасралтгүй хөгжиж байдаг. Энэхүү хөгжлийн зүй тогтлыг марксист улс төр эдийн засгийн ухаан болон социологи судалж байдаг.
Гэхдээ хүн төрөлхтний хөгжлийн социол зүй тогтол нь биологийн, тухайлбал, мутацийн зүй тогтлын үйлчлэлийг “орлодоггүй”, Бид социал төдийгүй хөгжлийн илүү ерөнхий зүй тогтолд захирагддаг гэдгийг үл тоох буюу ухамсартайгаар үгүйсгэн өөртөө учруулдаг онолын өрөөсгөл байдал, практик хохирлоос зайлсхийхийн тулд эдгээрийг заавал судлах ёстой.
Арга зүйн хувьд ийм судалгааг үйлдвэрлэлийн тодорхой аргаас санаатайгаар хөндийрөхөөс эхлэн явуулж болно. Энэхүү хийсвэрлэл нь зөв бөгөөд учир нь угсаатны нийлэгжилтийн шинж чанар нь хүн төрөлхтний нийгмийн хөгжлийн хэмнэлээс эрс ялгаатай байдагт оршино. Бидний итгэж байгаагаар ийм аргаар авч үзэхэд хүн байгалийн харилцааны механизмын зааг хүрээ илүү тодорхой болдог юм.
Техник хэчнээн хөгжсөн байлаа ч хүмүүсийн амьдралыг тэтгэхэд зайлшгүй бүх зүйлийг байгалаас авдаг. Эдгээр нь тропик сүлжээнд хүмүүс амьдран буй бүс нутгийн дээд, эцсийн салбар болон орж байдаг гэсэн хэрэг юм. Ямар байлаа ч гэсэн дэгээрээ бүтэц – системлэг бүхэллэг зүйлийн элементүүд болох ба үүнд хүний хамт доместикат (гэрийн амьтан болон таримал ургамал), хүний хувирган өөрчилсөн болон байгалийн онгон болон хэвлийн баялаг, эсвэл нөхөрсөг, эсвэл дайсагнасан хөршүүдтэйгээ харилцах харьцаа, социал хөгжлийн аль нэг хөдлөнги шинж, түүнчлэн материаллаг болон оюун санааны соёлын элементүүд болон хэлний (нэг буюу хэд хэдэн) ямар нэг нийцэл зэрэг ордог.
Энэ нь түүхэн цаг хугацаанд үүсэн бүрэлдэж, сарнин алга болдог бөгөөд чингэхдээ өөрийнхөө ард өөрийн хөгжлөөс ангижирсан, зөвхөн эвдрэн сүйрэх шинжтэй хүний үйл ажиллагааны хөшөө дурсгал, гомоестазын буюу зогсонги шатандаа хүрсэн угсаатны реликт хэмээх үлдэгдлийг үлдээдэг. Гэхдээ угсаатны нийлэгжилтийн үйл явц тус бүр нь дэлхийн гадаргуу дээр арилшгүй ул мөрөө үлдээдэг бөгөөд үүний тусламжтайгаар угсаатны түүхийн зүй тогтлын ерөнхий шинж чанарыг тогтоох боломжтой. Антропогенийн буюу хүн төрөлхтний сүйрүүлэгч үйлчлэлээс байгалийг аврах асуудал шинжлэх ухааны гол асуудал болоод буй өнөө үед хүний ямар төрлийн үйл ажиллагаа угсаатнуудыг багтаан буй ландшафт буюу амьдрах орчинд хөнөөлтэй вэ гэдгийн учрыг олох ёстой. Хүмүүст үхлийн уршигтай байгалийн сүйрэл нь зөвхөн манай үеийн гамшиг биш бөгөөд энэ нь соёлын хөгжил, түүнчлэн хүн амын өсөлтийг дандаа дагаж үүсээгүй байдаг.
Хөгжлийн зүй тогтлын хоёр хэлбэрийн харилцан үйлчлэлийн тухай асуудлыг тавихдаа аспект буюу зэрэгцээ асуудлуудыг ялган тогтоох хэрэгтэй. Эсвэл асуудлыг хүний үйл ажиллагаатай холбоо бүхий био хүрээний хөгжлийн тухайд ярьж болно. Эсвэл байгал орчин : лито хүрээ, тропо хүрээ зэрэг дэлхийн бусад бүрхүүлийг бүрдүүлдэг био хүрээ болон царцанги бодис зэрэг бүрэлдэхтэй холбоотой хүн төрөлхтний хөгжлийн тухай ярьж болно. Хүн байгалийн харилцамж нь тогтмол шинжтэй боловч орон зай, цаг хугацааны хувьд ч туйлын ялгавартай байдаг. Гэхдээ харагдаж буй олон талт байдлын ард ажиглан буй эдгээр бүх үзэгдлүүдэд хэвшмэл байх нэгдмэл зарчим нуугдан байдаг. Иймээс л асуудлыг чухамхүү ингэж тавьж байгаа юм.
Ертөнцийн мөн чанар маш олон янз, хүн төрөлхтний бусад сүүн тэжээлтнээс ялгагдах нь бас л олон янз, учир нь хүн байгалийн нутаг байхгүй, харин дээд палеолитийн үеэс эхлэн гаригийн бүхий л хуурай газарт тархан байршсан юм. Хүний дасан зохицох чадвар бусад амьтдынхаас хавьгүй илүү. Энэ нь газар зүйн янз бүрийн бүс нутгуудад, түүхийн янз бүрийн үеүдэд хүмүүс болон байгалийн бүрдлүүд (амьдрах орчин болон геобиоценоз) янз бүрээр харилцан үйлчилж ирсэн байна. Гэнэтийн зүйлийг судалж болдоггүй учраас энэ дүгнэлт өөрөө бол хэтийн төлөвгүй юм. Гэхдээ асуудлыг ангилалд оруулаад үзвэл бүх юм өөр болно. Материйн хөдөлгөөний байгалийн болон нийгмийн зүй тогтлуудын хооронд байнгын хамаарал байдаг. Тэгвэл үүний механизм нь юу вэ?, байгал болон нийгмийн огтлолцлын цэг нь хаана байна ? Ийм цэг байгаа бөгөөд өөрөөр байсан бол хүнээс байгалийг хамгаалах тухай асуудал үүсэхгүй байхсан билээ.
С.В.Калесник газар зүйг : 1) хүмүүсийн бүтээлийг судалдаг эдийн засгийн болон 2) дэлхийн бүрхүүлийг, түүний дотор био хүрээг судалдаг физик хэмээн хуваахыг санал болгосон билээ. Энэ нь маш ухаалаг ангилал юм. Байгал нь бидний бүтээж чадахгүй уул, ус, ой, тал, амьтан ургамлын шинэ төрлүүдийг бүтээдэг. Харин хүмүүс машин бүтээж, уран баримал барьж, ном зохиол бичдэг. Үүнийг байгал хийж чадахгүй юм.
Байгалийн болон хүний бүтээлүүдэд зарчмын ялгаа байдаг уу ? Тийм ээ ! Байгалийн элементүүд бие биедээ шилждэг. “Энэ чулуу эрт цагт дуугарч байсан, энэ ороонго өвс үүлэнд хөвж байсан”, Нар, манай Галактикийн одод, манай гаригийн гүн дэх радио задралаас эрчим хүчээ уудлан авч мөнхөд амьдардаг юм.
Дэлхий гаригийн био хүрээ нь тэлэлтэд тэмүүлж буй атомын биогенийн шилжилтийн замаар ертөнцийн энтропи буюу замбараагүй байдлыг ялж байдаг юм. Үүний эсрэгээр хүний бүтээсэн зүйлс хадгалагдах юмуу эвдрэн сүйрч болдог. Пирамидууд өнө удаан байж байна. Эйфелийн цамхаг тийм ч удаан байхгүй. Гэхдээ аль аль нь мөнхийн биш ажгуу. Чухам үүнд л био хүрээ болон ямар ч агуу хэмжээнд хүрсэн байлаа гэсэн техно хүрээний хоорондын зарчмын ялгаа байдаг
СУДЛАХ ЗҮЙЛ
Угсаатны тухай шинжлэх ухааны орчин үеийн төлөв байдлын тойм нь уншигчдыг гонсойлгодог. Энэ сэдвээр бичиж буй бүх зохиогчид, түүний дотор угсаатны зүйчид үнэн хэрэг дээрээ угсаатны жинхэнэ мөн чанарыг мэргэжлийн, язгуур угсааны зэргээр сольсоор байгаад угсаатны бодит байдлыг үгүйсгэсэнтэй адил болгодог. Угсаатан нь хүмүүст шууд үзэгдэл (феномен) байдлаар мэдрэгдэж байгаа зөвхөн тэр зүйлийг угсаатан оршин байгаа мэт ярьдаг. Нэгэн яруу найрагч “Өдөр ч, шөнө ч нар бидний өмнүүр хөдөлж байдаг гэсэн хэдий ч зөрүүд Галилей зөв юм” гэсэн байдаг. Үнэхээр ч угсаатанч буюу этнологи хүний хувьд анхлан харахад давагдашгүй мэт санагдах гутранги үзлийн зарим нэг үндэслэл байдаг юм.
Угсаатан судлал буюу этнологи бол шинээр төрж буй шинжлэх ухаан болно. Этнограф буюу угсаатны зүйн цуглуулга, ажиглалтын хуримтлал нь асуудал дэвшүүлээгүй шинжлэх ухаан учир утгагүй цуглуулга болон хувирах аюул нүүрлэсэн нь ойлгомжтой болж эхэлсэн ХХ зууны хоёрдугаар хагаст энэ шинжлэх ухааны хэрэгцээ үүссэн юм. Ийнхүү бидний нүдний өмнө нийгэм зүй болон угсаатан судлал хэмээх анх харахад хүн төрөлхтөн гэдэг нэг л зүйлийг сонирхдог, чингэхдээ огт өөр талаас нь судалдаг шинжлэх ухаанууд үүссэн юм. Энэ нь ч зүй ёсны билээ. Ямар ч хүн нэгэн зэрэг нийгмийн гишүүн, угсаатны гишүүн байдаг, гэхдээ энэ хоёр нь тийм ч адилхан биш юм. Иймээс угсаатан судлал буюу этнологи нь шинжлэх ухаан болохынхоо хувьд тодорхойлолт шаардана. Одоохондоо бол этнологи гэдгийг амьтны зан үйлийн тухай этологийн шинжлэх ухаантай төстэйгөөр угсаатны хамтлагийн зан үйлийн лугшилтийн тухай шинжлэх ухаан гээд хэлчихье. Энэхүү лугшилт нь бие хүний хувийн ухамсартай болон сэтгэл хөдлөлийн зүйл, уламжлал, хамт олны албадах үйлчлэл, гадаад нөхцөл байдал, газар зүйн орчны нөлөө, тэр ч байтугай түүхийн гэнэтийн хөгжил, давших явцаар өдөөгдөж болно. Ийм нарийн нийлмэл асуудлын учир начрыг олохын тулд үүнд нийцсэн арга зүй хэрэгтэй юм. Энэхүү арга зүй нь эсвэл, нийгмийн шинжлэх ухааны уламжлалт арга зүй, эсвэл байгал шинжлэлийн арга зүй байж болох юм. Шинжлэх ухааны цоо шинэ салбарыг барин авсан эрдэмтдийн өмнө тулгарсан бэрхшээлийг амжилттай даван туулахын тулд алийг нь сонгон авах ёстой вэ ?
Эхлээд “нийгмийн шинжлэх ухаан” гэдэг ойлголтыг тодотгоё. Дундад зууны үед Христианы ертөнцөд шинжлэх ухааны мэдээллийн цорын ганц төгс нэр хүнд бүхий эх сурвалж нь Библи ба Аристотелийн бүтээл гэсэн хоёр ном л байлаа. Шинжлэх ухаан нь яг таг гарган ирэх ёстой ишлэлүүдийг тайлбарлахтай дүйцэж байсан бөгөөд учир нь бичиг номгүй буруу номтнууд Христ болон Аристотелийн номлолоос иш татаж буй хэсгийг ямагт санаанаасаа ургуулан боддог байв. Эндээс л өнөө үе хүртэл тэсэн гарч чадсан сэдвээс иш татах систем үүссэн юм. Шинжлэх ухааны энэ түвшинг схоластик гэдэг бөгөөд XY зуун гэхэд энэ нь эрдэмтдийн санаанд нийцэхээ байжээ. Ингээд эх сурвалжийн хүрээг өргөтгөн эртний бусад зохиогчдын бүтээлээр нэмж, хэрэгцээтэй сэдвүүдийг шалган нягтлах болжээ. Ингэж сэдвүүдэд шүүмжлэлтэй ханддагаараа схоластикаас ялгаатай хүмүүнлэгийн (өөрөөр хэлбэл бурхны биш, хүний) шинжлэх ухаан болох филологи буюу хэл шинжлэл үүсчээ. Гэхдээ эх сурвалж бахь байдгаараа харь үгс байлаа. Сэргэн мандалтын эрин үеийн дараа томоохон байгал судлаачид мэдээлэл авах нийгмийн ухааны аргын оронд байгалийг ажиглах болон туршилтад үндэслэсэн байгал судлалыг сөргүүлэн тавьсан юм. “Эртний зохиогчид юу хэлэв ?” гэдэг асуултын оронд “үнэн хэрэг дээрээ ямар байв?” гэдгийг тайлбарлахыг оролдож, асуултын дарааллыг өөрчлөв. Судлах зүйл биш, харин хандлага, тухайлбал, арга зүй өөрчлөгдсөн нь харагдаж байна.
Шинэ арга зүйг удаан бөгөөд жигд бишээр хүлээн зөвшөөрсөн юм. Бүр 1633 онд Галилейг эсэргүүцэгчид ийм баримт бидэнд буй утга зохиолд байхгүй гэдгийг цохон зүтгэсний учраас тэрээр Дэлхий Нарыг тойрон эргэдэг гэсэн үгээсээ няцахад хүрсэн байна. XYIII зууны үед Францын шинжлэх ухааны академийн хуралдаан дээр Лавауазье солир унасан тухай мэдээллийг “шинжлэх ухааны бус” хэмээн мэдэгдээд “Тэнгэрээс чулуу унаж чадахгүй. Учир нь тэнгэрт чулуу байхгүй” гэж няцаасан байдаг.
Газар зүй нь зөвхөн XIX зуунд л амазонка хэмээх дайчин эмс, усан онгоцыг живүүлдэг аварга наймаалж, бусад домог зохиолуудаас ангижирсан бөгөөд эдгээрийг бүдүүлэг түвшинд буй уншигчид шууд утгаар нь хүлээн авдаг байсан юм. Туршилт хийж болохгүй, ажиглалтыг давтаж чадахгүй байсан түүхчид хамгийн ихээхэн зовлонтой байв. Гэтэл харьцуулсан болон дотоод эх сурвалжийн аль алийг нь шүүмжилж болох монист хандлага бий болов. Олон тооны цуцалтгүй судалгааны үр дүнд он цагийн хэлхээсэнд буй маргашгүй баримтуудын журмыг гаргаж, эргэлзээтэй мэдээллийн хэсгийг авч хаяв. Мэдлэгийн энэхүү асар их баялаг нь бидний сонирхож буй социал нийтлэг – анги юмуу, төр мэтийн улс төрийн бүхэллэг, эсвэл угсаатан зэрэг тодорхой объектод хавсаргасан тохиолдолд ашиг тусаа өгч байв. Угсаатантай холбоотой тохиолдолд түүхийн баримтууд нь “мэдээллийн архив” болон хувирч, газар зүй, биологи, биофизик, биохимийн зэрэг мэдээлэлтэй нийлэн этнологийн шинжлэх ухааны зорилгод үйлчилдэг бөгөөд бүтээлч нийлэгжилт байгаа тохиолдолд анхдагч материал хуримтлах үед тэмдэглэгдсэн хангалттай тооны, үнэн магад ажиглалт дээр үндэслэгдсэн байгалийн шинжлэх ухаан болох этнологийг тайлбарлах боломж өгөх юм.
Одоо бид угсаатны зүйг дүрслэлийн, эсвэл онолын гэж үзэж болох уу, газар зүй анхаарлын гадна үлдсэн үү, түүх шинжлэх “ухааны хүрээнд” бүхэлдээ хамаарагдах уу гэсэн тулгуур сэдэвдээ эргэн оръё. Үгүй, бас дахин үгүй. Ийм байр суурь нь бидний бодлоор ул суурьгүй, утга учиргүй юм. Энэ нь шинжлэх ухааныг явцууруулж, өөрөөр хэлбэл эрдэм шинжилгээний ажилтны мэдлэгийг багасгасан явцууралд хүргэнэ. Мэдээж хэрэг энэ нь түүнд амар байх боловч түүний ажил нь хэтийн шинжээ алдаж, уншигчдад уйтгартай байх болно. Энд дэвшүүлсэн сэдвийг хүчлэн зөвшөөрөхгүй байх нь боловсруулсан түүхийн арга зүйг өөр зүйлд хэрэглэн гуйвуулахад хүргэх төдийгүй, угсаатны судлалын шинжлэх ухаанд хөнөөлтэй юм. Иймээс энэ шинжлэх ухаанд хөгжлийн ганцхан арга зам байгаа бөгөөд энэ нь материал цуглуулж, дүрслэхтэй зэрэгцүүлэн тавигдан буй асуудлаар нөхцөлдсөн тэрхүү өнцгөөр түүнийгээ тайлбарладаг болгон угсаатны зүйг хувиргах явдал юм.
ГҮН УХААНД ХИЙХ АЯЛАЛ
Энд маш товч байх ёстой. Бид угсаатан бүрэлдэхдээ байгалийн үзэгдэл гэж үзсэн учраас түүнийг судлах үндэс нь зөвхөн гүн ухаан, байгал шинжлэл, өөрөөр хэлбэл зөвхөн диалектик материализм байж болно. Түүхэн материализм нь нийгмийн хөгжлийн зүй тогтлыг нээх зорилготой. Өөрөөр хэлбэл К.Марксын хэлснээр хүмүүсийн биенд буй байгалийн түүхийг биш, хүмүүсийн түүхэнд хамаарна. Энэ хоёр “түүх” нь харилцан сүлжилдсэн, харилцан холбоотой боловч шинжлэх ухааны шинжилгээ нь харах өнцгийг тодосгох, өөрөөр хэлбэл асуудлын талуудыг нарийсгахыг шаарддаг.
Бидний сонирхлыг татаж буй түүхэн материалууд нь бидний мэдээллийн архиваас илүү зүйл биш. Шинжилгээний зорилгын хувьд энэ нь зайлшгүй бөгөөд хангалттай. Энэ асуудлаар К.Маркс “Түүх нь өөрөө байгалийн, байгал хүнээр бий болсон түүхийн бодит хэсэг юм. Хожим нь байгал судлал хүний тухай шинжлэх ухааныг өөртөө оруулсан бөгөөд чингэхдээ хүний тухай шинжлэх ухаан байгал шинжлэлийг өөртөө оруулсан тэр хэмжээгээр л оруулсан юм. Энэ нь нэг шинжлэх ухаан болно” гэжээ. Эдүгээ бид ийм шинжлэх ухаан бүтээхийн босгон дээр зогсож байна.
Синтезийн тухай ярих болоход асуудалд хандах хандлага харгалзан өөрчлөгддөг юм. Гэхдээ хэн бүхний мэдэж байгаагаар синтезээс өмнө анализ хийдэг ёстой билээ, хэт түрүүлэх шаардлага энд байхгүй юм. Зөвхөн ингэж үзсэн тохиолдолд л шинжлэх ухааны материалист байгал шинжлэл гуйвшгүй үлдэж чадна. Нэр томъёоны утга, арга зүйн шинж чанарын талаар ингэж тохироод тавигдсан асуудал руугаа эргэж оръё.
ХҮН ТӨРӨЛХТӨН HOMO SAFIENS – ИЙН ТӨРӨЛ БОЛОХ НЬ
“Хүн ба Дэлхий”, эсвэл “Хүн ба байгал” гэж ярьж заншсан бөгөөд хэдийгээр дунд сургуулийн үеэс энэ бол Дундад зуунаас эхтэй энгийн, бүдүүлэг антропоцентризм буюу хүн төвт үзэл гэж тайлбарладаг байсан. Мэдээж хүн техникийг бүтээсэн, гэхдээ динозавр, мезозойн эринийг, мөн кайнозойн эриний махайродусыг бүтээгээгүй. Гэхдээ ХХ зууны бүхий л амжилтын үед ч бидний хүн бүр бодьгалийн хувьд ч, зүйлийн хувьд ч амьдралын агуулгыг бүрдүүлж байдаг байгалийг дотроо тээж явдаг юм. Бусад бүх нөхцөлд адил хүмүүсийн хэн нь ч амьсгалах, идэх, үхлээс зугтах, үр удмаа хамгаалах зэргээс татгалздаггүй. Хүн нь Дэлхий хэмээх гаригийн нэгэн бүрхүүл болох био хүрээний дотроо, амьтны зүйлийнхээ хүрээн дотроо үлдсэн байна. Хүн өөрт нь буй амьдралын хуулийг техникийн болон соёлын өвөрмөц үзэгдлүүдтэй хамтад нь багтааж, тэдгээрийг баяжуулдаг хэдий ч үүнийг төрүүлэгч байгалийн хүчний хамаарлаас салаагүй байна.
Хүн төрөлхтөн бол биологийн хэлбэр бөгөөд дэлхийн бөмбөрцгийн бүхий л гадаргууд мөстлөгийн дараах эрин үед тархсан, асар олон ялгамж бүхий нэгдмэл зүйл болно. Зүйлийн тархалтын нягтрал нь янз бүр хэдий ч туйлын мөснөөс бусад нийт Дэлхий бол хүний оршин амьдрах нутаг мөн. Хүний хөл хүрээгүй “онгон” газар хаа нэгтээ байгаа гэж бодох хэрэггүй. Өнөөгийн цөл газар, ширэнгэ зэрэг нь палеолитийн үеийн бууцаар дүүрэн байсан, Амазонкийн ой шугуй нь эртний хүмүүсийн газар тариалангаар эвдэрч атаршуулсан хөрсөн дээр ургаж байгаа, тэр ч бүү хэл Анд болон Гималайн нуруунд бидэнд ойлгомжгүй байгууламжууд олдож байна. Өөрөөр хэлбэл, өөрийн оршихуйн үеүдэд Homo Safiens дэлхийн гадаргуу дээрх тархалтаа хэд хэдэн удаа, байнга өөрчилж ирсэн байна. Тэрээр бусад олон зүйлийн нэгэн адил аль болох их нягт бүхий аль болох их орон зай эзэмшихийг эрмэлзэж иржээ. Гэхдээ л ямар нэг юм түүнд саад болж, түүний боломжийг хязгаарлаж байсан байна. Энэ нь юу вэ ?
Ихэнхи сүүн тэжээлтнээс ялгаатай нь Homo Safiens –ийг сүргийн амьтан ч гэж, орь ганц амьтан ч гэж нэрлэж болохгүй юм. Хүн нэг бол социум буюу нийтлэгийн гишүүн, нэг бол угсаатан гэж авч үзсэн өнцгөөсөө хамааралтайгаар хамт олны дотор оршин байдаг. Нарийн хэлбэл, хүн бүр нэгэн зэрэг нийгмийн гишүүн, овог аймгийн төлөөлөгч болж байдаг, гэхдээ энэ хоёр ойлголтыг жишиж болдоггүй бөгөөд жишээлбэл, урт болон жин, эсвэл халаалтын зэрэг болон цахилгаан цэнэг лугаа өөр өөр хавтгайд оршин байдаг.
Хүний социал хөгжлийг сайн судалсан, түүний зүй тогтлыг түүхэн материализм томъёолж өгсөн. Нийгэм–эдийн засгийн формацийг дамжсан социал хэлбэрүүдийн гэнэтийн хөгжил нь хамт олны дунд байх хүнд л зөвхөн заяасан бөгөөд түүний биологийн бүтэцтэй ямар ч холбоогүй юм. Эл асуудал нэн тодорхой бөгөөд үүн дээр зогсох нь утга учиргүй болно. Харин нэр томъёоны будлиан гарахаас сэргийлж бидний этнос буюу угсаатан гэж цаашид нэрлэх овог аймгийн асуудал нь нэн бүрхэг бөгөөд утгагүй зүйлээр дүүрэн байдаг. Харин угсаатнаас гадуур Дэлхий дээр нэг ч хүн байхгүй гэдэг ганц зүйл л эргэлзээгүй юм. “Чи хэн бэ ?” гэсэн асуултад “орос”, “франц”, “перс”, “масаи” гэх мэтээр ганц ч минут бодолгүйгээр хариулна. Эндээс ухамсар дахь угсаатны хамаарал нь түгээмэл үзэгдэл болдог байна. Ингээд бүх зүйл дуусахгүй ээ.
“УГСААТАН” ГЭХ ОЙЛГОЛТЫН ТОДОРХОЙЛОЛТ
Дээр дурдсан хариултдаа хүн бүр ямар утга холбогдол, хамгийн гол нь ямар санаа агуулсан бол ? Тэрээр өөрийн ард түмэн, үндэстэн, ястан гэж юуг нэрлэж байна, тэрээр хөршөөсөө юугаараа ялгаатайг хэрхэн харж байна гэдэг бол угсаатны задлан шинжилгээний өнөө болтол шийдвэрлэж чадаагүй асуудлууд юм. Ахуйн түвшинд бол гэрэл сүүдэр, халуун хүйтэн, гашуун амттайн ялгааг тодорхойлохыг шаарддаггүйн адил энд асуудал байдаггүй. Өөрөөр хэлбэл, энд шалгуур нь мэдрэхүй болно. Өдөр тутмын амьдралд энэ хангалттай боловч ойлгоход багадах юм. Эндээс тодорхойлох шаардлага үүсч байна. Гэхдээ энд олон янзын зүйл үүсдэг. “Угсаатан бол гарал үүслийн нийтлэгийг тодорхойлсон үзэгдэл мөн”, “Угсаатан бол нийтлэг хэлний суурь дээр соёл бий болох үйл явц”, “Угсаатан бол өөр хоорондоо төстэй хүмүүсийн бүлэг”, “Угсаатан бол нийтлэг өөрийн ухамсраараа нэгдсэн хүмүүсийн бөөгнөрөл”, “Угсаатан бол аль нэгэн формацаас хамаарах хүмүүсийг нэгтгэсэн нөхцөлт ангилал” (энэ нь угсаатны категори бодитой биш гэсэн үг юм), “угсаатан бол байгал бий болох үйл явц”, “угсаатан бол социал ойлголт”.
Хүний байгалийн болон нийгмийн зүйлүүдийн харьцааны талаарх зөвлөлтийн эрдэмтдийн янз бүрийн үзэл бодлыг нэгтгэн дүгнэж үзвэл түүнийг гурван хэсэгт хувааж болох байна. 1.“Нэгдсэн” газар зүй нь хүний бүхий л үйл ажиллагааг байгалийн зүй тогтолд оруулж байна. 2. Зарим түүхч, угсаатны зүйчид хүн төрөлхтөнтэй холбоотой бүхий л үзэгдлийг анатоми, зарим талаар физиологоос бусдыг нь социал үзэгдэл хэмээн үзэж байна. 3. Антропогенийн үйл явцад материйн хөдөлгөөний нийгмийн хэлбэр болон байгалийн бүрдлийн (механик, физик, хими, биологийн) бий болохыг ялган үздэг байна. Зохиогчийн хувьд сүүлчийн энэ концепци нь цорын ганц зөв үзэл юм.
Энд нэрт археологич, хазарын түүхч М.И. Артамоновын үзэл онцгой байр суурь эзэлдэг. Удаан жилийн археологийн, өөрөөр хэлбэл өөрийн хөгжлөөсөө хагацсан, гэхдээ цаг хугацааны туршид эвдрэн сөнөж буй сөнөсөн хөшөө дурсгалтай ноцолдсоны үр дүнд төрсөн түүний бодлоор “Угсаатан нь ангитай адил, гэхдээ социал байгууллага биш, харин төлөв байдал юм. Энэ үед хүний байгалаас хамаарах нь бага байх тусам түүний соёлын түвшин өндөр байдаг нь мэдээжийн үнэн юм. Үүнийг зөвшөөрөхөд хэцүү юм.
Сүүлчийн сэдвээс эхлэе. Хүний бие махбодь Дэлхийн био хүрээнд багтах ба биоценозын буюу ижил нөхцөлд буй амьтан ургамлын солилцоонд оролцдог. Профессор хүн бушменаас өөрөөр амьсгалдаг, эсвэл бэлгийн бус замаар үрждэг, эсвэл арьс нь хүхрийн хүчлийг мэдэрдэггүй, тэрээр юм иддэггүй, эсвэл 40 хүний хоол иддэг, эсвэл түүнд газрын татах хүч өөрөөр үйлчилдэг гэдгийг хэн ч нотолж чадахгүй юм.
Энэ бүхэн нь ажилладаг, сэтгэдэг, өөрчлөгдөн буй орчинд дасан зохицдог, орчноо өөрчилдөг, түүнийг өөрийн хэрэгцээнд зохицуулдаг, хамт олон болон нэгддэг, тэр бүрэлдэхүүндээ төр улсаа байгуулдаг хүний бие махбодийн байгалаас хамаарах хамаарал билээ. Сэтгэн боддог хувь чанар нь бие махбодьтой нэгэн бүхлийг бүрдүүлдэг, ингэснээрээ Дэлхий гаригийн нэгэн бүрхүүл болсон амьд байгалийн хүрээний чанадад гардаггүй болно. Үүний хамт хүн нь зэвсэг бэлтгэдгээрээ бусад амьдаас ялгагдах бөгөөд чанарын хувьд өөр хүрээ болох техно хүрээг бүтээдэг.
Хүн гараараа мохоо зүйлийг ч, амьд бодисын зүйлийг ч хийх болсноор (зэвсэг, урлагийн бүтээл, гэрийн амьтан, хүнсний ургамал) биоценезийн солилцооны хэмнэлээс салан гарсан юм. Эдгээр нь эсвэл хадгалагдан үлдэнэ, эсвэл хэрвээ зориудаар нөөцлөөгүй бол эвдрэн сүйрнэ. Сүүлчийн тохиолдолд тэдгээр нь байгалийн өвөрт буцан очно. Дайны талбарт хаясан сэлэм зэвэрч, төмрийн исэл болон хувирна. Эвдэрсэн цайз овоо болно. Зэрлэгшсэн нохой динго болно, харин адуу бол мустанг болно. Юмсын (техно хүрээний ) энэхүү үхэл нь байгал өөрөөс нь булаан авсан материалыг буцаан авч буй хэрэг юм. Байгал нь техникийн сүйтгэлийг хэдийгээр тэвчих боловч мэдээжийн эцсийн бүлэгт эргэлт буцалтгүй өөрчлөгдсөнөөс бусад өөрийн болгоноо авдаг гэдгийг эртний соёл иргэншлүүдийн түүх харуулж байна. Палеолитийн үеийн цахиур зэвсэг, Бальбект хийсэн хавтангууд, бетонжуулсан талбай, пластмасс эдлэл зэрэг ийм байдаг. Мөн энэ нь био хүрээ хэвлийдээ буцаан авч хүч хүрдэггүй шарил, муми зэрэг болно. Хэрэв манай гаригт сансрын сүйрэл болсон тохиолдолд химийн болон дулааны гэх мэт тогтонги бодисын үйл явц нь тэдгээрийг анхны байдалд нь буцааж чадна. Тэр болтол эдгээрийг соёл иргэншлийн дурсгалууд хэмээн нэрлэж болно, харин манай техникүүд ч хэзээ нэгэн цагт хөшөө дурсгал болно.
С.В.Калесникийн санал болгосон ангиллыг үндэс болгон бид түүнд дотор угсаатны үзэгдлийн байрыг олох ёстой. Урьдчилаад хэлэхэд угсаатан гэдэг бол био хүрээ, социо хүрээ хоёрын зааг дээр орших, Дэлхийн био хүрээний бүтцэд нэн өвөрмөц зориулалт бүхий үзэгдэл юм. Энэ тодорхойлолт тунхаглал мэт харагдаж байг яахав. Гэхдээ өнөөдөр уншигч энэ номыг юуны тулд бичсэнийг болон зохиогч зүгээр л томъёолол өгөхийг зорилгүй, харин энэ ном шинжлэх ухааны өнөөгийн түвшинд шинжлэх ухааны таамаглалд тавигдах бүхий л шаардлагад нийцсэн гэдэгт итгүүлэх суурь үндэст өөрийн хүрсэн бүх замыг үзүүлсэн гэдгийг уншигчид мэдэх болно.

